\chapter{Tabulation}\label{ch-tabulation}

\section{The First Tangle Datasets}

This chapter will describe the methodology used to answer two of the essential
questions detailed in Section~\ref{sec-intro-overview}.

\begin{quote}
  ``How do I systematically construct rational tangles?'', ``How do I tell two
  rational tangles I make apart?'', and ``How do I generate new
  rational tangles?''
\end{quote}

\begin{quote}
  ``How do I systematically construct Montesinos tangles?'', ``How do I tell two
  Montesinos tangles I make apart?'', and ``How do I generate new Montesinos
  tangles?''
\end{quote}

The methodologies outlined in this section are for relatively simple classes of
tangles, the rational and Montesinos tangles. As we progress, we will see a
common solution pattern that outlines the general approach to the more difficult
algebraic/arborescent case in Section~\ref{sec-arborescent}. That
approach takes the form of
the cadence:

\begin{enumerate}
  \item Define the object.
  \item Define equivalence for the object.
  \item Identify a unique representation.
  \item Generate those unique representatives\footnote{Each tangle
      can be represented by an infinite number of diagrams. A unique
      representative in this context means a particular ``flavor'' of
      diagram that
    exists for every tangle in the class we are concerned about.}.
\end{enumerate}

%  prettier-ignore-start

\subsection{Rational Tangles}\label{sec-rational}

%  prettier-ignore-end

We will see in this section that \textbf{rational tangles},
originally described by
Conway \citep{conwayEnumerationKnotsLinks1970}, have a deeply
combinatorial nature,
making them among the simplest classes of tangle. This simplicity leads to the
rational tangles, and their knot closures, being some of the most commonly
studied objects in knot theory.

\subsubsection{Construction}

In our development of the modified tangle calculus
(Section~\ref{subsec-tangle_operations})
we described a way to glue tangles together, allowing us to take simple objects
and build complex objects. That approach forms the basis for our construction of
the rational tangles.

We start, with an intuitive description of the construction. Imagine a zero
tangle (Figure~\ref{prelim-fig-basic_0}), now attach to that tangle a
crank on the right
(east) side (Figure~\ref{fig-rat_tang-crank}). We allow, for a
moment, the fixed points of
the tangle to move. If we crank a half turn clockwise or anti-clockwise, we
introduce a twist, if we turn the crank $n$ half turns we make an integral
$n$ tangle
(Section~\ref{subsec-integral_tangle}).

\begin{figure}[H]
  \centering
  \includegraphics[width=0.5\linewidth]{files/fig-rat_tang-crank-8529a02a85dbb7212e706bdd68e2a9b4.pdf}
  \caption[A set of three turns, changing a basic 0 tangle into an
  integral 3 tangle.]{A set of three turns, changing a basic 0 tangle
  into an integral 3 tangle.}
  \label{fig-rat_tang-crank}
\end{figure}

When we have completed $n$ turns we take the crank and move it to the bottom
(south) side of the tangle. We turn the crank to add $n$ half turns, this time
building a vertical integral tangle. Continuing this, alternating between right
and bottom sets of half turns, as seen in
Figure~\ref{fig-rat_tang-crank_many}, we create a \textbf{rational
tangle}. For a rational tangle $T$, the list of counts for right and bottom
twists is the \textbf{twist vector}
(Definition~\ref{rational-def-twistvector}) of the rational
tangle.

\begin{figure}[H]
  \centering
  \includegraphics[width=0.5\linewidth]{files/fig-rat_tang-crank_m-bbfbb8ad5a58ffb705d4dab5ff1ebff7.pdf}
  \caption[The set of alternating turns building a rational
  tangle.]{The set of alternating turns building a rational tangle
  $\LB3\ 2\ 1\RB$.}
  \label{fig-rat_tang-crank_many}
\end{figure}

Formalizing this intuitive construction requires the $\vee$ and $+$ operations
(Section~\ref{subsec-tangle_operations}), as well as the integral tangles
(Section~\ref{subsec-integral_tangle}). This formalization was
originally stated by Conway
\citep{conwayEnumerationKnotsLinks1970} but was ultimately proved by Burde and
Zieschang, we will give a later construction by Kauffman and Goldman
\citep{goldmanRationalTangles1997}. For convenience, we denote a
horizontal integral
tangle with $a\in\Z$ crossings as $t_a$, similarly, a vertical integral tangle
with $b\in \Z$ crossings as $t_b^\prime$.

\begin{definition}{Kauffman and Goldman, Page 310
    \textbf{\citep{goldmanRationalTangles1997,
  conwayEnumerationKnotsLinks1970}}}{rational-def-rational}
  To build a \textbf{rational tangle}, take one of the following constructions:

  \begin{enumerate}
    \item Start with a horizontal tangle $t_{a_0}$. Add by $\vee$ a
      vertical tangle
      $t_{b_0}^{\prime}$ on the bottom, then add by $+$ a horizontal tangle
      $t_{a_1}$ on the right, then add by $\vee$ a $t_{b_1}^{\prime}$ on the
      bottom, and so on, stopping after a finite number of such steps at a
      horizontal tangle $t_{a_n}$.
    \item Start with a vertical tangle $t_{b_0}^\prime$. Add by $+$ a horizontal
      tangle $t_{a_0}$ on the right, then add by $\vee$ a vertical tangle
      $t_{b_1}^\prime$ on the bottom, then add by $+$ a $t_{a_1}$ on
      the right, and
      so on, stopping after a finite number of such steps at a
      horizontal tangle $t_{a_n}$.
  \end{enumerate}

\end{definition}

\begin{note}
  In the first case the twist vector has an odd number of entries and in the
  second the twist vector has an even number of entries.
\end{note}

To see the alignment between Definition~\ref{rational-def-rational}
and our intuitive
construction, view the turning of the crank as corresponding to horizontal and
vertical integral tangles, and the alternating of right and bottom as
corresponding to alternating $+$ and $\vee$.

\paragraph{Correspondence With Extended Rational Numbers}

Now we formally address our first two essential questions by:

\begin{enumerate}
  \item Describing a notation for rational tangles.
  \item Demonstrating a correspondence between the rational tangles
    and the extended rational
    numbers.
  \item Showing that the correspondence distinguishes (tells apart)
    rational tangles.
\end{enumerate}

The answer to our first essential question, ``How do I systematically construct
rational tangles?'', is seen by formalizing the twist vector notational strategy
we saw in our intuitive formulation for rational tangles. This allows us to
systematically write down a rational tangle by a list of integers.

\begin{definition}{Conway, Page 332
    \textbf{\citep{conwayEnumerationKnotsLinks1970,
  goldmanRationalTangles1997}}}{rational-def-twistvector}
  The list of integers of the sets of tangles $t_{a_i}$ and $t_{b_j}^\prime$
  from Definition~\ref{rational-def-rational} ordered as in
  Equation~(\ref{rational-def-math-tv}) or
  (\ref{rational-def-math-tv2}) is called a
  \textbf{twist vector}.

  \begin{equation}
    \label{rational-def-math-tv}
    \LB a_0\ b_0\ a_1\ b_1\ \cdots\ a_n\RB
  \end{equation}
  \begin{equation}
    \label{rational-def-math-tv2}
    \LB b_0\ a_0\ b_1\ \cdots\ a_n\RB
  \end{equation}

\end{definition}We are now ready to answer the second of the
essential questions, ``How do I tell
two rational tangles I make apart?'' The critical observation to answer the
questions is due to Conway's use \citep{conwayEnumerationKnotsLinks1970} of the
entries of a twist vector as the entries for a continued fraction
(Definition~\ref{rational-def-frac}). Since the twist vector is of
finite length, this continued
fraction corresponds to a rational number \citep{rockettContinuedFractions1998}.

\begin{definition}{Conway, Page 332
  \textbf{\citep{conwayEnumerationKnotsLinks1970}}}{rational-def-frac}
  For a rational tangle $T$ we call
  Equation~(\ref{rational-math-frac}) the \textbf{fraction of a
  rational tangle} and denote it $F\LP T\RP$.
  \begin{equation}
    \label{rational-math-frac}
    \frac{p}{q} = a_{n} + \cfrac{1}{b_n + \cfrac{1}{a_n + \cfrac{1}{b_{n-1} +
    \cfrac{1}{a_{n-1} + \cfrac{1}{\ddots\,+\cfrac{1}{a_0}}}}}}
  \end{equation}

\end{definition}

\begin{note}
  The correspondance with the extended rational numbers means that
  $F\LP\LB1\ \m 1\ 0\RB\RP=\frac{1}{0}$ corresponds to the basic $\infty$ tangle
  (Figure~\ref{prelim-fig-basic_nc-inf}).
\end{note}

Kauffman and Goldman \citep{goldmanRationalTangles1997} prove
(Theorem~\ref{rational-thm-conways})
that this correspondence distinguishes, tells apart, rational tangles. Meaning,
given two rational tangles, if their fractions are the same the tangles are
isotopic, and if the fractions are different, the tangles are not isotopic. This
answers the second of our essential questions.

\begin{theorem}{Conway's Theorem, Kauffman and Goldman page 315
    \textbf{\citep{conwayEnumerationKnotsLinks1970, burdeKnots2013,
  goldmanRationalTangles1997}}}{rational-thm-conways}
  Let $T_1$ and $T_2$ be rational tangles. $F\LP T_1\RP=F\LP T_2\RP$ if and only
  if $T_1$ is ambient isotopic to $T_2$.

\end{theorem}Observe, Theorem~\ref{rational-thm-conways} does not
discount the possibility of two
non-equivalent twist vectors, those differing in at least one entry,
representing the same rational tangle.
Example~\ref{rational-ex-tvssamebutdiff} demonstrates
that the possibility is true. In fact, for each rational tangle, the set of
twist vectors representing it is infinite.

\begin{example}{}{rational-ex-tvssamebutdiff}
  \begin{equation}
    \begin{aligned}
      F\LP\LB 1\ 7\ 0\ 1\RB \RP &= \frac{9}{1}\\
      F\LP\LB \m 3\ 1\ \ \m 1\ \m 1\ \m 1\ 1\ 9\RB \RP &= \frac{9}{1}\\
      F\LP\LB 1\ \m1\ 1\ \m1\ 7\ 0\ 3\RB \RP &= \frac{9}{1}\\
      F\LP\LB 1\ \m1\ \m1\ 1\ \m1\ 1\ \m1\ 1\ \m1\ 9\RB \RP &= \frac{9}{1}\\
      F\LP\LB 1\ 8\RB \RP &= \frac{9}{1}\\
      F\LP\LB 9\RB \RP &= \frac{9}{1}\\
    \end{aligned}
  \end{equation}

\end{example}To effectively answer our third essential question,
``How do I generate rational
tangles?'', we will need to determine a unique twist vector representative for
each rational tangle. A unique representative allows us to simply and
efficiently write down each tangle without risk of duplicates showing up on our
list.

\subsubsection{Canonical Twist Vectors}

Identifying a unique representative will stem from properties of finite
continued fractions. We start by defining a specific subclass of finite
continued fractions with integer coefficients, the \textbf{regular continued
fractions}. We frame the definition in the context of twist vectors.

\begin{definition}{Kauffman and Goldman, page
    318\textbf{\citep{kauffmanClassificationRationalKnots2002,
  rockettContinuedFractions1998}}}{}
  A continued fraction with integer coefficients $c_i$ is called a
  \textbf{regular
  continued fraction} if $c_i< 0$ for every coefficient or $0< c_i$ for
  every coefficient except the last which may be 0.

\end{definition}Conveniently, each rational number corresponds to
exactly two regular continued
fractions \citep{rockettContinuedFractions1998}. The first is a twist
vector with the
leftmost element greater than or equal to 2, and the second with leftmost
element equal to 1, per Equation~(\ref{rational-math-twotv}).
Observe, one of these
twist vectors has an even number of entries, and one has an odd number of
entries.

\begin{equation}
  \label{rational-math-twotv}
  \begin{aligned}
    F\LP\LB 9\RB \RP &= \frac{9}{1}= 8+\frac{1}{1} = F\LP\LB 1\ 8\RB \RP \\
    F\LP\LB 9\ 0\RB \RP &= 0+\frac{1}{9}=0+\frac{1}{8+\frac{1}{1}}=
    F\LP\LB 1\ 8\ 0\RB \RP \\
  \end{aligned}
\end{equation}

\begin{note}
  The fraction $\frac{1}{0}$ corresponds to the $\LB 0\ 0\RB$ twist vector.
\end{note}

To identify a unique representative for a rational number and hence rational
tangle, we will select, for convenience, the odd length twist vector as our
unique representative.

\begin{definition}{Kauffman and Lambropoulou, Page 13
  \textbf{\citep{kauffmanClassificationRationalKnots2002}}}{rational-def-canonrat}
  A twist vector is called a \textbf{canonical twist vector} if it contains
  coefficients of a regular continued fraction and is of odd length or is
  $\LB0\ 0\RB$.

\end{definition}
\subsubsection{Computational Methods}

Armed with a unique representative for a rational tangle, we can construct our
computational answer to the third essential question.

\paragraph{Notation}

We start by describing how we will digitally store a rational tangle. In the
rational tangle case, the theoretical encoding strategy of twist vectors happens
to be well suited for computational storage. A twist vector can be
computationally stored identically to its written form, a list of space
separated integers delimited by a pair of square braces, $\LB\ \RB$. As we will
see in Section~\ref{sec-arborescent} this direct translation of
theoretical notation for tangles to
computational notation is not always the case.

\paragraph{Generation}

A common tactic in the knot tabulation space is to pare down the number of items
that must be tabulated by leveraging symmetries of the objects being tabulated.
For example, the $\LB 3\RB$ and $\LB \m 3\RB$ tangles are related to each other
by a minus operation (Section~\ref{subsubsec-conway_minus}). Meaning,
if we tabulate
$\LB 3\RB$, we can recover $\LB \m 3\RB$ with multiplication by $\m 1$. This
fact holds for all rational tangles as demonstrated by
Lemma~\ref{rationa-lema-pmtv}.
Allowing us to focus our efforts on the rational tangle with twist vectors
containing only positive entries.

\begin{lemma}{Kauffman and Lambropoulou, Page 19
  \textbf{\citep{kauffmanClassificationRationalKnots2002}}}{rationa-lema-pmtv}
  For a tangle $T$ with negative $-T$ (Section~\ref{subsubsec-conway_minus}), if
  $F\LP T\RP=\frac{p}{q}$ then $F\LP \m T\RP=\m\frac{ p}{q}$.

\end{lemma}We now begin our development of a generation strategy for
twist canonical twist
vectors of rational tangles. The method seen here utilizes a
common combinatorial method for defining compositions of integers,
the same is used for
rational tangle counting by Bryhtan \citep{bryhtanTabulating2stringTangles2024}.
Consider, for a given crossing number $n$, what is the most ``obvious'' twist
vector? A viable candidate for most ``obvious'' is a twist vector, of the form
seen in Equation~(\ref{rational-math-1s}), dropping the trailing 0 where needed.

\begin{equation}
  \label{rational-math-1s}
  \LB 1\ 1\ 1\ \cdots\ 1\RB
\end{equation}

The 1's twist vector (\ref{rational-math-1s}) is an ideal starting
point for developing
a generation strategy, as it distributes the data of a rational tangle as
broadly as possible. Next, we consider how we might transform the 1's twist
vector into a twist vector with a two crossing integral components. We can do
this by exchanging the space between the first and second 1 with a numeric
$+$.

\begin{equation}
  \label{rational-math-1plus}
  \begin{aligned}
    &\LB 1\square 1\ 1\ \cdots\ 1\RB\\
    &\LB 1+1\ 1\ \cdots\ 1\RB\\
    &\LB 2\ 1\ \cdots\ 1\RB\\
  \end{aligned}
\end{equation}

This process tells us that we can utilize the exchange of whitespace of a twist
vector for $+$ to generate new twist vectors. To complete our generation, we
must generate every combination of exchanges.

The 1's twist vector with crossing number $n$ has $n$ 1s and $n -1$ spaces. In
each space position, we have the option between a space and $+$. To enumerate
all $2^{n -1}$ combinations, we can simply count, in binary, from 1 to
$2^{n -1}$, as in Example~\ref{rational-ex-counting}.

\begin{example}{Combinations of exchanges for $n=4$ and their twist
  vector }{rational-ex-counting}
  \begin{equation}
    \begin{aligned}
      000\to \square \square \square\to &\LB1\ 1\ 1\ 1\RB \\
      001\to \square \square +\to &\LB1\ 1\ 2\RB \\
      010\to \square + \square\to &\LB1\ 2\ 1\RB \\
      011\to \square + +\to &\LB1\ 3\RB \\
      100\to + \square \square\to &\LB2\ 1\ 1\RB \\
      101\to + \square +\to &\LB2\ 2\RB \\
      110\to + + \square\to &\LB3\ 1\RB \\
      111\to + + +\to &\LB4\RB \\
    \end{aligned}
  \end{equation}

\end{example}Our final refinement in this process it to transform
this collection into
canonical twist vectors. Half of the twist vectors, those of odd length,
generated in this process are already canonical. To canonize the even length
twist vectors, we append 0 to the right most position of each list, turning
the even vectors into an odd canonical vectors.

\begin{note}
  Appending 0 to the even twist vectors makes each of the fractions for these
  twist vectors sit in the unit interval, $\LB 0,1\RP$.
\end{note}

We conclude the section with a set of algorithms that describe a method for
computationally generating all rational tangles up to a given crossing number.

\begin{remark}{Find all rational tangles of crossing number
  \textbf{$n$} }{find-rat-tang-of-n}
  \textbf{Input}

  \begin{itemize}
    \item A crossing number $n$
  \end{itemize}

  \textbf{Output}

  \begin{itemize}
    \item All collection $T$ of twist vectors
  \end{itemize}

  \textbf{Routine}

  \begin{enumerate}
    \item Generate $O$ the 1's twist vector as in
      Equation~(\ref{rational-math-1s}) for $n$
    \item for $i=0$ to $2^{n -1}$
      \begin{enumerate}
        \item Transform $i$ into its binary representation
        \item Exchange as in Equation~(\ref{rational-math-1plus})
          digits of $i$ with spaces where $i$
          is $O$ and with $+$ where $i$ is 1
        \item Apply operations to the 1s vector and store the
          resulting vector as $O_r$
        \item If $O_r$ is odd length, store $O_r$ in $T$
        \item Else, append 0 to $O_r$ and store
      \end{enumerate}
  \end{enumerate}

\end{remark}
\begin{remark}{Find all rational tangles up to crossing number
  \textbf{$n$}}{find-rat-tang-to-n}
  \textbf{Input}

  \begin{itemize}
    \item A crossing number $n$
  \end{itemize}

  \textbf{Output}

  \begin{itemize}
    \item All collection $T$ of twist vectors
  \end{itemize}

  \textbf{Routine}

  \begin{enumerate}
    \item Store the twist vectors $\LB 0\RB,\LB 0\ 0\RB,\LB 1\RB$
    \item for $i=2$ to $n$
      \begin{enumerate}
        \item Execute Algorithm~\ref{find-rat-tang-of-n} with $i$
      \end{enumerate}
  \end{enumerate}

\end{remark}
%  prettier-ignore-start

\subsection{Montesinos Tangles}\label{sec-monttang}

%  prettier-ignore-end

In this section, we will use the rational tangles to build a yet more complex
class of tangles, the Montesinos tangles. This building up process demonstrates
one of the core strategies for tangle tabulation.

\subsubsection{Construction}

With the rational tangles in hand, we wish to utilize that data to enumerate
additional tangles. One way we have seen to build simple objects into complex
objects is to combine two tangles with the $+$ or $\vee$ operation. To keep
complexity under control, we start with combining tangles with repeated $+$ sum.
When all summands, $R_i$ in Equation~(\ref{mont-math-def}) are rational,
we call the result of the sum a \textbf{Montesinos Tangle}
\citep{ernstTANGLEEQUATIONS1996, bonahonNewGeometricSplittings2016}.

\begin{equation}
  \label{mont-math-def}
  R_0+R_1+\cdots+R_n
\end{equation}

\begin{note}
  Under this characterization of the Montesinos tangles, every rational tangle
  including the integral tangles are Montesinos tangles with a single summand.
\end{note}

\paragraph{Unique Representative}

Next, we develop a classification of Montesinos tangles, allowing us to tell two
Montesinos tangles apart. For each rational summand $R_i$ in a
Montesinos tangle, we have four possibilities:

\begin{enumerate}
  \item $R_i$'s canonical twist vector is positive and ends in 0
  \item $R_i$'s canonical twist vector is positive and does not end in 0
  \item $R_i$'s canonical twist vector is negative and ends in 0
  \item $R_i$'s canonical twist vector is negative and does not end in 0
\end{enumerate}

Observe that in the second and fourth cases, $R_i$ terminates in a horizontal
integral tangle. In these cases, the tangle can be simplified by using the flype
(Section~\ref{subsubsec-opo-flype}) to move the horizontal crossings
to be the right most
summand, seen in Figure~\ref{mont-ex-flypesimple}. When this process
is carried out for each
summand of the type in cases two and four, the resulting summands all fall into
cases one and three.

\begin{figure}[H]
  \centering
  \includegraphics[width=\linewidth]{files/fig-mont-fc6488a6dd1ca987118be5ace13f3e88.pdf}
  \caption[A Montesinos tangle simplification.]{A Montesinos tangle
    \\$\LB1\ 2\ 2\RB+\LB \m1\ \m2\ 0\RB+\LB \m3\ \m2\ \m1\RB+\LB
    1\ 2\ 0\RB+\LB1\ 2\ 2\RB$
    simplifying to \\$\LB 1\ 2\ 0\RB+\LB \m 1\ \m 2\ 0\RB+\LB \m3\ \m
  2\ 0\RB+\LB1\ 2\ 2\RB+\LB1\ 2\ 0\RB+\LB 3\RB$}
  \label{mont-ex-flypesimple}
\end{figure}

We will now pare down to a single possibility, case one. Consider a summand
$R_i$ in case three, meaning $\m 1 <F\LP R_i\RP<0$ except for $R_n$, which may
be integral. Theorem~\ref{rational-thm-conways} tells us that if we
can find an alternative,
potentially non-canonical, twist vector that fits our needs, we are free to
exchange without impacting topology. What we would like is an odd length twist
vector, where every entry is positive, except for the rightmost, which is a
negative value, see Figure~\ref{mont-ex-equ-replace}.

\begin{figure}[H]
  \centering
  \includegraphics[width=0.5\linewidth]{files/3_2_-1-6b0c825b6d0643ae9067e9762d6f498c.pdf}
  \caption[A transformation of a canonical rational tangle.]{On the
    left rational tangle $\LB \m1\ \m2\ \m1\ \m1\ 0\RB$ and on the right
    $\LB 3\ 2\ \m1\RB$. These tangles have fractions
    $F\LP\LB \m1\ \m2\ \m1\ \m1\ 0\RB\RP=\frac{\m4}{7}$ and
  $F\LP\LB 3\ 2\ \m1\RB\RP=\frac{\m4}{7}$ showing the tangles to be isotopic.}
  \label{mont-ex-equ-replace}
\end{figure}

This ensures that the fraction is still negative but will allow us to flype the
terminal horizontal integral tangle to the right. Rockett and Sz\"usz
give a lemma
that establishes the existence of such a twist vector for each rational number.

\begin{lemma}{Rockett and Sz\"usz, Page 3
  \textbf{\citep{rockettContinuedFractions1998}}}{mont-lem-other_cfrac}
  Every rational number has a continued fraction with positive integer entries
  except for the first (rightmost twist vector entry) which is an integer.

\end{lemma}Full details on how the conversion from any twist vector to this
twist vector can be found in Rockett and Sz\"usz
\textbf{\citep{rockettContinuedFractions1998}}.
We have now shown that each case (2,3,4) can be transformed into the first case,
and we can define a canonical form for Montesinos tangles.

\begin{theorem}{Bonahon and Siebenmann, Theorem 11.7
  \textbf{\citep{nakanoEfficientGenerationPlane2002}}}
  Every non-rational Montesinos tangle $T$ admits a canonical diagram satisfying
  the following construction:
  \begin{equation}
    T \cong R_0+\cdots+R_m+\frac{k}{1}
  \end{equation}
  where each $R_i \cong \frac{p_i}{q_i}$ is a rational subtangle in
  canonical form with
  fraction satisfying $0<\frac{p_i}{q_i}<1$, and $\frac{k}{1}$ is a horizontal
  integer subtangle.

\end{theorem}
\subsubsection{Computational Methods}

\paragraph{Notation}

Before we can generate Montesinos tangles, we need to define an efficient
notation for computation and storage. Similar to what we saw in the rational
tangle case, the theoretical notation for Montesinos tangles is sufficient for
computation. However, with eyes on future computational work, we will generalize
our notation to increase reusability and the efficiency of storage..

Montesinos tangles are simple forms of the algebraic tangles
(Definition~\ref{def-algebraic}),
so we will build a notational strategy for general algebraic tangles.
The strategy seen here is similar to those found in Conolly
\citep{connollyClassificationTabulation2string2021}, Caudron
\citep{caudron1982classification}, and Gren, Sulkowska, and,
Gabrov\v{s}ek \citep{gren2025classificationalgebraictangles}. The
theoretical notation for algebraic tangles is outlined in
Section~\ref{subsec-tangle_operations} and seen in
Figure~\ref{tangle-alg-tree-2}.
We can simplify the notation, without losing fidelity, by substituting the
integral leaf tangles for rational tangle twist vectors. Additionally, we can
improve the storage overhead by storing the tree as a string in Polish notation
\citep{lukasiewiczElementyLogikiMatematycznej1929}. Storing in Polish
notation allows
us to drop all the parentheses from our notation, saving two bytes in each
instance.

\begin{figure}[H]
  \centering
  \begin{subfigure}[b]{0.45\textwidth}
    \centering
    \includegraphics[width=\textwidth]{files/mont-81522111e034b6335aedb3c82a3272ea.pdf}
    \caption{}
    \label{tangle-alg-tree-1}
  \end{subfigure}
  ~
  \begin{subfigure}[b]{0.45\textwidth}
    \centering
    \includegraphics[width=\textwidth]{files/mermaid-253f34ae-3e58cd025723b9ae2cdb797305599ac0.png}
    \caption{}
    \label{tangle-alg-tree-2}
  \end{subfigure}
  \caption[Conversion of a tangle to a algebraic tangle tree.]{The
    tangle in \ref{tangle-alg-tree-1}:
    \\$\cdot$Algebraically:$\LB1\ 2\ 0\RB+\LP\LB2\ 1\ 0\RB+\LB2\ 2\ 0\RB\RP$
    \\$\cdot$In polish
    notation:$+\LB1\ 2\ 0\RB+\LB2\ 1\ 0\RB\LB2\ 2\ 0\RB$ $\cdot$\\ As an
  algebraic tangle tree in \ref{tangle-alg-tree-2}}
  \label{tangle-alg-tree}

\end{figure}
\paragraph{Generation}

In this section, we will design an algorithm that allows us to efficiently
generate new Montesinos tangles up to a given crossing number. As we saw, the
construction of a Montesinos tangle is based on the repeated summation of
rational tangles. Consequently, our generation strategy will utilize our table
of rational tangles.

To start, we need a mechanism that allows us to select all possible combinations
of rational tangles with crossing numbers that sum to our target. For a
Montesinos tangle $T$, the set of rational components $\LS R_i\RS_i^n$ combined
with the integral component $k$ corresponds to an ordered list of crossing
numbers as in Equation~(\ref{mont-math-orderedlist}).

\begin{equation}
  \label{mont-math-orderedlist}
  CN\LP R_0\RP,\ \cdots,\ CN\LP R_n\RP,\  CN\LP k\RP
\end{equation}

We call a list of the form seen in (\ref{mont-math-orderedlist}) a
\textbf{stencil} for a
Montesinos tangle. By construction, every canonical Montesinos tangle relates to
precisely one stencil. Observe, that $2\leq CN\LP R_i\RP$ since $\frac{1}{2}$ is
the lowest crossing number rational tangle with fraction in the unit interval.

Generation for all Montesinos tangles of a given crossing number at this point
can be broken down into two steps:

\begin{enumerate}
  \item Generate all stencils
  \item Fill in the stencils with all rational tangles of the
    appropriate crossing
    number whose fraction is in the unit interval (plus an integral
    tangle in the rightmost position).
\end{enumerate}

For the first step, we require a mechanism for breaking an integer into all
possible combinations where the parts sum to the integer. Luckily, we have
already seen how to do this, in the context of a twist vector. We follow the
same counting algorithm outlined in
Algorithm~\ref{find-rat-tang-of-n} however, we modify the
algorithm to keep both the even and odd outputs, but filter out stencils with
entries less than two.

\begin{example}{}{ex-mont-stencils}The set of all possible stencils
  for crossing number 5:

  \bigskip\noindent
  \begin{tabular}{p{\dimexpr 0.250\linewidth-2\tabcolsep}p{\dimexpr
      0.250\linewidth-2\tabcolsep}p{\dimexpr
    0.250\linewidth-2\tabcolsep}p{\dimexpr 0.250\linewidth-2\tabcolsep}}
    \toprule
    $1\ 1\ 1\ 1\ 1$ & $2\ 1\ 1\ 1$ & $1\ 2\ 1\ 1$ & $1\ 1\ 2\ 1$ \\
    $1\ 1\ 1\ 2$ & $3\ 1\ 1$ & $1\ 3\ 1$ & $1\ 1\ 3$ \\
    $2\ 2\ 1$ & $2\ 1\ 2$ & $1\ 2\ 2$ & $3\ 2$ \\
    $2\ 3$ & $4\ 1$ & $1\ 4$ & $5$ \\
    \bottomrule
  \end{tabular}

  \bigskip$\!\,$The set of stencils for crossing number 5:

  \bigskip\noindent
  \begin{tabular}{p{\dimexpr 0.500\linewidth-2\tabcolsep}p{\dimexpr
    0.500\linewidth-2\tabcolsep}}
    \toprule
    $3\ 2$ & $2\ 3$ \\
    \bottomrule
  \end{tabular}

  \bigskip
\end{example}For the second step, for each entry in the stencil we
create a list of rational tangles in the unit interval with that entry for a
crossing number. For the rightmost entry of the stencil we also include the
horizontal integral tangles with crossing number equal entry. This creates all
combinations of input tangles given by the stencil.

\begin{example}{}{ex-mont-stencil_insert}The set of rational tangles
  of crossing numbers two and three:

  \bigskip\noindent
  \begin{tabular}{p{\dimexpr 0.500\linewidth-2\tabcolsep}p{\dimexpr
    0.500\linewidth-2\tabcolsep}}
    \toprule
    Two & Three \\
    \hline
    $[1\ 1\ 0]$ & $[2\ 1\ 0]$ \\
    $[2]$ & $[1\ 2\ 0]$ \\
    & $[3]$ \\
    \bottomrule
  \end{tabular}

  \bigskip$\!\,$The set of stencils for crossing number five, with
  rational tangles inserted in polish notation:

  \bigskip\noindent
  \begin{tabular}{p{\dimexpr 0.500\linewidth-2\tabcolsep}p{\dimexpr
    0.500\linewidth-2\tabcolsep}}
    \toprule
    $3\ 2$ & $2\ 3$ \\
    \hline
    $+[2\ 1\ 0][1\ 1\ 0]$ & $+[1\ 1\ 0][2\ 1\ 0]$ \\
    $+[1\ 2\ 0][1\ 1\ 0]$ & $+[1\ 1\ 0][1\ 2\ 0]$ \\
    \bottomrule
  \end{tabular}

  \bigskip
\end{example}We conclude the section with a set of algorithms that
describe this method for
computationally generating all Montesinos tangles up to a given crossing number.
\newpage
\begin{remark}{Find all stencils of crossing number
  \textbf{$n$}}{find-mont-sten}
  \textbf{Input}

  \begin{itemize}
    \item A crossing number
  \end{itemize}

  \textbf{Output}

  \begin{itemize}
    \item All collection $S$ stencils
  \end{itemize}

  \textbf{Routine}

  \begin{enumerate}
    \item Generate $O$ the 1's twist vector as in
      Equation~(\ref{rational-math-1s}) for $n$
    \item for $i=0$ to $2^{n -1}$
      \begin{enumerate}
        \item Transform $i$ into its binary representation
        \item Exchange as in Equation~(\ref{rational-math-1plus})
          digits of $i$ with spaces where $i$
          is $O$ and with $+$ where $i$ is 1
        \item Apply operations to the 1s vector and store the
          resulting vector as $O_r$
        \item Continue to the next iteration if $O_r$ has entries less than 2
        \item Add $O_r$ to $S$
      \end{enumerate}
  \end{enumerate}

\end{remark}
\begin{remark}{Find all Montesinos tangles of crossing number
  \textbf{$n$}}{find-mont-tang-of-n}
  \textbf{Input}

  \begin{itemize}
    \item All rational tangles of crossing number up to $n$
  \end{itemize}

  \textbf{Output}

  \begin{itemize}
    \item All collection $T$ of Montesinos tangles
  \end{itemize}

  \textbf{Routine}

  \begin{enumerate}
    \item Execute Algorithm~\ref{find-mont-sten} for $n$ and store in $S$
    \item for each stencil $s$ in in $S$
      \begin{enumerate}
        \item Retrieve lists $L=\LS L_i\RS_i^n$ of rational tangles
          for each stencil
          entry $s_i$.
        \item Add to the $L_n$ list the integral tangle $s_n$
        \item While there is a list in $L$ that is not exhausted.
          \begin{enumerate}
            \item Construct and store a Montesinos tangle from list entries.
          \end{enumerate}
      \end{enumerate}
  \end{enumerate}

\end{remark}
\begin{remark}{Find all non-rational Montesinos tangles up to
  crossing number \textbf{$n$}}{find-mont-tang-to-n}
  \textbf{Input}

  \begin{itemize}
    \item A crossing number
    \item All rational tangles up to $n$
  \end{itemize}

  \textbf{Output}

  \begin{itemize}
    \item All collection $T$ of twist vectors
  \end{itemize}

  \textbf{Routine}

  \begin{enumerate}
    \item for $i=4$ to $n$
      \begin{enumerate}
        \item Execute Algorithm~\ref{find-mont-tang-of-n} with $i$
      \end{enumerate}
  \end{enumerate}

\end{remark}
%  prettier-ignore-start

\section{Arborescent Tangles}\label{sec-arborescent}

%  prettier-ignore-end

This section describes the methodology we use to answer the final of the
essential questions detailed in Section~\ref{sec-intro-overview}.

\begin{quote}
  ``How do I systematically construct algebraic/arborescent
  tangles?'', ``How do I
  tell two algebraic/arborescent tangles I make apart?'', and ``How many
  algebraic/arborescent tangles can I create?''
\end{quote}

In this thesis so far we have worked with the algebraic tangles
(Definition~\ref{def-algebraic})
constructed with Conway's tangle arithmetic
(Section~\ref{subsubsec-conway_calc}).
In this section we will leverage a slightly different, but
equivalent, construction
given by Bonahon and Siebenmann \citep{bonahonNewGeometricSplittings2016} the
\textbf{arborescent tangles}. The one-to-one correspondence between
the classes will
become clear as we introduce the construction for arborescent tangles.

This section starts with an overview of Bonahon and Siebenmann's
\citep{bonahonNewGeometricSplittings2016} definition of arborescent
knots and tangles
(Section~\ref{prelim-arbor_def}). We then give their smooth and
combinatorial constructions
of arborescent knots and tangles (Section~\ref{subsec-wptt}). Next,
we give original work
extending Bonahon and Siebenmann's canonical construction to a local view
(Section~\ref{sec-CWPTT-def}). This local view is leveraged to define
a unique representative
for each class of arborescent tangles (Section~\ref{subsec-rlitt}).
Finally, we will describe
an original algorithm and notation that directly enumerates those unique
representatives (Section~\ref{subsec-computation}).

%  prettier-ignore-start

\subsection{Definition of Arborescent}\label{prelim-arbor_def}

%  prettier-ignore-end

We now give a high-level description of the manifold theory underpinning the
theory of arborescent knots and tangles. A full treatment of the manifold theory
can be found in Bonahon and Siebenmann
\citep{bonahonNewGeometricSplittings2016}.
Our first concept is that
of a \textbf{knot pair}, which serves as the underlying structure for
all the smooth
objects in this subsection.

\begin{definition}{Bonahon and Siebenmann, Page 15
  \textbf{\citep{bonahonNewGeometricSplittings2016}}}{prelim-def-arborescent-knot-pair}
  A \textbf{knot pair} is a pair $(M, K)$ where $M$ is an oriented
  connected compact 3
  manifold with (possibly empty) boundary, and where $K$ is a proper
  1-dimensional
  submanifold of $M$.

\end{definition}
For example a tangle $\LP B^3,T\RP$ is a knot pair.
\begin{definition}{Bonahon and Siebenmann, Page vi
  \textbf{\citep{bonahonNewGeometricSplittings2016}}}{prelim-def-arborescent-knot}
  A knot $(S^3, K)$ is \textbf{arborescent} if there exists a finite collection
  $F_1,\dots,F_n$ of disjoint Conway spheres such that, if $N$ is the
  closure (in the sense of a closure of a set) of
  any component of $S^3 - \cup_{i=1}F_i$ , then the pair $(N, K \cap
  N )$ takes the
  simple form of Figure~\ref{fig-arborescent_part} after suitable isotopic
  deformation in $S^3$.

  \begin{figure}[H]
    \centering
    \includegraphics[width=0.5\linewidth]{files/arborescent_knot-48d6fedd7bda45f063e639fe06adfa3e.pdf}
    \caption[A collection of Conway circles forming a arborescent
    vignette.]{A collection of Conway circles forming what we call an
    \textbf{arborescent vignette}.}
    \label{fig-arborescent_part}
  \end{figure}

\end{definition}
\begin{figure}[H]
  \centering
  \includegraphics[width=0.5\linewidth]{files/arborescent_band-5577f7920d0716dd044364979702873e.pdf}
  \caption[The arborescent vignette.]{The arborescent vignette from
  Figure~\ref{fig-arborescent_part} seen with Conway spheres.}
  \label{fig-arborescent_band}
\end{figure}

\begin{figure}[H]
  \centering
  \includegraphics[width=0.5\linewidth]{files/vin_1-d2a610c6959ee59775ba4a238da5e94a.pdf}
  \caption[The arborescent vignette showing a 1 crossing tangle.]{The
  arborescent vignette showing a 1 crossing tangle to be arborescent.}
  \label{fig-arborescent_vignette_1}
\end{figure}

\begin{note}
  The $F_i$ in Definition~\ref{prelim-def-arborescent-knot} are
  disjoint but may sit inside
  each other. This means we may have arborescent vignettes containing
  arborescent
  vignettes, but the closure of each individual component looks like
  Figure~\ref{fig-arborescent_part}.
\end{note}

Observe that arborescent knots are characterized by a collection of Conway
spheres (circles). Choosing to not fill one, or more, of these Conway spheres
yields a tangle.

\begin{definition}{Bonahon and Siebenmann, Page 144
  \textbf{\citep{bonahonNewGeometricSplittings2016}}}{intro-def-arbor-tangle}
  Define an \textbf{arborescent tangle} as one whose underlying knot
  pair $(M, K)$ (Definition~\ref{prelim-def-arborescent-knot-pair}) is
  arborescent in the sense defined in
  Definition~\ref{prelim-def-arborescent-knot}.

\end{definition}We see from the above the first portion of the
correspondence between the
algebraic and arborescent tangles. Each algebraic tangle can be naturally
decomposed into a collection of nested arborescent knot vignettes given by its
operations $+$ and $\vee$.

%  prettier-ignore-start

\subsection{Weighted Planar Trees}\label{subsec-wptt}

%  prettier-ignore-end

%  prettier-ignore-start

%  prettier-ignore-end

%  prettier-ignore-start

\subsubsection{Construction of Arborescent Knots from Weighted Planar
Trees}\label{construction_of_arbor}

%  prettier-ignore-end

This subsection begins with an introduction to Bonahon and Siebenmann's
\citep{bonahonNewGeometricSplittings2016} construction of arborescent knots and
tangles in the smooth setting. This is followed by the development of a
combinatorial representation for arborescent knots and tangles
\citep{bonahonNewGeometricSplittings2016}. We deviate slightly from Bonahon and
Siebenmann's introduction but ultimately arrive at the same structure. In our
introduction we develop partial solutions, then progressively modify those
partial solutions until they fit our needs. Next, we describe Bonahon and
Siebenmann's \citep{bonahonNewGeometricSplittings2016} operations on the
combinatorial structure, which allow us to systematically modify the structure,
without changing the topology. This subsection finishes with the classification
of arborescent knots and tangles given by Bonahon and Siebenmann
\citep{bonahonNewGeometricSplittings2016} as well as our extension
from a global to
local viewpoint.

\paragraph{Bands and Plumbing Squares}

Our first step in describing a notation for the arborescent knots
\citep{bonahonNewGeometricSplittings2016} is describing a plumbing operation on
bands. A band with a plumbing square is a band $S^1\times\LB 0,1\RB$, along with
an oriented square on the band such that two of the sides of the square
intersect the boundary of the band. Two examples of bands with plumbing squares
can be seen in Figure~\ref{wpt-construc-fig-band_sum}.

\begin{figure}[H]
  \centering
  \begin{subfigure}[b]{.45\textwidth}
    \centering
    \includegraphics[width=\textwidth]{files/bnd_sum_1-3b7426920de1e4efba9af1d3a5aba180.pdf}
    \caption[A band with a plumbing square facing the viewer.]{A band
    with a plumbing square facing the viewer.}
    \label{wpt-construc-fig-band_sum-1}
  \end{subfigure}
  ~
  \begin{subfigure}[b]{.45\textwidth}
    \centering
    \includegraphics[width=\textwidth]{files/bnd_sum_2-0f8a5a0218c04208316206db8595abf8.pdf}
    \caption[A band with the plumping square facing away from the
    viewer.]{A band with the plumping square facing away from the
    viewer. We are looking through the band.}
    \label{wpt-construc-fig-band_sum-2}
  \end{subfigure}
  \caption[Plumbing squares of bands.]{}
  \label{wpt-construc-fig-band_sum}
\end{figure}\paragraph{Plumbing bands}

We now glue the bands seen in Figure~\ref{wpt-construc-fig-band_sum}
together with an
operation called \textbf{plumbing}. Consider the orientation given in the green
band's plumbing square. We will call the blue arrow $X$ and the thicker red
arrow $Y$; similarly for the blue band with $X^\prime$ and $Y^\prime$. We
\textbf{plumb} the bands together along their plumbing squares, with
the requirement
that the orientation labels are mapped $X\to Y^\prime$ and $Y\to X^\prime$.
Finally, we forget the boundaries of the plumbing squares, leaving only the
joined boundaries of the bands. The result of plumbing as well as a local
picture for plumbing can be seen in Figure~\ref{wpt-construc-fig-band_sum_opo}.

\begin{figure}[H]
  \centering
  \begin{subfigure}[b]{0.45\textwidth}
    \centering
    \includegraphics[width=\textwidth]{files/bnd_sum_sum-5039ec3a2186e40e4a34a24bb86ae52c.pdf}
    \caption{}
    \label{wpt-construc-fig-band_sum_opo-1}
  \end{subfigure}
  ~
  \begin{subfigure}[b]{0.45\textwidth}
    \centering
    \includegraphics[width=\textwidth]{files/bnd_sum_patch-4165c83cc3f37aafe79d4b947ad0b900.pdf}
    \caption{}
    \label{wpt-construc-fig-band_sum_opo-2}
  \end{subfigure}
  \caption[Two bands plumbed.]{Two bands plumbed.}
  \label{wpt-construc-fig-band_sum_opo}
\end{figure}Our plumbing band construction can be turned into a knot,
by adding a series of
half-twists to our plumbing bands (Figure~\ref{wpt-construc-fig-6} and
Figure~\ref{wpt-construc-fig-25}). When forming the half twists, we
have two options for
direction relative to the band, we call one positive (left handed twists) and
one negative (right handed twists). We note that the twists appear in unique
regions of the band, determined by their position relative to the plumbing
squares.

\begin{figure}[H]
  \centering
  \begin{subfigure}[b]{0.45\textwidth}
    \centering
    \includegraphics[width=\textwidth]{files/arbor_band_with_twis-ae3d0a74a12df52afd907b081cba1c07.pdf}
    \caption[Band with two negative half twists.]{Band with two
      negative half twists\newline
    and three plumbing squares.}
    \label{wpt-construc-fig-6}
  \end{subfigure}
  ~
  \begin{subfigure}[b]{0.45\textwidth}
    \centering
    \includegraphics[width=\textwidth]{files/arbor_band_with_twis-5c07bffa285ece2cd9b0d471ca67f520.pdf}
    \caption[Band with three positive half twists.]{Band with three
      positive half twists \newline
    and one plumbing square.}
    \label{wpt-construc-fig-25}
  \end{subfigure}
  \caption[Plumbing bands with twists.]{}
\end{figure}Successive plumbing yields collections of bands like those seen in
Figure~\ref{wpt-construc-fig-10}. Finally, turning
Figure~\ref{wpt-construc-fig-10} into a knot is as
simple as removing the interior of the bands, leaving only the boundary, per
Figure~\ref{wpt-construc-fig-24}.
\begin{figure}[H]
  \centering
  \begin{subfigure}[b]{0.45\textwidth}
    \centering
    \includegraphics[width=\textwidth]{files/arbor_bands-d47febe50384b73ef69d7c9b9eb15c3f.pdf}
    \caption{}

    \label{wpt-construc-fig-10}
  \end{subfigure}
  ~
  \begin{subfigure}[b]{0.45\textwidth}
    \centering
    \includegraphics[width=\linewidth]{files/arbor_bound-a3d0541c15268c7cee37d83234237561.pdf}
    \caption{}

    \label{wpt-construc-fig-24}
  \end{subfigure}
  \caption[Plumbing and arborescent knots.]{A set of plumbed bands in
    \ref{wpt-construc-fig-10}
  and arborescent knot in \ref{wpt-construc-fig-24}}

\end{figure}
It is important to note that, for creating arborescent knots, we must restrict
plumbing from creating ``cycles'' of bands. That is a chain of
plumbing beginning
and ending with the same band, as seen in
Figure~\ref{wpt-construc-fig-cycle}. If we allow
cycles in the bands, we may create a polygonal tangle, defined in
Section~\ref{subsubsec-opo-insert}. These polygonal tangles contain
portions that do not
satisfy Definition~\ref{intro-def-arbor-tangle}, so are not arborescent.

\begin{figure}[H]
  \centering
  \includegraphics[width=0.5\linewidth]{files/band_cycle-7c0a613101506fe84b3e2dd906bec769.pdf}
  \caption[A collection of bands plumbed into a cycle.]{A collection
    of bands plumbed in such a way that the last band is plumbed to
  the first band in a cycle.}
  \label{wpt-construc-fig-cycle}
\end{figure}

We now establish some language for describing the relative positions of bands.
This language will be reused when we transition to the combinatorial setting,
and is widely used in graph theoretic settings.

\begin{definition}{}{wpt-construc-def-relationships_of_bands}
  Given a band $B$ with plumbing squares, we call the set $C$ of bands
  plumbed to $B$ the \textbf{children} of $B$. Additionally, for
  $c\in C$ we call $B$
  the \textbf{parent} of $c$ and the collection of $C -\LS c \RS$ the
  \textbf{siblings of $c$}.

\end{definition}We claim the plumbing band construction is in
correspondence with the definition
of arborescent seen in Definition~\ref{prelim-def-arborescent-knot}.
To see this, we take each
plumbing band and encapsulate it in a $S^2$ so that the corners of the plumbing
squares lie on the $S^2$, giving us the vignette seen in
Figure~\ref{fig-arborescent_band}.

\paragraph{Weighted Planar Trees}

The band construction we have developed for arborescent knots, as it stands, is
completely unsuited for machine computation. In this subsection, we lay out a
line of reasoning leading to a combinatorial encoding strategy
\citep{bonahonNewGeometricSplittings2016} for the plumbing band construction of
arborescent knots and tangles. The line of reasoning starts by presenting the
required data of arborescent knots and tangles that any combinatorial
representation must encode. We then propose partial solutions, each
progressively closer to the full encoding described by Bonahon and Siebenmann
\citep{bonahonNewGeometricSplittings2016}. As we will see, the encoding strategy
ultimately takes the form of a modified rooted plane tree, a specialized flavor
of graph theoretic tree.

Inventing a combinatorial encoding strategy means we first have to identify the
essential information that is needed to construct arborescent knots and tangles
from band plumbings. The two essential pieces of information that must be
encoded by any combinatorial strategy for notating plumbing of bands are the
following:

\begin{enumerate}
  \item The parent child relationship between bands
  \item The twists on bands, and their positions relative to the band's children
    (plumbing squares)
\end{enumerate}

Explicit details expanding on why these two pieces of information are essential
can be found in Bonahon and Siebenmann
\citep{bonahonNewGeometricSplittings2016}. We
will see in the following subsections ways in which these data are essential,
albeit in specialized cases.

Consider the first piece of essential information our combinatorial strategy
must encode, the parent child relationships between bands. Perhaps the most
commonly computationally utilized structure that encodes relational data is an
abstract graph. We imagine how an abstract graph might be used for encoding the
relationship of bands. One solution is to map bands to vertices and plumbing
relationships to edges. In the discussion of the band construction, we
restricted plumbing from forming cycles
(Figure~\ref{wpt-construc-fig-cycle}). A result of
this restriction is that any abstract graph must also have no cycles, meaning
all the graphs we will work with, abstract or otherwise, are actually trees. We
will call the data of a vertex and a collection of \textbf{bonds} (half-edges)
associated with plumbing squares the \textbf{local picture around a vertex}.

We have partially completed our goal of encoding the essential information of
band plumbing in a combinatorial object. Unfortunately, an abstract tree doesn't
encode all the essential data. Particularly, our second piece of information,
the positions of children and weights, is not easily seen in an abstract tree.
To solve this problem, we will instead use a modified version of an abstract
tree, a \textbf{rooted plane tree}, for our encoding.

\begin{definition}{}{wpt-construc-def-rooted_plane_tree}A
  \textbf{rooted plane tree} is an abstract tree imbued with a strict
  total order,
  indexed by the non-negative integers, on the vertices. We call the least
  vertex the \textbf{root} of the tree.

\end{definition}In a rooted plane tree $\Gamma$, at each vertex $v$,
the children of $v$ have an
ordering inherited from the total order of $\Gamma$, we call this ordering of
the children the \textbf{cyclic order} of the children. The cyclic
order gives us two
natural ways to draw $v$ and its children in the plane. We may choose to draw
the children anti-clockwise in one of increasing or decreasing order of index.
The realization of these two options can be seen in
Figure~\ref{wpt-construc-fig-order_tree}. A universal choice of
increasing or decreasing
yields a unique realization of a rooted plane tree in the plane.

\begin{figure}[H]
  \centering
  \begin{subfigure}[b]{0.45\textwidth}
    \centering
    \includegraphics[width=\textwidth]{files/arbor_graph_split_lo-45ac2a5c77ff7ba4ddbfbc87daddc9be.pdf}
    \caption[The local picture of a vertex in anti-clockwise
    order.]{The local picture of a vertex with child labels
    increasing in anti-clockwise order.}
    \label{wpt-construc-fig-order_1}
  \end{subfigure}
  ~
  \begin{subfigure}[b]{0.45\textwidth}
    \centering
    \includegraphics[width=\textwidth]{files/arbor_graph_split_lo-2249954de7fa775addf0b3e30c964706.pdf}
    \caption[The local picture of a vertex in anti-clockwise
    order.]{The local picture of a vertex with child labels
    decreasing in anti-clockwise order.}
    \label{wpt-construc-fig-order_2}
  \end{subfigure}
  \caption[Local pictures of a vertex of a tree. ]{}
  \label{wpt-construc-fig-order_tree}
\end{figure}
%  prettier-ignore-start

\subparagraph{Indexing the total order of a tree}\label{indexing-rpt}

%  prettier-ignore-end

We will now describe an \textbf{ideal indexing} for a rooted plane tree.

\begin{definition}{}{}Let $\Gamma$ be a rooted plane tree $r$, be the
  root of $\Gamma$. Additionally,
  let $v_i\neq r$ be a vertex with parent $v_p$ and children
  $v_{c_1},\,\cdots,\,v_{c_n}$. We call the indexing of a rooted
  plane \textbf{depth first} if
  it satisfies the following:

  \begin{itemize}
    \item $r$ is index 0
    \item $p<i<c_1<\cdots< c_n$
  \end{itemize}

\end{definition}

\begin{note}
  Two commonly seen orderings of a tree are the breadth and depth
  first orderings, both orderings
  are ideal orderings. For our purposes we will prefer the depth first ordering.
\end{note}

\begin{figure}[H]
  \centering
  \begin{subfigure}[b]{0.45\textwidth}
    \centering
    \includegraphics[width=\textwidth]{files/rpt_order-8d066d4b0623b0d11cb8e5223597ccfb.pdf}
    \caption[A rooted plane tree with ideal indexing.]{A rooted plane
      tree with ideal indexing. The index of each vertex is seen inside
    the vertex.}
    \label{wpt-construc-fig-rptorder_1}
  \end{subfigure}
  ~
  \begin{subfigure}[b]{0.45\textwidth}
    \centering
    \includegraphics[width=\textwidth]{files/rpt_order_ni-05d87f91294bbb47d9e46fbf563a337b.pdf}
    \caption[A rooted plane tree with indexing that is not depth
    first.]{A rooted plane tree with indexing that is not depth
    first. The index of is each vertex seen inside the vertex.}
    \label{wpt-construc-fig-rptorder_2}
  \end{subfigure}
  \caption[Indexing strategies of a rooted plane tree. ]{}
  \label{wpt-construc-fig-order}
\end{figure}

\begin{convention}
  For the remainder of this thesis we will adopt some conventions for
  rooted plane
  trees (and their derivatives the weighted planar tangle trees, CWPTT, and
  RLITT). When realizing a tree in the plane we select the universal
  anti-clockwise increasing
  order and assume that the tree has depth first indexing.
\end{convention}

The final data we need to record is the position and count of half twists
relative to plumbing squares. We observed earlier that the half twists on a band
must lie in a unique region determined by position relative to plumbing squares
on a band. This relationship can be recreated for a rooted plane tree by
annotating the local view of a vertex with an integer placed in the regions
between bonds. The relationship between a plumbing band and a weighted vertex in
a rooted plane tree can be seen in Figure~\ref{wpt-construc-fig-7}.
The weights placed in
regions between bonds inherit a cyclic order from the cyclic order of the bonds.
Each weight falls in the region between two bonds. We assign to each weight
(including zero weights) the lower of the two indices. This aligns with
assigning to the weight the index that appears before it in the anti-clockwise
planar realization of the cyclic order, per
Figure~\ref{wpt-construc-fig-weights-with-index}.

\begin{figure}[H]
  \centering
  \begin{subfigure}[b]{0.45\textwidth}
    \centering
    \includegraphics[width=\textwidth]{files/arbor_graph_split_lo-d4ac908e2b012b3aa3b53562353c28a8.pdf}
    \caption[The local view of a vertex with the weights $-2$, zero,
    and zero.]{The local view of a vertex with the weights $-2$,
    zero, and zero.}
    \label{wpt-construc-fig-7}
  \end{subfigure}
  ~
  \begin{subfigure}[b]{0.45\textwidth}
    \centering
    \includegraphics[width=\textwidth]{files/arbor_graph_split_lo-9fccfef14ced7a247f4d9f2d0b7199f1.pdf}
    \caption[The local view of a vertex with weights.]{The local view
      of a vertex with weight. \newline
      Notice the index of the weights come
      from the bond ``before'' it in the planar realized cyclic order
    given by convention.}
    \label{wpt-construc-fig-weights-with-index}
  \end{subfigure}
  \caption[Local view of a vertex with weights. ]{}
  \label{wpt-construc-fig-order_weights}
\end{figure}We can see a full example of a tree with its associated
plumbed construction in
Figure~\ref{wpt-construc-fig-27}. We call this fully realized
combinatorial recipe for an
arborescent knot a \textbf{weighted planar tree}.

\begin{definition}{Bonahon and Siebenmann, Page
  143\textbf{\citep{bonahonNewGeometricSplittings2016}}}{}
  A rooted plane tree $\Gamma$ augmented with weights is called a
  \textbf{weighted planar tree}.

\end{definition}
\begin{figure}[H]
  \centering
  \begin{subfigure}[b]{0.45\textwidth}
    \centering
    \includegraphics[width=\textwidth]{files/arbor_graph-f540218bf98a387a4e4b547a93445b49.pdf}
    \caption[The tree describing the plumbing of bands.]{The tree
      describing the plumbing of bands. Each vertex represents the band
    illustrated near it.}
    \label{wpt-construc-fig-27}
  \end{subfigure}
  ~
  \begin{subfigure}[b]{0.45\textwidth}
    \centering
    \includegraphics[width=\textwidth]{files/arbor_bands-d47febe50384b73ef69d7c9b9eb15c3f.pdf}
    \caption[The realization by plumbing bands of a tree.]{The
      realization by plumbing bands of the tree in
    Figure~\ref{wpt-construc-fig-27}}
    \label{wpt-construc-fig-28}
  \end{subfigure}
  \caption[Realization of plumbing of a tree.]{}
\end{figure}
%  prettier-ignore-start

\paragraph{Weighted Planar Tangle Trees}\label{wpt-construc-sec-wptt}

%  prettier-ignore-end

Our construction to this point has been concerned with the notation for knots
and links. We now give a modification of this notation for tangles. A weighted
planar tree, as in Figure~\ref{wpt-construc-fig-29}, can be modified
to represent a tangle
by allowing a \textbf{free bond} (half-edge), to be attached to a
vertex, that is, to
allow bands to have a non-plumbed plumbing square. We realize the non-plumbed
square, as a Conway circle for a two string tangle as in
Figure~\ref{wpt-construc-fig-tangle_trad}. To consistently orient the
Conway sphere's interior, we align the north boundary points of the Conway
sphere with the top (up in the band orientation) boundary component, and place
the NW corner first (following the orientation of the boundary), per
Figure~\ref{wpt-construc-fig-band_orientation}. Plumbing two bands
then corresponds to the
action of gluing a pair of tangles together on their Conway spheres so that
boundary points align.

A tree may have many free bonds, with each free bond representing a unique
boundary Conway sphere. Each boundary component serves as a location where a
tangle can be inserted to form a knot or link. For our efforts in enumerating
two string tangles, we restrict our focus to trees that have a single free bond.
In tangle trees with a single free bond, we designate the vertex with the free
bond as the root of the tree.

\begin{figure}[H]
  \centering
  \begin{subfigure}[b]{0.45\textwidth}
    \centering
    \includegraphics[width=\textwidth]{files/arbor_tangle-14a418154dfaf35f146b52dfa168bd93.pdf}
    \caption[The plumbing realization of an arborescent tangle.]{The
    plumbing realization of an arborescent tangle.}
    \label{wpt-construc-fig-29}
  \end{subfigure}
  ~
  \begin{subfigure}[b]{0.45\textwidth}
    \centering
    \includegraphics[width=0.7\textwidth]{files/example_tangle-4c128833ef1611bc12dcf29876e1e528.pdf}
    \caption[Realizing bands as a tangle.]{With an isotopy of the
      tangle and inversion of the Conway circle given by the
      non-plumbed square we have the realization of
      Figure~\ref{wpt-construc-fig-29} as a
    traditional orthogonally projected tangle.}
    \label{wpt-construc-fig-tangle_trad}
  \end{subfigure}
  \caption[Plumbing bands as a tangle.]{}
\end{figure}

\begin{figure}[H]
  \centering
  \includegraphics[width=.5\textwidth]{files/bnd_with_orientation-a963ba0fe226248533f2dad3ada2982d.pdf}
  \caption[The orientation of a Conway sphere.]{The orientation of a
    Conway sphere given by a plumbing square on a band of an
    arborescent tangle. The orientation of the underlying plumbing
    square is shown.
    This aligns with a left hand rule with $Y$ the thumb, $X$ the
    index finger, and
    $Z$ middle finger, with $Z$ pointing away from the center of the band,
  out of the page in this case. }
  \label{wpt-construc-fig-band_orientation}
\end{figure}

We will see that keeping track of the
location of the fixed points of the boundary sphere is important when
determining tangle equivalence. This is due to the need to maintain the rational
number (Definition~\ref{rational-def-frac}) associated with the
``rational tangle'' subtangles of
a tree, prompting us to assign rotation information to the free bonds. This
information takes the form of labels from the members of $V_4$ of the Klein
four-group $\iota,\xi,\zeta,\eta$. Each of these labels corresponds to a
rotation of the Conway sphere around an axis in $\R^3$, as seen in
Figure~\ref{wpt-construc-fig-v_4rotations} and
Figure~\ref{wpt-construc-fig-k4g}. Full details for the
manifold theory underpinning these markings are found in Bonahon and Siebenmann
\citep{bonahonNewGeometricSplittings2016}. We call such a labeled
tree a \textbf{weighted
planar tangle tree}.

\begin{figure}[H]
  \centering
  \begin{subfigure}[b]{0.2\textwidth}
    \centering
    \includegraphics[width=\textwidth]{files/v4_rotations_i-2a340230132ad100c9adc5eb2529d02f.pdf}
    \caption{}
    \label{wpt-construc-fig-k4g-rotationsi}
  \end{subfigure}
  \quad
  \quad
  \quad
  \begin{subfigure}[b]{0.4\textwidth}
    \centering
    \includegraphics[width=\textwidth]{files/v4_rotations-ac373932ba90a3a5b91b87418d43ff12.pdf}
    \caption{}
    \label{wpt-construc-fig-k4g-rotations}
  \end{subfigure}
  \caption[The effect of the $V_4$ rotations.]{The identity rotation
  (no rotation), and the effect of the $V_4$ rotations on each of the}
  \label{wpt-construc-fig-v_4rotations}
\end{figure}

\begin{definition}{Bonahon and Siebenmann, Page 165
  \textbf{\citep{bonahonNewGeometricSplittings2016}}}{}
  A weighted planar tree $\Gamma$ with free bonds labeled in $V_4$ is called a
  \textbf{weighted planar tangle tree (WPTT)}.

\end{definition}

\begin{figure}[H]
  \centering
  \begin{subfigure}[b]{0.4\textwidth}
    \centering
    \includegraphics[width=.5\textwidth]{files/iota-7ccd6540b4630d7bc532763aec0fa7fb.pdf}
    \caption[$\iota$ for no rotation.]{$\iota$ for no rotation}
    \label{wpt-construc-fig-k4g-i}
  \end{subfigure}
  \hfill
  \begin{subfigure}[b]{0.4\textwidth}
    \centering
    \includegraphics[width=.5\textwidth]{files/zeta-ccf154293b27a3af9396774196e3df8a.pdf}
    \caption[$\xi$ rotates around the $x$-axis.]{$\xi$ rotates around
    the $x$-axis}
    \label{wpt-construc-fig-k4g-x}
  \end{subfigure}
  \newline
  \centering
  \begin{subfigure}[b]{0.4\textwidth}
    \centering
    \includegraphics[width=.5\textwidth]{files/eta-8a2277d5eff21840ebb2ca47ad98f9ae.pdf}
    \caption[$\eta$ rotates around the $y$-axis.]{$\eta$ rotates
    around the $y$-axis}
    \label{wpt-construc-fig-k4g-y}
  \end{subfigure}
  \hfill
  \begin{subfigure}[b]{0.4\textwidth}
    \centering
    \includegraphics[width=.5\textwidth]{files/xi-1d75e1fe4635234febc0e30976942994.pdf}
    \caption[$\zeta$ rotates around the $z$-axis.]{$\zeta$ rotates
    around the $z$-axis}
    \label{wpt-construc-fig-k4g-z}
  \end{subfigure}
  \caption[Roations of a tangle. ]{}
  \label{wpt-construc-fig-k4g}
\end{figure}
%  prettier-ignore-start

\subsubsection{Anatomy of a tree}\label{wpt-construc-sec-subtrees}

%  prettier-ignore-end

In this subsection, we will describe several portions of weighted planar trees:
the ring subtree, essential vertex, and the sticks of a tree.

%  prettier-ignore-start

\paragraph{Ring subtree}\label{wpt-construc-sec-rings}

%  prettier-ignore-end

We will now describe the ring subtrees of a weighted planar tree, which locally
appear as Figure~\ref{wpt-construc-fig-17}.

\begin{figure}[H]
  \centering
  \includegraphics[width=0.5\linewidth]{files/arbor_graph_ring-d80406b3ccc9a75cbacd2e8eecfa706d.pdf}
  \caption[Positive and negative ring subtrees.]{Positive and
  negative ring subtrees}
  \label{wpt-construc-fig-17}
\end{figure}

Now, resolving the plumbing for the positive subtree, we arrive at bands as in
Figure~\ref{wpt-construc-fig-12}.

\begin{figure}[H]
  \centering
  \includegraphics[width=0.5\linewidth]{files/arbor_ring-f1ab6236c3e07c2dbd1fd86733fbfa89.pdf}
  \caption[Plumbed ring bands.]{Plumbed ring bands}
  \label{wpt-construc-fig-12}
\end{figure}

Notice that the boundary of these plumbed bands has three components, as seen in
Figure~\ref{wpt-construc-fig-13}.

\begin{figure}[H]
  \centering
  \includegraphics[width=0.5\linewidth]{files/arbor_ring_no_bnd-571778a714009a521aa0c17358a4d65e.pdf}
  \caption[Ring boundary.]{Ring boundary}
  \label{wpt-construc-fig-13}
\end{figure}

With an isotopy of the tangle and inversion of the Conway circle given by the
non-plumbed square, we can arrange our plumbed bands into the standard tangle
projection seen in Figure~\ref{wpt-construc-fig-14}. This tangle
projection tells us that
the subtree in Figure~\ref{wpt-construc-fig-12} is, depending on
location of $\text{NW}$,
either the zero or infinity tangle with a ring.

\begin{figure}[H]
  \centering
  \includegraphics[width=0.3\linewidth]{files/arbor_ring_tangle-a2936a53323b2187ce9cd2a9209f32db.pdf}
  \caption[Ring Tangle.]{Ring Tangle}
  \label{wpt-construc-fig-14}
\end{figure}

%  prettier-ignore-start

\paragraph{Essential Vertex}\label{wpt-construc-sec-essential_verts}

%  prettier-ignore-end

We now classify each vertex into one of two classes, the essential vertices and
the non-essential vertices.

\begin{definition}{Bonahon and Siebenmann, Page 159
  \textbf{\citep{bonahonNewGeometricSplittings2016}}}{apn-def-2}
  We define an \textbf{essential vertex} as any vertex with valence
  (count of the number of bonds)
  greater than 3.

\end{definition}
\begin{definition}{Bonahon and Siebenmann, Page 159
  \textbf{\citep{bonahonNewGeometricSplittings2016}}}{apn-def-3}
  A vertex is called non-essential if it has valence (count of the
  number of bonds) $0,1,2$.

\end{definition}As an example, consider the vertices seen in
Figure~\ref{wpt-construc-fig-18}.

\begin{figure}[H]
  \centering
  \includegraphics[width=\linewidth]{files/arbor_ring_essential-dc5f17b0ae16077f7beced9f4b581de3.pdf}
  \caption[A weighted planar tangle tree annotated with essential
  vertices.]{A weighted planar tree annotated with essential vertices
    in orange and
  non-essential in blue}
  \label{wpt-construc-fig-18}
\end{figure}

%  prettier-ignore-start

\paragraph{Sticks of a Tree}\label{wpt-construc-sec-sticks}

%  prettier-ignore-end

The final part of the anatomy of a tree we will consider is the
\textbf{sticks} of a
tree.

\begin{definition}{Bonahon and Siebenmann, Page 159
  \textbf{\citep{bonahonNewGeometricSplittings2016}}}{wpt-construc-def-sticks-of-a-tree}
  Let $\Gamma$ be a weighted planar tree and $\LS b_i\RS$ be the set
  of essential
  vertices of $\Gamma$ including their bonds (half-edges). We call the
  $\Gamma_s=\Gamma\setminus \LS b_i\RS$ the \textbf{sticks} of
  $\Gamma$ and every
  connected component of $\Gamma_s$ a \textbf{stick}.

\end{definition}As an example, consider the tree seen in
Figure~\ref{wpt-construc-fig-18}, the sticks of
which can be seen in Figure~\ref{wpt-construc-fig-19}.

\begin{figure}[H]
  \centering
  \includegraphics[width=\linewidth]{files/arbor_ring_noessenti-ac06ddb43820e892797452240f28d7d0.pdf}
  \caption[Sticks of a tree.]{Sticks of the tree from
    Figure~\ref{wpt-construc-fig-18}, six half-open sticks and one open
  stick.}
  \label{wpt-construc-fig-19}
\end{figure}

By construction, a stick subtree of $\Gamma$ may have 0, 1, or 2 free bonds
(seen in Figure~\ref{wpt-construc-fig-sticks}). We call a stick with
0 free bonds closed, 1
free bond half-open, and 2 free bonds open. Additionally, we call a stick where
each vertex has a single weight a \textbf{proper stick}, and we call
a vertex on the
end of a stick an \textbf{end vertex}.

\begin{figure}[H]
  \centering
  \includegraphics[width=0.5\linewidth]{files/sticks_open-a37c24f6b1180f1c4b02c7cf19d6eea4.pdf}
  \caption[Closed, half-open, and an open sticks.]{From top to
    bottom, a closed, half-open, and an open stick. Each end vertex
  is colored in red.}
  \label{wpt-construc-fig-sticks}
\end{figure}

%  prettier-ignore-start

\subparagraph{Integral Tangles}\label{wpt-construc-sec-integral}

%  prettier-ignore-end

When a weighted planar tangle tree is a half-open stick containing a single
vertex with a single weight $w_0$ we call it an \textbf{integral tangle tree}.

\begin{figure}[H]
  \centering
  \includegraphics[width=\linewidth]{files/watt_integral-d0566b70563422fdde261b042662a783.pdf}
  \caption[A stick realized as a integral tangle.]{A stick realized
  as a integral tangle.}
  \label{wpt-construc-fig-integral}
\end{figure}

%  prettier-ignore-start

\subparagraph{Rational Tangles}\label{wpt-construc-sec-rational}

%  prettier-ignore-end

Bonahon and Siebenmann \citep{bonahonNewGeometricSplittings2016} give a
correspondence between stick tangle trees (stick with a single free bond) and
Conway's rational tangles \citep{conwayEnumerationKnotsLinks1970}. An
example of the
correspondence can be seen in Figure~\ref{wpt-construc-fig-rat}.

\begin{figure}[H]
  \centering
  \includegraphics[width=\linewidth]{files/watt_rational-aab54569c87acbbe1143e6b14a6cbeeb.pdf}
  \caption[A stick tangle tree realized as a rational tangle.]{A
  stick tangle tree realized as a rational tangle.}
  \label{wpt-construc-fig-rat}
\end{figure}

%  prettier-ignore-start

\subparagraph{Tree Crossing Number}\label{wpt-construc-sec-TCN}

%  prettier-ignore-end

Finally, we define the \textbf{Tree Crossing Number (TCN)} of a weighted planar
tangle tree. This corresponds to the crossing number of the tangle diagram given
by the weighted planar tangle tree.

\begin{definition}{}{apn-def-tcn}For weighted planar tangle tree
  $\Gamma$, with weights $\LS w_i\RS$. We call
  \begin{equation}
    \text{TCN}=\sum |w_i|
  \end{equation}
  the \textbf{Tree Crossing Number (TCN)}.

\end{definition}
%  prettier-ignore-start

\subsubsection{Calculus of Weighted Planar Trees}\label{calculus_on_trees}

%  prettier-ignore-end

This subsection develops a set of moves, $F^\prime_3,\ F_2,\ F_1$, and $R^\pm$,
on the weighted planar trees described in Section~\ref{subsec-wptt}. We will
restrict our discussion to a subset of the calculus of weighted planar trees. A
full description of the calculus can be found in Bonahon and Siebenmann
\citep{bonahonNewGeometricSplittings2016}. The
$F^\prime_3,\ F_2,\ F_1,\text{ and}\ R_\pm$ moves allow us to systematically
modify, without changing the topology, a weighted planar tree and form the basis
for the classification of the arborescent knots.

\paragraph{The $F^\prime_3$ Move}

The first move, and as we will see, the most important in distinguishing
tangles, is the $F_3^\prime$ move. In this move we consider the
local picture of a vertex. In the local view, isolate a single bond
(half-edge corresponding to a plumbing
square of a band), then move a weight across that bond. The impact
of this movement of a weight propagates to the descendants of the
subtree attached to the bond
(plumbing square) but leaves unchanged all other weights and bonds
(including their
attached subtrees) in the local view of the object vertex.
\newpage
\begin{definition}{Bonahon and Siebenmann, Section 12.7.3
  \textbf{\citep{bonahonNewGeometricSplittings2016}}}{wpt-moves-def-f3p-move}
  The \textbf{$F_3^\prime$ move} on a weighted arborescent tree moves
  a weight $W$ as in
  Figure~\ref{wpt-moves-fig-f3p-move-indef} and, if $W$ is odd,
  reverses the cyclic order of
  weights and bonds at all vertices of the subtree attached to the
  purple bond (half-edge) lying at odd distance
  (count of edges between two vertices) from the vertex shown. Also, when $W$ is
  odd, apply $\xi$ ($X$-axis rotation
  Figure~\ref{wpt-construc-fig-k4g}) to all free bonds
  in the subtree attached to the purple bond (half-edge) that are
  attached to a vertex at even distance from the
  vertex shown, and $\eta$ ($Y$-axis rotation
  Figure~\ref{wpt-construc-fig-k4g-y}) to those
  at odd distance. The rotations are relative to the local orientations of the
  plumbing squares on the bands corresponding to vertices at odd distance from
  the vertex carrying weight $W$.

  \begin{figure}[H]
    \centering
    \includegraphics[width=\linewidth]{files/f3p_def-fea9c634e65e4c0469220121c3d52833.pdf}
    \caption[An $F_3^\prime$ move on a subtree.]{An $F_3^\prime$ move
      on a subtree. The purple bond indicates the subtree
      impacted by the move. The blue portion of the local view
      indicates all other bonds and weights
    of the vertex.}
    \label{wpt-moves-fig-f3p-move-indef}
  \end{figure}

\end{definition}The $F_3^\prime$ move is a derivative of the more
general $F_3$ move described
in Bonahon and Siebenmann \citep{bonahonNewGeometricSplittings2016}.
\newpage
\begin{definition}{Bonahon and Siebenmann, Section 12.7.1
  \textbf{\citep{bonahonNewGeometricSplittings2016}}}{wpt-moves-def-f3-move}
  The \textbf{$F_3$ move} replaces the left side of
  Figure~\ref{wpt-moves-fig-f3-move-indef} with the right side, where
  the cyclic order of bonds and weights is reversed at all vertices
  in the subtree attached to the purple bond (half-edge) of the
  vertex shown and at odd distances from this vertex. Also, apply
  $\xi$ ($X$-axis rotation) to all free bonds in the subtree attached
  to the purple bond (half-edge) that are attached to a vertex at
  even distance from the vertex shown,
  and $\eta$ ($Y$-axis rotation) to those at odd distance. The rotations are
  relative to the orientation of the  plumbing square (Conway sphere) of the the
  band the weight is moving across, per
  Figure~\ref{wpt-construc-fig-band_orientation}.

  \begin{figure}[H]
    \centering
    \includegraphics[width=\linewidth]{files/f3_def-73cd92acd310ffe76248c72f6e024950.pdf}
    \caption[An $F_3^\prime$ move on a subtree.]{An $F_3$ move on a
      subtree. The purple bond indicates the subtree
      impacted by the move. The blue portion of the local view
      indicates all other bonds and weights
    of the vertex.}
    \label{wpt-moves-fig-f3-move-indef}
  \end{figure}

\end{definition}

\begin{note}
  $F_3^\prime$ is equivalent, in
  Figure~\ref{wpt-moves-fig-f3-move-indef}, to setting
  $X=0$, then executing $F_3$ $W$ times, decreasing $\abs{W}$ with
  each iteration.
\end{note}

We will now consider some examples of the $F_3^\prime$ move. While we explore
these examples, we view $F_3^\prime$ from the perspective of an object vertex
(object band). The object vertex may have one or more children that we will act
on with $F^\prime_3$. When we translate $F_3$ and $F_3^\prime$ into practice we
are free to operate on children as well as the parent of a vertex (band).

\subparagraph{$F_3^\prime$ on bands}

The translation of crossings across child bands models the
traditional \textbf{flype}
move, of ``Tait Flyping Conjecture'' \citep{taitKnotsIIIII1900} fame. To see the
correspondence between $F_3^\prime$ and flype we need to view the plumbed child
(or parent) band as a tangle, $T$. We can then carry out $F_3^\prime$ over this
tangle. Tracking the parts of this operation, we can see the correspondence in
Figure~\ref{uc-c-f3-e-flype_and_bnd}.
\begin{figure}[H]
  \centering
  \includegraphics[width=\linewidth]{files/bnd_f3-57da5576983f5d6c9a55f15dc96a5397.pdf}
  \caption[Flype and $F_3^\prime$.]{Flype and $F_3^\prime$, with
    orientation of the Conway sphere given by
  Section~\ref{wpt-construc-sec-wptt}.}
  \label{uc-c-f3-e-flype_and_bnd}
\end{figure}

When this is carried out, for an odd number of crossings, the child band is
inverted so it lies inside the parent band
(Figure~\ref{uc-c-f3-e-flype_bnd}). The inversion
reverses the cyclic order of the child, as described in
Definition~\ref{wpt-moves-def-f3p-move}.
Applying $F_3^\prime$ to an even number of crossings is equivalent to applying
the move on two sets of an odd number of crossings. When the first set of odd
crossings is applied, the child band is inverted; the second set inverts the
child again, leaving it where it began
(Figure~\ref{wpt-moves-fig-example-f3-even-2}).
We expand to an example where the child band has descendants, as in
Figure~\ref{uc-c-f3-e-cor}. Observe that when the $F_3^\prime$ is
applied in this case, the
child band and every band even distance from it (odd distance from the parent)
is inverted.

\begin{figure}[H]
  \centering
  \includegraphics[width=\textwidth]{files/bnd_only_f3_even_2-ece02f57e951d9e680c8d28405e22790.pdf}
  \caption[$F_3^\prime$ on the band model.]{The even case of the $F_3^\prime$
    with the purple band and any descendants of the purple band
  remaining unchanged.}
  \label{wpt-moves-fig-example-f3-even-2}
\end{figure}

\begin{figure}[H]
  \centering
  \includegraphics[width=\textwidth]{files/bnd_only_f3-f34dfee65146995d01e338de8bd5c474.pdf}
  \caption[The odd $F_3^\prime$ case on a band model realization of
    the given portion of a
  tree.]{ The odd $F_3^\prime$ case on a band model realization of
    the given portion of a
    tree, with local $X$-axis in blue and $Y$-axis in red given
    relative to the purple band.
    Yielding a $\xi$ ($X$-axis rotation) to all bands plumbed to the purple band
    or plumbed at even distance (counting plumbing squares) from the
    orange band,
    and $\eta$ ($Y$-axis rotation) to the purple band and those
    plumbed at odd distance from the orange band.
    Note that the orientations of the plumbing squares must agree before and
    after $F_3^\prime$. Following the orientations with the left hand
    (Figure~\ref{wpt-construc-fig-band_orientation})
    rule shows the orientation of the purple band reverses in $X$ and
    $Z$ in the second image due to the rotation in $Y$
  (we are no longer looking through the band in the second image). }
  \label{uc-c-f3-e-flype_bnd}
\end{figure}

\begin{figure}[H]
  \centering
  \includegraphics[width=\linewidth]{files/bands_odd_change-707797bd70b675acb7b1ee4ff8e2c6c8.pdf}
  \caption[$F_3^\prime$ when applied to a band.]{$F_3^\prime$ when
    applied to a band (gray) with a child (orange) and grandchild
    (light blue). We read the bands following the local orientation
    of the plumbing
    squares. Before the move is applied, the child (orange) band is
    traversed as;
    the blue band, a green star, a green circle, and back to the parent. After
    $F_3^\prime$ it is traversed as; a green circle, a green star,
    the blue band,
    and back to the parent. The blue band is traversed as; yellow star, yellow
  circle, and back to orange band.}
  \label{uc-c-f3-e-cor}
\end{figure}

\subparagraph{$F_3^\prime$ Examples}

Consider the weighted planar tree in
Figure~\ref{wpt-moves-fig-example_f3-even-1} and
Figure~\ref{wpt-moves-fig-example_f3-odd_1}, the left trees in each
agree in all but a
weight of a single vertex, what we will call our object vertex which is marked
in orange. The weight of this object vertex has been changed from -2 in
Figure~\ref{wpt-moves-fig-example_f3-even-1} to -3 in
Figure~\ref{wpt-moves-fig-example_f3-odd_1}.

We will first walk through
Figure~\ref{wpt-moves-fig-example_f3-even-1}. In this example,
our object weight is e ven, applying $F_3^\prime$ to the tree, the impacted
subtree (purple subtree in Figure~\ref{wpt-moves-fig-f3p-move-indef})
is unchanged except
for free bonds, which are altered as described in
Definition~\ref{wpt-moves-def-f3p-move}.

\begin{figure}[H]
  \centering
  \includegraphics[width=0.7\linewidth]{files/watt_rooted_even-4d8585086a5084d3bb46059e31a94e1b.pdf}
  \caption[$F_3^\prime$ on a weighted planar tree.]{$F_3^\prime$ on a
    weighted planar tree with even weight. The weight of the
  object weight is even, so the impacted subtree is unchaged.}
  \label{wpt-moves-fig-example_f3-even-1}
\end{figure}

In Figure~\ref{wpt-moves-fig-example_f3-odd_1}, the object weight is
odd, applying
$F_3^\prime$ the cyclic order of vertices, of the impacted subtree at an odd
distance are reversed. Additionally, all free bonds in the impacted subtree are
altered as described in Definition~\ref{wpt-moves-def-f3p-move}.

\begin{figure}[H]
  \centering
  \includegraphics[width=0.7\linewidth]{files/watt_rooted_odd-f5793e212c4d6a0fa6d460f40db1f8c4.pdf}
  \caption[$F_3^\prime$ on a weighted planar tree.]{$F_3^\prime$ on a
    weighted planar tree with odd weight clockwise. Note the
    changes in the relative positions of subtrees after the application of
  $F_3^\prime$.}
  \label{wpt-moves-fig-example_f3-odd_1}
\end{figure}

\paragraph{The $F_2$ Move}

Our second move, $F_2$, is a special application of the general $F_3$ move.

\begin{definition}{Bonahon and Siebenmann, Section 12.7.1
  \textbf{\citep{bonahonNewGeometricSplittings2016}}}{uc-c-f2-d-f2-t}
  The \textbf{$F_2$ move} on a weighted arborescent tangle tree
  reverses the cyclic order
  of bonds and weights at one vertex on the tree and at every vertex
  at even distance from it; also
  apply $\eta$ ($Y$-axis rotation) to every free bond of a vertex at
  even (or zero) distance, and
  apply $\xi$ ($X$-axis rotation) to every free bond at odd distance.
  The rotations are relative to the orientation of the plumbing square (Conway
  sphere) of the band being acted on, per Section~\ref{subsec-wptt}.

\end{definition}$F_2$ is equivalent to applying $F_3$ to a vertex by
moving a $\pm 1$ weight
around a full cycle of the children (and parent), as in
Figure~\ref{wpt-moves-fig-example_f2_cycle}. If the vertex has no
weights, the zero weight
is split into a +1 and -1, one of which completes the cycle. The +1 and
-1 then cancel, returning the vertex to zero weight. The result of carrying
out $F_2$ on a weighted planar tree can be seen in
Figure~\ref{wpt-moves-fig-example_f2}.

\begin{figure}[H]
  \centering
  \includegraphics[width=\textwidth]{files/f2_local-a580dd826a40b46351b2f9bac9e0d6dd.pdf}
  \caption[$F_3$ moving a weight in a complete cycle.]{$F_3$ moving a
  weight in a complete cycle}
  \label{wpt-moves-fig-example_f2_cycle}
\end{figure}
\begin{figure}[H]
  \centering
  \includegraphics[width=\textwidth]{files/watt_rooted-f89a1c8a169f64c614d42e3085f20ec8.pdf}
  \caption[$F_2$ on a weighted planar tree.]{$F_2$ on a weighted
    planar tree. Observe the changes to the entire tree, as
  opposed to the changes of $F_3^\prime$ which impact only a subtree.}
  \label{wpt-moves-fig-example_f2}
\end{figure}
Observe that vertices can be partitioned into two equivalence classes. Those
changed by $F_2$ applied at an even distance from the root, and those changed by
$F_2$ applied at an odd distance from the root. We write $F_2$ on the even class
as $F_{2e}$ and odd as $F_{2o}$.

\paragraph{The $F_1$ Move}

The third of the $F$ moves is the $F_1$ move, which is a repeated application of
the $F_2$ move.

\begin{definition}{Bonahon and Siebenmann, Section 12.7.1
  \textbf{\citep{bonahonNewGeometricSplittings2016}}}{uc-c-f1-d-f1-t}
  The \textbf{$F_1$ move} on a weighted arborescent tangle tree
  reverses the cyclic order
  of bonds and weights at every vertex of the graph and applies
  $\zeta$ ($z$-axis
  rotation) to every free bond. The rotations are relative to the
  orientation of the plumbing square (Conway
  sphere) of the band being acted on, per Section~\ref{subsec-wptt}.

\end{definition}To realize $F_1$ as $F_2$ moves, we successively
apply $F_{2e}$ then $F_{2o}$ to
the tree. Observe that the combination of $F_{2e}$ and $F_{2o}$ modifies the
free bonds by $\xi\eta=\zeta$.

\paragraph{The $R^\pm$ moves}

The $R$ move, or ring move is the final move we will describe on weighted planar
tangle trees and deals with the ring subtrees of a tree. The result of a ring
move on a tangle can be seen in Figure~\ref{wpt-moves-fig-example-r}.

\begin{figure}[H]
  \centering
  \includegraphics[width=\linewidth]{files/example_ring-bbe1129008e26ffeb7c9523c9fa21b23.pdf}
  \caption[$R^ -$ on on a tangle representation of a tree.]{$R^ -$ on
  on a tangle representation of a tree.}
  \label{wpt-moves-fig-example-r}
\end{figure}

\begin{definition}{Bonahon and Siebenmann, Section 12.3
  \textbf{\citep{bonahonNewGeometricSplittings2016}}}{wpt-moves-def-rm}
  The \textbf{$R^ -$} replaces the left of Figure~\ref{rmove-n-pic}
  with the right, leaving the rest of the tree unchanged.

  \begin{figure}[H]
    \centering
    \includegraphics[width=0.5\linewidth]{files/def-1f585d3bf18ce302acf77bd52e61f5f0.pdf}
    \caption[A $\LP-\RP$-ring subtree moving around a vertex.]{A
      $\LP-\RP$-ring subtree moving around a vertex. Equivalent to
    Figure~\ref{wpt-moves-fig-example-r}.}
    \label{rmove-n-pic}
  \end{figure}

\end{definition}

\begin{note}
  In Figure~\ref{wpt-moves-fig-example-r} the ring moves from the
  right to the left of the
  tangle. This corresponds to the ring subtree in
  Definition~\ref{wpt-moves-def-rm} moving from
  top to bottom of the orange portion of the tree.
\end{note}

\begin{definition}{Bonahon and Siebenmann, Section 12.3
  \textbf{\citep{bonahonNewGeometricSplittings2016}}}{wpt-moves-def-rp}
  The \textbf{$R^+$} replaces the left of Figure~\ref{rmove-n-pic}
  with the right, leaving the rest of the tree unchanged.

  \begin{figure}[H]
    \centering
    \includegraphics[width=0.5\linewidth]{files/def-e0f5401b23b1e565a07876afd1239ac4.pdf}
    \caption[A $\LP+\RP$-ring subtree moving around a vertex.]{A
    $\LP+\RP$-ring subtree moving around a vertex.}
    \label{rmove-p-pic}
  \end{figure}

\end{definition}
%  prettier-ignore-start

\subsubsection{Canonical Weighted Planar Tangle Trees
(CWPTT)}\label{sec-CWPTT-def}

%  prettier-ignore-end

Observe that the weighted planar tangle trees we have seen are badly non-unique
representatives for arborescent tangles. The first step to finding a unique
preferred representative is to put some additional conditions on a weighted
planar tree, $\Gamma$. The following conditions pare down the equivalence class
of an arborescent tangle to a more manageable level.

\begin{definition}{Bonahon and Siebenmann, Section 12.8.2
  \textbf{\citep{bonahonNewGeometricSplittings2016}}}{wpt-equi-def-abcanon}
  A weighted planar tree is called a \textbf{canonical weighted planar
  tangle tree (CWPTT)} if it has a single free bond with a label from
  $V_4$ and satisfies the following conditions:

  \begin{itemize}
    \item \textbf{Weight Condition (W)} At each vertex of $\Gamma$, at
      most one weight is
      non-zero.

    \item \textbf{Stick Conditions}

      \begin{itemize}
        \item[\textbf{(S.1)}] On any stick the weights of the
          vertices are non-zero
          except for end vertices that have a bond free in $\Gamma$
          and for the case $\Gamma$ is
          Figure~\ref{wpt-construc-fig-stick_cond-1} or
          Figure~\ref{wpt-construc-fig-stick_cond-2}.
        \item[\textbf{(S.2)}] The non-zero weights along any stick are of
          alternating sign.
        \item[\textbf{(S.3)}] No end vertex of a stick has weight $\pm
          1$ unless it
          has a bond free in $\Gamma$.
      \end{itemize}

      \begin{figure}[H]
        \centering
        \begin{subfigure}[b]{0.45\textwidth}
          \centering
          \includegraphics[width=0.3\textwidth]{files/0-38f5b78e700440f6bd95db3013659000.pdf}
          \caption[The zero tangle. Where the indicated vertex is
          $v_0$.]{The zero tangle.}
          \label{wpt-construc-fig-stick_cond-1}
        \end{subfigure}
        ~
        \begin{subfigure}[b]{0.45\textwidth}
          \centering
          \includegraphics[width=0.3\textwidth]{files/00-ff78b5394e9bf86a9ab4f9dd4fcec2d4.pdf}
          \caption[The infinity tangle.]{The infinity tangle. Where
          the indicated vertex is $v_0$ and $v_1$.}
          \label{wpt-construc-fig-stick_cond-2}
        \end{subfigure}
        \caption{Where the given sticks are the entire tree $\Gamma$}
        \label{wpt-construc-fig-stick_cond}
      \end{figure}
    \item One of:

      \begin{itemize}
        \item \textbf{Positivity Condition (P)} Except for those with a
          free bond, there are no sticks in $\Gamma$ of the forms
          Figure~\ref{wpt-construc-fig-positivity_cond-1} or
          Figure~\ref{wpt-construc-fig-positivity_cond-2}.

          \begin{figure}[H]
            \centering
            \begin{subfigure}[b]{0.45\textwidth}
              \centering
              \includegraphics[width=0.3\textwidth]{files/am2-aaefc226f30ccf19cdbae75e67deae18.pdf}
              \caption[The -2 integral tangle.]{The -2 integral
              tangle. Where the indicated vertex is $v_i$ with $i\neq 0$}
              \label{wpt-construc-fig-positivity_cond-1}
            \end{subfigure}
            ~
            \begin{subfigure}[b]{0.45\textwidth}
              \centering
              \includegraphics[width=0.3\textwidth]{files/am2a-16bd870e6cc41ef82a45bae5879e4226.pdf}
              \caption[A fully open stick with -2 crossings.]{A fully
                open stick with -2 crossings. Where the indicated
              vertex is $v_i$ with $i\neq 0$}
              \label{wpt-construc-fig-positivity_cond-2}
            \end{subfigure}
            \caption[Positivity condition.]{}
            \label{wpt-construc-fig-positivity_cond}
          \end{figure}
        \item \textbf{Negativity Condition (N)} Except for those with a
          free bond, there are no sticks in $\Gamma$ of the forms
          Figure~\ref{wpt-construc-fig-negativity_cond-1} or
          Figure~\ref{wpt-construc-fig-negativity_cond-2}.

          \begin{figure}[H]
            \centering
            \begin{subfigure}[b]{0.45\textwidth}
              \centering
              \includegraphics[width=0.3\textwidth]{files/a2-2b741f8ba3f1cb8d8849231ce6357e1b.pdf}
              \caption[The 2 integral tangle.]{The 2 integral tangle.
              Where the indicated vertex is $v_i$ with $i\neq 0$}
              \label{wpt-construc-fig-negativity_cond-1}
            \end{subfigure}
            ~
            \begin{subfigure}[b]{0.45\textwidth}
              \centering
              \includegraphics[width=0.3\textwidth]{files/a2a-ec06a2a1420a8190c1216d90a131ca77.pdf}
              \caption[A fully open stick with 2 crossings.]{A fully
                open stick with 2 crossings.W here the indicated vertex
              is $v_i$ with $i\neq 0$}
              \label{wpt-construc-fig-negativity_cond-2}
            \end{subfigure}
            \caption[Negativity condition.]{}
            \label{wpt-construc-fig-negativity_cond}
          \end{figure}
      \end{itemize}
  \end{itemize}

\end{definition}

\begin{note}
  The set of CWPTT for an arborescent tangle can be large.
\end{note}
\begin{note}
  The positivity and negativity conditions are a consequence of the behavior of
  two crossing tangles seen in Figure~\ref{minimal-fig-nonmin}. We
  will adopt the
  $\LP+\RP$ as our prefered form.
\end{note}
Bonahon and Siebenmann show in Corollary~\ref{wpt-equi-lemma-exist}
that these conditions are
sufficient to realize every arborescent tangle. In fact, every weighted planar
tangle tree can be turned into a CWPTT by a series of moves in an extended
calculus on weighted planar trees
\citep{bonahonNewGeometricSplittings2016}. We call
this process \textbf{canonization} of a weighted planar tangle tree.

\begin{corollary}{Existance of CWPTT, Bonahon and Siebenmann
    Corollary 12.20
  \textbf{\citep{bonahonNewGeometricSplittings2016}}}{wpt-equi-lemma-exist}
  Every arborescent tangle is obtained by plumbing operations from
  arborescent tangles associated to positively (or negatively)
  canonical weighted planar trees (with labels in $V_4$ at free bonds).

\end{corollary}We note some consequences of the positivity and
negativity condition. First, a
positive CWPTT ($\LP +\RP$-CWPTT) can be transformed by a sequence of moves in
the extended calculus of weighted planar trees into a negative CWPTT
($\LP -\RP$-CWPTT). Similarly, a negative CWPTT can be transformed into a
positive CWPTT. Second, we note that a CWPTT, with no modification, can be both
positive and negative. We will refer to these trees as \textbf{neutral} trees.

Bonahon and Siebenmann give a classification of arborescent tangles via moves on
CWPTT.

\begin{theorem}{Classification Theorem for Canonical Weighted Planar
    Tangle Trees, Bonahon and Siebenmann, Theorem 12.21
  \textbf{\citep{bonahonNewGeometricSplittings2016}}}{wpt-equi-thm-classi}
  Consider two positive (or negative) CWPTT
  $\Gamma^{\,}$ and $\Gamma^\prime$, with free bonds labeled by
  elements of $V_4$. Plumbing according to $\Gamma$ and $\Gamma^\prime$
  gives isomorphic arborescent tangles if and only if $\Gamma$ and
  $\Gamma^\prime$ can be deduced from each other by a sequence of moves
  ($F_1$), ($F_2$), ($F_3^\prime$), and the modified ring moves $\LP\pm R\RP$.

\end{theorem}Further, Bonahon and Siebenmann describe an algorithm
for producing these
sequences of moves. This algorithm will be useful to us in
Section~\ref{sec-rlitt-generation}.

\begin{theorem}{Bonahon and Siebenmann, Theorem 12.19
  \textbf{\citep{bonahonNewGeometricSplittings2016}}}{wpt-equi-cor-algo}
  There exists an effective algorithm which, for any weighted planar
  tree $\Gamma$
  with free bonds labeled by elements of $V_4$, alters $\Gamma$ by a sequence of
  moves of the calculus of arborescent tangles to produce its collection of
  positively (or negatively) canonical weighted planar trees.

\end{theorem}\paragraph{Canonical Vertex}

The conditions for a weighted planar tree to be a CWPTT are phrased for the
global context of a weighted planar tangle tree. We now recontextualize those
conditions for a local view, a single vertex of the tree.

\begin{definition}{}{vertex-canon-def}
  A vertex $v_i$ of a weighted planar tangle tree $\Gamma$ with a single free
  bond labeled from $V_4$ is said to be \textbf{$\LP+\RP$-canonical}
  if $v_i$ has at
  most one non-zero weight $w_i$, and $i$ is zero (the root) with no
  conditions or
  $i$ is not zero with the following conditions satisfied:

  \begin{enumerate}[I.]

    \item {If the valence of $v_i$ is 1 and all of:
        \begin{itemize}
          \item{Stick Condition:
              \begin{enumerate}[1.]
                \item {$w_i\neq 0$ unless $i=1$ and $w_0=0$ (the
                  weight of the root)}
                \item {If the valence of $v_{i -1}$ (the parent) is 2
                    then $\text{sign}\LP w_i\RP\neq\text{sign}\LP w_{i
                  -1}\RP$ unless $i=1$ and $w_0=0$}
                \item {$w_i\neq \pm 1$}
              \end{enumerate}
            }
          \item {Positivity Condition:
            \begin{enumerate}[1)]
              \item{If the valence of $v_{i -1}$ (the parent) is
                greater than 2 then $w_i\neq -2$}
            \end{enumerate}
          }
      \end{itemize}
    }

  \item {If the valence of $v_i$ is 2 and all of:
      \begin{enumerate}
        \item{Stick Condition:
            \begin{enumerate}[1.]
              \item {$w_i\neq 0$}
              \item{
                \begin{enumerate}[i)]
                  \item{ If valence of the child is 2 then
                      $\text{sign}\LP w_i\RP\neq\text{sign}\LP
                    w_{i+1}\RP$ (the child)}
                  \item {If valence of the parent is 2 then
                      $\text{sign}\LP w_i\RP\neq\text{sign}\LP w_{i
                    -1}\RP$ (the parent)}
                \end{enumerate}
              }
            \item {If valence of the parent or valence of the child
              is greater than 2 (is essential) then $w_i\neq \pm1$}
          \end{enumerate}
        }
      \item {Positivity Condition:
          \begin{enumerate}[1.]
            \item {If valence of the parent and valence of the child
              is greater than 2 (both are essential) then $w_i\neq \m2$}
          \end{enumerate}
        }
    \end{enumerate}
  }
\end{enumerate}

\end{definition}From this definition, we now show that these
conditions are identical to those
in the global context of Definition~\ref{wpt-equi-def-abcanon}.

\begin{theorem}{}{vertex-and-cannon}$\Gamma$ is a $\LP+\RP$-CWPTT if
and only if all the vertices of $\Gamma$ are
$\LP+\RP$-canonical.

\end{theorem}
\begin{proof}
Checking each canonicity condition locally shows both directions.
\end{proof}
%  prettier-ignore-start

The definition and proof for $\LP -\RP$-canonical vertices are identical.
Similarly to the canonical tree case, we define a third positivity class for a
vertex, the \textbf{neutral vertex}, a vertex that is both $\LP
-\RP$-canonical and
$\LP+\RP$-canonical.

\subsubsection{Minimalization of CWPTT}\label{sec-minimalization}

%  prettier-ignore-end

\paragraph{CWPTT are Not Minimal}

A common measure for the complexity of knots and their relatives is the
\textbf{minimal crossing number}. That being the least number of
crossings needed to
realize the object in a diagram, we call that diagram the
\textbf{minimal diagram}.
It is natural to ask if our CWPTT are minimal representatives among either all
representations or arborescent representations. A quick analysis of the process
of canonization demonstrates that CWPTT are unfortunately far from minimal even
among arborescent representatives. An example of canonizing a tangle, making
that tangle non-minimal, is seen in Figure~\ref{minimal-fig-nonmin_min}.

\begin{figure}[H]
\centering
\begin{subfigure}[b]{\textwidth}
\centering
\includegraphics[width=.6\textwidth]{files/non-minimal_minimali-5d080586fc68b86cb67ded41def430c0.pdf}
\caption[A minimal presentation of a arborescent tangle.]{A minimal
  presentation of a arborescent tangle in both its orthogonal projection
as well as its weighted planar tree.}
\label{minimal-fig-nonmin_min}
\end{subfigure}
\newline
\centering
\begin{subfigure}[b]{\textwidth}
\centering
\includegraphics[width=.6\textwidth]{files/non-minimal-b7d96efc2cab4d59d3fdbe8e4e6fd919.pdf}
\caption[A non-minimal presentation of the same arborescent
tangle.]{A non-minimal presentation of the same arborescent tangle as
  Figure~\ref{minimal-fig-nonmin_min} in both its orthogonal projection
as well as its CWPTT.}
\label{minimal-fig-nonmin_non}
\end{subfigure}
\caption[Minimal and non-minimal trees.]{}
\label{minimal-fig-nonmin}
\end{figure}\paragraph{Canonization Can Increase Complexity}

As we have seen, a CWPTT often does not realize a minimal crossing
representative for an arborescent tangle. Since minimal crossing number is such
a common measure for complexity, we should understand how canonization impacts
the crossing number complexity of a CWPTT. We will accomplish this by
identifying a (non-unique) minimal arborescent representative for each tangle.
That is, a weighted planar tangle tree with minimal TCN among all weighted
planar tangle trees in its equivalence class. To begin we expand our
understanding of the moves in the calculus of arborescent tangles to those that
alter weights arithmetically. These moves are related to the arithmetic
operations on continued fractions \citep{bonahonNewGeometricSplittings2016}.

\begin{definition}{Bonahon and Siebenmann, Section 12.3
\textbf{\citep{bonahonNewGeometricSplittings2016}}}{minimal-def-arithmetic}
When carrying out the following \textbf{arithmetic moves} the
relative positions of weights are
critical to the invariance of the underlying knot pair.

\begin{itemize}
\item (0.1) The \textbf{0.1 move} replaces the left side with the
  right side of Figure~\ref{minimal-fig-move01}.

  \begin{figure}[H]
    \centering
    \includegraphics[width=0.625\linewidth]{files/def-06392bbc82305bf238bf19fae97e05ba.pdf}
    \caption[Move 0.1 on a weighted planar tree.]{Move 0.1 on a
    weighted planar tree.}
    \label{minimal-fig-move01}
  \end{figure}

\item (0.2) The \textbf{0.2 move} replaces the left side with the
  right side of Figure~\ref{minimal-fig-move02}. Additionally, the
  cyclic order of all descendants in the purple subtree is reversed
  and $\zeta$ ($Z$-axis rotation
  Figure~\ref{wpt-construc-fig-k4g-z}). The vertices $a$ and $b$ need
  not be valence two, either or both may have a valence greater than two.

  \begin{figure}[H]
    \centering
    \includegraphics[width=0.625\linewidth]{files/def-c1d1c36f8c5b6b3cf209c56ae164e0e1.pdf}
    \caption[Move 0.2 on a weighted planar tree.]{Move 0.2 on a
    weighted planar tree.}
    \label{minimal-fig-move02}
  \end{figure}

\item (1.1) The \textbf{1.1 move} replaces the left side with the
  right side of Figure~\ref{minimal-fig-move11}.

  \begin{figure}[H]
    \centering
    \includegraphics[width=0.625\linewidth]{files/def-c88ffe87967c699fe0df6b57f440c690.pdf}
    \caption[Move 1.1 on a weighted planar tree.]{Move 1.1 on a
    weighted planar tree.}
    \label{minimal-fig-move11}
  \end{figure}

\item (1.2) The \textbf{1.2 move} replaces the left side with the
  right side of Figure~\ref{minimal-fig-move12}. The vertices $a$ and
  $b$ need not be valence two, either or both may have a valence
  greater than two.

  \begin{figure}[H]
    \centering
    \includegraphics[width=0.625\linewidth]{files/def-7fcc9ebbcf1bc20bcad55cabc45a67d1.pdf}
    \caption[Move 1.2 on a weighted planar tree.]{Move 1.2 on a
    weighted planar tree.}
    \label{minimal-fig-move12}
  \end{figure}

\item (2.1) Replace the left side with the right side of
  Figure~\ref{minimal-fig-move21}

  \begin{figure}[H]
    \centering
    \includegraphics[width=0.625\linewidth]{files/def-0d1c913041ef32fcc5f274b63cf2c009.pdf}
    \caption[Move 2.1 on a weighted planar tree.]{Move 2.1 on a
    weighted planar tree.}
    \label{minimal-fig-move21}
  \end{figure}

\item (2.2) Replace the left side with the right side of
  Figure~\ref{minimal-fig-move22}. The vertices $a$ and $b$ need not
  be valence two, either or both may have a valence greater than two.

  \begin{figure}[H]
    \centering
    \includegraphics[width=0.625\linewidth]{files/def-53b898a2529b73e08acc755c45e36375.pdf}
    \caption[Move 2.2 on a weighted planar tree.]{Move 2.2 on a
    weighted planar tree.}
    \label{minimal-fig-move22}
  \end{figure}
\end{itemize}

\end{definition}
\begin{note}
The 2.1 and 2.2 moves are what allow us to pass between the $\LP+\RP$ and
$\LP -\RP$ canonical classes of trees.
\end{note}

From here we will show that canonization of minimal trees increases TCN
complexity in a controlled manner.

\begin{theorem}{}{minimal-thm-minimal}A minimal tree canonizes to a
$\LP+\RP$-CWPTT, with only the moves 1.2, 2.1,
and 2.2 increasing TCN.

\end{theorem}
\begin{proof}Let $\Gamma$ be a minimal TCN arborescent representative
of its equivalence
class. Starting with the weight condition
(W) maximally apply $F_3$ consolidating the weights of each vertex.
$F_3$ may need to be reapplied after application of arithmetic moves
and does not impact TCN.

Next we handle the stick condition starting with condition S.2 concerning
sticks having non-zero weights. Maximally apply
to $\Gamma$ moves 0.1, and 0.2; this removes zero weights of
sticks. In the general canonization process S.1, the alternating
stick condition, is handled by application
of move 1.2. However, on a pair of vertices violating condition
S.1, move 1.2
decreases the TCN. This means by minimality of $\Gamma$, condition
S.1 must already be
satisfied. Finally to obtain condition S.3, we apply moves 1.1 and
1.2, where move 1.2 may increase crossing number by 1.

The last condition to enforce is P, the positvity condition, which is
done by modifying $\Gamma$ by application of moves 2.1
and 2.2. Similarly to our S.1 case, $\Gamma$ is minimal, 2.1,
and 2.2 cannot decrease TCN. However, 2.1 and 2.2 may increase TCN
by 1 and 2 respectively.

\end{proof}

We note that
Theorem~\ref{minimal-thm-minimal} indicates that
any opportunities to decrease TCN in a CWPTT are found on the end
vertices of sticks adjacent to
an essential vertex. Particularly, at least one weight
used in the execution of 1.2, 2.1, and 2.2 must be carried by an essential
vertex. Reversing the sequence of moves in
Theorem~\ref{minimal-thm-minimal} tells us that a
minimal tree can be constructed from a CWPTT by application of TCN
decreasing moves 1.2,
2.1, and 2.2. It is important to note that this does not guarantee that
every set of applications of the 1.2, 2.1, and 2.2 moves to a CWPTT
minimizes the TCN, only the existence of a path to a minimal tree.

\begin{note}
The reversed sequence of moves in Theorem~\ref{minimal-thm-minimal}
taking us from a CWPTT to a minimal weighted planar tangle tree does
not need to increase TCN to obtain minimality.
\end{note}

\paragraph{Bounding complexity}

We now introduce a bound on complexity between a CWPTT and a minimal
representation of that tree.  We build the bound by identifying the
maximum number of subtrees of a CWPTT which admit
a 1.2, 2.1, or 2.2 move. We begin by identifying the smallest TCN of a
subtree that admits each move. These minimal subtrees follow directly from
Definition~\ref{minimal-def-arithmetic} and our essential vertex
requirement. The smallest, by TCN, canonical subtree admitting move 1.2 is
7, as in Figure~\ref{minimal-fig-minimize_move12}.
For move 2.1 the smallest TCN for a canonical subtree is 5, as in
Figure~\ref{minimal-fig-minimize_move21}, and for 2.2 the smallest TCN is
10, as in Figure~\ref{minimal-fig-minimize_move22}. Combining these
smallest TCN subtrees with the how each move decreases TCN gives the
bound in Equation~(\ref{bound-arbor-cwptt}), where $\Gamma_m$ is a
minimal representative for the
tangle class of $\Gamma$.

\begin{equation}
\label{bound-arbor-cwptt}
\begin{aligned}
\text{TCN}\LP\Gamma\RP-\text{TCN}\LP\Gamma_m\RP&\leq
1\cdot\left\lfloor\frac{\text{TCN}\LP\Gamma\RP}{7}\right\rfloor+1\cdot\left\lfloor\frac{\text{TCN}\LP\Gamma\RP}{5}\right\rfloor+2\cdot\left\lfloor\frac{\text{TCN}\LP\Gamma\RP}{10}\right\rfloor\\
&\leq
\frac{\text{TCN}\LP\Gamma\RP}{7}+\frac{\text{TCN}\LP\Gamma\RP}{5}+2\cdot\frac{\text{TCN}\LP\Gamma\RP}{10}\\
&=\frac{\text{TCN}\LP\Gamma\RP\cdot 19}{35}\\
\end{aligned}
\end{equation}

\begin{figure}[H]
\centering
\begin{subfigure}[b]{.3\textwidth}
\centering
\includegraphics[width=\textwidth]{files/minimal-382709d139690600e8a2d4027e6ae77c.pdf}
\caption[Admits move 1.2.]{Admits move 1.2}
\label{minimal-fig-minimize_move12}
\end{subfigure}
~
\begin{subfigure}[b]{.2\textwidth}
\centering
\includegraphics[width=\textwidth]{files/minimal-8a6f09e8a725817c264714a97dc2e1f8.pdf}
\caption[Admits move 2.1]{Admits move 2.1}
\label{minimal-fig-minimize_move21}
\end{subfigure}
~
\begin{subfigure}[b]{.3\textwidth}
\centering
\includegraphics[width=\textwidth]{files/minimal-d307983510930002fc016efa47e0e062.pdf}
\caption[Admits move 2.2]{Admits move 2.2}
\label{minimal-fig-minimize_move22}
\end{subfigure}

\label{minimal-cont-move-subtrees}
\caption[Subtrees for minimalization moves.]{Examples of
canoncial subtrees of a $\LP+\RP$-CWPTT which admit the given moves.
Note: These are subtrees admiting the moves, but not the only subtrees admiting
the moves. }
\end{figure}
%  prettier-ignore-start

\subsection{Right Leaning Identity CWPTT}\label{subsec-rlitt}

%  prettier-ignore-end

The CWPTT are sufficient for distinguishing any two arborescent tangles via
moves on their trees. Unfortunately, the equivalence class of CWPTT is still too
large for computational enumeration to be feasible. The time required for
pairwise comparisons grows badly exponentially. Luckily, from the class of CWPTT
for an arborescent tangle, we can select a unique preferred form that allows for
efficient direct enumeration by computer. To achieve this we will define two
additional conditions for CWPTT, first the right leaning condition, and second,
the identity condition. We call these preferred CWPTT \textbf{Right
Leaning Identity
Canonical Weighted Planar Tangle Trees (RLITT)}.

\subsubsection{Existence of Right Leaning CWPTT}

We start our construction of RLITT by defining what conditions make a CWPTT a
right leaning CWPTT.

\begin{definition}{}{rli-const-def-rl}A CWPTT is called \textbf{right
leaning} if all weights are in the highest indexed
region (as in Section~\ref{indexing-rpt}) of each vertex.
Additionally, any ring subtrees
that are children of a vertex are the highest indexed children of that vertex.

\end{definition}Our next step is to show that every arborescent
tangle has a right leaning
representative.

\begin{theorem}{}{rli-const-thm-rl-exists}Every arborescent tangle
has a right leaning CWPTT representative.

\end{theorem}
\begin{proof}Let $\Gamma$ be a CWPTT representative for a tangle $T$.
If every weight
$w_i$ of $\Gamma$ is in the highest indexed region of $\Gamma_{w_i}$, we
are done. Otherwise, we will follow a similar algorithm to that outlined by
Bonahon and Siebenmann \citep{bonahonNewGeometricSplittings2016} for
distinguishing
CWPTT. Let $w_i$ be the weight for the lowest indexed vertex $v_i$ not in its
highest indexed region of $\Gamma_{v_i}$. With move $F_3^\prime$ shift $w_i$ so
that it lies in the highest indexed region. Further, choose to shift $w_i$
anti-clockwise, as this ensures that $v_j$ with $j<i$ are unchanged
when $w_i$ is
odd. Additionally, with ring moves position the ring subtrees in the right most
region before the weight. We repeat this process for any $v_k$ with
$i<k$ where the weight $w_k$ not
in the highest indexed region. Since $\Gamma$ has finite vertices, the
algorithm terminates with a $\Gamma$ transformed into a right leaning tree
completing the proof.

\end{proof}
\subsubsection{Existence of Identity CWPTT}

Our second step in the construction of RLITT is to define what conditions make a
CWPTT an identity CWPTT.

\begin{definition}{}{rli-const-def-identity}A CWPTT is called an
\textbf{identity tree} if its free bond is marked by
$\iota\in V_4$.

\end{definition}Again, we must show that every arborescent tangle has
an identity
representative.

\begin{theorem}{}{rli-const-thm-ident_exists}Every arborescent tangle
has an identity CWPTT representative.

\end{theorem}
\begin{proof}Let $\Gamma$ be a CWPTT representative for a tangle $T$.
If the label
$\alpha$ for the free bond of $\Gamma$ is $\iota$ we are done. Otherwise,
we fall into one of three cases:

\begin{itemize}
\item $\alpha=\zeta$: In the case $\alpha$ is $\zeta$ we apply move $F_1$. This
  modifies $\alpha$ by $\zeta$ yielding $\alpha\zeta=\zeta\zeta=\iota$.
\item $\alpha=\eta$: In the case $\alpha$ is $\eta$ we apply move $F_{2e}$. This
  modifies $\alpha$ by $\eta$ yielding $\alpha\eta=\eta\eta=\iota$.
\item $\alpha=\xi$: In the case $\alpha$ is $\xi$ we apply move $F_{2o}$. This
  modifies $\alpha$ by $\xi$ yielding $\alpha\xi=\xi\xi=\iota$.
\end{itemize}

This transforms $\Gamma$ into an identity tree completing the proof.

\end{proof}
\subsubsection{Existence of Right Leaning Identity CWPTT (RLITT)}

What we have shown is that every arborescent tangle has at least one right
leaning CWPTT and at least one identity CWPTT representative. Combining these
two ideas, we will show that every arborescent tangle has at least one CWPTT
that is right leaning and identity, we call such a CWPTT a RLITT.

\begin{definition}{}{rli-const-def-rlident}A CWPTT is called a
\textbf{right leaning identity tangle tree (RLITT)} if it's a
right leaning and identity tree.

\end{definition}
\begin{theorem}{}{rli-const-thm-rightident_exists}Every CWPTT has a
right leaning identity representative.

\end{theorem}
\begin{proof}Let $\Gamma$ be an identity CWPTT representative for a tangle $T$.
Applying the algorithm described in the proof of
Theorem~\ref{rli-const-thm-rl-exists} transforms $\Gamma$ into a right leaning
tree. Our requirement that $F_3^\prime$ be anti-clockwise ensures that the
resulting tree retains label $\iota$. This shows that $\Gamma$ can be
represented as a right leaning identity CWPTT.

\end{proof}
\subsubsection{Uniqueness of Right Leaning Identity CWPTT}

Our final step at identifying a preferred representative CWPTT of an arborescent
tangle is to show that a $\LP +\RP$-RLITT is unique in the set of CWPTT
representing an arborescent tangle. We will utilize the key proposition from
Bonahon and Siebenmann \citep{bonahonNewGeometricSplittings2016} a
consequence of
Theorem~\ref{wpt-equi-thm-classi} and Theorem~\ref{wpt-equi-cor-algo}.

\begin{proposition}{Bonahon and Siebenmann, Proposition 12.22
\textbf{\citep{bonahonNewGeometricSplittings2016}}}{rli-const-prop-not_iso}
Let $\Gamma$ and $\Gamma^\prime$ be $\LP +\RP$-CWPTT tangle trees
with isomorphic
underlying abstract trees. Further, let $\varphi$ be a sequence of moves of the
calculus of arborescent tangles ($F_1$, $F_2$, $F_3^\prime$, and the
modified ring moves $\LP\pm R\RP$). Assume that there is an $i > 0$ such that
$\varphi$ respects the cyclic orders of weight and bonds at each vertex $v_j$
with $j < i$, and that the labels of the free bond $\alpha$ in
$\Gamma$ and of $\varphi\LP \alpha\RP$ in $\Gamma^\prime$ are identical. Assume
moreover that one of:

\begin{enumerate}
\item $\varphi$ reverses the cyclic order of bonds at $v_i$
\item $\varphi$ does not respect the label in $V_4$ of some free bond of a
  vertex $v_j$ with $1 \leq j < i$.
\end{enumerate}

Then $\varphi$ is not a sequence of moves of the calculus of arborescent tangles
taking $\Gamma^\prime$ to $\Gamma$.

\end{proposition}
\begin{theorem}{}{rli-const-thm-rlitt_unique}The $\LP +\RP$-RLITT
representative is unique in the class of CWPTT.

\end{theorem}
\begin{proof}Let $\Gamma$ and $\Gamma^\prime$ be two $\LP +\RP$-RLITT
representatives for an arborescent
tangle $T$. Assume for the sake of contradiction that
$\Gamma\neq \Gamma^\prime$, meaning $T$ has two distinct $\LP +\RP$-RLITT. The
classification result in Theorem~\ref{wpt-equi-thm-classi} and algorithm given
by Theorem~\ref{wpt-equi-cor-algo} we produce a sequence of moves in the
calculus of arborescent tangles that takes $\Gamma^\prime$ to $\Gamma$. Let
$\varphi$ be such a sequence. By construction the labels in $V_4$ of $\Gamma$
and $\Gamma^\prime$ agree. Now, since $\Gamma\neq \Gamma^\prime$ there must be a
first, in the total order, vertex $v_i$ where $\Gamma$ and $\Gamma^\prime$
disagree. As $\LP +\RP$-RLITT the location of weights for $v_i$ in $\Gamma$ and
$\Gamma^\prime$ must appear in the same region. This requires that the
disagreement at $v_i$ must be in cyclic order of its children. We find ourselves
in the first case of Proposition~\ref{rli-const-prop-not_iso}, making
$\varphi$ not a
sequence of moves of the calculus of arborescent tangles taking $\Gamma^\prime$
to $\Gamma$.

\end{proof}
%  prettier-ignore-start

\subsection{Computational Methods}\label{subsec-computation}

%  prettier-ignore-end

%  prettier-ignore-start

\subsubsection{An Encoding Strategy for Arborescent Knots and
Tangles}\label{sec-arborescent-linear}

%  prettier-ignore-end

The various flavors of weighted planar trees we have seen thus far are a useful
tool for manipulation of arborescent tangles by humans or machines.
Unfortunately, the tree structure is quite difficult to store directly in a
computer database. We will rectify this by introducing a linearization strategy
for weighted planar trees. This linearization strategy is designed to encode not
only CWPTT but arbitrary weighted planar tangle trees. If a weighted planar
tangle tree has more than one free bond we list the label as a subtree. We will
omit this from our algorithm description as we are primarily concerned with
RLITT.

We will descend the tree following the indexing of the total order
(Section~\ref{indexing-rpt}), where the total ordering is an ideal
ordering, specifically the
depth first ordering. As we descend the tree, we annotate the
linearization with two
sets of delimiters. Each delimiter communicates extra information about the type
of subtree it is delimiting, the two sets of delimiters are as follows:

\begin{itemize}
\item $\LB\ \RB$: Corresponds to a half-open proper stick and is
interpreted as a
twist vector for a rational tangle
\cite{kauffmanClassificationRationalKnots2002}.
\item $\LP\ \RP$: Corresponds to a vertex not on a half-open stick.
\end{itemize}

We will now walk through an example of the linearization algorithm. Let $\Gamma$
be a weighted planar tangle tree seen in Figure~\ref{wpt-rli-fig-23}.
As we walk the tree,
the vertex currently being linearized will be called the \textbf{object vertex}.

\paragraph{Tangle Linearization Example}

We begin by adding the $V_4$ label for our tangle to the linearization. We then
start the following algorithm with the root as the object vertex.

For the object vertex, we add a `$\LP\ \right.$' delimiter to our linearization.
Adding to our linearization the weights and children of the object vertex in an
anti-clockwise order. When a child bond is encountered, we descend to that
child. When we descend, we have two cases to consider, the child is a half-open
proper stick or otherwise.

\paragraph{Case 1: The Child Is The Root Of A Half-Open And Proper Stick}

When the child is the root of a is proper and half-open (contains a leaf
vertex), we append that stick as the twist vector
(Definition~\ref{rational-def-twistvector}) for the corresponding rational
tangle. Let the stick consist of the vertices, $v_i\cdots v_{i+k}$, and weights,
$w_i\cdots w_{i+k}$. We delimit the stick with $\LB\ \RB$, with each weight
separated by a space, and the leaf weight as the left most entry of the twist
vector. Further, every other entry is multiplied by -1, forcing the sign of
all entries to match, as in (\ref{linear-math-tv}).

\begin{equation}
\label{linear-math-tv}
\LB w_{i+k}\ \m w_{i+k-1}\cdots\ \m w_{i-1}\ w_{i}\RB
\end{equation}

\paragraph{Case 2: The Child Is Essential, On A Open Stick, Or On A
Non-Proper Stick}

In this case, we restart the algorithm from the beginning with the current
vertex as the object vertex.

When we have exhausted the children for the object vertex, we close our
linearization for that vertex with the delimiter `$\LN\ \RP$'. We then return to
the parent linearization, repeating until all vertices have been exhausted. An
example of a tree encoded with this strategy can be seen in
Figure~\ref{wpt-rli-fig-23}.

\begin{figure}[H]
\centering
\includegraphics[width=\linewidth]{files/watt_walk_tangle-18fc70625153f60d62d4bd17ca1eac74.pdf}
\caption[Encoded tree subtrees are indicated by color.]{Encoded tree
subtrees are indicated by color. Encoding follows the indicated path.}
\label{wpt-rli-fig-23}
\end{figure}

%  prettier-ignore-start

\subsubsection{Generation of Right Leaning Identity Weighted Planar
Tangle Trees}\label{sec-rlitt-generation}

%  prettier-ignore-end

\paragraph{Generation of Rooted Plane Tree}

Just as rooted plane trees serve as the scaffolding we built WPTT on, a rooted
plane tree algorithm will serve as the backbone for our RLITT generation
algorithm. We will now give a brief description of the generation algorithm for
rooted plane trees given by Nakano
\citep{nakanoEfficientGenerationPlane2002}. We
begin by defining the \textbf{rightmost path} of a rooted plane tree $\Gamma$.

\begin{definition}{Nakano, Section 2
\textbf{\citep{nakanoEfficientGenerationPlane2002}}}{}
Let $v_0$ be the root and $v_i$ be the highest indexed leaf (vertex of valence
$\leq 1$) of a rooted plane tree $\Gamma$. The unique path
$\LP v_0,\,\cdots,\,v_i \RP$, in the standard graph theoretic sense, is called
the \textbf{rightmost path} of $\Gamma$.

\end{definition}Next, we define a grafting operation on rooted plane
trees $\Gamma$ and
$\Gamma^\prime$.

\begin{definition}{}{rli-gen-def-grafting_op}Let
$\mathcal{T}^{p}_{n}$ be the set of rooted plane trees on $n$ vertices,
$\Gamma_r\in \mathcal{T}^{p}_{n}$ , and $\Gamma_s\in \mathcal{T}^{p}_{m}$.
Define the \textbf{grafting operation} as follows.

\begin{equation}
\begin{aligned}
  \star_i:\mathcal{T}^{rp}_{n}\times\mathcal{T}^{rp}_{m}&\to\mathcal{T}^{rp}_{n+m}\\
  \Gamma_r\times\Gamma_s&\mapsto\Gamma_r\star_i\Gamma_s
\end{aligned}
\end{equation}

At the vertex $v_i$ of $\Gamma_r$, introduce an edge to the root of $\Gamma_s$.
Now, adjust the indexing of a vertex $s_k$ of $\Gamma_s$ as $v_{n+k}$, placing
$\Gamma_s$ as the rightmost child of $v_i$. This results in a rooted plane tree
$\Gamma\in \mathcal{T}^{rp}_{n+m}$.

When grafting at the root $v_0$ we omit the index label in the grafting
operation, that is, $\star_0$ is written simply as $\star$. We call
$\Gamma_r$ the
\textbf{rootstock} and $\Gamma_s$ the \textbf{scion}.

\end{definition}
\begin{figure}[H]
\centering
\includegraphics[width=0.6\linewidth]{files/arbor_graph_grafting-61926ec067da7853030611f875cb397e.pdf}
\caption[Grafting a scion.]{Grafting a scion $\Gamma_s$ to a
rootstock $\Gamma_r$ with $\Gamma_r\star_3\Gamma_s$}
\label{rli-gen-fig-scion_grafting}
\end{figure}
Now we are prepared to give the algorithm to generate all rooted plane trees of
a given size that are created from $\Gamma$ by a sequence of $\star_\ast$
operations on the rightmost path of $\Gamma$.
\newpage
\begin{remark}{Find all rooted plane trees of a given size created
from \textbf{$\Gamma$
\citep{nakanoEfficientGenerationPlane2002}}}{find-all-related-trees}
\textbf{Input}

\begin{itemize}
\item A tree $\Gamma\in \mathcal{T}^{rp}_{i}$, $i<n$
\item A target size $n\in T$
\end{itemize}

\textbf{Output}

\begin{itemize}
\item All collections of trees $\Gamma\in \mathcal{T}^{rp}_{n}$
  created from $\Gamma$
\end{itemize}

\textbf{Routine}

\begin{enumerate}
\item Set $P$ to the rightmost path of $\Gamma$
\item Set $\Gamma_1$ as the single vertex rooted plane tree
\item If the number of vertices of $\Gamma$ is $n$ exit the algorithm
\item For each vertex $v_i$ in $P$
  \begin{enumerate}
    \item Graft $\Gamma_1$ onto $\Gamma$ as $\Gamma\star_i
      \Gamma_1=\Gamma^\prime$
    \item Start a new instance of Algorithm~\ref{find-all-related-trees} with
      $\Gamma=\Gamma^\prime$ and size=size
  \end{enumerate}
\end{enumerate}

\end{remark}This algorithm can be extended to an algorithm that finds
all rooted plane trees
up to a given size as follows.

\begin{remark}{Efficient generation of rooted plane trees
\textbf{\citep{nakanoEfficientGenerationPlane2002}}}{find-efficient-trees}
\textbf{Input}

\begin{itemize}
\item A target number of vertices of a tree $n\in \Z$
\end{itemize}

\textbf{Output}

\begin{itemize}
\item All trees $\Gamma\in \mathcal{T}^{rp}_{n}$
\end{itemize}

\textbf{Routine}

\begin{enumerate}
\item Set $\Gamma_1$ as the single vertex rooted plane tree
\item Execute Algorithm~\ref{find-all-related-trees} with
  $\Gamma=\Gamma_1$ and $n=n$
\end{enumerate}

\end{remark}
\newpage
\paragraph{Modification for RLITT}

The algorithm described above serves as the inspiration for the algorithm we
will build now for the enumeration of the arborescent tangles. Building this
algorithm begins with modifying the grafting $\star_i$ operation to operate on
weighted planar tangle trees as follows.

\begin{definition}{}{rli-gen-def-grafting_op-wpt}The \textbf{grafting
operation} $\star_i$ for weighted planar tangle trees, we require
that the free bond of the scion be grafted to $v_i$. We also specify that the
scion be grafted so that the rightmost weight of $v_i$ remains to the right of
the scion after grafting; this can be seen in
Figure~\ref{rli-gen-fig-scion_grafting_wth_weight}.

\end{definition}
\begin{figure}[H]
\centering
\begin{subfigure}[b]{\textwidth}
\centering
\includegraphics[width=0.6\textwidth]{files/awptt_graph_pregraft-28141689ce1fed14e0a3223ce9a0eede.pdf}
\caption[A WPTT rootstock and RLITT scion grafted.]{A rootstock
  $\Gamma_r=\iota\LP \LP2\LP2\LB3\RB2\LB -3\RB3\RP \LB3\RB4\RP 4\RP$
  in grey and scion $\Gamma_s=\iota\LB 10\ 9 \RB$ in orange. Each vertex is
labeled with its index in the order on $\Gamma$.}
\label{rli-gen-fig-scion_grafting_wth_weight_1}
\end{subfigure}
\newline
\begin{subfigure}[b]{\textwidth}
\centering
\includegraphics[width=0.6\textwidth]{files/awptt_graph_grafted-e265b8e668aca3adac89f27cd9529cd2.pdf}
\caption[Grafting at $v_2$.]{Grafting at $v_2$ yields
  $\Gamma_r\star_2\Gamma_s=\iota\LP \LP2\LP2\LB3\RB2\LB -3\RB\LB 10\ 9
\RB 3\RP \LB3\RB4\RP 4\RP$}
\label{rli-gen-fig-scion_grafting_wth_weight_2}
\end{subfigure}
\caption[Grafting trees.]{}
\label{rli-gen-fig-scion_grafting_wth_weight}
\end{figure}Just as we have adjusted the grafting operator, we must
adjust the Nakano
algorithm so it is aware of the extra data in an RLITT. The initial reaction to
this problem may be to simply annotate the scion of the grafting operation with
the weights necessary to reach a target TCN. Unfortunately, this method quickly
runs into issues generating even just the Montesinos tangles, a smaller class of
tangle described by \citep{bonahonNewGeometricSplittings2016}. We must make a
slightly more radical change to the Nakano algorithm, that being, grafting an
entire RLITT to only the root $v_0$ of the rootstock. We will now show that a
list of integral tangles (single vertex RLITT
Section~\ref{subsec-integral_tangle})
combined with grafting at the root generates all $\LP+\RP$-RLITT. To start, we
will prove that each $\LP+\RP$-RLITT is integral or the result of grafting
weighted planar tangle trees at the root.

\begin{theorem}{}{rli-gen-const-thm-acnm1toacn}Every $\Gamma$
$\LP+\RP$-RLITT of TCN $n$ is one of two forms:

\begin{enumerate}
\item $\Gamma$ is a single vertex with weight $\pm n$.
\item $\Gamma$ is the result of grafting at the root of some
  rootstock $\Gamma_r$
  and $\LP+\RP$-RLITT scion $\Gamma_s$ where:
  \begin{enumerate}
    \item In $\Gamma_r$, $v_0$ is valence two, and $v_1$ is canonical
      except for violating the stick condition by
      $\text{Sign}\LP v_0\RP=\text{Sign}\LP v_1\RP$. Each vertex in
      $\LS v_i\RS_{i=2}^n$ of $\Gamma_r$ is $\LP +\RP$-canonical.
    \item $\Gamma_r$ is $\LP+\RP$-RLITT
  \end{enumerate}
\end{enumerate}

\end{theorem}
\begin{proof}Let $\Gamma$ be a $\LP +\RP$-RLITT, we have three cases
based on the valence of
$v_0\in \Gamma$.

Valence of $v_0$ is:

\begin{enumerate}
\item \textbf{One:} $\Gamma$ is integral
  (Section~\ref{subsec-integral_tangle}), we fall into
  the first condition.

\item \textbf{Two:} $\Gamma$ is a stick at the root, we let
  $\Gamma_r=\iota\LB w_0\RB$
  and $\Gamma_s=\iota\LP\alpha w_1\RP$, where $\alpha$ is the remaining
  vertices, weights, and bonds of $\Gamma$. $\Gamma_r$ is integral, and as such
  is $\LP+\RP$-RLITT. Now to show that $\Gamma_s$ is $\LP+\RP$-RLITT. Since
  $\Gamma$ is $\LP+\RP$-RLITT, each $v_i$ in $\Gamma$ is canonical. Each vertex
  in $\Gamma_s$ is also in $\Gamma$, so $\Gamma_s$ has only canonical vertices
  and by Theorem~\ref{vertex-and-cannon} is $\LP+\RP$-canonical.
  Finally, since locations of
  weights are unchanged and grafting requires the scion to have identity label
  $\Gamma_s$ is $\LP+\RP$-RLITT.

\item \textbf{Three:} The root of $\Gamma$ has two children, we let
  $\Gamma_s$ be the tree
  with the right child of $v_0$ as a root and
  $\Gamma_r=\iota\LP \alpha w_0\RP$, where $\alpha$ is the remaining vertices,
  weights, and bonds of $\Gamma$. Showing $\Gamma_s$ is $\LP+\RP$-RLITT follows
  identically as it did in the first form.

  We now show $\Gamma_r$ is $\LP+\RP$-RLITT. First, given that $\Gamma$ is
  $\LP+\RP$-RLITT, each $v_i$ in $\Gamma$ is canonical, consequently each
  vertex that is unchanged in $\Gamma_r$ ($\LS v_i\RS_{i=2}^n$) is canonical.
  The vertices that differ in $\Gamma_r$ are $v_0$ and $v_1$, $v_0$ remains
  the root of $\Gamma_r$ so is canonical. Showing the canonicity of $v_1$
  depends on the valence of $v_1$. When the valence is 0 or is greater
  than 2, $\Gamma_r$ is $\LP+\RP$-RLITT by the same argument as $\Gamma_s$.
  When $v_1$ is valence 1 or 2, $\Gamma_r$ begins with a stick, and $v_1$
  continues to satisfy the weight, positivity, and the $\pm 1$ and 0 portion
  of the stick condition. For the sign condition, when the signs agree, we are
  in form 2.1 of this theorem, and when they disagree, we are in form 2.2.
  \newpage
\item \textbf{Greater than Three:} The final case is when the valence
  of $v_0$ is greater
  than three. We let $\Gamma_s$ be the tree with the right child of $v_0$ as a
  root and $\Gamma_r=\iota\LP \alpha w_0\RP$, where $\alpha$ is the remaining
  vertices, weights, and bonds of $\Gamma$. Both $\Gamma_r$ and $\Gamma_s$
  follow the same argument used for $\Gamma_s$ of the valence 1 case.
\end{enumerate}

\end{proof}An identical theorem can be phrased for the $\LP
-\RP$-RLITT case. With this
result, we can start our construction of a grafting algorithm for arborescent
tangles.

\begin{remark}{Find weighted planar trees by grafting RLITT scions to
the root of RLITT rootstocks}{find-grafted-trees}\textbf{Input}

\begin{itemize}
\item A collection of of RLITT scions $T_s$
\item A collection of RLITT rootstocks $T_r$
\end{itemize}

\textbf{Output}

\begin{itemize}
\item A collection of weighted planar trees
\end{itemize}

\textbf{Routine}

\begin{enumerate}
\item for each combination of $\Gamma_r\in T_r$ and $\Gamma_s \in T_s$
  \begin{enumerate}
    \item Compute $\Gamma_r\star \Gamma_s$
  \end{enumerate}
\end{enumerate}

\end{remark}Executing the algorithm with $\Gamma_r=\LS \iota\LB
1\RB\RS$ as the rootstock
and $\Gamma_s=\LS \iota\LB 2\ 0\RB\RS$ produces the tangle
$\iota\LB 2\ 0\ 1\RB$. Unfortunately, this resultant tangle violates the stick
condition and hence is not canonical. The remainder of this
subsection will refine
the grafting algorithm to satisfy each of the RLITT conditions.

\subparagraph{Weight Condition}

The simplest condition to verify is the weight condition. By construction,
grafting the rootstock and scion introduces no additional weights at the
grafting vertex. Meaning, the weight condition is satisfied with no adjustment
to the algorithm.

\subparagraph{Identity Condition}

The next RLITT condition to address is the identity condition. We note that the
$\star_i$ operation does not modify the $V_4$ label of the rootstock. This
observation means if the rootstock is identity, the grafted tree will also be
identity.

%  prettier-ignore-start

\subparagraph{Stick Condition}\label{rli-gen-sec-stick-con}

%  prettier-ignore-end

To start with the stick condition we will prove that if grafting produces a
non-canonical tree the non-canonical vertex must be adjacent to the root.

\begin{theorem}{}{only-the-root-matters}For $\Gamma_r$ a $\LP+\RP
-$RLITT and $\Gamma_s$ a $\LP+\RP -$RLITT scion,
the result of $\Gamma=\Gamma_r\star\Gamma_s$ is canonical, if all
$v_i$ at distance 1 or less from the root are canonical.

\end{theorem}
\begin{proof}We need to show that each vertex $v_i$ at a distance
greater than one from the
root of $\Gamma=\Gamma_r\star\Gamma_s$ is canonical. The vertex $v_i$ is also a
vertex of either $\Gamma_r$ or $\Gamma_s$. If the vertex is in $\Gamma_r$ then
$v_i$ has a parent in $\Gamma_r$ and if the valence of $v_i$ is 2 or more
$v_i$ also has children in $\Gamma_r$. The parent and children in $\Gamma$ are
the same as the parent and children in $\Gamma_r$. Since $\Gamma_r$ is RLITT
$v_i$ is canonical in $\Gamma_r$ and since it shares a parent and children in
$\Gamma$ it is also canonical in $\Gamma$. Similarly, for $v_i$ in the scion,
showing canonicity of grafting is dependent only on the canonicity of the
vertices at a distance up to one from the root.

\end{proof}The $\LP -\RP$-RLITT case is shown identically, covering
the majority of
possibilities. However, we need to take special care when grafting a
non-negative to a non-positive tree (or the reverse). Before we address that
case we define a restricted class of scions that, after grafting, satisfy the
nonzero portion of the stick condition.

\begin{definition}{}{}A $\LP+\RP -$RLITT (respectively $\LP+\RP
-$RLITT) $\Gamma$ with root weight $w_0$
is called a \textbf{good scion} when either:

\begin{enumerate}
\item $w_0\neq0$
\item $w_0=0$ and the valence of $v_0$ is greater than 2 (essential)
\end{enumerate}

\end{definition}
\begin{theorem}{}{positive-and-negative-dont-mix}For $\Gamma_r$ a
non-negative $\LP+\RP -$RLITT, and $\Gamma_s$ a good
non-positive $\LP -\RP -$RLITT scion, the result of
$\Gamma=\Gamma_r\star\Gamma_s$ is non-canonical.

\end{theorem}
\begin{proof}$\Gamma_r$ is a non-neutral $\LP+\RP -$RLITT, so it has
a non-root vertex $v_{i}$,
which is a stick of the form Figure~\ref{wpt-construc-fig-positivity_cond-1} or
Figure~\ref{wpt-construc-fig-positivity_cond-2}. Similarly,
$\Gamma_s$ is a non-neutral
$\LP+\RP -$RLITT, so it has a non-root vertex $v_{j}$, which is a stick of the
form Figure~\ref{wpt-construc-fig-negativity_cond-1} or
Figure~\ref{wpt-construc-fig-negativity_cond-2}.
Since, $v_i$ and $v_j$ are not at the root, they remain sticks of the form
Figure~\ref{wpt-construc-fig-positivity_cond-1},
Figure~\ref{wpt-construc-fig-positivity_cond-2},
Figure~\ref{wpt-construc-fig-negativity_cond-1}, or
Figure~\ref{wpt-construc-fig-negativity_cond-2}
after grafting. Making $\Gamma$ neither $\LP+\RP -$RLITT nor $\LP
-\RP -$RLITT, as
we desired.

\end{proof}

\newpage
\begin{remark}{Find weighted planar trees by grafting RLITT good
scions to the root of RLITT rootstocks}{find-grafted-good-trees}\textbf{Input}

\begin{itemize}
\item A collection of RLITT good scions $T_s$
\item A collection of RLITT rootstocks $T_r$
\end{itemize}

\textbf{Output}

\begin{itemize}
\item A collection of weighted planar trees (still not guaranteed to be RLITT)
\end{itemize}

\textbf{Routine}

\begin{enumerate}
\item for each combination of $\Gamma_s\in T_s$ and $\Gamma_r \in T_r$
  \begin{enumerate}
    \item Compute $\Gamma = \Gamma_r\star \Gamma_s$
    \item for each vertex $v_i$ at distance 1 from the root of $\Gamma$
      \begin{enumerate}
        \item Continue to the next iteration of the outer loop if
          $v_i$ fails to satisfy the stick condition
      \end{enumerate}

    \item Report $\Gamma$
  \end{enumerate}
\end{enumerate}

\end{remark}
%  prettier-ignore-start

\subparagraph{Positivity/Negativity Condition}\label{rli-gen-sec-pm-con}

%  prettier-ignore-end

Our approach to the positivity and negativity condition follows our approach to
the stick condition. We will leverage
Theorem~\ref{only-the-root-matters} to add a check for
positvity and negativity in our algorithm.

\begin{remark}{Find weighted planar trees by grafting
\textbf{$\LP+\RP$-RLITT good scions to the root of $\LP+\RP$-RLITT
rootstocks}}{find-grafted-good-trees-p}
\textbf{Input}

\begin{itemize}
\item A collection of $\LP+\RP$-RLITT good scions $T_s$
\item A collection of $\LP+\RP$-RLITT rootstocks $T_r$
\end{itemize}

\textbf{Output}

\begin{itemize}
\item A collection of weighted planar trees (still not guaranteed to be RLITT)
\end{itemize}

\textbf{Routine}

\begin{enumerate}
\item for each combination of $\Gamma_s\in T_s$ and $\Gamma_r \in T_r$
  \begin{enumerate}
    \item Compute $\Gamma = \Gamma_r\star \Gamma_s$
    \item for each vertex $v_i$ at distance 1 from the root of $\Gamma$
      \begin{enumerate}
        \item Continue to the next iteration of the outer loop if
          $v_i$ fails to satisfy the stick condition
        \item Continue to the next iteration of the outer loop if
          $v_i$ fails to satisfy the positivity condition
      \end{enumerate}

    \item Report $\Gamma$
  \end{enumerate}
\end{enumerate}

\end{remark}
\begin{remark}{Find weighted planar trees by grafting \textbf{$\LP
  -\RP$-RLITT good scions to the root of $\LP -\RP$-RLITT
rootstocks}}{find-grafted-good-trees-n}
\textbf{Input}

\begin{itemize}
\item A collection of $\LP -\RP$-RLITT good scions $T_s$
\item A collection of $\LP -\RP$-RLITT rootstocks $T_r$
\end{itemize}

\textbf{Output}

\begin{itemize}
\item A collection of weighted planar trees (still not guaranteed to be RLITT)
\end{itemize}

\textbf{Routine}

\begin{enumerate}
\item for each combination of $\Gamma_s\in T_s$ and $\Gamma_r \in T_r$
  \begin{enumerate}
    \item Compute $\Gamma = \Gamma_r\star \Gamma_s$
    \item for each vertex $v_i$ at distance 1 from the root of $\Gamma$
      \begin{enumerate}
        \item Continue to the next iteration of the outer loop if
          $v_i$ fails to satisfy the stick condition
        \item Continue to the next iteration of the outer loop if
          $v_i$ fails to satisfy the negativity condition
      \end{enumerate}

    \item Report $\Gamma$
  \end{enumerate}
\end{enumerate}

\end{remark}\subparagraph{Right Leaning Condition}

Satisfying the right leaning condition is a consequence of the modified
$\star_i$ operation. Our definition of the $\star_i$ operation grafts the scion
in such a way that weights at $v_i$ are always right of the scion. To fully
satisfy the right leaning condition, we need to ensure that any stick subtrees
of are in the right most postions. This is accomplished with a slight
modification of our grafting algorithms.

\begin{remark}{Find weighted planar trees by grafting
\textbf{$\LP+\RP$-RLITT good scions to the root of $\LP+\RP$-RLITT
rootstocks}}{find-grafted-good-trees-p-r}
\textbf{Input}

\begin{itemize}
\item A collection of $\LP+\RP$-RLITT good scions $T_s$
\item A collection of $\LP+\RP$-RLITT rootstocks $T_r$
\end{itemize}

\textbf{Output}

\begin{itemize}
\item A collection of RLITT
\end{itemize}

\textbf{Routine}

\begin{enumerate}
\item for each combination of $\Gamma_s\in T_s$ and $\Gamma_r \in T_r$
  \begin{enumerate}
    \item Compute $\Gamma = \Gamma_r\star \Gamma_s$
    \item Shift ring subtrees of the root of $\Gamma$ to the right
    \item for each vertex $v_i$ at distance 1 from the root of $\Gamma$
      \begin{enumerate}
        \item Continue to the next iteration of the outer loop if
          $v_i$ fails to satisfy the stick condition
        \item Continue to the next iteration of the outer loop if
          $v_i$ fails to satisfy the positivity condition
      \end{enumerate}

    \item Report $\Gamma$
  \end{enumerate}
\end{enumerate}

\end{remark}
\begin{remark}{Find weighted planar trees by grafting \textbf{$\LP
  -\RP$-RLITT good scions to the root of $\LP -\RP$-RLITT
rootstocks}}{find-grafted-good-trees-n-r}
\textbf{Input}

\begin{itemize}
\item A collection of $\LP -\RP$-RLITT good scions $T_s$
\item A collection of $\LP -\RP$-RLITT rootstocks $T_r$
\end{itemize}

\textbf{Output}

\begin{itemize}
\item A collection of RLITT
\end{itemize}

\textbf{Routine}

\begin{enumerate}
\item for each combination of $\Gamma_s\in T_s$ and $\Gamma_r \in T_r$
  \begin{enumerate}
    \item Compute $\Gamma = \Gamma_r\star \Gamma_s$
    \item Shift ring subtrees of the root of $\Gamma$ to the right
    \item for each vertex $v_i$ at distance 1 from the root of $\Gamma$
      \begin{enumerate}
        \item Continue to the next iteration of the outer loop if
          $v_i$ fails to satisfy the stick condition
        \item Continue to the next iteration of the outer loop if
          $v_i$ fails to satisfy the negativity condition
      \end{enumerate}

    \item Report $\Gamma$
  \end{enumerate}
\end{enumerate}

\end{remark}\paragraph{Full Generation Algorithm}

The algorithm we have developed so far generates new RLITT from two restricted
collections of trees. Unfortunately, it doesn't yet tell us how to select the
collections that guarantee the generation of all arborescent tangles up to a
target TCN are represented. The final step in the generation scheme is
describing how to build these collections. With computer enumeration in mind, we
would like for our strategy to be easily split into jobs that can be run in
parallel.

We observe that when grafting $\Gamma_r\star_i\Gamma_s=\Gamma$, the TCN $r$ of
$\Gamma_r$ and the TCN $s$ of $\Gamma_s$, sum to the TCN of $\Gamma$. This
observation is the key underpinning of the strategy we use to define discrete
generation jobs. For a target TCN we have the integer pairs seen in
(\ref{rli-gen-eq-buckets}) that sum to the target TCN. Each of these
pairs defines two classes, determined by TCN, of RLITT that can be grafted.

\begin{equation}
\label{rli-gen-eq-buckets}
\begin{aligned}
(0&,\text{TCN})\\
( 1&,\text{TCN}-1)\\
&\vdots\\
(\text{TCN}-1&,1)\\
(\text{TCN}&,0)
\end{aligned}
\end{equation}

The next question we should ask is if we can simplify the list at all. First, as
we saw in the discussion of the stick condition
Section~\ref{rli-gen-sec-stick-con}, we
need our scions to be good. This means we cannot have 1 or 0 in the second
position of the pair, excluding $(\text{TCN} -1,1)$ and
$(\text{TCN},0)$ from the
list. Second, the pair $(0,\text{TCN})$ will be excluded from our list, since
the zero-crossing tangle $\iota[0\ 0]$ can't serve as rootstock; grafting any
scion would violate the stick condition. We will recover tangles with root
weight 0 with a post-processing step.

The following is the recursive algorithm used to take us from the set of RLITT
with $\text{TCN} -1$ to the set of RLITT of the target TCN.

\begin{remark}{Find RLITT of given TCN from all RLITT of
TCN-1}{find-rlitt-from-acnm1toacn}\textbf{Input}

\begin{itemize}
\item A target TCN
\item All RILTT up to TCN-1
\end{itemize}

\textbf{Output}

\begin{itemize}
\item A set $T$ of all RLITT of TCN
\end{itemize}

\textbf{Routine}

\begin{enumerate}
\item Set $i=1$
\item Set $N=\text{TCN} -1$
\item Set $T$ to the set $\LS \iota[\text{TCN}], \iota[0\ \text{TCN}]\RS$
\item For each pair $(i,N -i)$
  \begin{enumerate}
    \item Set $T_{s+}$ to be the set of $+$-RLITT good scions with TCN
      equal to $\text{N} -i$
    \item Set $T_{s -}$ to be the set of $( -)$-RLITT good scions with TCN
      equal to $\text{N} -i$
    \item Set $T_{r+}$ to be the set of $(+)$-RLITT TCN equal to $i$
    \item Set $T_{r -}$ to be the set of $( -)$-RLITT TCN equal to $i$
    \item Execute Algorithm~\ref{find-grafted-good-trees-n-r} with
      input $T_{r -}$ and $T_{s -}$
    \item Add the results to $T$
    \item Execute Algorithm~\ref{find-grafted-good-trees-p-r} with
      input $T_{r+}$ and $T_{s+}$
    \item Add the results to $T$
  \end{enumerate}

\item For every RLITT $\Gamma$ in $T$
  \begin{enumerate}
    \item Continue to the next iteration of the loop if root of
      $\Gamma$ is valence two with weight zero
    \item Compute $\iota[0]\star\Gamma$ and add to T
  \end{enumerate}
\end{enumerate}

\end{remark}
\begin{remark}{Find RLITT up to a given TCN}{find-rlitt-up-to-acn}\textbf{Input}

\begin{itemize}
\item A target TCN
\end{itemize}

\textbf{Output}

\begin{itemize}
\item A set $T$ of all RLITT up to TCN
\end{itemize}

\textbf{Routine}

\begin{enumerate}
\item Set $T$ to be the set $\LS
  \iota[0],\ \iota[0\ 0],\ \iota[1],\ \iota[ -1],\ \iota[2],\ \iota[
  -2],\,\ \iota[2\ 0],\ \iota[ -2\ 0]\RS$
\item for i from 3 to TCN
  \begin{enumerate}
    \item Execute Algorithm~\ref{find-rlitt-from-acnm1toacn} with
      input TCN $i$ and RLITT set
      $T$.
    \item Add the results to $T$
  \end{enumerate}
\end{enumerate}

\end{remark}It is important to note that the output of
Algorithm~\ref{find-rlitt-up-to-acn} includes
duplicates in the form of $\LP+\RP$-RLITT and $\LP -\RP$-RLITT pairs. To
deduplicate our list so it contains only topologically unique objects, we select
from the list the collection of $\LP+\RP$-RLITT.
