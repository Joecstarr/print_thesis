\chapter{Future Directions and Undergraduate
Research}\label{ch-future-directions}

In this chapter, we describe the future research directions for the tabulation
of tangles. The future directions take two forms. First, the direct next steps
to the work of this thesis
(Section~\ref{sec-future_work-continued_tabulation}). Second, an
undergraduate research research experience program with a collection
of undergraduate
problems (Section~\ref{sec-future_work-tabulation}).

%  prettier-ignore-start

\section{Continued Tabulation}\label{sec-future_work-continued_tabulation}

%  prettier-ignore-end

In this section, we describe the next steps for what has been discussed in
Chapter~\ref{ch-tabulation}. The first item to tackle in our future tabulation
work is the reconciliation of our table of arborescent tangles with
the algebraic table produced by Gren, Sulkowska, and,
Gabrov\v{s}ek \citep{gren2025classificationalgebraictangles}. Following this our
efforts take two forms, first a collection of minimalization problems for
the data we have generated. Second, is the expansion of this work
to cover the complete set of two string tangles, the arborescent and
the polygonal.

\subsection{Minimalization of Arborescent Tangles}

As we discussed in \nf{sec-minimalization}, that RLITT are often non-minimal
arborescent representatives of a tangle. In fact, minimal arborescent
representations may not be minimal tangle representations. Conway gives an
example \citep{conwayEnumerationKnotsLinks1970} where an arborescent
(algebraic) knot
is transformed into a minimal polygonal knot (\nf{subsubsec-opo-insert}). In
Figure~\ref{fig-6starstar_nonminimal} we rephrase Conway's example for tangles.

\begin{figure}[H]
  \centering
  \includegraphics[width=\linewidth]{files/fig-6starstar_nonmin-065532cac05a0e9000a7c014da0e0610.pdf}
  \caption[An arborescent tangle being turned into a polygonal
  tangle.]{An arborescent tangle being turned into a polygonal tangle
    via a sequence of
  interpolated Reidemister moves.}
  \label{fig-6starstar_nonminimal}
\end{figure}

This leads to two items that must be addressed to ensure the list of tangles
contains minimal diagrams.

\subsubsection{Identification of Minimal Arborescent Tangle Representations}

The first item, and most straightforward to address, is the identification of a
minimal arborescent representative of a given RLITT $\Gamma$. This requires the
identification of the weighted planar tree related to $\Gamma$ whose TCN is
minimal. We saw in \nf{sec-minimalization} the ways canonization can increase
complexity in a weighted planar tangle tree. To take $\Gamma$ to its minimal
form, we will need to develop the theory for and an implementation of an
efficient algorithm to systematically de-canonize $\Gamma$ into its minimal
arborescent form.

\subsubsection{Identification of Minimal Representations for
Arborescent Tangles}

Second, and more challenging, is the identification of
the minimal representative of a given arborescent tangle. That is, identifying
the minimal representative, arborescent or otherwise as in
Figure~\ref{fig-6starstar_nonminimal}. This task requires the development of a
classification of the subtrees of a weighted planar tree that correspond to
moves of the type similar to that seen in
Figure~\ref{fig-6starstar_nonminimal}. The
complexity of this task is compounded by the fact that the family of polygon
graphs allowing these types of moves is infinite (easily shown via an
induction). Further, the moves that enable arborescent tangles that are
minimally polygonal are not limited to the moves on the marked $6^{**}$
(Figure~\ref{fig-6starstar_nonminimal}). We can see a second class in
Figure~\ref{fig-other_nonminimal}.

\begin{figure}[H]
  \centering
  \includegraphics[width=\linewidth]{files/fig-other_nonminimal-665b635bba53dc590e590ae2271ee742.pdf}
  \caption[An arborescent tangle turned into a polygonal tangle.]{An
  arborescent tangle turned into a polygonal tangle.}
  \label{fig-other_nonminimal}
\end{figure}

\subsection{Polygonal Tangles}

We now discuss the expansion of the tangle tables to include all polygonal
tangles up to a target crossing number. Expanding tangle tables to include the
polygonal tangles is useful as at high crossing numbers, the polygonal tangles
dominate the arborescent tangles. Unfortunately, for the polygonal case, we lack
a general classification result. Meaning, as it stands, we have no theoretical
mechanism for direct generation of unique representatives for a polygonal tangle
as we have done with the classes of tangle in this thesis. The development of a
general classification for polygonal tangles is difficult, at least as difficult
as a general classification of knots. With this in mind, we will discuss two
possible directions for expanding the polygonal tangles.

\subsubsection{Ad Hoc Classification of Constellations}

In \nf{subsubsec-opo-insert}, we discussed constellations
\citep{connollyClassificationTabulation2string2021} used for
generation of polygonal
tangles via insertion. The crossing number of a polygonal tangle is bounded
below by the vertex count of its constellation. So the number of constellations
represented at reasonable crossing numbers is small. Additionally, when the
difference between crossing number and vertex count is low, many of the inserted
tangles will have low crossing numbers, again bounding complexity. In his thesis
work, Connolly \citep{connollyClassificationTabulation2string2021}
enumerates the ten
smallest constellations, those with ten or fewer vertices.

\begin{note}
  It's worth noting that expanding the table of constellations for a
  polygon graph is computationally
  hard, shown NP-Complete by
  Cook\citep{cookComplexityTheoremprovingProcedures1971}.
\end{note}

With low vertex count polygons, and at low crossing number, developing an ad hoc
classification result for each constellation may be a fruitful approach. For
example, consider the constellation seen in
Figure~\ref{fig-6starstar_const}, we can
enumerate the possible crossing numbers and locations for tangles to be
inserted, Table~\ref{sec-fw-adhoc-tab-to7}.

\begin{figure}[H]
  \centering
  \includegraphics[width=0.5\linewidth]{files/fig-6starstar_const-5b51b8bba46d02ef6ff52c23ba10c900.pdf}
  \caption[The unique constellation for $6^{\ast\ast}$.]{The unique
  constellation for $6^{\ast\ast}$}
  \label{fig-6starstar_const}
\end{figure}

We see that up to 7 crossings this constellation only admits rational
insertions, at 8 crossings we see our first Montesinos. Completing an analysis
of the possibilities for insertion may reveal patterns that allow a
classification of this $6^{**}$ constellation.
% Even partial results in this
% arena may yield more efficient heuristics when utilizing the methodology in
% \nf{subsubsec-fw-brute}.

\begin{figure}
  \centering
  \caption[Possible insertions of $6^{\ast\ast}$ by crossing
  number.]{Possible insertions of $6^{\ast\ast}$ by crossing number.}
  \label{sec-fw-adhoc-tab-to7}
  \begin{tabular}{p{\dimexpr 0.500\linewidth-2\tabcolsep}p{\dimexpr
    0.500\linewidth-2\tabcolsep}}
    \toprule
    Crossing Number & Possible Crossing numbers for insertion \\
    \hline
    5 & $\ast.1.1.1.1.1$ \\
    6 & \(\displaystyle
      \begin{aligned}\ast.2.1.1.1.1,\ast.1.2.1.1.1\\\ast.1.1.2.1.1,\ast.1.1.1.2.1,\\\ast.1.1.1.1.2
    \end{aligned} \) \\
    7 & \(\displaystyle
      \begin{aligned}\ast.1.1.1.2.2,*.1.1.2.1.2\\\ast.1.1.2.2.1,*.1.2.1.1.2,\\\ast.1.2.1.2.1,*.1.2.2.1.1,\\\ast.2.1.1.1.2,*.2.1.1.2.1,\\\ast.2.1.2.1.1,*.2.2.1.1.1,\\\ast.3.1.1.1.1,*.1.3.1.1.1,\\\ast.1.1.3.1.1,*.1.1.1.3.1,\\\ast.1.1.1.1.3
    \end{aligned} \) \\
    \bottomrule
  \end{tabular}
\end{figure}

%  prettier-ignore-start

% \subsubsection{Brute Force Tabulation}\label{subsubsec-fw-brute}

% %  prettier-ignore-end

% Without a full or partial classification of the polygonal tangles,
% we must take
% an alternative approach to what we have seen in this thesis. That
% alternative is
% the brute force, two pass enumeration strategy used in previous
% knot tabulation
% efforts \citep{dowkerClassificationKnotProjections1983,
% hosteFirst1701936Knots1998, burtonNext350Million2020} and outlined
% in Section~\ref{sec-history-of-tabulation}. The key
% difficulty in this methodology is the selection of invariants that combine to
% uniquely identify tangles. This difficulty is due to the fast growth rate for
% the count of tangles of a given crossing number. This growth rate means any
% invariant that is selected must have an efficient computation
% strategy. If, for
% example, as Burton did \citep{burtonNext350Million2020}, we select
% hyperbolic volume,
% we may be able to distinguish a large portion of tangles, however computation
% for crossing numbers as low as 10 will be intractable due to the
% per tangle time
% to compute the volume. This says nothing about the raw storage needed to hold
% the computed and partial data. A seemingly better choice are invariants
% (Section~\ref{subsec-invariant}) which have polynomial time
% computations, such as those
% introduced by van der Veen and Bar-Natan (to be published
% \citep{vanderveenKnotInvariantsFinite}). While weaker than
% hyperbolic volume, these
% invariants are stronger than the polynomial invariants, and faster
% than both to
% compute. Statistics for polynomial invariants can be found in Maguire's thesis
% work \citep{maguireKhovanovHomologyLegendrian}.

%  prettier-ignore-start

\section{Tabulation as Undergraduate Research}\label{sec-future_work-tabulation}

%  prettier-ignore-end

\subsection{A Research Experience Program for Undergraduates}

The accessibility of knot theory was discussed in
Section~\ref{sec-intro-intuit_knot_theory}.
This section elaborates on how that accessibility can be leveraged to engage
undergraduates in research. Throughout this thesis, we have investigated and
observed the depth and complexity of tabulation. We have seen how easily
portions of complex tabulation problems can be ``peeled off'' and decomposed as
self-contained problems. Additionally, we discussed product management training
(Section~\ref{sec-product-management}) and developed a software
enginering life cycle
(Section~\ref{sec-life-cycle}) for use in organizing undergraduate
research. These self
contained problems combined with our processes, produce a research experience
program ideal for undergraduates. The research experience program can be
enhanced by sequencing problems with a gradual release of responsibility model
as described by Fisher and Frey \citep{fisherBetterLearningStructured2013}.

We now outline a multi-semester program for training of undergraduates,
beginning when those students have only lower division (college algebra level)
maturity. To begin, we engage in directed reading with low level accessible
texts such as this thesis or The Knot Book: An Elementary Introduction to the
Mathematical Theory of Knots by Adams
\citep{adamsKnotBookElementary2004}. When the
student has gained a basic understanding of knots, a structured play problem
(potentially non-original work) can be introduced. Fitting the need here are
problems such as: the sculpting and 3D printing of stick
knots\footnote{A stick knot is a knot made of straight lines of a
unit length.} in a
program such as Blender or OpenSCAD \nf{sec-proj-sticks}, the creation of knot
mosaics \nf{sec-proj-mosaic}, or quilting Celtic knots
\nf{sec-proj-quilt}. The goal of
such an activity is to build wonder and cultivate confidence in the students
investigation skills.

As the student matures, their investigation skills improve, and their
knowledgebase deepens, they can be presented with more complex reading and novel
research problems. Depending on the student's interest, we present additional
but more advanced readings such as LinKnot
\citep{jablanLinKnotKnotTheory2007} or
accessible research papers such as Burton
\citep{burtonNext350Million2020}. We should
encourage freedom in these readings, allowing students to select sections and
formulate questions of their own. At this level, problems should still be well
structured having a clear path from start to finish but requiring the filling in
of gaps with original work. We should consider problems such as the translation
of notations as in \nf{sec-proj-notations}, or the computation of a
well understood
invariant, as in \nf{sec-projs-invariants}.

The program culminates with a mature undergraduate researcher ready to tackle
complex problems. At this point we expect the student to have mature reading and
reasoning skills, but perhaps lack skills such as literature review. Support for
reading at this stage should be focused on assisting students in finding answers
rather than answers being provided. Students may be prepared to formulate a
research question of their own and this should be encouraged; however presenting
students with ideas to build on or select from is beneficial. An ideal problem
here should fit student's interests and have a clear goal but perhaps no clear
starting point, for example the random tangle sampling seen in
\nf{sec-proj-rand}.

\subsection{Infrastructure of a Tabulation Program}

One key issue that must be addressed in a tabulation research program is
computational needs. Computing on knots and tangles is not necessarily a
computationally challenging task. Many problems are simple to describe
computationally and have efficient implementations. The primary challenge for
tabulation research stems from the raw scale of the dataset, as both the knot
and tangle datasets grow exponentially. This exponential growth of the data
gives us two primary areas of concern, time to compute and space to store. Even
if a problem has a nice constant time or linear time solution, doing a
computation on every knot or tangle in our dataset turns the problem into an
exponential one, a computational ``death by a thousand cuts''.

Generally when research questions run up against computational time constraints,
solutions take one of two forms, vertical scaling or horizontal scaling. We will
explain the two by analogy. Imagine we are trying to move a large boulder with a
bulldozer that is just not powerful enough for the job. We can trade in our
bulldozer for a bigger more powerful bulldozer that will push the rock without a
problem, this would be vertical scaling. Alternatively, we can blow up the
boulder and trade in our bulldozer for multiple smaller bulldozers, each able to
simultaneously move bits of the boulder, this would be horizontal scaling. In
our tabulation case, we could feasibly use either solution. If we vertically
scale, we could complete each computation faster. This makes sense for some
computations where each computation is slow such as hyperbolic volume. For other
computations, like the grafting seen in \nf{sec-rlitt-generation},
each computation
requires such little computational effort we would quickly run up against data
retrieval bottlenecks. In cases like this, distributing the effort horizontally
means, with some infrastructure effort, on aggregate we have less idle time in
the system. Another key feature of horizontal scaling in our case is the common
availability of clusters, at primarily undergraduate institutions. These
primarily undergraduate institutions often have no access to the large super
computers available at large institutions.

The planning and design of infrastructure leads us to address the second
challenge, space to store data. We will touch on two points: first the actual
storage of the data, and second, accessing and adding to the dataset. As we have
discussed, knot and tangle data grows exponentially, as expected, the space
needed to store that data also grows exponentially. As a benchmark, we use
arborescent tangles and the space required to store them in their linearized
form (\nf{sec-arborescent-linear}).

\bigskip\noindent
\begin{tabular}{p{\dimexpr 0.333\linewidth-2\tabcolsep}p{\dimexpr
  0.333\linewidth-2\tabcolsep}p{\dimexpr 0.333\linewidth-2\tabcolsep}}
  \toprule
  Tree Crossing Number & Projected Total number of tangles up to TCN
  & Projected Total size of notations up to TCN \\
  \hline
  20 &
  20178846455.0426 &
  744.93 GB\\
  21&
  77404113447.2751&
  3.02 TB\\
  22&
  296920571662.606&
  12.24 TB\\
  23&
  1138987289416.26&
  49.59 TB\\
  24&
  4369161597793.56&
  200.98 TB\\
  25&
  16760135017593&
  814.55 TB\\
  26&
  64292004387526.9&
  3.30 PB\\
  27&
  246624621285968&
  13.38 PB\\
  28&
  946053943972148&
  54.23 PB\\
  29&
  3629070212865634&
  219.77 PB\\

  \bottomrule
\end{tabular}

\bigskip$\,$ As we can see, the space required for storing tangles
quickly becomes
large. For perspective, a basic storage solution holds at least two copies of
the data, meaning to store arborescent tangles up to 21 crossings, we need
$3\times4\text{TB}=12\text{TB}$ of disk space (using 4TB as it's a common disk
size). More robust would be a solution that allows for two drive failures such
as RAIDZ2, in this case, we require $\LP2+3\RP\times4\text{TB}=20\text{TB}$. At
approximately $\$15$ per TB, putting us at around $\$200$ for the basic solution
and $\$300$ for the robust solution. It's important to remember this is the
required space and cost to store only a sequential list of notations for tangles
as in Example~\ref{sequential-list} making retrieval of a specific
tangle difficult.

\begin{example}{}{sequential-list}
  $$\iota\LP\LP\LB 2\RB \LB \m 3\RB \RP1\RP\iota\LP\LP\LB 2\RB \LB
  2\RB \m 1\RP1\RP\iota\LP\LP\LB 2\RB \LB 2\RB 1\RP1\RP\cdots$$
  A sequential list of tangles.
\end{example}

To expand the list, we require infrastructure that allows for random access, the
ability to search the data, and the ability to add additional data. This will
require that the data be stored in a database system. Unfortunately, a database
does come with a downside, it increases the storage requirements for the data
with mandatory overhead. With an effective data model and some consideration of
what should be stored and what should be computed on demand, we can mitigate
some of this overhead. However, even at low crossing number $\leq 12$, we will
quickly run into storage issues as we complete undergraduate computation
problems.

We will now discuss the selection and design of a database for a table of
tangles. Our data will be used by undergraduates, so an ideal database system is
one with a data model that has a shallow learning curve. Additionally, our data
is largely non-relational, meaning we don't need a database system geared to
relational data. These two items make a noSQL database system ideal. Just as we
had options for vertical scaling and horizontal scaling for carrying out the
computations we have the same two options for serving our database. However, in
the service case our choice is significantly more clear. Based on the size of
our data if we select vertical scaling our cost for a server will be in the
$\$10k -\$20k$ (not including storage cost) range and we may still end up with
bottlenecks. Therefore, for our needs, horizontal scaling is ideal. In this case
we can use a few low-cost cloud servers to store portions of the data, with a
coordinator balancing load across the system. If we encounter a bottleneck,
instead of buying a whole new expensive server, we simply add another low cost
server to our system. This horizontal scaling concept is called sharding, a
feature of MongoDB an ideal choice for our needs.

%  prettier-ignore-start

\subsection{Selection of Undergraduate Projects}\label{sec-selection_projects}

%  prettier-ignore-end

In this section we will provide a curated collection of undergraduate research
problem statements. We will also give a brief outline for each, contextualizing
the problem and describing what phase of the research experience program the
problem may be appropriate for:

\begin{enumerate}
  \item \textbf{Lower Division Student:} A lower division student is
    a student with little
    to no research or abstract math experience. A student at this
    level should be
    expected to have completed a college algebra course and started a calculus
    sequence. For students with a computational background we should expect the
    student to have started an introduction to programming course.
  \item \textbf{Intermediate Student:} An intermediate student is a
    student who has some
    exposure to abstract math. This could take the form of solving a lower
    division problem. These students should be well into a calculus sequence,
    having completed calculus II (advanced integration) or calculus III (vector
    calculus). For students with a computational background we should expect the
    student to have completed an introduction to programming sequence
    and started
    a course on algorithms and data structures.
  \item \textbf{Upper Division Student:} An upper division student is
    a student who can be
    expected to work semi-independently. They have solved one or more
    intermediate student problems, have completed the standard
    calculus sequence,
    and have begun abstract math courses. For students with a computational
    background we should expect the student to have completed a discrete methods
    course, ideally covering computational complexity theory.
\end{enumerate}

The remainder of the section gives statements for problems appropriate for
undergraduate research. The problems in the list fall into five types:
visualizations (\nf{sec-proj-visual}), invariants
(\nf{sec-projs-invariants}), notations
(\nf{sec-proj-notations}), generation (\nf{sec-proj-gen}), and
potpourri\footnote{Definition: A miscellaneous collection}
(\nf{sec-proj-pot}).

%  prettier-ignore-start

\subsubsection{Visualization}\label{sec-proj-visual}

%  prettier-ignore-end

Visualzation and spatial reasoning is a critically important for work in knot
theory. Problems of the visualization type develop specific visualizations or
general visualization tools for knots and tangles.

%  prettier-ignore-start

\paragraph{Create Knot Mosaics: Lower Division Student}\label{sec-proj-mosaic}

%  prettier-ignore-end

\subparagraph{Problem Statement}

Create a knot mosaic that has a particular property.

\subparagraph{Brief}

Knot mosaics are a simple method for creating knots from a collection of tiles.
Creating mosaics with a particular property, a specific writhe, for example, is
a fun and engaging activity where abstraction of a concept can be explored.
Modifying the tile set can add additional complexity to the task.

%  prettier-ignore-start

\paragraph{Create Stick Knots: Lower Division Student}\label{sec-proj-sticks}

%  prettier-ignore-end

\subparagraph{Problem Statement}

Create stick knots with desired properties.

\subparagraph{Brief}

Stick knots are knots built from a collection of unit sticks. Creating a
physical model by hand or with computer design and 3D printing develops spatial
reasoning skills needed for work in higher knot theory.

%  prettier-ignore-start

\paragraph{Creating Celtic Knots: Lower Division Student}\label{sec-proj-quilt}

%  prettier-ignore-end

\subparagraph{Problem Statement}

Create Celtic knots with desired properties.

\subparagraph{Brief}

Celtic knots are common artistic knots. Exploring the creation of a unique
ruleset for creating Celtic knots is an opportunity to develop a unique
understanding of the diagrammatic nuances of knot theory.

%  prettier-ignore-start

\paragraph{Compute Diagram for General Notations: Intermediate
Student}\label{sec-proj-diagram_att}

%  prettier-ignore-end

\subparagraph{Problem Statement}

Create an interface for plotting knots in an arbitrary notation in KnotPlot.

\subparagraph{Brief}

An aspect of knot theory that makes it among the most accessible higher math
domains is the ability for anyone to draw pictures of knots. A continuing theme
of this thesis is that computations/operations are easy by hand, with the
qualifier, ``up to reasonable crossing number''. This carries through with the
drawing of the diagrams. A common drawing tool in knot theory is KnotPlot
\citep{schareinInteractiveTopologicalDrawing1998}, unfortunatly KnotPlot has no
interface for drawing knots in ``arbitrary'' notations. The tool however, does
have a Lua scripting interface in which an arbitrary notation decoder can be
designed.
\newpage
\paragraph{Create VR Band Plumbing Visualizer: Upper Division Student}

\subparagraph{Problem Statement}

Create a 3D VR visualizer for the band plumbing construction for arborescent
knots.

\subparagraph{Brief}

The plumbing construction for arborescent knots and tangles is easiest
visualized in 3D. The ideal for visualizing these objects is in VR as this is
reduces the need for spatial reasoning. While the theory exists for creating the
objects, the linear algebra required makes this an upper division problem.

%  prettier-ignore-start

\subsubsection{Invariants}\label{sec-projs-invariants}

%  prettier-ignore-end

One way to build conjecture is by the analysis of patterns in data, these
conjectures often lead to the development of new theory. Problems in this
section create the collections of data that can be used for developing those
conjectures and theory.

%  prettier-ignore-start

\paragraph{Compute Polynomial From A Tangle Notation: Intermediate
Student or Upper Division Student}\label{sec-proj-homflypt}

%  prettier-ignore-end

\subparagraph{Problem Statement}

Develop the theory needed for efficiently computing polynomials of tangles

\subparagraph{Brief}

One of the most important advancements in knot theory was the discovery of knot
polynomials as a class of knot invariants. As an example, one of the most
powerful of these polynomials is the HOMFLYPT polynomial
\citep{freydNewPolynomialInvariant1985} constructed from the skein
relations equation
Figure~\ref{fig-future_work-skein_homfly}.

\begin{equation}
  \begin{aligned}
    P\LP\text{unknot}\RP&=1\\
    \ell P\left(L_{+}\right)+\ell^{-1} P\left(L_{-}\right)+m
    P\left(L_0\right)&=0\\
  \end{aligned}
\end{equation}

Conveniently, the data needed to apply the skein relations is precisely the data
encoded by RLITT, relative crossing data.

\begin{figure}[H]
  \centering
  \includegraphics[width=0.5\linewidth]{files/Skein_HOMFLY-6ebea249863dfdbae5e3432af0414ba9.pdf}
  \caption[The skein relation for the HOMFLYPT polynomial.]{The skein
    relation for the HOMFLYPT polynomial.  (Public domain, via
    Wikimedia Commons\citep{pbroks13SkeinHOMFLYPublic2008})
    % Similar to that seen for the Kauffman bracket Section~\ref{subsec-kauff}.
  }
  \label{fig-future_work-skein_homfly}
\end{figure}

Depending on the polynomial selected, the problem is appropriate for
intermediate students or upper division students. When the polynomial has a
developed tangle theory, the solution will have a well defined start and end
point and is appropriate for intermediate students. Otherwise, the full tangle
theory must be developed. This requires experience with the development of
original abstract theory, making the problem appropriate for upper division
students.

\paragraph{Compute Finite Type Invariant From A Tangle Notation:
Intermediate Student or Upper Division Student}

\subparagraph{Problem Statement}

Develop the theory needed for efficiently computing a finite type invariant of
tangles

\subparagraph{Brief}

Similar to the computation of polynomial invariants the computation of finite
type invariants expands our table with data useful for binning future tangles
and knots. Depending on the invariant selected, the problem is appropriate for
intermediate or upper division students. When the invariant has a developed
tangle theory the solution will have a well defined start and end point and be
appropriate for intermediate students. Otherwise, the full tangle theory must be
developed. This requires experience with the development of original abstract
theory, making the problem appropriate for upper division students.

%  prettier-ignore-start

\subsubsection{Notations}\label{sec-proj-notations}

%  prettier-ignore-end

There are many ways to encode the data of a knot, each with advantages and
disadvantages. Throughout this thesis, the primary target notation was the RLITT
Section~\ref{subsec-rlitt}. This subsection discusses several useful
notations where a
computational tool translating from and to RLITT is desired. Since the source
and destination notation in each problem are well understood each is appropriate
for intermediate students.

%  prettier-ignore-start

\paragraph{Notation Description for Extended Gauss Notation:
Intermediate Student}\label{sec-proj-note_gauss}

%  prettier-ignore-end

\subparagraph{Problem Statement}

Develop the theory translating RLITT to extended Gauss notation. Additionally,
develop the software needed for storing and translating per theory.

%  prettier-ignore-start

\paragraph{Notation Description For Planar Diagram (PD) Notation:
Intermediate Student}\label{sec-proj-note_pd}

%  prettier-ignore-end

\subparagraph{Problem Statement}

Develop the theory translating RLITT to PD notation. Additionally, develop the
software required for storing and translating per theory.

%  prettier-ignore-start

\paragraph{Notation Description For DT Notation: Intermediate
Student}\label{sec-proj-note_dt}

%  prettier-ignore-end

\subparagraph{Problem Statement}

Develop the theory translating RLITT to DT notation. Additionally, develop the
software needed for storing and translating per theory.

\subsubsection{Generation}\label{sec-proj-gen}

%  prettier-ignore-end

This section expands the census of tangles to more abstract classes. These
expanded lists increase accessibility of complex objects allowing for the
creation of new theory.

\paragraph{Create a table of virtual tangles: Upper Division Student}

\subparagraph{Problem Statement}

Create the theory needed to construct a table of virtual tangles.

\subparagraph{Brief}

Virtual knots, developed by Kauffman
\citep{kauffmanVirtualKnotTheory1999}, are an
extension of the knot concept where a knot shadow need not be planar. Some work
has been done on classifying the virtual tangles by Mellor and Nevin
\citep{mellorVirtualRationalTangles2020}. The full tangle theory must
be developed to
solve this problem. This requires experience with the development of original
abstract theory, making the problem appropriate for upper division students

\paragraph{Create a table of $n$ string tangles: Upper Division Student}

\subparagraph{Problem Statement}

Create the theory needed to construct a table of $n$ string tangles.

\subparagraph{Brief}

The tangles we have worked with in this thesis are the two string tangles, those
with four fixed points on the Conway circle. A natural extension to this concept
is the $n$ string tangle. Recently, Kwon
\citep{kwonClassificationRational3tangles2025}, the 3 strings
rational tangles, have
been classified. For the remaining cases, the full tangle theory must be
developed to solve the problem, this requires experience with the development of
original abstract theory, making the problem appropriate for upper division
students

%  prettier-ignore-start

\subsubsection{Potpourri}\label{sec-proj-pot}

%  prettier-ignore-end

Problems in this section are those that do not fit into other classes of
problems.

%  prettier-ignore-start

\paragraph{Random Tangle Sampling: Upper Division Student}\label{sec-proj-rand}

%  prettier-ignore-end

\subparagraph{Problem Statement}

Create the theory and software to select from a collection or generate a tangle
at random with an understood distribution.

\subparagraph{Brief}

In the introduction chapter Section~\ref{sec-intro-applications},
applications of knots to the
hard sciences were discussed. When working in the hard sciences, being able to
sample with an understood distribution from a collection is important. Similarly
to the previous notational projects, describing and implementing random sampling
methodologies is a class of extremely useful tabulation projects. The full
tangle theory must be developed to solve this problem. This requires experience
with the development of original abstract theory, making the problem appropriate
for upper division students
\newpage
\paragraph{Develop a tangle analogue for petal knots: Upper Division Student}

\subparagraph{Problem Statement}

Create the theory needed for a well defined tangle analogue of the petal
knots\citep{adamsKnotProjectionsSingle2015}.

\subparagraph{Brief}

Petal knots, first developed by Adams\citep{adamsKnotProjectionsSingle2015}, are
knots in which all crossings are colinear in the orthogonal projection, an
``ubercrossing''. Converting these objects to a braid is
straightforward, however
less obvious is converting to a two string tangle. Identifying a tangle analogue
for the petal knots may allow for computation of a whole new family of tangle
data. The full tangle theory must be developed to solve this problem This
requires experience with the development of original abstract theory, making the
problem appropriate for upper division students
