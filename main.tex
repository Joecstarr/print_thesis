\documentclass[10pt,twoside,openright]{uithesis}
\usepackage{graphicx}
\usepackage{geometry}
\setstocksize{9.25in}{6.125in}
\geometry{papersize={6.125in,9.25in},inner=.625in,outer=0.625in,top=0.625in,bottom=0.875in,bindingoffset=0.4in}
\usepackage{silence}
%Disable all warnings issued by latex starting with "You have..."
\WarningFilter{latex}{You have requested package}
\WarningFilter{pdf backend}{(pdf backend): ignore duplicate}
\WarningFilter{latex}{Hyper reference}
\WarningFilter{latex}{Reference `ex:}

\bibliography{main.bib}

\usepackage{listings}
\usepackage{xcolor}

\colorlet{punct}{red!60!black}
\definecolor{background}{HTML}{EEEEEE}
\definecolor{delim}{RGB}{20,105,176}
\colorlet{numb}{magenta!60!black}

\lstdefinelanguage{json}{
  basicstyle=\normalfont\ttfamily,
  numbers=left,
  numberstyle=\scriptsize,
  stepnumber=1,
  numbersep=8pt,
  showstringspaces=false,
  breaklines=true,
  frame=lines,
  backgroundcolor=\color{background},
  literate=
  *{0}{{{\color{numb}0}}}{1}
  {1}{{{\color{numb}1}}}{1}
  {2}{{{\color{numb}2}}}{1}
  {3}{{{\color{numb}3}}}{1}
  {4}{{{\color{numb}4}}}{1}
  {5}{{{\color{numb}5}}}{1}
  {6}{{{\color{numb}6}}}{1}
  {7}{{{\color{numb}7}}}{1}
  {8}{{{\color{numb}8}}}{1}
  {9}{{{\color{numb}9}}}{1}
  {:}{{{\color{punct}{:}}}}{1}
  {,}{{{\color{punct}{,}}}}{1}
  {\{}{{{\color{delim}{\{}}}}{1}
  {\}}{{{\color{delim}{\}}}}}{1}
  {[}{{{\color{delim}{[}}}}{1}
  {]}{{{\color{delim}{]}}}}{1},
}
\usepackage[inkscapeformat=png]{svg}
\usepackage{lmodern}
\usepackage[T1]{fontenc}    % use 8-bit T1 fonts
\usepackage{url}            % simple URL typesetting
\usepackage{booktabs}       % professional-quality tables
\usepackage{microtype}      % microtypography
\usepackage{doi}
\usepackage[twitter]{coloremoji}
\usepackage{subcaption}
\usepackage{float}
\usepackage{listings}
\usepackage{amssymb}
\usepackage{amsfonts}
\usepackage{mathrsfs}
\usepackage{times}
\usepackage{mathtools}
\usepackage{bbm}
\usepackage{fancyhdr}
\usepackage{tikz-cd}
\usepackage{tikz}

%%%%%%%%%%%%%%%%%%%%%%%%%%%%%%%%%%%%%%%%%%%%%%%%%%
%%%%%%%%%%%%%%%%%%%%  imports  %%%%%%%%%%%%%%%%%%%
\usepackage{amsmath}
\usepackage{amsthm}
\usepackage{framed}
\usepackage{subcaption}
%%%%%%%%%%%%%%%%%  math commands  %%%%%%%%%%%%%%%%
\newcommand{\R}{\mathbb{R}}
\newcommand{\m}{\scalebox{0.5}[1.0]{\(\ -\ \)\!}}
\newcommand{\LP}{\left(}
\newcommand{\RP}{\right)}
\newcommand{\LA}{\left\langle}
\newcommand{\RA}{\right\rangle}
\newcommand{\img}[1]{\includegraphics[height=2ex, width=2ex]{#1}}
\newcommand{\bkt}[1]{\LA \img{ #1}\RA}
\newcommand{\LB}{\left[}
\newcommand{\RB}{\right]}
\newcommand{\LS}{\left\lbrace}
\newcommand{\RS}{\right\rbrace}
\newcommand{\Z}{\mathbb{Z}}
\newcommand{\abs}[1]{\left\vert#1\right\vert}
\newcommand{\LN}{\left.}
%%%%%%%%%%%%%%%%%%%%%%%%%%%%%%%%%%%%%%%%%%%%%%%%%%
%%%%%%%%%%%%%%%%%%%%%%%%%%%%%%%%%%%%%%%%%%%%%%%%%%
%%%%%%%%%%%%%%%%%%%%  theorem  %%%%%%%%%%%%%%%%%%%
% \newtheorem{theorem}{Theorem}[section]
% \newtheorem{corollary}{Corollary}[theorem]
% \newtheorem{lemma}[theorem]{Lemma}
% \newtheorem{proposition}{Proposition}[section]
% \newtheorem{definition}{Definition}[section]
% \newtheorem{example}{Example}[section]
% \newtheorem{remark}{Algorithm}[section]
% \newtheorem{axiom}{Axiom}[section]
% \newtheorem{conjecture}{Conjecture}[section]
% \newtheorem{observation}{Figure}[section]
%%%%%%%%%%%%%%%%%%%%%%%%%%%%%%%%%%%%%%%%%%%%%%%%%%

\newcommand{\nf}[1]{\nameref{#1} (page~\pageref{#1})}

\setsecnumdepth{subsection}
\date{December 2025}
% \year{2025}
\title{The Tanglenomicon: Tabulation of Arborescent Tangles}

\author{Joseph C. Starr}

\definecolor{clrsnavy}{HTML}{001F3F}
\definecolor{clrsblue}{HTML}{0074D9}
\definecolor{clrsaqua}{HTML}{7FDBFF}
\definecolor{clrsteal}{HTML}{39CCCC}
\definecolor{clrsolive}{HTML}{3D9970}
\definecolor{clrsgreen}{HTML}{2ECC40}
\definecolor{clrslime}{HTML}{01FF70}
\definecolor{clrsyellow}{HTML}{FFDC00}
\definecolor{clrsorange}{HTML}{FF851B}
\definecolor{clrsred}{HTML}{FF4136}
\definecolor{clrsfuchsia}{HTML}{F012BE}
\definecolor{clrspurple}{HTML}{B10DC9}
\definecolor{clrsmaroon}{HTML}{85144B}
\definecolor{clrswhite}{HTML}{FFFFFF}
\definecolor{clrssilver}{HTML}{DDDDDD}
\definecolor{clrsgray}{HTML}{AAAAAA}
\definecolor{clrsdarkgray}{HTML}{616161}
\definecolor{clrsblack}{HTML}{111111}

\usepackage[many]{tcolorbox}
\tcbset{
  base/.style={
    enhanced,
    breakable,  % Allows page break.
    sharp corners,
    fonttitle=\bfseries,
    colbacktitle=#1!5!white,
    coltitle=#1,
    toptitle=0mm,
    bottomtitle=0mm,
    boxrule=0mm,
    left=2.5mm,
    leftrule=1mm,
    right=3.5mm,
    boxrule=0pt,frame hidden,
    colback=#1!5,
    colframe=#1!80,
    borderline west={2pt}{0pt}{#1},
    label separator=
  }
}

\newtcbtheorem[auto counter,number within=chapter]{theorem}{Theorem}{
  base=clrsnavy
}{}

\newtcbtheorem[auto counter,number within=chapter]{lemma}{Lemma}{
  base=clrsnavy
}{}

\newtcbtheorem[auto counter,number within=chapter]{corollary}{Corollary}{
  base=clrsnavy
}{}

\newtcbtheorem[auto counter,number within=chapter]{proposition}{Proposition}{
  base=clrsnavy
}{}

\newtcbtheorem[auto counter,number within=chapter]{definition}{Definition}{
  base=clrsolive
}{}

\newtcbtheorem[auto counter,number within=chapter]{remark}{Algorithm}{
  base=clrspurple
}{}

\newtcbtheorem[auto counter,number within=chapter]{requirement}{Requirement}{
  base=clrsfuchsia
}{}

\newtcbtheorem[auto counter,number within=chapter]{usecase}{Use Case}{
  base=clrsfuchsia
}{}

\newtcbtheorem[auto counter,number within=chapter]{testcard}{Test}{
  base=clrsred
}{}

\newtcbtheorem[auto counter,number within=chapter]{example}{Example}{
  base=clrsblack
}{}

\newtcolorbox{note}[1][]{
  title={Note:},
  base=clrsdarkgray
}

\newtcolorbox{convention}[1][]{
  title={Convention:},
  base=clrsorange
}
\usepackage[normalem]{ulem}
%%% --- The following two lines are what needs to be added --- %%%
\setcounter{biburllcpenalty}{7000}
\setcounter{biburlucpenalty}{8000}

\usepackage{hyperref}
\hypersetup{
  colorlinks,
  linkcolor=[rgb]{0.62,0.22,0.24},
  urlcolor=[rgb]{0.16,0.20,0.36},
  citecolor=[rgb]{0.95,0.65,0.07},
  pdftitle={\thetitle},
  pdfauthor={\theauthor},
  breaklinks=true
}

\degree{Doctor of Philosophy}
\concentration{Mathematics}

\ThesisSupervisor{Isabel Darcy}

\CommitteeMember{Francis Bonahon}
\CommitteeMember{Keiko Kawamuro }
\CommitteeMember{Colleen Mitchell}
\CommitteeMember{Radmila Sazdanovi{\'c}}

\setcounter{tocdepth}{2}

\begin{document}

\titlepage

\copyrightPage

\frontmatter

\begin{acknowledgments}
  Completing this PhD has been the hardest thing I've ever done. The
  last four years were filled with stress, doubt, and exhaustion. I
  would have never been able to tackle this alone. I feel grateful to
  have had a team around me and supported me through each milestone.

  I want to express my deepest thanks to my best friend and wife
  Valeria, who has been with me at every step. She sacrificed so much
  to make this happen. She moved across the country away from family,
  worked as our primary bread winner, ate way too many frozen dinners
  when I just couldn't do more that day, missed out on four years of
  vacations, and just felt every ounce of my stress and exhaustion.
  Valeria, te amo, I'm happy to have you as my partner, you're my
  rock. I also need to acknowledge what kept Valeria and I from
  losing it when the stress maxed out, Topper the pug, the best dog
  in the world. Thank you for sitting with (on) me for hours on end
  while I studied for qualifying exams, prepared for my comprehensive
  exam, prepared for my defense, and wrote this thesis.

  To my family, you've all been there for me. Mom and Dad, I'm
  grateful that you have always pushed me to do hard things and
  supported me through all my successes and failures. Mom, thank you
  for walking this path before me, I truly don't understand how you
  had two children while completing a PhD. I'm grateful for all the
  conversations and advice you've given me and for always reassuring
  me that it's normal and will be okay. Dad, thank you for everything
  you've done for us. Thank you for helping us move, for driving to
  Iowa to be here while I worked through exam weeks, for helping us
  with the garden, and for so many other things. To my siblings,
  thank you for being fun and keeping my spirits up. To Aunt Shirley,
  thank you for being in my corner. To my in-laws Gume and Marisol,
  thank you for all you've done for Valeria, I couldn't ask for
  better people to be my in-laws.

  School hasn't always been easy for me; as a child I experienced an
  environment where faculty were cruel. Moving beyond that experience
  has been hard, but I've been lucky to have had a wonderful
  collection of mentors and friends to help me. My high school
  physics teacher, Dr. Deano Smith, thank you for believing in me. My
  high school biology teachers, Martha and Tom Friedlander, thank you
  for always encouraging me. My math professor, Dr. Alan Wiggins,
  thank you for putting The Knot Book
  \citep{adamsKnotBookElementary2004} in my hands and showing me that
  I can do math. My computer science professor, Dr. Bruce Elenbogen,
  thank you for challenging me; you are missed.

  I would like to also thank the community I've built and been
  accepted into at Iowa and in the broader math community. To my
  advisor Isabel Darcy, thank you for all you've done for me in the
  last few years. Thank you for bringing me into your community and
  being one of my biggest cheerleaders and advocates. I look forward
  to our continued collaboration and hope we build great things. To
  my committee members, Francis Bonahon, Keiko Kawamuro, Colleen
  Mitchell, and Radmila Sazdanovic thank you for serving on my
  committee and for all the time you've given me as I've matured as a
  mathematician. To the Iowa Topology group, without the support of
  the RTG (DMS-2038103), I would never have had the opportunity to
  meet Dr. Bonahon and Dr. Sazdanovic. To my academic siblings and
  extended applied topology friends; Ethan, George, Jacob, Paria, and
  Robert, thanks for all the brainstorming and good conversation. To
  Nick Connolly, ``The Tanglenomicon''
  \citep{connollyClassificationTabulation2string2021} is an amazing name for the
  series of Iowa applied topology tanlge tabulation projects, I'm excited to add
  my subtile to this collection. To the
  ISA, particularly Lori Adams, thank you for helping me develop
  as an instructor and giving me a venue to express myself. The Iowa
  community is full of wonderful people, unfortunately I can't thank
  everyone individually, so to those who I havn't listed I simply thank
  you for being you.

  Finally, It may seem strange, but I would like to also acknowledge
  my 2019 fusion. Over my four years in Iowa, my car safely carried
  me as I drove around 1000 hours of commute.

\end{acknowledgments}

\begin{abstract}
  The enumeration of provably unique mathematical knots, tabulation, has been a
  core area of research for over 150 years. One important advancement
  in this area
  was Conway's development of a building block for knots, the tangles. Formally,
  an $n$-string tangle is a portion of a knot diagram enclosed in a Jordan curve
  that intersects the knot in exactly $2n$ points. The tabulation of the two
  string tangles, Conway's building blocks for knots, has been called:
  \begin{quote}
    ``The most important missing infrastructure project in knot
    theory'' -Dr. Dror
    Bar-Natan\citep{bar-natanMostImportantMissing}
  \end{quote}
  \vspace{-2mm}
  Without such a table of tangles, researchers are in the
  position of a chemist who possesses a table of fatty acids but no periodic
  table. In this thesis we answer, in part, this call.

  This thesis develops the theory needed to
  tabulate tables of successively more complex classes of tangles. We start with
  the rational and Montesinos tangles, and then conclude with the arborescent
  tangles, often called the algebraic tangles. For each of
  these classes of
  tangles, we additionally develop a collection of software designs used to
  efficiently and scalably compute tables of these tangles to high crossing
  number (19). We will also
  discuss ways that the accessibility of knot theory and in
  particular tabulation make the domain a candidate for undergraduate
  research. As
  part of our discussion of undergraduate research, we will outline a software
  engineering process particularly suited for undergraduate research in knot
  theory, as well as a model life cycle for an undergraduate research experience
  in computational knot theory.

  \vspace{1cm}
  \noindent
  This material is based upon work supported by the National Science
  Foundation under
  Award~No.~(DMS\nobreakdash2038103).
  Any opinions, findings, and conclusions or recommendations
  expressed in this material
  are those of the author and do not necessarily reflect the views
  of the National Science
  Foundation.
\end{abstract}

\begin{publicAbstract}
  The tabulation, creation of a list, of mathematical
  knots has been a core area of mathematics research for over 150 years. A key
  breakthrough in knot tabulation was Conway's development of a
  building block for
  knots, the two string tangle. A two string tangle is a portion of a
  mathematical knot bound within a ball and intersecting that
  ball in exactly
  four points. A table of these objects has been called:
  \begin{quote}
    ``The most important missing infrastructure project in knot
    theory'' -Dr. Dror
    Bar-Natan\citep{bar-natanMostImportantMissing}
  \end{quote}
  \vspace{-2mm}
  Without such a table of tangles, mathematicians and scientists are in the
  position of a chemist who possesses a table of fatty acids but no periodic
  table. In this thesis we answer, in part, this call.

  As we progress through this thesis, we produce
  successively larger and more complete tables of tangles. Each table we produce
  requires the development of the mathematical theory demonstrating that what we
  have produced is what we say it is. Then, for each tangle type we develop the
  theory for, we also develop a collection of software designs used
  to convince a
  computer to quickly generate large tables of tangles. We will also
  discuss ways that the accessibility of knot theory and in
  particular tabulation make this domain a candidate for undergraduate
  research. As
  part of our discussion of undergraduate research, we will outline a software
  engineering process particularly suited for undergraduate research in knot
  theory, as well as a model life cycle for an undergraduate research experience
  in computational knot theory.

\end{publicAbstract}

\tableofcontents

\listoffigures

\mainmatter

\chapter{Introduction}

There exists a common view of advanced mathematics as esoteric nonsense only
understandable by elite thinkers. While that sentiment is perhaps deserved by
some mathematical disciplines, knot theory is an area of advanced mathematics
that offers deep significance as well as general accessibility. A lay person can
be easily taught the basic structures of knot theory, while it takes years to
master the details.

%  prettier-ignore-start

\section{An Intuitive Grounding in Knot
Theory}\label{sec-intro-intuit_knot_theory}

%  prettier-ignore-end

A knot, when used in everyday life, is a tool whether it's the
``bunny ear'' knot
holding on your shoe, a decorative monkey's fist on your keychain, or a climbing
hitch securing yourself to a rock wall. Now, when thinking of these tool knots,
we should note one critical attribute, they're made of a single string with open
ends (Figure~\ref{fig-intro-everyday_knot}). Notice with this
construction, no matter how
``knotted'' the string, we can always pull on loops to remove the
knot, leaving us
with only an unknotted string.

\begin{figure}[H]
  \centering
  \includegraphics[width=0.8\linewidth]{files/everyday_knot-3d0a3f087e460add29008825581bbd36.pdf}
  \caption[An everyday knot with open ends.]{An everyday knot with open ends.}
  \label{fig-intro-everyday_knot}
\end{figure}

This leaves us with a somewhat unsatisfying construction, exactly one object, a
string. How might we add some spice to our construction? What if we turn the
string into a circle by gluing the ends as in
Figure~\ref{fig-intro-everyday_knot_closed}?
When we try to wiggle the closed up string around, trying to remove the knot, we
stretch and pull on the string, but regardless of what we do, we end up with a
knot. Constructing a physical model of
Figure~\ref{fig-intro-everyday_knot_closed} and
playing with it should convince you that the knot can't be removed. We call
these closed up versions of knots \textbf{mathematical knots}.

\begin{figure}[H]
  \centering
  \includegraphics[width=0.8\linewidth]{files/everyday_knot_closed-dc24f338fa5dd2633dcba3c6e31156ee.pdf}
  \caption[Closing the ends of the everyday knot.]{Closing the ends
    of the everyday knot in Figure~\ref{fig-intro-everyday_knot} to form a
  mathematical knot.}
  \label{fig-intro-everyday_knot_closed}
\end{figure}

With this, when compared to an everyday knot, we are already in a better place.
You may discover that then gluing the ends yields knots that can't be maneuvered
to look the same as Figure~\ref{fig-intro-everyday_knot_closed} or
Figure~\ref{fig-intro-circle}. Continuing to experiment with your physical
model, twisting the ends of the string around each other in different ways
(Figure~\ref{fig-intro-torus}).
\begin{figure}[H]
  \centering
  \includegraphics[width=0.3\linewidth]{files/circle-5a8ae67a7c8a7a4fc197ac67fe7ba70a.pdf}
  \caption[A representation of a simple loop.]{A representation of a
  simple loop made from gluing the ends of a rope together.}
  \label{fig-intro-circle}
\end{figure}

In fact, this twisting gives us a whole infinite family of odd crossing torus
knots, a sampling of which can be seen in Figure~\ref{fig-intro-torus_knots}.

\begin{figure}[H]
  \centering
  \includegraphics[width=0.5\textwidth]{files/torus_physical-fa039c19edb7484321bde650f450164b.pdf}
  \caption[A representation of twisting the ends of a rope.]{A
  representation of twisting the ends of a rope around itself before gluing.}
  \label{fig-intro-torus}
\end{figure}

\begin{figure}[H]
  \includegraphics[width=\textwidth]{files/torus-689573cf371cfec2180887a2654c1f43.pdf}
  \caption[Three knots built from twisting.]{Three knots built from
    the operation seen in Figure~\ref{fig-intro-torus}. From left
  to right: three, five, and seven crossing torus knots.}
  \label{fig-intro-torus_knots}
\end{figure}

Our experimentation with the physical model may conjure some questions, three
important questions we may ask are:

\begin{enumerate}
  \item ``How do I systematically construct knots?''
  \item ``How do I tell two knots I make apart?''
  \item ``How do I generate new knots?''
\end{enumerate}

Attempting to answer these questions, even just in part or with restrictions, is
the bread and butter of knot theory and the focus of the rest of this thesis.
Some great texts for continued reading on knot theory in order of accessibility
are: The Knot Book: An Elementary Introduction to the Mathematical Theory of
Knots by Adams \citep{adamsKnotBookElementary2004}, LinKnot: Knot
Theory by Computer
by Jablan and Sazdanovi{\'c} \citep{jablanLinKnotKnotTheory2007},
Knots and Links by
Rolfsen \citep{rolfsenKnotsLinks2003}, and An Introduction to Knot Theory by
Lickerish \citep{lickorishIntroductionKnotTheory1997}.

The remainder of this thesis quickly exchanges the idea of a knot for that of a
tangle, introduced by Conway \citep{conwayEnumerationKnotsLinks1970}.
A tangle can be
thought of as slamming a cookie cutter onto a knot diagram pinning down four
points. Once the knot is pinned down we then cut off the parts of the knot
laying outside the cookie cutter, seen in Figure~\ref{fig-intro-tangle_maker}.

\begin{figure}[H]
  \centering
  \includegraphics[width=\linewidth]{files/tangle_maker-d959b9e7c360bb1bd923d6177c09c044.pdf}
  \caption[The creation of a tangle from a knot.]{The creation of a
    tangle from a knot by cutting a section out of the knot but
  fixing four points.}
  \label{fig-intro-tangle_maker}
\end{figure}

%  prettier-ignore-start

\section{Brief Discussion on Applications}\label{sec-intro-applications}

%  prettier-ignore-end

As we saw in the Section~\ref{sec-intro-intuit_knot_theory},
mathematical knots can be easily constructed as physical objects. It should be
no surprise then that mathematical knots and tangles appear in the hard
sciences, particularly in the realms of physics, chemistry, and biology. In this
section, we will discuss one of the most commonly discussed applications of knot
theory.

\subsection{Tangles in DNA}

One of the fundamental features that identifies life as life is the ability to
self-replicate. In order to self-replicate, life must have a mechanism to pass
information to successive generations. Consider first the most basic self
replicating component of life, called a cell. The information of a cell is
stored as double-stranded DNA (dsDNA), as described by Crick, Franklin, Gosling,
and Watson
\citep{watsonMolecularStructureNucleic1953,
franklinMolecularConfigurationSodium1953},
a polymer consisting of two strands constructed from a sugar-phosphate connected
to one of the monomers \citep{watsonMolecularStructureNucleic1953}:

\begin{enumerate}
  \item Adenine (A)
  \item Guanine (G)
  \item Cytosine (C)
  \item Thymine (T)
\end{enumerate}

\begin{figure}[H]
  \centering
  \includegraphics[width=0.4\linewidth]{files/DNA_chemical_structu-d5653e324ba48edb00fc63567a2cf1c0.pdf}
  \caption[A schematic diagram demonstrating the structure of a dsDNA
  polymer.]{A schematic diagram demonstrating the structure of a
    dsDNA polymer (Ball CC BY-SA 2.5 via Wikimedia
  Commons\citep{madeleinepriceballDNAChemicalStructure2007})}
  \label{fig-intro-dna_chemical_structure}
\end{figure}

The two strands of dsDNA connect to each other to form the
``double-helix'' where
the monomers bind to each other with guanine (G) binding to cytosine (C), and
adenine (A) binding to thymine (T) \citep{watsonMolecularStructureNucleic1953}
(Figure~\ref{fig-intro-dna_chemical_structure}). When replicating,
the cell will duplicate
the dsDNA by splitting the dsDNA into two single strands with the enzyme DNA
helicase. Once the DNA is split, two new complementary single strands are
constructed to be paired with each of the original single strands.

\begin{figure}[H]
  \centering
  \includegraphics[width=\linewidth]{files/DNA_replication_en-7ba25b2f51c348cb7d031e6a172d6abd.pdf}
  \caption[A schematic diagram demonstrating the splitting of a
    double strand of DNA into
  two new double strands.]{A schematic diagram demonstrating the
    splitting of a double strand of DNA into
    two new double strands. (Ruiz Public domain via Wikimedia
  Commons\citep{ruizDNAReplicationFork})}
  \label{fig-intro-dna_split}
\end{figure}

The dsDNA of a cell needs to be physically stored inside the cell. Cell volume
is limited, so organizing the dsDNA to fit in that volume requires several
complex cellular mechanisms. One issue solved by these mechanisms is that of
local knotting, which becomes a problem when the cell attempts to replicate.
During the replication process, the DNA helicase begins splitting the dsDNA, but
when the DNA helicase reaches the locally knotted portion, it becomes stuck, the
replication will be unable to continue
\citep{albertsMolecularBiologyCell2022}, and
the cell will die. If this local knotting is allowed to happen unchecked, every
cell would eventually be unable to replicate and would ultimately die. Famously,
life finds a way, and one cellular mechanism that mitigates this local knotting
problem is the enzyme type II
topoisomerase\citep{albertsMolecularBiologyCell2022}.

The enzyme attempts to solve this local knotting by cutting one of
the dsDNA where
two segments cross, then moving the top double strand to the bottom, this action
can be seen in Figure~\ref{fig-intro-topo}.

\begin{figure}[H]
  \begin{subfigure}[t]{.35\textwidth}
    \centering
    \includegraphics[width=\textwidth]{files/Gyrase_structure_Dmi-1d125d277339d7c31499be4e7766a1ef.png}
    \caption[A schematic diagram of the enzyme type II
    topoisomerase.]{A schematic diagram of the enzyme type II
      topoisomerase. (Sutor CC BY-SA 4.0 via Wikimedia
    Commons\citep{sutorSchemeDNAGyrase})}
    \label{fig-intro-topo2schem}
  \end{subfigure}
  \hfill
  \centering
  \begin{subfigure}[t]{.6\textwidth}
    \centering
    \includegraphics[width=\textwidth]{files/topo3-31473ad625ab67bf8b08a2069a9bf0fb.pdf}
    \caption[Type II topoisomerase doing a crossing exchange.]{Type
      II topoisomerase doing a crossing exchange. From top left to bottom
  right: 1) A crossing of two dsDNA segments. 2) The enzyme grabs the
  under segment.
3) The enzyme splits the under segment. 4) The enzyme passes the over segment
through the gap. 5) The crossing with the segments exchanged.}
\label{fig-intro-topo}
\end{subfigure}
\caption[Schematic diagrams of the enzyme type II
topoisomerase.]{Schematic diagrams of the enzyme type II topoisomerase.}
\end{figure}

In mammals (and many other animal groups), dsDNA takes the form of long strings,
which can only become everyday knots. However, it was discovered by Dulbecco and
Vogt\citep{dulbeccoEVIDENCERINGSTRUCTURE1963,
weilCYCLICHELIXCYCLIC1963, vinogradTwistedCircularForm1965}
that in some viruses (Polyoma) the dsDNA is a closed loop, allowing it to form
into a mathematical knot (Figure~\ref{fig-intro-circ_dna_fig}) .

\begin{figure}
\begin{subfigure}[t]{.45\textwidth}
\includegraphics[width=\textwidth]{files/dna_circle-64775fa5ef9cccacd98295c0fecc35ba.pdf}
\caption[A schematic diagram of circular dsDNA.]{A schematic diagram
of circular dsDNA.}
\label{fig-intro-dna_circle}
\end{subfigure}
~
\begin{subfigure}[t]{.45\textwidth}
\centering
\includegraphics[width=\textwidth]{files/sem_knot-dee1106bb3b0e30c06d0f9d136d6f76f.png}
\caption[A scanning electron microscope image of knotted dsDNA.]{A
scanning electron microscope image of knotted dsDNA (Arsuaga CC
BY-ND\citep{arsuagaDNAKnotSeen2013}).}
\label{fig-intro-sem_knot-png}
\end{subfigure}
\caption[Schematic diagrams of circular and knotted dsDNA.]{Schematic
diagrams of circular and knotted dsDNA.}
\label{fig-intro-circ_dna_fig}
\end{figure}

From here one may ask, ``If the dsDNA is knotted and type II topoisomerase makes
a change, what kind of new knot can this make?'', this question was addressed
first by Ernst and Sumners in the 1990s \citep{ernstCalculusRationalTangles1990}
\citep{ernstTANGLEEQUATIONS1996}. Their approach considers the dsDNA
to be bounded by
two ``areas'', the first area is created by drawing a circle around the crossing
that type II topoisomerase is working on (right side of
Figure~\ref{fig-intro-tangle_equation_start}), and the second by
drawing a circle around the
remainder (left side of
Figure~\ref{fig-intro-tangle_equation_start}). From here, the
crossing change from type II topoisomerase can be modeled by changing the tangle
bound in the area on the right (Figure~\ref{fig-intro-tangle_calc}),
\citep{darcy3DVisualizationSoftware2008}.

\begin{note}
After the change, there may be many crossings that type II
topoisomerase could ``choose'' to work on next. A program
like TopoICE-X \citep{darcy3DVisualizationSoftware2008} (built into KnotPlot
\citep{schareinInteractiveTopologicalDrawing1998}) can help visualize
the results of
making these choices.
\end{note}

\begin{figure}
\begin{subfigure}[t]{.45\textwidth}
\centering
\includegraphics[width=\textwidth]{files/tangle_equation_star-893577b81dd84ec6fe27e7f4781c5418.pdf}
\caption[A knot diagram showing two areas containing knot data.]{A
knot diagram showing two areas containing knot data. The right side contains
the crossing that type II topoisomerase will work on.}
\label{fig-intro-tangle_equation_start}
\end{subfigure}
~
\begin{subfigure}[t]{.45\textwidth}
\centering
\includegraphics[width=\textwidth]{files/tangle_equation_resu-e4d9cd036ada95414b3b525d38b63b2d.pdf}
\caption[A knot diagram showing two areas containing knot data.]{A
knot diagram showing two areas containing knot data. The right side contains
the crossing that type II topoisomerase has worked on.}
\label{fig-intro-tangle_equation_result}
\end{subfigure}
\caption[A tangle model for a crossing change in a knot.]{A tangle
model for a crossing change in a knot.}
\label{fig-intro-tangle_calc}

\end{figure}
The modeling of the action of type II topoisomerase is just one of the many
applications for the theory of knots and tangles found in biology. For further
reading on applications, a good source is the ``Encyclopedia of knot
theory''\citep{adamsEncyclopediaKnotTheory2021}, a survey of many
subdisciplines of
knot theory with a chapter devoted to applications.

%  prettier-ignore-start

\section{Overview of this Thesis}\label{sec-intro-overview}

%  prettier-ignore-end

As discussed in the previous section, the goal of this thesis is to address, in
part, the three questions ``How do I systematically construct
knots?'', ``How do I
tell two knots I make apart?'', and ``How do I generate new knots?''
Particularly,
we will address a restricted version of these questions, for objects that can be
thought of as the building blocks of knots, the tangles introduced by Conway
\citep{conwayEnumerationKnotsLinks1970}. Through the
middle(Chapter~\ref{ch-tabulation} and Chapter~\ref{ch-software}) of
this thesis, we will
describe several strategies that have been employed to answer even further
restricted versions of these questions:

\begin{itemize}
\item ``How do I systematically construct rational tangles?'', ``How
do I tell two
rational tangles I make apart?'', and ``How do I generate new rational
tangles?''
\item ``How do I systematically construct Montesinos tangles?'' and
``How do I tell
two Montesinos tangles I make apart?'', and ``How do I generate new Montesinos
tangles?''
\item ``How do I systematically construct algebraic/arborescent
tangles?'' and ``How
do I tell two algebraic/arborescent tangles I make apart?'', and ``How do I
generate new algebraic/arborescent tangles?''
\end{itemize}

While we could answer these questions about tangles with pen and
paper brute force, as Tait, Little, Kirkman, Conway, and Caudron
\cite{taitTenfoldKnottiness1885,
kirkmanEnumerationDescriptionConstruction1885,
littleKnotsCensusOrder1885,conwayEnumerationKnotsLinks1970,caudron1982classification}
did for knots.  Beyond a
reasonable ``crossing number'' (Section~\ref{subsec-knot_def}), as
small as 8, the time
and effort needed makes pen and paper untenable. To achieve our goal in a
reasonable time, we will follow a similar framework to that of Hoste,
Thistlethwaite, Weeks, and Burton \citep{hosteFirst1701936Knots1998,
hosteEnumerationClassificationKnots2005,burtonNext350Million2020}.
Utilizing computer methods to generate a tangle table of algebraic/arborescent
tangles to twelve crossings. To effectively utilize computers for this
tabulation work requires the design and implementation of a knot theory specific
software toolbox.

\subsection{Chapter Summary}

We will now summarize each of the chapters of this thesis. Organizationally,
this thesis is partitioned into five chapters, with each chapter further divided
into sections, and subsections.

\subsubsection{Chapter 1: Introduction}

This introduction gave an intuitive description of the basics of knot theory,
and a discussion of an application for knots and tangles. We now finish with
this description of the content in the remainder of this thesis.

\subsubsection{Chapter 2: Preliminaries}

This chapter gives the preliminaries in knot and tangle theory needed
for the rest of the
thesis. Included are the historical background of tabulation in knot theory and
a grounding in knot and tangle theory. We will see definitions for knots and
tangles, as well as some notations and invariants for those knots and tangles.

\subsubsection{Chapter 3: Tabulation}

This chapter describes the theoretical methodology for the tabulation of
algebraic tangles. It is divided into two sections, the first
contains the method used for
tabulation of two well-understood classes of tangles: the rational and
Montesinos tangles. The second describes the methodology for the tabulation of
the more general type of tangle the arborescent (algebraic) tangles. For each
class of tangle we tabulated, a definition and classification is given, followed
by a theoretical generation strategy.

\subsubsection{Chapter 4: Software and its Engineering}

This chapter addresses the computational and engineering aspects of software for
mathematics research. The chapter begins with an overview of product management
and software engineering practices. In our discussion of software engineering we
develop a process designed for use by professional and undergraduate researchers
in computational knot theory. With our process, we create a product design for a
general purpose ``knot theory software toolbox''. The chapter concludes with the
software design (unit description) for the tools developed to realize the
solutions in the tabulation section.

\subsubsection{Chapter 5: Future Directions}

The final chapter of this thesis gives an overview of future work to be done in
the tangle tabulation domain. This takes two forms, first the direct next steps
for tangle tabulation at the professional research level. Second, we outline a
collection of topics and research directions at various levels appropriate for
undergraduate researchers.


\chapter{Preliminaries}

%  prettier-ignore-start

\section{History of Knot Tabulation}\label{sec-history-of-tabulation}

%  prettier-ignore-end

Mathematical interest in knots was brought to the forefront of mathematics and
physics during the 1860s by Lord Kelvin. A central area of research in physics
during the 1860s was the investigation of the fundamental building blocks of
matter: the atoms. Lord Kelvin hypothesized that atoms were knotted vortices in
the aether \citep{thomsonVortexAtoms1867}. With this hypothesis, a
natural next step
is the creation of a table of the elements which, by hypothesis, was a table of
knots.

\subsection{Knot Tabulation by Hand}

With Lord Kelvin's vortex hypothesis as the driving force for knot tabulation,
the first knot table was produced, via hand computation, by P.G. Tait. Tait's
first table, completed in the 1860s, contained prime knots (described in
\nf{subsec-prime_knot}) up to seven crossings
\citep{taitFirstSevenOrders1884} (described in \nf{subsec-knot_def}), a table
of the first seven knots can be seen in Figure~\ref{fig-hist-7table}

\begin{figure}[H]
  \centering
  \includegraphics[height=0.4\textheight]{files/knots_to_7-f13b9b5556c173ab3d955b74f82b2e53.pdf}
  \caption[A table of the first seven prime knots.]{A table of the
  first seven prime knots. \citep{schareinInteractiveTopologicalDrawing1998}}
  \label{fig-hist-7table}
\end{figure}

With a table of seven crossing knots complete, Tait's work in knot tabulation
continued, alongside Kirkman and Little
\citep{taitTenfoldKnottiness1885,
kirkmanEnumerationDescriptionConstruction1885, littleKnotsCensusOrder1885},
for the next 25 years. The trio ultimately constructed a complete list of prime
knots up to ten crossings (250 knots + 1 repeat). When the knot
theoretic machinery
available at the time is considered, the completion of these tables with such
high accuracy was a Herculean task. The tables contained a single error, two
equivalent (described in \nf{subsec-knot_equivalence}) ten crossing knots
(Figure~\ref{fig-hist-perko}), identified in 1974 by Perko, an
amateur mathematician
\citep{perkoClassificationKnots1974}.

\begin{figure}[H]
  \centering
  \includegraphics[width=0.5\linewidth]{files/perko_pair-f143cdf72c123a1835bc5e48f71d8245.pdf}
  \caption[The Perko pair.]{The Perko pair, a pair of equivalent ten
    crossing knots appearing as an
    error in early knot tables
    \citep{schareinInteractiveTopologicalDrawing1998,
  perkoClassificationKnots1974}.}
  \label{fig-hist-perko}
\end{figure}

After the completion of the ten crossing tables, efforts in knot tabulation
stagnated, with few concerted efforts and little progress being made in
expanding the tables. The next researcher to take up the tabulation torch was
Conway in the 1960s \citep{conwayEnumerationKnotsLinks1970}. Conway,
in ``a few hours''
\citep{conwayEnumerationKnotsLinks1970}, tabulated knots to eleven
crossings, with
only four omissions\citep{caudron1982classification}. Conway's work
continued by hand
computation, but employed a novel approach to tabulation. He described
decompositions of knots into building blocks, which he called
tangles\citep{conwayEnumerationKnotsLinks1970}. Conway paired this
with a calculus to
glue the blocks together. Under Conway's tangle calculus, the combinatorial work
of knot tabulation became a game of building from simple to complex. Inspired by
Conway's strategies, a second effort to enumerate eleven crossing knots was
carried out by Caudron \citep{caudron1982classification}, verifying
Conway's findings
and rectifying the four omissions. Caudron's confirmation of the eleven crossing
tables marked the final chapter in the hand computation era of knot tabulation.

\subsection{Knot Tabulation by Computer}

With advancements in manufacturing in the early 1980s, electronic computers
became closer to commodity products. This allowed for researchers of diverse
backgrounds and interests to take their crack at time-consuming computational
tasks. One of these computational tasks was the construction of knot tables. The
first to construct a knot table by computer were Dowker and Thistlethwaite in
1983 \citep{dowkerClassificationKnotProjections1983}, who produced a
table of all
prime knots to thirteen crossings. The pair implemented a novel two pass
approach that has served as an outline for all major efforts that have followed.
The process begins with a first pass to enumerate all possible knot diagrams
(described in \nf{subsec-knot_def}). This is followed by a second pass where
``sufficient invariants to distinguish them (knots) from each other''
\citep{dowkerClassificationKnotProjections1983} are computed. This effectively
assigns knots to equivalence classes (bins \footnote{In software
    development, the mathematical concept of an equivalence class is
often called a ``bin''.}), hence finding and removing
all duplicate entries from the list (deduping \footnote{Short for
    deduplicating, meaning the removal of duplicate entries from the
list.}).

The next table produced by computer, knots up to sixteen crossing, was given by
Hoste, Thistlethwaite, and Weeks \citep{hosteFirst1701936Knots1998}
in 1998. Their
process deviates only slightly from the earlier approach of Dowker and
Thistlethwaite, by leaning on heuristics\footnote{When discussing
  algorithms, a heuristic describes a special case that, when
seen, short circuits the algorithm, reducing unnecessary work.} to
limit the duplicates
found in their first pass. This preprocessing in the first pass allowed Hoste,
Thistlethwaite, and Weeks to compute their table in one to two weeks of wall
time\footnote{The real effective time of a clock on a wall, this is
  different from CPU
time which is a relative measure of time.}.

The most recent computational effort was carried out by Burton in 2020
\citep{burtonNext350Million2020}, finding knots to nineteen crossings. Burton's
program followed closely to the two pass processes, with further heuristic work
to preprocess in the first pass and heavily relying on the hyperbolic volume
invariant for the second pass. The computation of the nineteen crossing table
required months of wall time on a cluster\footnote{An aggregation of
  many computers, each with many computational cores. We'll
  see later in this thesis that tabulation is massively
parallelizable.}, serving as an important
signpost for the time requirement problems of knot tabulation.

\subsection{Tangle Tabulation}

Conway's tangle construction allowed him to quickly and effectively tabulate
knots by hand. These building blocks of knots are interesting mathematical
objects in their own right, however tabulation efforts for tangles have been
sparse. The importance of the creation of a large table of tangles has been
called:

\begin{quote}
  ``The Most Important Missing Infrastructure Project in Knot
  Theory'' - Bar-Natan
  \citep{bar-natanMostImportantMissing}
\end{quote}

As it stands, tables of tangles have been generated by hand up to seven
crossings by Kanenobu, Saito, and Satoh
\citep{kanenobuTanglesSevenCrossings2003a}.
Computer driven efforts have been undertaken by several members of the
University of Iowa Applied Topology (UIAT) group namely Conolly
\citep{connollyClassificationTabulation2string2021}, Bryhtan
\citep{bryhtanTabulating2stringTangles2024}, and this thesis.
Separately, a table of
algebraic tangles has been produced by Gren, Sulkowska, and,
Gabrov\v{s}ek \citep{gren2025classificationalgebraictangles}. Their tabulation
is built on a binary operation tree based on Conway's
\citep{conwayEnumerationKnotsLinks1970} tangle calculus. Similar binary
operation trees can be seen in Conolly
\citep{connollyClassificationTabulation2string2021}, Caudron
\citep{caudron1982classification}, and discussed in
Section~\ref{sec-monttang}.

\section{Foundations of Knots}

We now begin a more formal description of the foundations of the theory of
knots. Our treatment will begin with the general definition of knots, as well as
similar knot-like objects. Next, we will discuss ways in which two knots can be
considered equivalent. With this, we'll give an example of a common invariant
for knots. Finally, we conclude with descriptions of the notational strategy
that inspired the rest of this thesis, the Conway notation.

%  prettier-ignore-start

\subsection{Definition of a Knot}\label{subsec-knot_def}

%  prettier-ignore-end

As with anything, we must start with a definition, here we give one for a
\textbf{proper knot}.

\begin{definition}{From Jablan and Sazdanovi{\'c}, Definition 1.17
  \textbf{\citep{jablanLinKnotKnotTheory2007}}}{def-knot}
  A \textbf{proper knot} is a smooth embedding of a circle $S^1$ into Euclidean
  3-dimensional space $\R^3$ (or the 3-dimensional sphere
  $S^3$).

\end{definition}With some consideration, we can see this definition
aligns with our intuitive
description of a knot given in
Section~\ref{sec-intro-intuit_knot_theory}. We note that this
definition gives two choices for ambient space, for this thesis we will prefer
$S^3$ as an ambient space, the preference will become clear in later sections. A
natural extension of the concept of a knot is to allow for more than a single
$S^1$ component to be embedded into the ambient space. This allowance gives us
the concept of a \textbf{$c$ component link}.

\begin{definition}{From Jablan and Sazdanovi{\'c}, Definition 1.17
  \textbf{\citep{jablanLinKnotKnotTheory2007}}}{def-link}
  A \textbf{$c$ component link} is a smooth embedding of $c$ disjoint
  copies of a circle
  $S^1$ into $\R^3$ (or $S^3$), where the embeddings of circles $S_i^1$ are its
  components ($i = 1, 2,\dots, c$).

\end{definition}
\begin{convention}
  For convenience, and brevity, in the remainder of this thesis we
  will adopt the
  following convention. The term \textbf{knot} refers to the collection of
  all proper knots and
  all $c$ component links. If we find the need to exclude the $c$
  component links
  from consideration, we will use the term proper knot.
\end{convention}

Playing with this three-dimensional construction for knots, it will quickly
become apparent that three-dimensional knots are unwieldy to work with. To
simplify our work, we will now build a two-dimensionally encoded model for
knots, a \textbf{knot diagram}. We start by taking a knot $K\subset
S^3$. We then
select an $S^2$ such that $K$ lies fully in the interior. Now, for any plane
that lies tangent to $S^2$, we take an orthogonal projection of the knot onto
the plane. We require that the projection have no \textbf{degenerate crossings},
intersections of the knot projection where more than two points are collinear or
where the crossing is not transverse.
We call this projection a \textbf{knot shadow}, an example can be seen in
Figure~\ref{fig-knot_def-shadow}. A knot shadow is interpreted as a planar
graph\footnote{A graph in the mathematical sense, is a set of
  vertices combined with a set of
  relationships between those vertices called edges. A planar graph is a graph
that when drawn in the plane has no overlapping edges.}, with points
where strands overlap (are collinear in the
projection) as vertices, and edges of the shadow as the strands between the
overlaps.

\begin{figure}[H]
  \centering
  \includegraphics[width=0.5\linewidth]{files/knot_shadow-bca368cd9027f5ce97cd68c0bb57f1be.pdf}
  \caption[A schematic diagram demonstrating a knot and its
  shadow.]{A schematic diagram demonstrating a knot (orange) and its
    shadow (grey).
    Imagine a light shining from above the knot onto a piece of paper. The knot
  shadow is the shadow cast on the paper.}
  \label{fig-knot_def-shadow}
\end{figure}

Taking only the shadow of a knot we lose some data that is intuitively
important, the crossings of a knot (the relative distance of collinear points).
To recover this data in a diagram, we split the edges of the shadow where the
strand closer to the projection plane appears to travel under the edge
corresponding to the strand further from the plane. We call the edge that is
split the \textbf{under strand}, and the non-split strand the
\textbf{over strand}. These
augmented knot shadows are called knot diagrams and will serve as our primary
schematic model for knots throughout this thesis. We call the count of crossings
in a knot diagram the \textbf{crossing number} of the knot diagram.

We finish with naming a knot with particular significance, the knot with no
crossings in its diagram is called the \textbf{unknot}.

%  prettier-ignore-start

\subsection{Knot Equivalence}\label{subsec-knot_equivalence}

%  prettier-ignore-end

Armed with the formal definition of a knot, we can make our first progress in
answering the overarching question from
Section~\ref{sec-intro-intuit_knot_theory}.

\begin{quote}
  How do I tell two knots I make apart?
\end{quote}

To tell two knots apart, we need to discuss the concept of sameness, that is,
what is equivalence in knots. Our concept of equivalence for knots is given by
\textbf{ambient isotopy}, and equal knots are said to be
\textbf{ambient isotopic}.

\begin{definition}{From Jablan and Sazdanovi{\'c}, Definition 1.20
  \textbf{\citep{jablanLinKnotKnotTheory2007}}}{def-ambient_isotopic}
  Knots $K$ and $K_1$ are \textbf{ambient isotopic} if there exists a
  continuous function $H: \R^3 \times[0,1] \rightarrow \R^3$ such that:
  \begin{itemize}
    \item{$h_0=H((x, y, z), 0)$ is the identity $\R^3 \rightarrow \R^3$,}
    \item{for all $t \in[0,1], h_t=H((x, y, z), t)$ is a
      homeomorphism $\R^3 \rightarrow \R^3$,}
    \item{if $h_1=H((x, y, z), 1)$, then $h_1(K)=K_1$.}
  \end{itemize}
\end{definition}When working with the three-dimensional model of a
knot, writing down explicit
ambient isotopies is, in general, quite difficult. As we did in
\nf{subsec-knot_def}, we can simplify the concept of equality by moving ambient
isotopy to an equivalence of knot diagrams. Taking the orthogonal projection
model for knot diagrams given in \nf{subsec-knot_def}, ambient isotopy can be
modeled as three Reidemeister moves on diagrams
\citep{reidemeisterElementareBegruendungKnotentheorie1927}. Meaning,
two knots are
ambient isotopic if and only if their diagrams are equal under a chain of
Reidemeister moves
\citep{reidemeisterElementareBegruendungKnotentheorie1927} and isotopies.

The first Reidemeister move we will define is the Type I move
\citep{reidemeisterElementareBegruendungKnotentheorie1927}. To carry
out the Type I
move (Figure~\ref{fig-knot_def-r1}), start by taking a portion of a
diagram with no
crossings, then add a half twist. When adding the twist, we have two choices;
twist into (left handed) or out of (right handed) the plane the diagram lies in.
In either, we can freely remove the new crossing by twisting in the opposite
direction.

\begin{figure}[H]
  \centering
  \includegraphics[width=0.3\linewidth]{files/R1-1068ad57841d108de41eaba727a15cd2.pdf}
  \caption[Executing the two flavors of type I move.]{Executing the
    two flavors of type I move on a knot diagram. On the left, we have
    a twist into the plane, also called a positive or left-handed
    twist. On the right,
  we have a twist into the plane, also called a negative or right-handed twist.}
  \label{fig-knot_def-r1}
\end{figure}

The next Reidemeister move is the Type II
move\citep{reidemeisterElementareBegruendungKnotentheorie1927}, seen in
Figure~\ref{fig-knot_def-r2}. When we carry out the type II move, we
need two strands, each
with no crossings. We then pull one strand on top of the other, inducing two new
crossings in our diagram. Similarly to the type I move, the type II move can be
freely undone by pulling the strands apart.

\begin{figure}[H]
  \centering
  \includegraphics[width=0.3\linewidth]{files/R2-4010ce2b2b2ae0fefc84be521d714e22.pdf}
  \caption[Executing the two type II moves.]{Executing the two type
    II moves with a pair of strands. In the top image, the
    bottom strand is pulled over the upper. In the bottom image, the
    bottom strand
  is pulled under the top strand.}
  \label{fig-knot_def-r2}
\end{figure}

The final Reidemeister move is the Type III
move\citep{reidemeisterElementareBegruendungKnotentheorie1927}. In
the type III move,
we take three strands, two that form a crossing and a third that lies in one of
three possible positions:

\begin{enumerate}
  \item above the over strand
  \item between the over and under strands
  \item below the under strand
\end{enumerate}

We now execute the type III by taking the third strand (not part of the center
crossing) and passing it across the center crossing. As with type I and type II,
we're free to reverse the type III move.

\begin{figure}[H]
  \centering
  \includegraphics[width=0.3\linewidth]{files/R3-053c00ec76f615349542d57514de7583.pdf}
  \caption[Executing the three type III moves.]{Executing the three
    type III moves with a set of three strands. Top
    to bottom, the third strand is: $\ \bullet$ on top of the
    crossing strands $\ \bullet$ between
  the crossing strands $\ \bullet$ under the crossing strands.}
  \label{fig-knot_def-r3}
\end{figure}

We should note here that with a concept of equivalence comes equivalence classes
of knot diagrams. Historically, of particular interest in the tabulation of
knots, are the knot diagrams that have minimal crossing number, we call these
\textbf{minimal diagrams}, knot diagrams where crossing number cannot be
decreased by Reidemeister moves.

%  prettier-ignore-start

\subsection{Prime Knots}\label{subsec-prime_knot}

%  prettier-ignore-end

With the goal of enumerating objects, we should be clear on what types of
objects should be enumerated and which should be left uncounted. We now describe
the class of knot that tabulators are interested in, the
\textbf{prime knots}. The
first step is to define an operation on knots, called the \textbf{connect sum}.

\begin{definition}{}{}For knots $J$ and $K$, the \textbf{connect sum}
  $J\#K$ is produced by:

  \begin{enumerate}
    \item Excising an arc from both $J$ and $K$
    \item Gluing endpoints of $J$ to endpoints of $K$ so no new
      crossings are added.
  \end{enumerate}

  \begin{figure}[H]
    \centering
    \includegraphics[width=\linewidth]{files/fig-prime_knot-conne-dc193ded4f495df17dda8990f19f876c.pdf}
    \caption[An example of the connect sum of two trefoil knots.]{An
    example of the connect sum of two trefoil knots.}
    \label{fig-prime_knot-connect_sum}
  \end{figure}

\end{definition}With the connect sum operation defined, we are now
prepared to give the
definitions for prime and composite knots.

\begin{definition}{}{}
  A knot is called \textbf{prime} if, in every decomposition into a
  connect sum, one
  of the factors is unknotted. Otherwise, the knot is called \textbf{composite}.

\end{definition}
%  prettier-ignore-start

\subsection{Knot Invariants}\label{subsec-invariant}

%  prettier-ignore-end

Our next topic of interest is that of a knot invariant. In general, an invariant
for an object is a datum that is computed deterministically for the object and
remains unchanged within an equivalence class. In the knot case, we
will take the
concept of equality to be that given in \nf{subsec-knot_equivalence}.
As discussed
in Section~\ref{sec-history-of-tabulation}, invariants play an
important role in computer
tabulation. We will now describe a simple invariant we first
introduced in \nf{subsec-knot_equivalence}.

% The remainder of this section will be devoted to the construction of
% two historically significant knot invariants, the minimal crossing number and
% the \textbf{Jones polynomial}.

\subsubsection{Minimal Crossing Number}

We saw at the end of \nf{subsec-knot_equivalence} the definition of
the minimal crossing number for a knot. That being the minimal number of
crossings needed to represent the knot as a diagram. Somewhat intuitively,
this number is an invariant for a knot. If a knot has minimal crossing number
$4$, we will never be able to represent it with three crossings, so it has to
be different from the trefoil knot (a knot with three crossings).  However, we
can see from the table of the first seven knot (seen in
Figure~\ref{fig-hist-7table}) that having
equivalent crossing number does not give us equivalent knots.

% \subsubsection{Jones Polynomial}

% In this section we will describe a more complex invariant for knots. While we
% won't use this construction for our tangle tabulation it provides an example
% for the kind of theory available for future work by undergraduates
% (see Section~\ref{sec-selection_projects}).
% The Jones polynomial (Definition~\ref{def-jones}) is due to Vaughan Jones
% \citep{jonesPolynomialInvariantKnots1985}, which he discovered in
% 1985 during his
% work on Von Neumann algebras. A more combinatorial approach to the
% construction,
% due to Kauffman \citep{kauffmanStateModelsJones1987}, is given here
% and uses the
% \textbf{Kauffman bracket}.

% \begin{definition}{From Lickorish, Definition 3.6
% \textbf{\citep{kauffmanStateModelsJones1987a,
% jonesPolynomialInvariantKnots1985,
% lickorishIntroductionKnotTheory1997}}}{def-jones}
% The Jones Polynomial $V\LP \mathscr{K}\RP$ of an oriented knot
% $\mathscr{K}$ is
% the Laurent polynomial\footnote{A Laurent polynomial is a
% polynomial that is allowed to have both positive
% and negative powers.} with integer coefficients in $t^{1/2}$. Defined by

% \begin{equation}
% V\LP \mathscr{K}\RP=\LP\LP-A\RP^{\ \m 3w(K)}\LA K\RA\RP_{t^{1/2}=A^{\ \m 2}}
% \end{equation}
% where $K$ is any oriented diagram for $\mathscr{K}$.

% \end{definition}
% %  prettier-ignore-start

% \paragraph{Kauffman Bracket}\label{subsec-kauff}

% %  prettier-ignore-end

% The Kauffman bracket is a function that takes knot diagrams as
% input and outputs
% a Laurent polynomial[\^lp] (Definition~\ref{kb-def-kaufb}).

% \begin{definition}{Kauffman, Definition 2.1
% \textbf{\citep{kauffmanStateModelsJones1987,
% lickorishIntroductionKnotTheory1997}}}{kb-def-kaufb}
% The \textbf{Kauffman bracket} of an unoriented knot $K$, $\LA
% K\RA$, is the Laurent
% polynomial with integer coefficients in $A$ given by the following relations:

% \begin{enumerate}
% \item $\LA\img{media/kauf_bkt/unknot} \RA=1$
% \item $\LA K \sqcup \img{media/kauf_bkt/unknot} \RA=\LP -A^2
% -A^{\m2}\RP \LA K \RA$
% \item $\LA \img{media/kauf_bkt/crossing/crossing_un} \RA=A\LA
% \img{media/kauf_bkt/type2/6a} \RA+A^{\ \m 1}\LA
% \img{media/kauf_bkt/type2/6b} \RA$
% \end{enumerate}

% \end{definition}Since the number one and the unknot serve important
% roles for polynomials and
% knots respectively, the first criterion is intuitive to select,
% when inventing a
% knot polynomial invariant. For now, we focus our discussion on the third
% criterion. The action we see in three is called \textbf{smoothing a
% crossing} and is
% difficult to see. We will now present an intuitive model (originally by Jones
% \citep{jonesJonesPolynomialDummies2014}) for what happens in the
% smoothing process.

% Consider a crossing as a tangle, imagine that the over strand of
% the crossing is
% the slot of a giant flathead screw, and attach a marker on either end of the
% slot. Taking a screwdriver, we may turn the screw clockwise or anti-clockwise.
% When the screw is turned, the markers (arrows in
% Figure~\ref{fig-jp-screw-model}) trace out
% two arcs on the page. We create two new tangles by placing arcs inside each
% tangle and connecting the endpoints joined by the marker trace. We can see, by
% inspection, that the resulting pictures match the components of the third
% condition of Definition~\ref{kb-def-kaufb}.

% \begin{figure}[H]
% \centering
% \includegraphics[width=0.5\linewidth]{files/fig-jp-screw-model_i-c5ffbb95a016d5b69741e75d01c6beb3.pdf}
% \caption[A diagram depicting the screwdriver model of crossing
% smoothing.]{A diagram depicting the screwdriver model of crossing
% smoothing. On top, we have
% the original, pre smoothed crossing. On the bottom left, we have an
% anti-clockwise
% smoothing. The result connects the top point to the left point and
% the bottom point
% to the right point. On the bottom right we have a clockwise smoothing. The
% result connects the top point to the right point and the bottom point
% to the left point.}
% \label{fig-jp-screw-model}
% \end{figure}

% We will now move to proving that $\LA\,\RA$ is invariant under Reidemeister
% moves. That is, when we start with two equivalent (by Reidemeister move) local
% knot diagrams, the brackets of the two diagrams are equivalent.
% First, we prove
% that $\LA\,\RA$ is invariant under the type II move. For a moment,
% we will write
% the third condition of Definition~\ref{kb-def-kaufb} as
% $\LA \img{media/kauf_bkt/crossing/crossing_un} \RA=A\LA
% \img{media/kauf_bkt/type2/6a} \RA+B\LA \img{media/kauf_bkt/type2/6b} \RA$
% and the second as
% $\LA K \sqcup \img{media/kauf_bkt/unknot} \RA=\LP -A^2 -A^{\m2}\RP \LA K \RA$.

% \begin{theorem}{}{thm-typeii_bkt}The equality of equation
% (\ref{thm-kb-math-t2}) holds.
% \begin{equation}
% \label{thm-kb-math-t2}
% \LA \img{media/kauf_bkt/type2/1}\RA=\LA \img{media/kauf_bkt/type2/6b} \RA
% \end{equation}

% \end{theorem}\begin{proof}We begin by smoothing the crossing on the
% right side of the diagram, we then smooth
% the other crossing, this process yields the following chain of equalities,
% combining like terms where appropriate.

% \begin{equation}
% \begin{aligned}
%  \bkt{media/kauf_bkt/type2/1}
% &=A\bkt{media/kauf_bkt/type2/2a}+\LP
%   B\RP\bkt{media/kauf_bkt/type2/2b}\\
% &=A \LP
%     A\bkt{media/kauf_bkt/type2/3a}+\LP
%     B\RP\bkt{media/kauf_bkt/type2/4}\RP
%  +\LP B\RP\LP A\bkt{media/kauf_bkt/type2/6b}+\LP
%     B\RP\bkt{media/kauf_bkt/type2/6a}\RP\\
% &=\LP A^2+ABC+B^2\RP\bkt{media/kauf_bkt/type2/6a}
%  +\LP AB\RP\bkt{media/kauf_bkt/type2/6b}\\
% \end{aligned}
% \end{equation}
% To ensure that the bracket polynomial is an invariant we need
% to select $A, \ B, \ C$ so that $A^2+ABC+B^2=0$ and $AB=1$. The
% conditions are satisfied by
% $A=A$, $B=A^{\m 1}$, and $C=\LP -A^2 -A^{\m2}\RP$ (justifying the
% second criteria
% of Definition~\ref{kb-def-kaufb}). We then compute from the
% beginning (\ref{sec-kb-math-t2ftb}),
% showing the desired result.

% \begin{equation}
% \label{sec-kb-math-t2ftb}
% \begin{aligned}
%  \bkt{media/kauf_bkt/type2/1}
% &=A\bkt{media/kauf_bkt/type2/2a}+\LP
% B\RP\bkt{media/kauf_bkt/type2/2b}\\
%  &=A \LP
% A\bkt{media/kauf_bkt/type2/3a}+\LP
% A^{\ \m 1}\RP\bkt{media/kauf_bkt/type2/4}\RP+\LP
% A^{\ \m 1}\RP\LP A\bkt{media/kauf_bkt/type2/6b}+\LP
% A^{\ \m 1}\RP\bkt{media/kauf_bkt/type2/6a}\RP\\
%  &= \LP
% A^2+A^{\ \m
% 2}\RP\bkt{media/kauf_bkt/type2/6a}+\bkt{media/kauf_bkt/type2/4}+\bkt{media/kauf_bkt/type2/6b}\\
%  &= \LP
% A^2+A^{\ \m 2}\RP\bkt{media/kauf_bkt/type2/6a}+\LP-A^{2}-A^{\ \m
% 2}\RP\bkt{media/kauf_bkt/type2/6a}+\bkt{media/kauf_bkt/type2/6b}\\
%  &=\bkt{media/kauf_bkt/type2/6b}\\
% \end{aligned}
% \end{equation}

% \end{proof}Now, utilizing the fact that $\LA\,\RA$ is invariant for
% type II, we will prove
% the same for type III.

% \begin{theorem}{}{thm-typeiii_bkt}The equality of equation
% (\ref{thm-kb-math-t3}) holds.
% \begin{equation}
% \label{thm-kb-math-t3}
% \LA \img{media/kauf_bkt/type3/1}\RA=\LA \img{media/kauf_bkt/type3/6} \RA
% \end{equation}

% \end{theorem}\begin{proof}We begin by smoothing the crossing at the
% center of each diagram in the
% equality, yielding

% \begin{equation}
% \label{eq-kbkt-t3-1}
% \bkt{media/kauf_bkt/type3/1}
% =A\bkt{media/kauf_bkt/type3/2b}+A^{\ \m 1}\bkt{media/kauf_bkt/type3/2a}
% \end{equation}

% \begin{equation}
% \label{eq-kbkt-t3-2}
% \bkt{media/kauf_bkt/type3/6}
% =A\bkt{media/kauf_bkt/type3/3b}+A^{\ \m 1}\bkt{media/kauf_bkt/type3/3a}
% \end{equation}
% Now, on the clockwise smoothing equation (\ref{eq-kbkt-t3-1}), we
% execute a type II
% move and obtain Figure~\ref{fig-inv-t3-c}.

% \begin{figure}[H]
% \centering
% \includegraphics[width=0.5\linewidth]{files/2b_color-1bfc5e5306fab67c2456967cbbdb0f6b.pdf}
% \caption[First smoothing of a type III move.]{On the left is a
% clockwise smoothing of the type III move type crossing
% Figure~\ref{fig-knot_def-r3}. On the right is an application of the
% type II move on the
% under strand.}
% \label{fig-inv-t3-c}
% \end{figure}

% Similarly, the anti-clockwise smoothing from equation
% (\ref{eq-kbkt-t3-2}) we execute a type II
% move to obtain Figure~\ref{fig-inv-t3-cc}.

% \begin{figure}[H]
% \centering
% \includegraphics[width=0.5\linewidth]{files/2a_color-5837b7c8e662b5e2c0f8448cce0ff0fa.pdf}
% \caption[Second smoothing of a type III move.]{On the left is an
% anti-clockwise smoothing of the type III move type crossing
% Figure~\ref{fig-knot_def-r3}. On the right is an application of the
% type II move on the
% under strand.}
% \label{fig-inv-t3-cc}
% \end{figure}

% Since the type II move is invariant for the bracket polynomial
% (Theorem~\ref{thm-typeii_bkt})
% the chain of equalities in equation (\ref{eq-kbkt-t3-final}) shows
% the desired result.

% \begin{equation}
% \label{eq-kbkt-t3-final}
% \begin{aligned}
% \bkt{media/kauf_bkt/type3/1}
% &=A\bkt{media/kauf_bkt/type3/2b}+A^{\ \m 1}\bkt{media/kauf_bkt/type3/2a}\\
% &=A\bkt{media/kauf_bkt/type3/3b}+A^{\ \m
% 1}\bkt{media/kauf_bkt/type3/3a}&&\text{(by type II move)} \\
% &=\bkt{media/kauf_bkt/type3/6}
% \end{aligned}
% \end{equation}

% \end{proof}Now for the final Reidemeister move, the type I move, we
% compute two chains of
% equality, (\ref{sec-kb-math-t11}) and (\ref{sec-kb-math-t12}).

% \begin{equation}
% \label{sec-kb-math-t11}
% \begin{aligned}
%  \bkt{media/kauf_bkt/type1/1b} & =
% A\bkt{media/kauf_bkt/type1/2a}+A^{\ \m 1}\bkt{media/kauf_bkt/type1/2b}\\
% & = A\bkt{media/kauf_bkt/type1/3}+A^{\ \m 1}\LP
% -A^2-A^{\ \m 2}\RP\bkt{media/kauf_bkt/type1/3}\\
% & = \LP A+
% -A^-A^{\ \m 3}\RP\bkt{media/kauf_bkt/type1/3}\\
% & = -A^{\ \m 3}\bkt{media/kauf_bkt/type1/3}\\
% \end{aligned}
% \end{equation}

% \begin{equation}
% \label{sec-kb-math-t12}
% \begin{aligned}
%  \bkt{media/kauf_bkt/type1/1} & =
% A\bkt{media/kauf_bkt/type1/2b}+A^{\ \m 1}\bkt{media/kauf_bkt/type1/2a}\\
%  & = A\LP
% -A^2-A^{\ \m 2}\RP\bkt{media/kauf_bkt/type1/3}+A^{\ \m
% 1}\bkt{media/kauf_bkt/type1/3}\\
% & = \LP
% -A^3-A^{\ \m 1}+A^{\ \m 1}\RP\bkt{media/kauf_bkt/type1/3}\\
% & = -A^{3}\bkt{media/kauf_bkt/type1/3}\\
% \end{aligned}
% \end{equation}

% Observe, with (\ref{sec-kb-math-t11}) and (\ref{sec-kb-math-t12}),
% that the bracket polynomial
% as it stands is not invariant under Reidemeister moves. We must augment the
% bracket polynomial to find the invariance we would like.

% \subparagraph{Writhe}

% Before we can develop the concept of the \textbf{writhe} of a knot diagram, we
% need to define an \textbf{orientation} for a knot diagram. Imagine
% the strands of a knot to be a roller
% coaster, it's clear that the rollercoster car has two choices for
% direction. One
% direction we intuitively call forward and its oposite backward. Just like most
% rollercosters, each component of a knot is a circle we can choose
% to traverse in
% one of two directions. On a knot diagram we call this choice of
% forward, an orientation,
% see (Figure~\ref{fig-writhe-orientation}).

% \begin{figure}[H]
% \centering
% \includegraphics[width=0.3\linewidth]{files/5_1-8d4df8b4b3c3dbcd7f8dfc4314714dc7.pdf}
% \caption[An orientation applied to the knot.]{An orientation
% applied to the knot 5\textsubscript{1}. Following the strand in the direction
% of the arrow, you'll find that you arrive where you started.}
% \label{fig-writhe-orientation}
% \end{figure}

% We now zoom in on our oriented diagram, and focus on each crossing. Since the
% whole diagram is oriented, each strand in the local crossing has an induced
% orientation, indicated by arrow heads on the strands. We arrange the crossing,
% by rotating, so that the over strand arrowhead points up, notice then that
% there are two possibilities for the under strand; pointing west or pointing
% east. We call the crossing with a west pointing under strand a positive (+)
% crossing (Figure~\ref{fig-inv-or-positive}), while an east pointing
% under strand is called a
% negative (-) crossing (Figure~\ref{fig-inv-or-negative}).

% \begin{figure}[H]
%     \centering
% \begin{subfigure}[c]{.2\textwidth}
% \centering
% \includegraphics[width=\textwidth]{files/crossing_+-a152d2a53dd5f6118d5896ef3bd810a3.pdf}
% \caption[A positive crossing.]{A positive crossing.}
% \label{fig-inv-or-positive}
% \end{subfigure}
% ~
% \begin{subfigure}[c]{.2\textwidth}
% \centering
% \includegraphics[width=\textwidth]{files/crossing_--eed911694e1575f898c0b8af79213f77.pdf}
% \caption[A negative crossing.]{A negative crossing.}
% \label{fig-inv-or-negative}
% \end{subfigure}
% \caption[The two crossing orientations.]{The two crossing orientations.}
% \end{figure}

% \begin{note}
% A hint for determining the orientation of a crossing in practice is
% a right-hand
% rule. After the knot has been oriented, imagine your right hand with your
% fingers extended and palm towards you. Align your hand with the over strand so
% that your thumb points with the arrowhead. Now if your fingers are
% pointing with the
% arrowhead of the under strand, the crossing is positive, if your fingers are
% opposite the arrowhead of the under strand, the crossing is negative.
% \end{note}

% Given an oriented knot diagram, we can now classify each crossing
% as either positive or
% negative, providing us with enough information to define the writhe of a knot.

% \begin{definition}{}{def-writhe}The writhe, $w\LP K\RP$, of an
% oriented knot diagram $K$ is defined to be:
% \begin{equation}
% \begin{aligned}
% w\LP K\RP&=\LP\text{count of positive crossings in } K\RP\\
% &-\LP\text{count of negative crossings in } K\RP
% \end{aligned}
% \end{equation}

% \end{definition}
% \begin{note}
% The writhe of a diagram is not an invariant for the knot. We can see this by
% adding a number of type I moves to a diagram, which will give us two different
% writhes for the same knot.
% \end{note}

% The writhe is fairly simple to calculate, we will now carry out an example
% calculation for knot 5\textsubscript{1}.

% We start by taking the oriented version of knot 5\textsubscript{1} seen in
% Figure~\ref{fig-writhe-orientation} and modifying the diagram with
% local pictures for each
% crossings seen in Figure~\ref{fig-writhe-local}.

% \begin{figure}[H]
% \centering
% \includegraphics[width=0.4\linewidth]{files/fig-writhe-local-4733d4e765087c4c45e8c8937260d970.pdf}
% \caption[An oriented five crossing knot.]{An oriented five crossing
% knot, each local picture around a vertex is magnified
% to show local crossing signs.}
% \label{fig-writhe-local}
% \end{figure}

% Classifying each crossing as positive or negative, we can count finding 5
% positive and 0 negative crossing giving

% \begin{equation}
% w\LP\img{media/kauf_bkt/orientation/5_1}\RP=5+0=5
% \end{equation}

% \subparagraph{Kauffman Bracket: Type I}

% With the writhe, we're ready to augment the bracket polynomial applied to the
% type I move. We do this with the addition of multiplication by a
% $-A^{\ \m 3w\LP K\RP}$ term, obtaining Theorem~\ref{thm-typei}.

% \begin{theorem}{}{thm-typei}The equality of equation
% (\ref{thm-kb-math-t1}) holds.
% \begin{equation}
% \label{thm-kb-math-t1}
%     \begin{aligned}
%         -A^{\ \m 3w\LP
%             \img{media/kauf_bkt/type1/1b}\RP}\bkt{media/kauf_bkt/type1/1b}
%             & = \bkt{media/kauf_bkt/type1/3}\\
%         -A^{\ \m 3w\LP
%             \img{media/kauf_bkt/type1/1}\RP}\bkt{media/kauf_bkt/type1/1}
%             & = \bkt{media/kauf_bkt/type1/3}\\
%     \end{aligned}
% \end{equation}

% \end{theorem}\begin{proof}We first compute our writhes
% \begin{equation}
%     \begin{aligned}
%             -w\LP \img{media/kauf_bkt/type1/1b}\RP=\ \m 1\\
%             -w\LP \img{media/kauf_bkt/type1/1}\RP=1\\
%     \end{aligned}
% \end{equation}
% now computing each flavor of type I move in turn
% \begin{equation}
% \begin{aligned}
%         -A^{\ \m 3w\LP
%     \img{media/kauf_bkt/type1/1b}\RP}\bkt{media/kauf_bkt/type1/1b}
%     & =
%     \LP-A^{3}\RP\LP-A^{\ \m 3}\bkt{media/kauf_bkt/type1/3}\RP \\
%         & =
%     \LP-A^{3}\RP\LP-A^{\ \m 3}\bkt{media/kauf_bkt/type1/3} \RP\\
%         & = \bkt{media/kauf_bkt/type1/3} \\
% \end{aligned}
% \end{equation}
% \begin{equation}
% \begin{aligned}
%         -A^{\ \m 3w\LP
%     \img{media/kauf_bkt/type1/1}\RP}\bkt{media/kauf_bkt/type1/1}
%     & =
%     \LP-A^{\ \m 3}\RP\LP-A^{3}\bkt{media/kauf_bkt/type1/3}\RP\\
%         & =
%     \LP-A^{\ \m 3}\RP\LP-A^{3}\bkt{media/kauf_bkt/type1/3} \RP\\
%         & = \bkt{media/kauf_bkt/type1/3} \\
% \end{aligned}
% \end{equation}
% the desired sets of equality.

% \end{proof}Having gained each equality needed for an invariant of
% knots, we finish the
% section with the following theorem.

% \begin{theorem}The Jones polynomial (Definition~\ref{def-jones}) is
% an invariant of oriented knots.

% \end{theorem}\begin{proof}Combining Theorem~\ref{thm-typeii_bkt},
% Theorem~\ref{thm-typeiii_bkt}, and Theorem~\ref{thm-typei},
% substituting $A$ for
% $t$ we obtain the result.

% \end{proof}
\subsection{Knot Notations}

The final topic to cover in our treatment of foundations for knot theory is
notational strategies for knots. In \nf{subsec-knot_def}, we came
across our first
notational strategy for knots, the knot diagrams. While diagrams are a great
human-readable way to note a knot, when the tasks of enumeration and computation
(by hand) are considered, knot diagrams quickly show deficiencies. These
deficiencies become intractable when a computer is brought into the picture. As
a remedy for this issue, knot theorists have invented several combinatorial
notations for knots. Perhaps the most historically important knot notation for
use in tabulation by computer is the Dowker-Thistlethwaite (DT) notation
(\nf{sec-proj-note_dt}), developed by its namesakes specifically for use in
computational tabulation. Each notational strategy used in knot theory has
strengths and weaknesses. For example, using DT notation for computation of the
Jones polynomial may be more cumbersome
than using the Planar Diagram
(PD) notation (\nf{sec-proj-note_pd}) for the same task, as PD directly encodes
crossings while DT encodes a walk on a strand. The remainder of this section
will be the development of the Conway notation, which lays the foundation for
the work in this thesis.

%  prettier-ignore-start

\subsubsection{Conway Notation}\label{sec-conway}

%  prettier-ignore-end

In Section~\ref{sec-history-of-tabulation}, we saw that Conway
claimed to have enumerated
knots up to 11 crossings in ``a few hours''. Conway accomplished this
by breaking
knots into building blocks he called tangles. This section gives an outline of
the tools he developed and used to achieve those ``few hours'' of amazing
efficiency.

%  prettier-ignore-start

\paragraph{Definition of a Tangle}\label{subsubsec-deftangles}

%  prettier-ignore-end

Our first step in unlocking Conway's tabulation secrets is the definition of a
tangle. We will give Conway's original definition followed by a description of
what this looks like for a three dimensional embedding for a knot.

\begin{definition}{Conway, Page 330
  \textbf{\citep{conwayEnumerationKnotsLinks1970}}}{}
  We define a \textbf{tangle} as a portion of a knot diagram from
  which there emerge just 4
  arcs pointing in the compass directions $NW, \ NE, \ SW, \ \text{and }SE$.

\end{definition}These boundaries that split knots at four points are
called \textbf{Conway circles},
and we call the points $NW,\ NE,\ SW,\ \text{and }SE$ \textbf{boundary points}.
Formally, we can consider a Conway circle to be a Jordan
curve\footnote{A Jordan curve is a simple closed curve. This can be
  thought of as a curve
drawn on a piece of paper that has: 1) No end points. 2) No self
intersections.}
meeting the knot diagram in exactly four points
\citep{bonahonNewGeometricSplittings2016}. In general, we prefer our
Conway circles
to actually be circles in the colloquial sense. Luckily, the circle and Jordan
curve constructions are equivalent. This can be seen by a straightforward
isotopy of one into the other, per Figure~\ref{fig-generic_jordan_iso}.

\begin{figure}[H]
\centering
\includegraphics[width=\linewidth]{files/conway_circ_isotopy-9a8c6805166c373f6a290fdb7e952bb9.pdf}
\caption[An isotopy turning Jordan curves into circles.]{An isotopy
turning Jordan curves into circles.}
\label{fig-generic_jordan_iso}
\end{figure}

We move our attention to the three dimensional analog for a Conway circle, the
\textbf{Conway sphere}. A Conway sphere, similar to the Conway
circle, is an $S^2$
that encapsulates a portion of a knot so that the knot intersects the sphere in
exactly four points. Here we see the first example of our preference for ambient
space to be $S^3$ as opposed to $\R^3$. When a knot in $S^3$ is split by a
Conway sphere, the ambient $S^3$ is decomposed into two $B^3$, each with a
portion of the knot. Meaning, a single Conway sphere splits a knot into a pair
of tangles.

%  prettier-ignore-start

\paragraph{Basic Tangles}\label{subsubsec-basic_tangles}

%  prettier-ignore-end

Often, when thinking about a new construction, we focus on the simplest object
that can be created with the construction. In the case of drawing Conway circles
to build tangles, the simplest tangles are a tangle with no crossings (the
0 tangle Figure~\ref{prelim-fig-basic_0}) and a tangle with a single
crossing (the +1
tangle Figure~\ref{prelim-fig-basic_1}).

\begin{figure}[H]
\centering
\begin{subfigure}[c]{0.3\textwidth}
\centering
\includegraphics[width=\textwidth]{files/0-4d1017ef91a5026c6771eb9f6ca80be3.pdf}
\caption[A tangle with no crossings.]{A tangle with no crossings,
called the 0 tangle.}
\label{prelim-fig-basic_0}
\end{subfigure}
\quad
\quad
\quad
\begin{subfigure}[c]{0.3\textwidth}
\centering
\includegraphics[width=\textwidth]{files/1-8c32584b33e2a4295948c6caf7fd3d15.pdf}
\caption[A tangle with a single crossing.]{A tangle with a single
crossing, called the 1 tangle.}
\label{prelim-fig-basic_1}
\end{subfigure}
\caption[Two basic tangles.]{Two basic tangles.}
\label{fig-basic_tangles}
\end{figure}
%  prettier-ignore-start

\paragraph{Rotation and Mirroring of Tangles}\label{subsubsec-tangle_flips}

%  prettier-ignore-end

Consider a \textbf{generic tangle}, as seen in
Figure~\ref{fig-generic_tangle}, where
orientation (the position of the NW point) of data in the interior of the Conway
circle is indicated by a broken $T$.

\begin{figure}[H]
\centering
\includegraphics[width=0.3\linewidth]{files/generic_tangle-a689b8d843d000777715cd84c327d565.pdf}
\caption[A generic tangle.]{A generic tangle with a broken T.}
\label{fig-generic_tangle}
\end{figure}

We can manipulate this tangle by the set of rotations, clockwise or
anti-clockwise. Each rotation in turn gives a new arrangement of the interior
data. We can also manipulate the tangle by the set of flips, one around the core
x-axis and one around the y-axis. Each flip gives an arrangement of the interior
data. Pairing flips with rotations gives the table seen in
Figure~\ref{fig-tangle_flips}.

\begin{figure}[H]
\centering
\includegraphics[width=0.7\linewidth]{files/fig-tangle_flips-3e9facf2c059882951c81a73f9e178fa.pdf}
\caption[A table with all unique rotations and flips for a generic
tangle.]{A table with all unique rotations and flips for a generic tangle.
From top to bottom in the first column:$\ \bullet$ No Flip$\ \bullet$
Flip around the north south axis.
From left to right in each row:$\ \bullet$ No rotation$\ \bullet$
rotation quarter turn clockwise$\ \bullet$ rotation half turn clockwise
$\ \bullet$ rotation three quarter turn clockwise$\ \bullet$ rotation
quarter turn clockwise}
\label{fig-tangle_flips}
\end{figure}

When we apply this set of flips and rotations to the basic tangles seen in
\nf{subsubsec-basic_tangles}, we obtain the two additional basic tangles seen in
Figure~\ref{fig-basic_tangles_extra}.

\begin{figure}[H]
\centering
\begin{subfigure}[c]{0.3\textwidth}
\centering
\includegraphics[width=\textwidth]{files/inf-6468e70ef1d9a2cf5e8722b2b4f72d73.pdf}
\caption[A tangle with no crossings.]{A tangle with no crossings,
called the $\infty$ tangle.}
\label{prelim-fig-basic_nc-inf}
\end{subfigure}
\quad
\quad
\quad
\begin{subfigure}[c]{0.3\textwidth}
\centering
\includegraphics[width=\textwidth]{files/m1-0b2e6635a4f891072ad82f405b21419d.pdf}
\caption[A tangle with a single crossing.]{A tangle with a single
crossing, called the $\m 1$ tangle.}
\label{prelim-fig-basic_c-m1}
\end{subfigure}
\caption[Two additional basic tangles.]{Two additional basic tangles.}
\label{fig-basic_tangles_extra}
\end{figure}
%  prettier-ignore-start

\paragraph{Operations on Tangles}\label{subsubsec-conway_calc}

%  prettier-ignore-end

In addition to the rotations and flips, Conway introduced a calculus on tangles
\citep{conwayEnumerationKnotsLinks1970}. This calculus allowed Conway
to build the
simple basic tangles into iteratively more complex tangles.

%  prettier-ignore-start

\paragraph{Minus Tangle}\label{subsubsec-conway_minus}

%  prettier-ignore-end

For a generic tangle $T$, we call the tangle generated from a clockwise rotation
and flip around the y-axis the negative of $T$, notated $-T$. Equivalently, this
can be thought of as rotating the tangle around the $NW$ and $SE$ axis
(Figure~\ref{fig-opo-minus}).

\begin{figure}[H]
\centering
\includegraphics[width=0.7\linewidth]{files/fig-opo-minus-1fdce40a8fa385c913446181d227f7bf.pdf}
\caption[The ``-'' of a tangle]{Rotating a tangle around the $NW$
$SE$ diagonal, yielding the negative of the
tangle.}
\label{fig-opo-minus}
\end{figure}

\begin{note}
Observe that the minus of the +1 tangle
(Figure~\ref{prelim-fig-basic_1}) is the $\m 1$
tangle (Figure~\ref{prelim-fig-basic_c-m1}).
\end{note}

%  prettier-ignore-start

\paragraph{Tangle Addition}\label{subsubsec-opo-plus}

%  prettier-ignore-end

For a pair of generic tangles, $A$ and $B$, we construct their sum $A+B$ by
first aligning $A$ and $B$ horizontally. We then connect the $NE$ and $SE$ of
$A$ to the $NW$ and $SW$ of $B$, as seen in Figure~\ref{fig-opo-plus}.

\begin{figure}[H]
\centering
\includegraphics[width=\linewidth]{files/fig-opo-plus-5f318d6baa2485438c40832900598a5b.pdf}
\caption[The sum of two generic tangles.]{The sum of two generic tangles.}
\label{fig-opo-plus}
\end{figure}

The class of tangles built by successive addition of the $\pm 1$ basic tangles
are called the \textbf{integral tangles}.

\paragraph{Tangle Multiplication}

For a pair of generic tangles, $A$ and $B$, we construct their product, $A*B$
(or $A\ B$) by first aligning $A$ and $B$ horizontally. We then take $-A$ and
sum the two resulting tangles, equivalent to $-A+B$, as seen in
Figure~\ref{fig-opo-times}.

\begin{figure}[H]
\centering
\includegraphics[width=\linewidth]{files/fig-opo-times-0b4f2e944f3018fb2228367498107e23.pdf}
\caption[The product of two generic tangles.]{The product of two
generic tangles.}
\label{fig-opo-times}
\end{figure}

\begin{note}
Notice that
\begin{equation}
A*0=-A+0=-A
\end{equation}
\end{note}

\paragraph{Tangle Ramification}

For a pair of generic tangles, $A$ and $B$, we construct their ramification
$A,B$ by first aligning $A$ and $B$ horizontally. We then take $-A$ and $-B$ and
sum the resulting tangles. This makes ramification equivalent to $-A -B$ or
$A0+B0$, as seen in Figure~\ref{fig-opo-ramification}.

\begin{figure}[H]
\centering
\includegraphics[width=\linewidth]{files/fig-opo-ramification-ab32acc0ffde9c4a27ffdeb51ac9edaa.pdf}
\caption[The ramification of two generic tangles.]{The ramification
of two generic tangles.}
\label{fig-opo-ramification}
\end{figure}

%  prettier-ignore-start

\paragraph{Indicating Precedence}\label{subsubsec-opo-precedence}

%  prettier-ignore-end

With a set of operations comes the desire to chain multiple operations together.
The precedence for operations on tangles is indicated by parentheses in the
obvious way, as seen in Figure~\ref{fig-opo-prec}.

\begin{figure}[H]
\centering
\includegraphics[width=\linewidth]{files/fig-opo-prec-d6df532c9a7248cf9c91aaf460b986e2.pdf}
\caption[Multiple tangle operations chained together.]{Multiple
operations chained together with precedence indicated by parentheses.}
\label{fig-opo-prec}
\end{figure}

%  prettier-ignore-start

\paragraph{The Flype}\label{subsubsec-opo-flype}

%  prettier-ignore-end

When working in this calculus of tangles, a common situation you find yourself
in is one where the 1 (or $\ \m 1$) tangle is added to a tangle. In this
situation, we can move the 1 crossing from one side of $T$ to the other by a
\textbf{flype}. To complete a flype, we grab the top (north) and
bottom (south) of
the tangle and rotate (opposite the handedness of the crossing) as in
Figure~\ref{fig-opo-flype}.

\begin{figure}[H]
\centering
\includegraphics[width=\linewidth]{files/flype-c1c5e6a071b468e3e84bc83a5a2255ad.pdf}
\caption[Demonstrating the flype move.]{A $(+)$-flype on the top and
a $( -)$-flype on the bottom. Note that the generic
tangle is flipped around the $x$-axis during the flype.}
\label{fig-opo-flype}
\end{figure}

\paragraph{Closures}

Since Conway's interest was in knots, he naturally needed ways to close up a
tangle to form a knot. In this section, we will introduce two ways that this can
be accomplished. The first closure method is the simple tangle closure, where
points on a tangle are connected. The second closure method is the insertion of
multiple tangles into a graph.

\subparagraph{Simple Tangle Closures}

The first closure method is the simple tangle closure. For a generic tangle $T$,
we have two options for how to simply close up the tangle. One option is to
connect a strand from $NW$ to $NE$ and a strand from $SW$ to $SE$
(Figure~\ref{fig-closure-num}), called the numerator closure. The
alternative is the
denominator closure, formed by connecting a strand from $NW$ to $SW$ and a
strand from $NE$ to $SE$ (Figure~\ref{fig-closure-den}). In both cases, we
introduce no additional crossings.

\begin{figure}[H]\label{fig-closure-prec}
\centering
\begin{subfigure}[c]{0.4\textwidth}
\centering
\includegraphics[height=\textwidth]{files/fig-closure-num-b689f8228388f8fb25c454811afe92c1.pdf}
\caption[The numerator closure of a tangle.]{The numerator closure of a tangle.}
\label{fig-closure-num}
\end{subfigure}
\quad
\quad
\quad
\begin{subfigure}[c]{0.4\textwidth}
\centering
\includegraphics[width=\textwidth]{files/fig-closure-den-82a12a03e602f75e4b812dd59cbaae3d.pdf}
\caption[The denominator closure of a tangle.]{The denominator
closure of a tangle.}
\label{fig-closure-den}
\end{subfigure}
\caption[The two simple closures of a tangle.]{The two simple
closures of a tangle.}

\end{figure}
%  prettier-ignore-start

\subparagraph{Tangle Insertions}\label{subsubsec-opo-insert}

%  prettier-ignore-end

With the calculus of tangles and simple closures, Conway was able to enumerate a
substantial number of knots, but not all. We should notice a common theme with
the calculus, when starting with basic tangles every operation forms
a bigon\footnote{A bigon is a polygon with two sides. In the same way
that an octagon has
eight sides or a trigon (triangle) has three.} between two tangles in the
knot shadow. We can collapse bigons in the knot shadow by deleting edges and
merging the two vertices of a bigon. For a knot formed by only the operations
on basic tangles and with simple closures, if we iteratively collapse
all bigons we obtain a
four-valent planar\footnote{A planar graph is one that can be drawn
in the plane without edges
overlapping.} graph\footnote{A graph is said to be four valent if
each vertex has four edge ends
connected to it. In Figure~\ref{fig-bigon_collapse}, the result is
four valent.} with one vertex, per
Figure~\ref{fig-bigon_collapse}. The class of knots who have a
presentation where bigons can
be collapsed to a single vertex with two self edges are called the
\textbf{algebraic
knots}.

\begin{figure}[H]
\centering
\includegraphics[width=\linewidth]{files/fig-bigon_collapse-61482270433167ccce09792d12223a13.pdf}
\caption[Bigon collapsing.]{The collapsing of the bigons in a knot
shadow from left to right: $\ \bullet$ A trefoil knot. $\ \bullet$ A
knot shadow for a trefoil knot with a
bigon highlighted. $\ \bullet$ The previously highlighted bigon
collapsed, and a new
bigon highlighted. $\ \bullet$ A graph with no bigons.}
\label{fig-bigon_collapse}
\end{figure}

To obtain a knot that has non-bigon connections between inputs, we will first
identify a four-valent planar graph that has non-bigon connections between
vertices. The class of graph that is most useful here are the polygon graphs.
For example, the $6^{**}$ graph (or octahedron) can be seen in
Figure~\ref{fig-6starstar}. Within the
$6^{**}$ graph, we notice triangular regions between vertices.

\begin{figure}[H]
\centering
\includegraphics[width=0.5\linewidth]{files/fig-6starstar-0873a710a7445b1d9dea336c329f5d83.pdf}
\caption[A 6 vertex polygon. ]{A four valent planar graph with six
vertices and triangular regions between
vertices. When the graph is placed on the surface of an $S^2$ we get the
octahedron.}
\label{fig-6starstar}
\end{figure}

The simplest thing we can do from here is consider the graph as a knot shadow,
and for each vertex, choose an over and under strand. While that method would
give us a knot, it is limiting. Less limiting is a process of tangle insertion.
In this process we consider each vertex a boundary of a Conway circle in which
we can place a tangle generated with the Conway calculus. When we insert the
tangle into the Conway circle (vertex in the graph), the
$NW,\,NE,\,SW,\,\text{and }SE$ points of the tangle are connected to marked
points of the Conway circle (the four edges of the vertex). An example of a
tangle insertion into $6^{**}$ can be seen in
Figure~\ref{fig-6starstar_insurtion}.

\begin{figure}[H]
\centering
\includegraphics[width=0.5\linewidth]{files/fig-6starstar_insurt-a3fdbf3ad9bd55c3bcc22b2d2b2dcff6.pdf}
\caption[Tangle insertions.]{Tangles inserted into the $6^{**}$
tangle, with Conway notation
\\$6^{**}\ast.1\ 2\ 2\ 3\ 1.1\ 2\ 2\ 3\ 1.1\ 2\ 2\ 3\ 1.1\ 2\ 2\ 3\ 1.1\ 2\ 2\ 3\ 1$
The $\ast$ labeled vertex defines the four boundary points of the
resulting tangle. }
\label{fig-6starstar_insurtion}
\end{figure}

When each vertex has a tangle inserted, the result is a knot. When $n$
vertices are left empty, the result is a tangle with $n$ boundary Conway
circles. If $n$ is 1, we have a tangle in the sense we've been discussing. To
reduce ambiguity, we mark in the graph with a $\ast$ in that empty vertex. On
each polygon graph, we select a canonical ordering of the vertices,
per Figure~\ref{fig-6starstar_ordered}. When
notating the tangle insertions, we list the subtangles we wish to insert in
graph canonical order, each separated by a period and an empty vertex indicated
with $\ast$. Following the terminology outlined by Conolly
\citep{connollyClassificationTabulation2string2021}, we call these
singularly marked
polygonal graphs \textbf{constellations}. We call the tangle diagrams
generated by
this approach the \textbf{polygonal diagrams}. If a tangle has no algebraic
representative we call it a \textbf{polygonal tangle}.

\begin{figure}[H]
\centering
\includegraphics[width=0.5\linewidth]{files/fig-6starstar_orderd-c5a9af08f3c2d72cc2ea67857cfdc271.pdf}
\caption[Order of a polygon.]{The $6^{\ast\ast}$ graph with an order applied.}
\label{fig-6starstar_ordered}
\end{figure}

\section{Foundations of Tangles}

So far we have used tangles as a building block for knots, we will switch gears
slightly to consider a tangle as our main object of interest. This section will
give the foundations of the theory of tangles needed for the remainder of this
thesis.

\subsection{Tangle Equivalence}

As we saw for knots in \nf{subsec-knot_equivalence}, if we want to
tell two tangles
apart, we first need to be able to identify when tangles are the same. Our
development of the concept of equality of tangles follows closely to that of
knots. From \nf{subsubsec-deftangles}, we have definitions for tangles in the
context of knot diagrams and three-dimensional embeddings of knots. This tells
us that the concepts of equality developed in
\nf{subsec-knot_equivalence}, namely
ambient isotopy and Reidemeister moves, will apply in the tangle case with two
key differences.

The first difference with tangles when compared to knots is that we restrict
ambient isotopy and Reidemeister moves from pushing a strand through the
Conway sphere, or Conway circle. The second difference is the handling of the
boundary points. There are two conventions for how to handle equality with the
boundary points; first allowing the boundary points to move on the Conway
sphere, and second fixing the boundary points on the Conway sphere.

\subsubsection{Moveable Boundary Points}

Our first case for handling the boundary of a tangle is allowing the boundary
points to move freely on the Conway sphere. In this case, a generic tangle, is
equivalent to each of its rotations and flips (\nf{subsubsec-tangle_flips}). In
addition to the rotation and flip equivalence, a moveable boundary allows us to
unwind the outermost integral components of a tangle
(Figure~\ref{fig-tangl_eq-unwinding}).

\begin{figure}[H]
\centering
\includegraphics[width=\linewidth]{files/fig-tangl_eq-unwindi-273d573f9a22c37519018e3a651b4f17.pdf}
\caption[Unwinding of integral tangles.]{The progressive unwinding of
integral tangles, leaving a basic 0 tangle.}
\label{fig-tangl_eq-unwinding}
\end{figure}

\begin{note}
In the moveable boundary case, there is only one basic tangle, the 0 tangle.
\end{note}

For this thesis, we will assume tangles have a fixed boundary unless explicitly
mentioned.

\subsubsection{Non-moveable Boundary Points}

The non-moveable case is the more straightforward of the two boundary concepts
of equality for tangles. In the fixed boundary world, we have four distinct
basic tangles $1,\ \ \m 1,\ 0,\ \text{and } \infty$ seen in
Figure~\ref{fig-basic_tangles}
and Figure~\ref{fig-basic_tangles_extra}. These are all distinct when
the boundary is fixed. In Figure~\ref{fig-tangl_eq-fixed}
we find two tangles, each with two crossings, but not equivalent by Reidemeister
moves.

\begin{figure}
\centering
\begin{subfigure}[t]{.45\textwidth}
\centering
\includegraphics[width=\textwidth]{files/fig-tang_eq-fixed-2-b02757c89b1831f1ec4678fa05e05464.pdf}
\caption[A horizontal integral tangle with two crossings.]{A
horizontal integral tangle with two crossings.}
\label{fig-tang_eq-fixed-2}
\end{subfigure}
~
\begin{subfigure}[t]{.45\textwidth}
\centering
\includegraphics[width=\linewidth]{files/fig-tang_eq-fixed-v2-cb810de4fba3b5df9ca4eab5dcecff9d.pdf}
\caption[A vertical integral tangle with two crossings.]{A vertical
integral tangle with two crossings.}
\label{fig-tang_eq-fixed-v2}
\end{subfigure}

\caption[Two nonequivalent tangles with two crossings.]{Two no
equivalent tangles with two crossings.}
\label{fig-tangl_eq-fixed}
\end{figure}
%  prettier-ignore-start

\subsection{Modified Tangle Operations}\label{subsec-tangle_operations}

%  prettier-ignore-end

In \nf{subsubsec-conway_calc} we saw Conway's calculus of tangles. While this
construction is powerful and flexible, it's rather unintuitive and cumbersome
when combined with computer methods. In this section, we will describe a
slightly modified, but equivalent, version of the calculus. This version of
tangle arithmetic is due to Kauffman and Goldman
\citep{goldmanRationalTangles1997}.

Instead of Conway's three operations on the two basic tangles 1 and 0, this
arithmetic needs the two basic operations and all four basic tangles
(Figure~\ref{fig-basic_tangles} and
Figure~\ref{fig-basic_tangles_extra}). The first operation in this
arithmetic is exactly Conway's horizontal sum, $+$
(Figure~\ref{fig-opo-plus}). The second
operation is the vertical sum, $\vee$, sometimes written
$*$\citep{goldmanRationalTangles1997} or
$+^\prime$\citep{kauffmanClassificationRationalKnots2002}. For
generic tangle $A$ and
$B$, $A\vee B$ is built analogously to the $+$ but stacking $A$ vertically on
top of $B$ instead of horizontally (Figure~\ref{fig-opo-vee}).

\begin{figure}[H]
\centering
\includegraphics[width=\linewidth]{files/fig-opo-vee-0108b868def1c0b26ad7e858e31be367.pdf}
\caption[The vertical sum of two generic tangles.]{The vertical sum
of two generic tangles.}
\label{fig-opo-vee}
\end{figure}

These two operations combined with parentheses as in
\nf{subsubsec-opo-precedence}
give a natural arithmetic structure to the combinations of tangles. We'll see in
later sections how this structure is easily encoded on a computer as a data
structure. We conclude the section by redefining the \textbf{algebraic tangles}.

\begin{definition}{Conway, Page
331\textbf{\citep{conwayEnumerationKnotsLinks1970}}}{def-algebraic}
Any tangle that can be produced by the two binary operations $+$ and $\vee$ on
the four basic tangles is called an \textbf{algebraic tangle}.

\end{definition}
%  prettier-ignore-start

\subsection{Integral Tangles}\label{subsec-integral_tangle}

%  prettier-ignore-end

We finish the chapter with a description of a class of tangles we first
encountered in \nf{subsubsec-opo-plus}. The integral tangles are the
simplest class
of tangle that are built from the basic operations on basic tangles. We start by
defining the horizontal integral tangles.

\begin{definition}{}{}A tangle built from the successive sum of $n +1$
tangles is called a
\textbf{horizontal integral} (or simply integral) $n$ tangle.
Similarly, a sum of $\m 1$ tangles is a
horizontal integral $\m n$ tangle.

\end{definition}It is convenient to notate the horizontal integral
tangles simply by their
corresponding integer, $\pm n$. A similar construction can be defined for the
$\vee$ operation, yielding the vertical integral tangles.

\begin{definition}{}{}A tangle built from the successive vertical sum of
$n+1$ tangles is called a
vertical integral $n$ tangle. Similarly, a vertical sum of $\m 1$ tangles is a
vertical integral $\m n$ tangle.

\end{definition}We will notate the vertical
tangles by $\pm\frac{1}{n}$.


\chapter{Tabulation}\label{ch-tabulation}

\section{The First Tangle Datasets}

This chapter will describe the methodology used to answer two of the essential
questions detailed in Section~\ref{sec-intro-overview}.

\begin{quote}
  ``How do I systematically construct rational tangles?'', ``How do I tell two
  rational tangles I make apart?'', and ``How do I generate new
  rational tangles?''
\end{quote}

\begin{quote}
  ``How do I systematically construct Montesinos tangles?'', ``How do I tell two
  Montesinos tangles I make apart?'', and ``How do I generate new Montesinos
  tangles?''
\end{quote}

The methodologies outlined in this section are for relatively simple classes of
tangles, the rational and Montesinos tangles. As we progress, we will see a
common solution pattern that outlines the general approach to the more difficult
algebraic/arborescent case in Section~\ref{sec-arborescent}. That
approach takes the form of
the cadence:

\begin{enumerate}
  \item Define the object.
  \item Define equivalence for the object.
  \item Identify a unique representation.
  \item Generate those unique representatives\footnote{Each tangle
      can be represented by an infinite number of diagrams. A unique
      representative in this context means a particular ``flavor'' of
      diagram that
    exists for every tangle in the class we are concerned about.}.
\end{enumerate}

%  prettier-ignore-start

\subsection{Rational Tangles}\label{sec-rational}

%  prettier-ignore-end

We will see in this section that \textbf{rational tangles},
originally described by
Conway \citep{conwayEnumerationKnotsLinks1970}, have a deeply
combinatorial nature,
making them among the simplest classes of tangle. This simplicity leads to the
rational tangles, and their knot closures, being some of the most commonly
studied objects in knot theory.

\subsubsection{Construction}

In our development of the modified tangle calculus
(Section~\ref{subsec-tangle_operations})
we described a way to glue tangles together, allowing us to take simple objects
and build complex objects. That approach forms the basis for our construction of
the rational tangles.

We start, with an intuitive description of the construction. Imagine a zero
tangle (Figure~\ref{prelim-fig-basic_0}), now attach to that tangle a
crank on the right
(east) side (Figure~\ref{fig-rat_tang-crank}). We allow, for a
moment, the fixed points of
the tangle to move. If we crank a half turn clockwise or anti-clockwise, we
introduce a twist, if we turn the crank $n$ half turns we make an integral
$n$ tangle
(Section~\ref{subsec-integral_tangle}).

\begin{figure}[H]
  \centering
  \includegraphics[width=0.5\linewidth]{files/fig-rat_tang-crank-8529a02a85dbb7212e706bdd68e2a9b4.pdf}
  \caption[A set of three turns, changing a basic 0 tangle into an
  integral 3 tangle.]{A set of three turns, changing a basic 0 tangle
  into an integral 3 tangle.}
  \label{fig-rat_tang-crank}
\end{figure}

When we have completed $n$ turns we take the crank and move it to the bottom
(south) side of the tangle. We turn the crank to add $n$ half turns, this time
building a vertical integral tangle. Continuing this, alternating between right
and bottom sets of half turns, as seen in
Figure~\ref{fig-rat_tang-crank_many}, we create a \textbf{rational
tangle}. For a rational tangle $T$, the list of counts for right and bottom
twists is the \textbf{twist vector}
(Definition~\ref{rational-def-twistvector}) of the rational
tangle.

\begin{figure}[H]
  \centering
  \includegraphics[width=0.5\linewidth]{files/fig-rat_tang-crank_m-bbfbb8ad5a58ffb705d4dab5ff1ebff7.pdf}
  \caption[The set of alternating turns building a rational
  tangle.]{The set of alternating turns building a rational tangle
  $\LB3\ 2\ 1\RB$.}
  \label{fig-rat_tang-crank_many}
\end{figure}

Formalizing this intuitive construction requires the $\vee$ and $+$ operations
(Section~\ref{subsec-tangle_operations}), as well as the integral tangles
(Section~\ref{subsec-integral_tangle}). This formalization was
originally stated by Conway
\citep{conwayEnumerationKnotsLinks1970} but was ultimately proved by Burde and
Zieschang, we will give a later construction by Kauffman and Goldman
\citep{goldmanRationalTangles1997}. For convenience, we denote a
horizontal integral
tangle with $a\in\Z$ crossings as $t_a$, similarly, a vertical integral tangle
with $b\in \Z$ crossings as $t_b^\prime$.

\begin{definition}{Kauffman and Goldman, Page 310
    \textbf{\citep{goldmanRationalTangles1997,
  conwayEnumerationKnotsLinks1970}}}{rational-def-rational}
  To build a \textbf{rational tangle}, take one of the following constructions:

  \begin{enumerate}
    \item Start with a horizontal tangle $t_{a_0}$. Add by $\vee$ a
      vertical tangle
      $t_{b_0}^{\prime}$ on the bottom, then add by $+$ a horizontal tangle
      $t_{a_1}$ on the right, then add by $\vee$ a $t_{b_1}^{\prime}$ on the
      bottom, and so on, stopping after a finite number of such steps at a
      horizontal tangle $t_{a_n}$.
    \item Start with a vertical tangle $t_{b_0}^\prime$. Add by $+$ a horizontal
      tangle $t_{a_0}$ on the right, then add by $\vee$ a vertical tangle
      $t_{b_1}^\prime$ on the bottom, then add by $+$ a $t_{a_1}$ on
      the right, and
      so on, stopping after a finite number of such steps at a
      horizontal tangle $t_{a_n}$.
  \end{enumerate}

\end{definition}

\begin{note}
  In the first case the twist vector has an odd number of entries and in the
  second the twist vector has an even number of entries.
\end{note}

To see the alignment between Definition~\ref{rational-def-rational}
and our intuitive
construction, view the turning of the crank as corresponding to horizontal and
vertical integral tangles, and the alternating of right and bottom as
corresponding to alternating $+$ and $\vee$.

\paragraph{Correspondence With Extended Rational Numbers}

Now we formally address our first two essential questions by:

\begin{enumerate}
  \item Describing a notation for rational tangles.
  \item Demonstrating a correspondence between the rational tangles
    and the extended rational
    numbers.
  \item Showing that the correspondence distinguishes (tells apart)
    rational tangles.
\end{enumerate}

The answer to our first essential question, ``How do I systematically construct
rational tangles?'', is seen by formalizing the twist vector notational strategy
we saw in our intuitive formulation for rational tangles. This allows us to
systematically write down a rational tangle by a list of integers.

\begin{definition}{Conway, Page 332
    \textbf{\citep{conwayEnumerationKnotsLinks1970,
  goldmanRationalTangles1997}}}{rational-def-twistvector}
  The list of integers of the sets of tangles $t_{a_i}$ and $t_{b_j}^\prime$
  from Definition~\ref{rational-def-rational} ordered as in
  Equation~(\ref{rational-def-math-tv}) or
  (\ref{rational-def-math-tv2}) is called a
  \textbf{twist vector}.

  \begin{equation}
    \label{rational-def-math-tv}
    \LB a_0\ b_0\ a_1\ b_1\ \cdots\ a_n\RB
  \end{equation}
  \begin{equation}
    \label{rational-def-math-tv2}
    \LB b_0\ a_0\ b_1\ \cdots\ a_n\RB
  \end{equation}

\end{definition}We are now ready to answer the second of the
essential questions, ``How do I tell
two rational tangles I make apart?'' The critical observation to answer the
questions is due to Conway's use \citep{conwayEnumerationKnotsLinks1970} of the
entries of a twist vector as the entries for a continued fraction
(Definition~\ref{rational-def-frac}). Since the twist vector is of
finite length, this continued
fraction corresponds to a rational number \citep{rockettContinuedFractions1998}.

\begin{definition}{Conway, Page 332
  \textbf{\citep{conwayEnumerationKnotsLinks1970}}}{rational-def-frac}
  For a rational tangle $T$ we call
  Equation~(\ref{rational-math-frac}) the \textbf{fraction of a
  rational tangle} and denote it $F\LP T\RP$.
  \begin{equation}
    \label{rational-math-frac}
    \frac{p}{q} = a_{n} + \cfrac{1}{b_n + \cfrac{1}{a_n + \cfrac{1}{b_{n-1} +
    \cfrac{1}{a_{n-1} + \cfrac{1}{\ddots\,+\cfrac{1}{a_0}}}}}}
  \end{equation}

\end{definition}

\begin{note}
  The correspondance with the extended rational numbers means that
  $F\LP\LB1\ \m 1\ 0\RB\RP=\frac{1}{0}$ corresponds to the basic $\infty$ tangle
  (Figure~\ref{prelim-fig-basic_nc-inf}).
\end{note}

Kauffman and Goldman \citep{goldmanRationalTangles1997} prove
(Theorem~\ref{rational-thm-conways})
that this correspondence distinguishes, tells apart, rational tangles. Meaning,
given two rational tangles, if their fractions are the same the tangles are
isotopic, and if the fractions are different, the tangles are not isotopic. This
answers the second of our essential questions.

\begin{theorem}{Conway's Theorem, Kauffman and Goldman page 315
    \textbf{\citep{conwayEnumerationKnotsLinks1970, burdeKnots2013,
  goldmanRationalTangles1997}}}{rational-thm-conways}
  Let $T_1$ and $T_2$ be rational tangles. $F\LP T_1\RP=F\LP T_2\RP$ if and only
  if $T_1$ is ambient isotopic to $T_2$.

\end{theorem}Observe, Theorem~\ref{rational-thm-conways} does not
discount the possibility of two
non-equivalent twist vectors, those differing in at least one entry,
representing the same rational tangle.
Example~\ref{rational-ex-tvssamebutdiff} demonstrates
that the possibility is true. In fact, for each rational tangle, the set of
twist vectors representing it is infinite.

\begin{example}{}{rational-ex-tvssamebutdiff}
  \begin{equation}
    \begin{aligned}
      F\LP\LB 1\ 7\ 0\ 1\RB \RP &= \frac{9}{1}\\
      F\LP\LB \m 3\ 1\ \ \m 1\ \m 1\ \m 1\ 1\ 9\RB \RP &= \frac{9}{1}\\
      F\LP\LB 1\ \m1\ 1\ \m1\ 7\ 0\ 3\RB \RP &= \frac{9}{1}\\
      F\LP\LB 1\ \m1\ \m1\ 1\ \m1\ 1\ \m1\ 1\ \m1\ 9\RB \RP &= \frac{9}{1}\\
      F\LP\LB 1\ 8\RB \RP &= \frac{9}{1}\\
      F\LP\LB 9\RB \RP &= \frac{9}{1}\\
    \end{aligned}
  \end{equation}

\end{example}To effectively answer our third essential question,
``How do I generate rational
tangles?'', we will need to determine a unique twist vector representative for
each rational tangle. A unique representative allows us to simply and
efficiently write down each tangle without risk of duplicates showing up on our
list.

\subsubsection{Canonical Twist Vectors}

Identifying a unique representative will stem from properties of finite
continued fractions. We start by defining a specific subclass of finite
continued fractions with integer coefficients, the \textbf{regular continued
fractions}. We frame the definition in the context of twist vectors.

\begin{definition}{Kauffman and Goldman, page
    318\textbf{\citep{kauffmanClassificationRationalKnots2002,
  rockettContinuedFractions1998}}}{}
  A continued fraction with integer coefficients $c_i$ is called a
  \textbf{regular
  continued fraction} if $c_i< 0$ for every coefficient or $0< c_i$ for
  every coefficient except the last which may be 0.

\end{definition}Conveniently, each rational number corresponds to
exactly two regular continued
fractions \citep{rockettContinuedFractions1998}. The first is a twist
vector with the
leftmost element greater than or equal to 2, and the second with leftmost
element equal to 1, per Equation~(\ref{rational-math-twotv}).
Observe, one of these
twist vectors has an even number of entries, and one has an odd number of
entries.

\begin{equation}
  \label{rational-math-twotv}
  \begin{aligned}
    F\LP\LB 9\RB \RP &= \frac{9}{1}= 8+\frac{1}{1} = F\LP\LB 1\ 8\RB \RP \\
    F\LP\LB 9\ 0\RB \RP &= 0+\frac{1}{9}=0+\frac{1}{8+\frac{1}{1}}=
    F\LP\LB 1\ 8\ 0\RB \RP \\
  \end{aligned}
\end{equation}

\begin{note}
  The fraction $\frac{1}{0}$ corresponds to the $\LB 0\ 0\RB$ twist vector.
\end{note}

To identify a unique representative for a rational number and hence rational
tangle, we will select, for convenience, the odd length twist vector as our
unique representative.

\begin{definition}{Kauffman and Lambropoulou, Page 13
  \textbf{\citep{kauffmanClassificationRationalKnots2002}}}{rational-def-canonrat}
  A twist vector is called a \textbf{canonical twist vector} if it contains
  coefficients of a regular continued fraction and is of odd length or is
  $\LB0\ 0\RB$.

\end{definition}
\subsubsection{Computational Methods}

Armed with a unique representative for a rational tangle, we can construct our
computational answer to the third essential question.

\paragraph{Notation}

We start by describing how we will digitally store a rational tangle. In the
rational tangle case, the theoretical encoding strategy of twist vectors happens
to be well suited for computational storage. A twist vector can be
computationally stored identically to its written form, a list of space
separated integers delimited by a pair of square braces, $\LB\ \RB$. As we will
see in Section~\ref{sec-arborescent} this direct translation of
theoretical notation for tangles to
computational notation is not always the case.

\paragraph{Generation}

A common tactic in the knot tabulation space is to pare down the number of items
that must be tabulated by leveraging symmetries of the objects being tabulated.
For example, the $\LB 3\RB$ and $\LB \m 3\RB$ tangles are related to each other
by a minus operation (Section~\ref{subsubsec-conway_minus}). Meaning,
if we tabulate
$\LB 3\RB$, we can recover $\LB \m 3\RB$ with multiplication by $\m 1$. This
fact holds for all rational tangles as demonstrated by
Lemma~\ref{rationa-lema-pmtv}.
Allowing us to focus our efforts on the rational tangle with twist vectors
containing only positive entries.

\begin{lemma}{Kauffman and Lambropoulou, Page 19
  \textbf{\citep{kauffmanClassificationRationalKnots2002}}}{rationa-lema-pmtv}
  For a tangle $T$ with negative $-T$ (Section~\ref{subsubsec-conway_minus}), if
  $F\LP T\RP=\frac{p}{q}$ then $F\LP \m T\RP=\m\frac{ p}{q}$.

\end{lemma}We now begin our development of a generation strategy for
twist canonical twist
vectors of rational tangles. The method seen here utilizes a
common combinatorial method for defining compositions of integers,
the same is used for
rational tangle counting by Bryhtan \citep{bryhtanTabulating2stringTangles2024}.
Consider, for a given crossing number $n$, what is the most ``obvious'' twist
vector? A viable candidate for most ``obvious'' is a twist vector, of the form
seen in Equation~(\ref{rational-math-1s}), dropping the trailing 0 where needed.

\begin{equation}
  \label{rational-math-1s}
  \LB 1\ 1\ 1\ \cdots\ 1\RB
\end{equation}

The 1's twist vector (\ref{rational-math-1s}) is an ideal starting
point for developing
a generation strategy, as it distributes the data of a rational tangle as
broadly as possible. Next, we consider how we might transform the 1's twist
vector into a twist vector with a two crossing integral components. We can do
this by exchanging the space between the first and second 1 with a numeric
$+$.

\begin{equation}
  \label{rational-math-1plus}
  \begin{aligned}
    &\LB 1\square 1\ 1\ \cdots\ 1\RB\\
    &\LB 1+1\ 1\ \cdots\ 1\RB\\
    &\LB 2\ 1\ \cdots\ 1\RB\\
  \end{aligned}
\end{equation}

This process tells us that we can utilize the exchange of whitespace of a twist
vector for $+$ to generate new twist vectors. To complete our generation, we
must generate every combination of exchanges.

The 1's twist vector with crossing number $n$ has $n$ 1s and $n -1$ spaces. In
each space position, we have the option between a space and $+$. To enumerate
all $2^{n -1}$ combinations, we can simply count, in binary, from 1 to
$2^{n -1}$, as in Example~\ref{rational-ex-counting}.

\begin{example}{Combinations of exchanges for $n=4$ and their twist
  vector }{rational-ex-counting}
  \begin{equation}
    \begin{aligned}
      000\to \square \square \square\to &\LB1\ 1\ 1\ 1\RB \\
      001\to \square \square +\to &\LB1\ 1\ 2\RB \\
      010\to \square + \square\to &\LB1\ 2\ 1\RB \\
      011\to \square + +\to &\LB1\ 3\RB \\
      100\to + \square \square\to &\LB2\ 1\ 1\RB \\
      101\to + \square +\to &\LB2\ 2\RB \\
      110\to + + \square\to &\LB3\ 1\RB \\
      111\to + + +\to &\LB4\RB \\
    \end{aligned}
  \end{equation}

\end{example}Our final refinement in this process it to transform
this collection into
canonical twist vectors. Half of the twist vectors, those of odd length,
generated in this process are already canonical. To canonize the even length
twist vectors, we append 0 to the right most position of each list, turning
the even vectors into an odd canonical vectors.

\begin{note}
  Appending 0 to the even twist vectors makes each of the fractions for these
  twist vectors sit in the unit interval, $\LB 0,1\RP$.
\end{note}

We conclude the section with a set of algorithms that describe a method for
computationally generating all rational tangles up to a given crossing number.

\begin{remark}{Find all rational tangles of crossing number
  \textbf{$n$} }{find-rat-tang-of-n}
  \textbf{Input}

  \begin{itemize}
    \item A crossing number $n$
  \end{itemize}

  \textbf{Output}

  \begin{itemize}
    \item All collection $T$ of twist vectors
  \end{itemize}

  \textbf{Routine}

  \begin{enumerate}
    \item Generate $O$ the 1's twist vector as in
      Equation~(\ref{rational-math-1s}) for $n$
    \item for $i=0$ to $2^{n -1}$
      \begin{enumerate}
        \item Transform $i$ into its binary representation
        \item Exchange as in Equation~(\ref{rational-math-1plus})
          digits of $i$ with spaces where $i$
          is $O$ and with $+$ where $i$ is 1
        \item Apply operations to the 1s vector and store the
          resulting vector as $O_r$
        \item If $O_r$ is odd length, store $O_r$ in $T$
        \item Else, append 0 to $O_r$ and store
      \end{enumerate}
  \end{enumerate}

\end{remark}
\begin{remark}{Find all rational tangles up to crossing number
  \textbf{$n$}}{find-rat-tang-to-n}
  \textbf{Input}

  \begin{itemize}
    \item A crossing number $n$
  \end{itemize}

  \textbf{Output}

  \begin{itemize}
    \item All collection $T$ of twist vectors
  \end{itemize}

  \textbf{Routine}

  \begin{enumerate}
    \item Store the twist vectors $\LB 0\RB,\LB 0\ 0\RB,\LB 1\RB$
    \item for $i=2$ to $n$
      \begin{enumerate}
        \item Execute Algorithm~\ref{find-rat-tang-of-n} with $i$
      \end{enumerate}
  \end{enumerate}

\end{remark}
%  prettier-ignore-start

\subsection{Montesinos Tangles}\label{sec-monttang}

%  prettier-ignore-end

In this section, we will use the rational tangles to build a yet more complex
class of tangles, the Montesinos tangles. This building up process demonstrates
one of the core strategies for tangle tabulation.

\subsubsection{Construction}

With the rational tangles in hand, we wish to utilize that data to enumerate
additional tangles. One way we have seen to build simple objects into complex
objects is to combine two tangles with the $+$ or $\vee$ operation. To keep
complexity under control, we start with combining tangles with repeated $+$ sum.
When all summands, $R_i$ in Equation~(\ref{mont-math-def}) are rational,
we call the result of the sum a \textbf{Montesinos Tangle}
\citep{ernstTANGLEEQUATIONS1996, bonahonNewGeometricSplittings2016}.

\begin{equation}
  \label{mont-math-def}
  R_0+R_1+\cdots+R_n
\end{equation}

\begin{note}
  Under this characterization of the Montesinos tangles, every rational tangle
  including the integral tangles are Montesinos tangles with a single summand.
\end{note}

\paragraph{Unique Representative}

Next, we develop a classification of Montesinos tangles, allowing us to tell two
Montesinos tangles apart. For each rational summand $R_i$ in a
Montesinos tangle, we have four possibilities:

\begin{enumerate}
  \item $R_i$'s canonical twist vector is positive and ends in 0
  \item $R_i$'s canonical twist vector is positive and does not end in 0
  \item $R_i$'s canonical twist vector is negative and ends in 0
  \item $R_i$'s canonical twist vector is negative and does not end in 0
\end{enumerate}

Observe that in the second and fourth cases, $R_i$ terminates in a horizontal
integral tangle. In these cases, the tangle can be simplified by using the flype
(Section~\ref{subsubsec-opo-flype}) to move the horizontal crossings
to be the right most
summand, seen in Figure~\ref{mont-ex-flypesimple}. When this process
is carried out for each
summand of the type in cases two and four, the resulting summands all fall into
cases one and three.

\begin{figure}[H]
  \centering
  \includegraphics[width=\linewidth]{files/fig-mont-fc6488a6dd1ca987118be5ace13f3e88.pdf}
  \caption[A Montesinos tangle simplification.]{A Montesinos tangle
    \\$\LB1\ 2\ 2\RB+\LB \m1\ \m2\ 0\RB+\LB \m3\ \m2\ \m1\RB+\LB
    1\ 2\ 0\RB+\LB1\ 2\ 2\RB$
    simplifying to \\$\LB 1\ 2\ 0\RB+\LB \m 1\ \m 2\ 0\RB+\LB \m3\ \m
  2\ 0\RB+\LB1\ 2\ 2\RB+\LB1\ 2\ 0\RB+\LB 3\RB$}
  \label{mont-ex-flypesimple}
\end{figure}

We will now pare down to a single possibility, case one. Consider a summand
$R_i$ in case three, meaning $\m 1 <F\LP R_i\RP<0$ except for $R_n$, which may
be integral. Theorem~\ref{rational-thm-conways} tells us that if we
can find an alternative,
potentially non-canonical, twist vector that fits our needs, we are free to
exchange without impacting topology. What we would like is an odd length twist
vector, where every entry is positive, except for the rightmost, which is a
negative value, see Figure~\ref{mont-ex-equ-replace}.

\begin{figure}[H]
  \centering
  \includegraphics[width=0.5\linewidth]{files/3_2_-1-6b0c825b6d0643ae9067e9762d6f498c.pdf}
  \caption[A transformation of a canonical rational tangle.]{On the
    left rational tangle $\LB \m1\ \m2\ \m1\ \m1\ 0\RB$ and on the right
    $\LB 3\ 2\ \m1\RB$. These tangles have fractions
    $F\LP\LB \m1\ \m2\ \m1\ \m1\ 0\RB\RP=\frac{\m4}{7}$ and
  $F\LP\LB 3\ 2\ \m1\RB\RP=\frac{\m4}{7}$ showing the tangles to be isotopic.}
  \label{mont-ex-equ-replace}
\end{figure}

This ensures that the fraction is still negative but will allow us to flype the
terminal horizontal integral tangle to the right. Rockett and Sz\"usz
give a lemma
that establishes the existence of such a twist vector for each rational number.

\begin{lemma}{Rockett and Sz\"usz, Page 3
  \textbf{\citep{rockettContinuedFractions1998}}}{mont-lem-other_cfrac}
  Every rational number has a continued fraction with positive integer entries
  except for the first (rightmost twist vector entry) which is an integer.

\end{lemma}Full details on how the conversion from any twist vector to this
twist vector can be found in Rockett and Sz\"usz
\textbf{\citep{rockettContinuedFractions1998}}.
We have now shown that each case (2,3,4) can be transformed into the first case,
and we can define a canonical form for Montesinos tangles.

\begin{theorem}{Bonahon and Siebenmann, Theorem 11.7
  \textbf{\citep{nakanoEfficientGenerationPlane2002}}}
  Every non-rational Montesinos tangle $T$ admits a canonical diagram satisfying
  the following construction:
  \begin{equation}
    T \cong R_0+\cdots+R_m+\frac{k}{1}
  \end{equation}
  where each $R_i \cong \frac{p_i}{q_i}$ is a rational subtangle in
  canonical form with
  fraction satisfying $0<\frac{p_i}{q_i}<1$, and $\frac{k}{1}$ is a horizontal
  integer subtangle.

\end{theorem}
\subsubsection{Computational Methods}

\paragraph{Notation}

Before we can generate Montesinos tangles, we need to define an efficient
notation for computation and storage. Similar to what we saw in the rational
tangle case, the theoretical notation for Montesinos tangles is sufficient for
computation. However, with eyes on future computational work, we will generalize
our notation to increase reusability and the efficiency of storage..

Montesinos tangles are simple forms of the algebraic tangles
(Definition~\ref{def-algebraic}),
so we will build a notational strategy for general algebraic tangles.
The strategy seen here is similar to those found in Conolly
\citep{connollyClassificationTabulation2string2021}, Caudron
\citep{caudron1982classification}, and Gren, Sulkowska, and,
Gabrov\v{s}ek \citep{gren2025classificationalgebraictangles}. The
theoretical notation for algebraic tangles is outlined in
Section~\ref{subsec-tangle_operations} and seen in
Figure~\ref{tangle-alg-tree-2}.
We can simplify the notation, without losing fidelity, by substituting the
integral leaf tangles for rational tangle twist vectors. Additionally, we can
improve the storage overhead by storing the tree as a string in Polish notation
\citep{lukasiewiczElementyLogikiMatematycznej1929}. Storing in Polish
notation allows
us to drop all the parentheses from our notation, saving two bytes in each
instance.

\begin{figure}[H]
  \centering
  \begin{subfigure}[b]{0.45\textwidth}
    \centering
    \includegraphics[width=\textwidth]{files/mont-81522111e034b6335aedb3c82a3272ea.pdf}
    \caption{}
    \label{tangle-alg-tree-1}
  \end{subfigure}
  ~
  \begin{subfigure}[b]{0.45\textwidth}
    \centering
    \includegraphics[width=\textwidth]{files/mermaid-253f34ae-3e58cd025723b9ae2cdb797305599ac0.png}
    \caption{}
    \label{tangle-alg-tree-2}
  \end{subfigure}
  \caption[Conversion of a tangle to a algebraic tangle tree.]{The
    tangle in \ref{tangle-alg-tree-1}:
    \\$\cdot$Algebraically:$\LB1\ 2\ 0\RB+\LP\LB2\ 1\ 0\RB+\LB2\ 2\ 0\RB\RP$
    \\$\cdot$In polish
    notation:$+\LB1\ 2\ 0\RB+\LB2\ 1\ 0\RB\LB2\ 2\ 0\RB$ $\cdot$\\ As an
  algebraic tangle tree in \ref{tangle-alg-tree-2}}
  \label{tangle-alg-tree}

\end{figure}
\paragraph{Generation}

In this section, we will design an algorithm that allows us to efficiently
generate new Montesinos tangles up to a given crossing number. As we saw, the
construction of a Montesinos tangle is based on the repeated summation of
rational tangles. Consequently, our generation strategy will utilize our table
of rational tangles.

To start, we need a mechanism that allows us to select all possible combinations
of rational tangles with crossing numbers that sum to our target. For a
Montesinos tangle $T$, the set of rational components $\LS R_i\RS_i^n$ combined
with the integral component $k$ corresponds to an ordered list of crossing
numbers as in Equation~(\ref{mont-math-orderedlist}).

\begin{equation}
  \label{mont-math-orderedlist}
  CN\LP R_0\RP,\ \cdots,\ CN\LP R_n\RP,\  CN\LP k\RP
\end{equation}

We call a list of the form seen in (\ref{mont-math-orderedlist}) a
\textbf{stencil} for a
Montesinos tangle. By construction, every canonical Montesinos tangle relates to
precisely one stencil. Observe, that $2\leq CN\LP R_i\RP$ since $\frac{1}{2}$ is
the lowest crossing number rational tangle with fraction in the unit interval.

Generation for all Montesinos tangles of a given crossing number at this point
can be broken down into two steps:

\begin{enumerate}
  \item Generate all stencils
  \item Fill in the stencils with all rational tangles of the
    appropriate crossing
    number whose fraction is in the unit interval (plus an integral
    tangle in the rightmost position).
\end{enumerate}

For the first step, we require a mechanism for breaking an integer into all
possible combinations where the parts sum to the integer. Luckily, we have
already seen how to do this, in the context of a twist vector. We follow the
same counting algorithm outlined in
Algorithm~\ref{find-rat-tang-of-n} however, we modify the
algorithm to keep both the even and odd outputs, but filter out stencils with
entries less than two.

\begin{example}{}{ex-mont-stencils}The set of all possible stencils
  for crossing number 5:

  \bigskip\noindent
  \begin{tabular}{p{\dimexpr 0.250\linewidth-2\tabcolsep}p{\dimexpr
      0.250\linewidth-2\tabcolsep}p{\dimexpr
    0.250\linewidth-2\tabcolsep}p{\dimexpr 0.250\linewidth-2\tabcolsep}}
    \toprule
    $1\ 1\ 1\ 1\ 1$ & $2\ 1\ 1\ 1$ & $1\ 2\ 1\ 1$ & $1\ 1\ 2\ 1$ \\
    $1\ 1\ 1\ 2$ & $3\ 1\ 1$ & $1\ 3\ 1$ & $1\ 1\ 3$ \\
    $2\ 2\ 1$ & $2\ 1\ 2$ & $1\ 2\ 2$ & $3\ 2$ \\
    $2\ 3$ & $4\ 1$ & $1\ 4$ & $5$ \\
    \bottomrule
  \end{tabular}

  \bigskip$\!\,$The set of stencils for crossing number 5:

  \bigskip\noindent
  \begin{tabular}{p{\dimexpr 0.500\linewidth-2\tabcolsep}p{\dimexpr
    0.500\linewidth-2\tabcolsep}}
    \toprule
    $3\ 2$ & $2\ 3$ \\
    \bottomrule
  \end{tabular}

  \bigskip
\end{example}For the second step, for each entry in the stencil we
create a list of rational tangles in the unit interval with that entry for a
crossing number. For the rightmost entry of the stencil we also include the
horizontal integral tangles with crossing number equal entry. This creates all
combinations of input tangles given by the stencil.

\begin{example}{}{ex-mont-stencil_insert}The set of rational tangles
  of crossing numbers two and three:

  \bigskip\noindent
  \begin{tabular}{p{\dimexpr 0.500\linewidth-2\tabcolsep}p{\dimexpr
    0.500\linewidth-2\tabcolsep}}
    \toprule
    Two & Three \\
    \hline
    $[1\ 1\ 0]$ & $[2\ 1\ 0]$ \\
    $[2]$ & $[1\ 2\ 0]$ \\
    & $[3]$ \\
    \bottomrule
  \end{tabular}

  \bigskip$\!\,$The set of stencils for crossing number five, with
  rational tangles inserted in polish notation:

  \bigskip\noindent
  \begin{tabular}{p{\dimexpr 0.500\linewidth-2\tabcolsep}p{\dimexpr
    0.500\linewidth-2\tabcolsep}}
    \toprule
    $3\ 2$ & $2\ 3$ \\
    \hline
    $+[2\ 1\ 0][1\ 1\ 0]$ & $+[1\ 1\ 0][2\ 1\ 0]$ \\
    $+[1\ 2\ 0][1\ 1\ 0]$ & $+[1\ 1\ 0][1\ 2\ 0]$ \\
    \bottomrule
  \end{tabular}

  \bigskip
\end{example}We conclude the section with a set of algorithms that
describe this method for
computationally generating all Montesinos tangles up to a given crossing number.
\newpage
\begin{remark}{Find all stencils of crossing number
  \textbf{$n$}}{find-mont-sten}
  \textbf{Input}

  \begin{itemize}
    \item A crossing number
  \end{itemize}

  \textbf{Output}

  \begin{itemize}
    \item All collection $S$ stencils
  \end{itemize}

  \textbf{Routine}

  \begin{enumerate}
    \item Generate $O$ the 1's twist vector as in
      Equation~(\ref{rational-math-1s}) for $n$
    \item for $i=0$ to $2^{n -1}$
      \begin{enumerate}
        \item Transform $i$ into its binary representation
        \item Exchange as in Equation~(\ref{rational-math-1plus})
          digits of $i$ with spaces where $i$
          is $O$ and with $+$ where $i$ is 1
        \item Apply operations to the 1s vector and store the
          resulting vector as $O_r$
        \item Continue to the next iteration if $O_r$ has entries less than 2
        \item Add $O_r$ to $S$
      \end{enumerate}
  \end{enumerate}

\end{remark}
\begin{remark}{Find all Montesinos tangles of crossing number
  \textbf{$n$}}{find-mont-tang-of-n}
  \textbf{Input}

  \begin{itemize}
    \item All rational tangles of crossing number up to $n$
  \end{itemize}

  \textbf{Output}

  \begin{itemize}
    \item All collection $T$ of Montesinos tangles
  \end{itemize}

  \textbf{Routine}

  \begin{enumerate}
    \item Execute Algorithm~\ref{find-mont-sten} for $n$ and store in $S$
    \item for each stencil $s$ in in $S$
      \begin{enumerate}
        \item Retrieve lists $L=\LS L_i\RS_i^n$ of rational tangles
          for each stencil
          entry $s_i$.
        \item Add to the $L_n$ list the integral tangle $s_n$
        \item While there is a list in $L$ that is not exhausted.
          \begin{enumerate}
            \item Construct and store a Montesinos tangle from list entries.
          \end{enumerate}
      \end{enumerate}
  \end{enumerate}

\end{remark}
\begin{remark}{Find all non-rational Montesinos tangles up to
  crossing number \textbf{$n$}}{find-mont-tang-to-n}
  \textbf{Input}

  \begin{itemize}
    \item A crossing number
    \item All rational tangles up to $n$
  \end{itemize}

  \textbf{Output}

  \begin{itemize}
    \item All collection $T$ of twist vectors
  \end{itemize}

  \textbf{Routine}

  \begin{enumerate}
    \item for $i=4$ to $n$
      \begin{enumerate}
        \item Execute Algorithm~\ref{find-mont-tang-of-n} with $i$
      \end{enumerate}
  \end{enumerate}

\end{remark}
%  prettier-ignore-start

\section{Arborescent Tangles}\label{sec-arborescent}

%  prettier-ignore-end

This section describes the methodology we use to answer the final of the
essential questions detailed in Section~\ref{sec-intro-overview}.

\begin{quote}
  ``How do I systematically construct algebraic/arborescent
  tangles?'', ``How do I
  tell two algebraic/arborescent tangles I make apart?'', and ``How many
  algebraic/arborescent tangles can I create?''
\end{quote}

In this thesis so far we have worked with the algebraic tangles
(Definition~\ref{def-algebraic})
constructed with Conway's tangle arithmetic
(Section~\ref{subsubsec-conway_calc}).
In this section we will leverage a slightly different, but
equivalent, construction
given by Bonahon and Siebenmann \citep{bonahonNewGeometricSplittings2016} the
\textbf{arborescent tangles}. The one-to-one correspondence between
the classes will
become clear as we introduce the construction for arborescent tangles.

This section starts with an overview of Bonahon and Siebenmann's
\citep{bonahonNewGeometricSplittings2016} definition of arborescent
knots and tangles
(Section~\ref{prelim-arbor_def}). We then give their smooth and
combinatorial constructions
of arborescent knots and tangles (Section~\ref{subsec-wptt}). Next,
we give original work
extending Bonahon and Siebenmann's canonical construction to a local view
(Section~\ref{sec-CWPTT-def}). This local view is leveraged to define
a unique representative
for each class of arborescent tangles (Section~\ref{subsec-rlitt}).
Finally, we will describe
an original algorithm and notation that directly enumerates those unique
representatives (Section~\ref{subsec-computation}).

%  prettier-ignore-start

\subsection{Definition of Arborescent}\label{prelim-arbor_def}

%  prettier-ignore-end

We now give a high-level description of the manifold theory underpinning the
theory of arborescent knots and tangles. A full treatment of the manifold theory
can be found in Bonahon and Siebenmann
\citep{bonahonNewGeometricSplittings2016}.
Our first concept is that
of a \textbf{knot pair}, which serves as the underlying structure for
all the smooth
objects in this subsection.

\begin{definition}{Bonahon and Siebenmann, Page 15
  \textbf{\citep{bonahonNewGeometricSplittings2016}}}{prelim-def-arborescent-knot-pair}
  A \textbf{knot pair} is a pair $(M, K)$ where $M$ is an oriented
  connected compact 3
  manifold with (possibly empty) boundary, and where $K$ is a proper
  1-dimensional
  submanifold of $M$.

\end{definition}
For example a tangle $\LP B^3,T\RP$ is a knot pair.
\begin{definition}{Bonahon and Siebenmann, Page vi
  \textbf{\citep{bonahonNewGeometricSplittings2016}}}{prelim-def-arborescent-knot}
  A knot $(S^3, K)$ is \textbf{arborescent} if there exists a finite collection
  $F_1,\dots,F_n$ of disjoint Conway spheres such that, if $N$ is the
  closure (in the sense of a closure of a set) of
  any component of $S^3 - \cup_{i=1}F_i$ , then the pair $(N, K \cap
  N )$ takes the
  simple form of Figure~\ref{fig-arborescent_part} after suitable isotopic
  deformation in $S^3$.

  \begin{figure}[H]
    \centering
    \includegraphics[width=0.5\linewidth]{files/arborescent_knot-48d6fedd7bda45f063e639fe06adfa3e.pdf}
    \caption[A collection of Conway circles forming a arborescent
    vignette.]{A collection of Conway circles forming what we call an
    \textbf{arborescent vignette}.}
    \label{fig-arborescent_part}
  \end{figure}

\end{definition}
\begin{figure}[H]
  \centering
  \includegraphics[width=0.5\linewidth]{files/arborescent_band-5577f7920d0716dd044364979702873e.pdf}
  \caption[The arborescent vignette.]{The arborescent vignette from
  Figure~\ref{fig-arborescent_part} seen with Conway spheres.}
  \label{fig-arborescent_band}
\end{figure}

\begin{figure}[H]
  \centering
  \includegraphics[width=0.5\linewidth]{files/vin_1-d2a610c6959ee59775ba4a238da5e94a.pdf}
  \caption[The arborescent vignette showing a 1 crossing tangle.]{The
  arborescent vignette showing a 1 crossing tangle to be arborescent.}
  \label{fig-arborescent_vignette_1}
\end{figure}

\begin{note}
  The $F_i$ in Definition~\ref{prelim-def-arborescent-knot} are
  disjoint but may sit inside
  each other. This means we may have arborescent vignettes containing
  arborescent
  vignettes, but the closure of each individual component looks like
  Figure~\ref{fig-arborescent_part}.
\end{note}

Observe that arborescent knots are characterized by a collection of Conway
spheres (circles). Choosing to not fill one, or more, of these Conway spheres
yields a tangle.

\begin{definition}{Bonahon and Siebenmann, Page 144
  \textbf{\citep{bonahonNewGeometricSplittings2016}}}{intro-def-arbor-tangle}
  Define an \textbf{arborescent tangle} as one whose underlying knot
  pair $(M, K)$ (Definition~\ref{prelim-def-arborescent-knot-pair}) is
  arborescent in the sense defined in
  Definition~\ref{prelim-def-arborescent-knot}.

\end{definition}We see from the above the first portion of the
correspondence between the
algebraic and arborescent tangles. Each algebraic tangle can be naturally
decomposed into a collection of nested arborescent knot vignettes given by its
operations $+$ and $\vee$.

%  prettier-ignore-start

\subsection{Weighted Planar Trees}\label{subsec-wptt}

%  prettier-ignore-end

%  prettier-ignore-start

%  prettier-ignore-end

%  prettier-ignore-start

\subsubsection{Construction of Arborescent Knots from Weighted Planar
Trees}\label{construction_of_arbor}

%  prettier-ignore-end

This subsection begins with an introduction to Bonahon and Siebenmann's
\citep{bonahonNewGeometricSplittings2016} construction of arborescent knots and
tangles in the smooth setting. This is followed by the development of a
combinatorial representation for arborescent knots and tangles
\citep{bonahonNewGeometricSplittings2016}. We deviate slightly from Bonahon and
Siebenmann's introduction but ultimately arrive at the same structure. In our
introduction we develop partial solutions, then progressively modify those
partial solutions until they fit our needs. Next, we describe Bonahon and
Siebenmann's \citep{bonahonNewGeometricSplittings2016} operations on the
combinatorial structure, which allow us to systematically modify the structure,
without changing the topology. This subsection finishes with the classification
of arborescent knots and tangles given by Bonahon and Siebenmann
\citep{bonahonNewGeometricSplittings2016} as well as our extension
from a global to
local viewpoint.

\paragraph{Bands and Plumbing Squares}

Our first step in describing a notation for the arborescent knots
\citep{bonahonNewGeometricSplittings2016} is describing a plumbing operation on
bands. A band with a plumbing square is a band $S^1\times\LB 0,1\RB$, along with
an oriented square on the band such that two of the sides of the square
intersect the boundary of the band. Two examples of bands with plumbing squares
can be seen in Figure~\ref{wpt-construc-fig-band_sum}.

\begin{figure}[H]
  \centering
  \begin{subfigure}[b]{.45\textwidth}
    \centering
    \includegraphics[width=\textwidth]{files/bnd_sum_1-3b7426920de1e4efba9af1d3a5aba180.pdf}
    \caption[A band with a plumbing square facing the viewer.]{A band
    with a plumbing square facing the viewer.}
    \label{wpt-construc-fig-band_sum-1}
  \end{subfigure}
  ~
  \begin{subfigure}[b]{.45\textwidth}
    \centering
    \includegraphics[width=\textwidth]{files/bnd_sum_2-0f8a5a0218c04208316206db8595abf8.pdf}
    \caption[A band with the plumping square facing away from the
    viewer.]{A band with the plumping square facing away from the
    viewer. We are looking through the band.}
    \label{wpt-construc-fig-band_sum-2}
  \end{subfigure}
  \caption[Plumbing squares of bands.]{}
  \label{wpt-construc-fig-band_sum}
\end{figure}\paragraph{Plumbing bands}

We now glue the bands seen in Figure~\ref{wpt-construc-fig-band_sum}
together with an
operation called \textbf{plumbing}. Consider the orientation given in the green
band's plumbing square. We will call the blue arrow $X$ and the thicker red
arrow $Y$; similarly for the blue band with $X^\prime$ and $Y^\prime$. We
\textbf{plumb} the bands together along their plumbing squares, with
the requirement
that the orientation labels are mapped $X\to Y^\prime$ and $Y\to X^\prime$.
Finally, we forget the boundaries of the plumbing squares, leaving only the
joined boundaries of the bands. The result of plumbing as well as a local
picture for plumbing can be seen in Figure~\ref{wpt-construc-fig-band_sum_opo}.

\begin{figure}[H]
  \centering
  \begin{subfigure}[b]{0.45\textwidth}
    \centering
    \includegraphics[width=\textwidth]{files/bnd_sum_sum-5039ec3a2186e40e4a34a24bb86ae52c.pdf}
    \caption{}
    \label{wpt-construc-fig-band_sum_opo-1}
  \end{subfigure}
  ~
  \begin{subfigure}[b]{0.45\textwidth}
    \centering
    \includegraphics[width=\textwidth]{files/bnd_sum_patch-4165c83cc3f37aafe79d4b947ad0b900.pdf}
    \caption{}
    \label{wpt-construc-fig-band_sum_opo-2}
  \end{subfigure}
  \caption[Two bands plumbed.]{Two bands plumbed.}
  \label{wpt-construc-fig-band_sum_opo}
\end{figure}Our plumbing band construction can be turned into a knot,
by adding a series of
half-twists to our plumbing bands (Figure~\ref{wpt-construc-fig-6} and
Figure~\ref{wpt-construc-fig-25}). When forming the half twists, we
have two options for
direction relative to the band, we call one positive (left handed twists) and
one negative (right handed twists). We note that the twists appear in unique
regions of the band, determined by their position relative to the plumbing
squares.

\begin{figure}[H]
  \centering
  \begin{subfigure}[b]{0.45\textwidth}
    \centering
    \includegraphics[width=\textwidth]{files/arbor_band_with_twis-ae3d0a74a12df52afd907b081cba1c07.pdf}
    \caption[Band with two negative half twists.]{Band with two
      negative half twists\newline
    and three plumbing squares.}
    \label{wpt-construc-fig-6}
  \end{subfigure}
  ~
  \begin{subfigure}[b]{0.45\textwidth}
    \centering
    \includegraphics[width=\textwidth]{files/arbor_band_with_twis-5c07bffa285ece2cd9b0d471ca67f520.pdf}
    \caption[Band with three positive half twists.]{Band with three
      positive half twists \newline
    and one plumbing square.}
    \label{wpt-construc-fig-25}
  \end{subfigure}
  \caption[Plumbing bands with twists.]{}
\end{figure}Successive plumbing yields collections of bands like those seen in
Figure~\ref{wpt-construc-fig-10}. Finally, turning
Figure~\ref{wpt-construc-fig-10} into a knot is as
simple as removing the interior of the bands, leaving only the boundary, per
Figure~\ref{wpt-construc-fig-24}.
\begin{figure}[H]
  \centering
  \begin{subfigure}[b]{0.45\textwidth}
    \centering
    \includegraphics[width=\textwidth]{files/arbor_bands-d47febe50384b73ef69d7c9b9eb15c3f.pdf}
    \caption{}

    \label{wpt-construc-fig-10}
  \end{subfigure}
  ~
  \begin{subfigure}[b]{0.45\textwidth}
    \centering
    \includegraphics[width=\linewidth]{files/arbor_bound-a3d0541c15268c7cee37d83234237561.pdf}
    \caption{}

    \label{wpt-construc-fig-24}
  \end{subfigure}
  \caption[Plumbing and arborescent knots.]{A set of plumbed bands in
    \ref{wpt-construc-fig-10}
  and arborescent knot in \ref{wpt-construc-fig-24}}

\end{figure}
It is important to note that, for creating arborescent knots, we must restrict
plumbing from creating ``cycles'' of bands. That is a chain of
plumbing beginning
and ending with the same band, as seen in
Figure~\ref{wpt-construc-fig-cycle}. If we allow
cycles in the bands, we may create a polygonal tangle, defined in
Section~\ref{subsubsec-opo-insert}. These polygonal tangles contain
portions that do not
satisfy Definition~\ref{intro-def-arbor-tangle}, so are not arborescent.

\begin{figure}[H]
  \centering
  \includegraphics[width=0.5\linewidth]{files/band_cycle-7c0a613101506fe84b3e2dd906bec769.pdf}
  \caption[A collection of bands plumbed into a cycle.]{A collection
    of bands plumbed in such a way that the last band is plumbed to
  the first band in a cycle.}
  \label{wpt-construc-fig-cycle}
\end{figure}

We now establish some language for describing the relative positions of bands.
This language will be reused when we transition to the combinatorial setting,
and is widely used in graph theoretic settings.

\begin{definition}{}{wpt-construc-def-relationships_of_bands}
  Given a band $B$ with plumbing squares, we call the set $C$ of bands
  plumbed to $B$ the \textbf{children} of $B$. Additionally, for
  $c\in C$ we call $B$
  the \textbf{parent} of $c$ and the collection of $C -\LS c \RS$ the
  \textbf{siblings of $c$}.

\end{definition}We claim the plumbing band construction is in
correspondence with the definition
of arborescent seen in Definition~\ref{prelim-def-arborescent-knot}.
To see this, we take each
plumbing band and encapsulate it in a $S^2$ so that the corners of the plumbing
squares lie on the $S^2$, giving us the vignette seen in
Figure~\ref{fig-arborescent_band}.

\paragraph{Weighted Planar Trees}

The band construction we have developed for arborescent knots, as it stands, is
completely unsuited for machine computation. In this subsection, we lay out a
line of reasoning leading to a combinatorial encoding strategy
\citep{bonahonNewGeometricSplittings2016} for the plumbing band construction of
arborescent knots and tangles. The line of reasoning starts by presenting the
required data of arborescent knots and tangles that any combinatorial
representation must encode. We then propose partial solutions, each
progressively closer to the full encoding described by Bonahon and Siebenmann
\citep{bonahonNewGeometricSplittings2016}. As we will see, the encoding strategy
ultimately takes the form of a modified rooted plane tree, a specialized flavor
of graph theoretic tree.

Inventing a combinatorial encoding strategy means we first have to identify the
essential information that is needed to construct arborescent knots and tangles
from band plumbings. The two essential pieces of information that must be
encoded by any combinatorial strategy for notating plumbing of bands are the
following:

\begin{enumerate}
  \item The parent child relationship between bands
  \item The twists on bands, and their positions relative to the band's children
    (plumbing squares)
\end{enumerate}

Explicit details expanding on why these two pieces of information are essential
can be found in Bonahon and Siebenmann
\citep{bonahonNewGeometricSplittings2016}. We
will see in the following subsections ways in which these data are essential,
albeit in specialized cases.

Consider the first piece of essential information our combinatorial strategy
must encode, the parent child relationships between bands. Perhaps the most
commonly computationally utilized structure that encodes relational data is an
abstract graph. We imagine how an abstract graph might be used for encoding the
relationship of bands. One solution is to map bands to vertices and plumbing
relationships to edges. In the discussion of the band construction, we
restricted plumbing from forming cycles
(Figure~\ref{wpt-construc-fig-cycle}). A result of
this restriction is that any abstract graph must also have no cycles, meaning
all the graphs we will work with, abstract or otherwise, are actually trees. We
will call the data of a vertex and a collection of \textbf{bonds} (half-edges)
associated with plumbing squares the \textbf{local picture around a vertex}.

We have partially completed our goal of encoding the essential information of
band plumbing in a combinatorial object. Unfortunately, an abstract tree doesn't
encode all the essential data. Particularly, our second piece of information,
the positions of children and weights, is not easily seen in an abstract tree.
To solve this problem, we will instead use a modified version of an abstract
tree, a \textbf{rooted plane tree}, for our encoding.

\begin{definition}{}{wpt-construc-def-rooted_plane_tree}A
  \textbf{rooted plane tree} is an abstract tree imbued with a strict
  total order,
  indexed by the non-negative integers, on the vertices. We call the least
  vertex the \textbf{root} of the tree.

\end{definition}In a rooted plane tree $\Gamma$, at each vertex $v$,
the children of $v$ have an
ordering inherited from the total order of $\Gamma$, we call this ordering of
the children the \textbf{cyclic order} of the children. The cyclic
order gives us two
natural ways to draw $v$ and its children in the plane. We may choose to draw
the children anti-clockwise in one of increasing or decreasing order of index.
The realization of these two options can be seen in
Figure~\ref{wpt-construc-fig-order_tree}. A universal choice of
increasing or decreasing
yields a unique realization of a rooted plane tree in the plane.

\begin{figure}[H]
  \centering
  \begin{subfigure}[b]{0.45\textwidth}
    \centering
    \includegraphics[width=\textwidth]{files/arbor_graph_split_lo-45ac2a5c77ff7ba4ddbfbc87daddc9be.pdf}
    \caption[The local picture of a vertex in anti-clockwise
    order.]{The local picture of a vertex with child labels
    increasing in anti-clockwise order.}
    \label{wpt-construc-fig-order_1}
  \end{subfigure}
  ~
  \begin{subfigure}[b]{0.45\textwidth}
    \centering
    \includegraphics[width=\textwidth]{files/arbor_graph_split_lo-2249954de7fa775addf0b3e30c964706.pdf}
    \caption[The local picture of a vertex in anti-clockwise
    order.]{The local picture of a vertex with child labels
    decreasing in anti-clockwise order.}
    \label{wpt-construc-fig-order_2}
  \end{subfigure}
  \caption[Local pictures of a vertex of a tree. ]{}
  \label{wpt-construc-fig-order_tree}
\end{figure}
%  prettier-ignore-start

\subparagraph{Indexing the total order of a tree}\label{indexing-rpt}

%  prettier-ignore-end

We will now describe an \textbf{ideal indexing} for a rooted plane tree.

\begin{definition}{}{}Let $\Gamma$ be a rooted plane tree $r$, be the
  root of $\Gamma$. Additionally,
  let $v_i\neq r$ be a vertex with parent $v_p$ and children
  $v_{c_1},\,\cdots,\,v_{c_n}$. We call the indexing of a rooted
  plane \textbf{depth first} if
  it satisfies the following:

  \begin{itemize}
    \item $r$ is index 0
    \item $p<i<c_1<\cdots< c_n$
  \end{itemize}

\end{definition}

\begin{note}
  Two commonly seen orderings of a tree are the breadth and depth
  first orderings, both orderings
  are ideal orderings. For our purposes we will prefer the depth first ordering.
\end{note}

\begin{figure}[H]
  \centering
  \begin{subfigure}[b]{0.45\textwidth}
    \centering
    \includegraphics[width=\textwidth]{files/rpt_order-8d066d4b0623b0d11cb8e5223597ccfb.pdf}
    \caption[A rooted plane tree with ideal indexing.]{A rooted plane
      tree with ideal indexing. The index of each vertex is seen inside
    the vertex.}
    \label{wpt-construc-fig-rptorder_1}
  \end{subfigure}
  ~
  \begin{subfigure}[b]{0.45\textwidth}
    \centering
    \includegraphics[width=\textwidth]{files/rpt_order_ni-05d87f91294bbb47d9e46fbf563a337b.pdf}
    \caption[A rooted plane tree with indexing that is not depth
    first.]{A rooted plane tree with indexing that is not depth
    first. The index of is each vertex seen inside the vertex.}
    \label{wpt-construc-fig-rptorder_2}
  \end{subfigure}
  \caption[Indexing strategies of a rooted plane tree. ]{}
  \label{wpt-construc-fig-order}
\end{figure}

\begin{convention}
  For the remainder of this thesis we will adopt some conventions for
  rooted plane
  trees (and their derivatives the weighted planar tangle trees, CWPTT, and
  RLITT). When realizing a tree in the plane we select the universal
  anti-clockwise increasing
  order and assume that the tree has depth first indexing.
\end{convention}

The final data we need to record is the position and count of half twists
relative to plumbing squares. We observed earlier that the half twists on a band
must lie in a unique region determined by position relative to plumbing squares
on a band. This relationship can be recreated for a rooted plane tree by
annotating the local view of a vertex with an integer placed in the regions
between bonds. The relationship between a plumbing band and a weighted vertex in
a rooted plane tree can be seen in Figure~\ref{wpt-construc-fig-7}.
The weights placed in
regions between bonds inherit a cyclic order from the cyclic order of the bonds.
Each weight falls in the region between two bonds. We assign to each weight
(including zero weights) the lower of the two indices. This aligns with
assigning to the weight the index that appears before it in the anti-clockwise
planar realization of the cyclic order, per
Figure~\ref{wpt-construc-fig-weights-with-index}.

\begin{figure}[H]
  \centering
  \begin{subfigure}[b]{0.45\textwidth}
    \centering
    \includegraphics[width=\textwidth]{files/arbor_graph_split_lo-d4ac908e2b012b3aa3b53562353c28a8.pdf}
    \caption[The local view of a vertex with the weights $-2$, zero,
    and zero.]{The local view of a vertex with the weights $-2$,
    zero, and zero.}
    \label{wpt-construc-fig-7}
  \end{subfigure}
  ~
  \begin{subfigure}[b]{0.45\textwidth}
    \centering
    \includegraphics[width=\textwidth]{files/arbor_graph_split_lo-9fccfef14ced7a247f4d9f2d0b7199f1.pdf}
    \caption[The local view of a vertex with weights.]{The local view
      of a vertex with weight. \newline
      Notice the index of the weights come
      from the bond ``before'' it in the planar realized cyclic order
    given by convention.}
    \label{wpt-construc-fig-weights-with-index}
  \end{subfigure}
  \caption[Local view of a vertex with weights. ]{}
  \label{wpt-construc-fig-order_weights}
\end{figure}We can see a full example of a tree with its associated
plumbed construction in
Figure~\ref{wpt-construc-fig-27}. We call this fully realized
combinatorial recipe for an
arborescent knot a \textbf{weighted planar tree}.

\begin{definition}{Bonahon and Siebenmann, Page
  143\textbf{\citep{bonahonNewGeometricSplittings2016}}}{}
  A rooted plane tree $\Gamma$ augmented with weights is called a
  \textbf{weighted planar tree}.

\end{definition}
\begin{figure}[H]
  \centering
  \begin{subfigure}[b]{0.45\textwidth}
    \centering
    \includegraphics[width=\textwidth]{files/arbor_graph-f540218bf98a387a4e4b547a93445b49.pdf}
    \caption[The tree describing the plumbing of bands.]{The tree
      describing the plumbing of bands. Each vertex represents the band
    illustrated near it.}
    \label{wpt-construc-fig-27}
  \end{subfigure}
  ~
  \begin{subfigure}[b]{0.45\textwidth}
    \centering
    \includegraphics[width=\textwidth]{files/arbor_bands-d47febe50384b73ef69d7c9b9eb15c3f.pdf}
    \caption[The realization by plumbing bands of a tree.]{The
      realization by plumbing bands of the tree in
    Figure~\ref{wpt-construc-fig-27}}
    \label{wpt-construc-fig-28}
  \end{subfigure}
  \caption[Realization of plumbing of a tree.]{}
\end{figure}
%  prettier-ignore-start

\paragraph{Weighted Planar Tangle Trees}\label{wpt-construc-sec-wptt}

%  prettier-ignore-end

Our construction to this point has been concerned with the notation for knots
and links. We now give a modification of this notation for tangles. A weighted
planar tree, as in Figure~\ref{wpt-construc-fig-29}, can be modified
to represent a tangle
by allowing a \textbf{free bond} (half-edge), to be attached to a
vertex, that is, to
allow bands to have a non-plumbed plumbing square. We realize the non-plumbed
square, as a Conway circle for a two string tangle as in
Figure~\ref{wpt-construc-fig-tangle_trad}. To consistently orient the
Conway sphere's interior, we align the north boundary points of the Conway
sphere with the top (up in the band orientation) boundary component, and place
the NW corner first (following the orientation of the boundary), per
Figure~\ref{wpt-construc-fig-band_orientation}. Plumbing two bands
then corresponds to the
action of gluing a pair of tangles together on their Conway spheres so that
boundary points align.

A tree may have many free bonds, with each free bond representing a unique
boundary Conway sphere. Each boundary component serves as a location where a
tangle can be inserted to form a knot or link. For our efforts in enumerating
two string tangles, we restrict our focus to trees that have a single free bond.
In tangle trees with a single free bond, we designate the vertex with the free
bond as the root of the tree.

\begin{figure}[H]
  \centering
  \begin{subfigure}[b]{0.45\textwidth}
    \centering
    \includegraphics[width=\textwidth]{files/arbor_tangle-14a418154dfaf35f146b52dfa168bd93.pdf}
    \caption[The plumbing realization of an arborescent tangle.]{The
    plumbing realization of an arborescent tangle.}
    \label{wpt-construc-fig-29}
  \end{subfigure}
  ~
  \begin{subfigure}[b]{0.45\textwidth}
    \centering
    \includegraphics[width=0.7\textwidth]{files/example_tangle-4c128833ef1611bc12dcf29876e1e528.pdf}
    \caption[Realizing bands as a tangle.]{With an isotopy of the
      tangle and inversion of the Conway circle given by the
      non-plumbed square we have the realization of
      Figure~\ref{wpt-construc-fig-29} as a
    traditional orthogonally projected tangle.}
    \label{wpt-construc-fig-tangle_trad}
  \end{subfigure}
  \caption[Plumbing bands as a tangle.]{}
\end{figure}

\begin{figure}[H]
  \centering
  \includegraphics[width=.5\textwidth]{files/bnd_with_orientation-a963ba0fe226248533f2dad3ada2982d.pdf}
  \caption[The orientation of a Conway sphere.]{The orientation of a
    Conway sphere given by a plumbing square on a band of an
    arborescent tangle. The orientation of the underlying plumbing
    square is shown.
    This aligns with a left hand rule with $Y$ the thumb, $X$ the
    index finger, and
    $Z$ middle finger, with $Z$ pointing away from the center of the band,
  out of the page in this case. }
  \label{wpt-construc-fig-band_orientation}
\end{figure}

We will see that keeping track of the
location of the fixed points of the boundary sphere is important when
determining tangle equivalence. This is due to the need to maintain the rational
number (Definition~\ref{rational-def-frac}) associated with the
``rational tangle'' subtangles of
a tree, prompting us to assign rotation information to the free bonds. This
information takes the form of labels from the members of $V_4$ of the Klein
four-group $\iota,\xi,\zeta,\eta$. Each of these labels corresponds to a
rotation of the Conway sphere around an axis in $\R^3$, as seen in
Figure~\ref{wpt-construc-fig-v_4rotations} and
Figure~\ref{wpt-construc-fig-k4g}. Full details for the
manifold theory underpinning these markings are found in Bonahon and Siebenmann
\citep{bonahonNewGeometricSplittings2016}. We call such a labeled
tree a \textbf{weighted
planar tangle tree}.

\begin{figure}[H]
  \centering
  \begin{subfigure}[b]{0.2\textwidth}
    \centering
    \includegraphics[width=\textwidth]{files/v4_rotations_i-2a340230132ad100c9adc5eb2529d02f.pdf}
    \caption{}
    \label{wpt-construc-fig-k4g-rotationsi}
  \end{subfigure}
  \quad
  \quad
  \quad
  \begin{subfigure}[b]{0.4\textwidth}
    \centering
    \includegraphics[width=\textwidth]{files/v4_rotations-ac373932ba90a3a5b91b87418d43ff12.pdf}
    \caption{}
    \label{wpt-construc-fig-k4g-rotations}
  \end{subfigure}
  \caption[The effect of the $V_4$ rotations.]{The identity rotation
  (no rotation), and the effect of the $V_4$ rotations on each of the}
  \label{wpt-construc-fig-v_4rotations}
\end{figure}

\begin{definition}{Bonahon and Siebenmann, Page 165
  \textbf{\citep{bonahonNewGeometricSplittings2016}}}{}
  A weighted planar tree $\Gamma$ with free bonds labeled in $V_4$ is called a
  \textbf{weighted planar tangle tree (WPTT)}.

\end{definition}

\begin{figure}[H]
  \centering
  \begin{subfigure}[b]{0.4\textwidth}
    \centering
    \includegraphics[width=.5\textwidth]{files/iota-7ccd6540b4630d7bc532763aec0fa7fb.pdf}
    \caption[$\iota$ for no rotation.]{$\iota$ for no rotation}
    \label{wpt-construc-fig-k4g-i}
  \end{subfigure}
  \hfill
  \begin{subfigure}[b]{0.4\textwidth}
    \centering
    \includegraphics[width=.5\textwidth]{files/zeta-ccf154293b27a3af9396774196e3df8a.pdf}
    \caption[$\xi$ rotates around the $x$-axis.]{$\xi$ rotates around
    the $x$-axis}
    \label{wpt-construc-fig-k4g-x}
  \end{subfigure}
  \newline
  \centering
  \begin{subfigure}[b]{0.4\textwidth}
    \centering
    \includegraphics[width=.5\textwidth]{files/eta-8a2277d5eff21840ebb2ca47ad98f9ae.pdf}
    \caption[$\eta$ rotates around the $y$-axis.]{$\eta$ rotates
    around the $y$-axis}
    \label{wpt-construc-fig-k4g-y}
  \end{subfigure}
  \hfill
  \begin{subfigure}[b]{0.4\textwidth}
    \centering
    \includegraphics[width=.5\textwidth]{files/xi-1d75e1fe4635234febc0e30976942994.pdf}
    \caption[$\zeta$ rotates around the $z$-axis.]{$\zeta$ rotates
    around the $z$-axis}
    \label{wpt-construc-fig-k4g-z}
  \end{subfigure}
  \caption[Roations of a tangle. ]{}
  \label{wpt-construc-fig-k4g}
\end{figure}
%  prettier-ignore-start

\subsubsection{Anatomy of a tree}\label{wpt-construc-sec-subtrees}

%  prettier-ignore-end

In this subsection, we will describe several portions of weighted planar trees:
the ring subtree, essential vertex, and the sticks of a tree.

%  prettier-ignore-start

\paragraph{Ring subtree}\label{wpt-construc-sec-rings}

%  prettier-ignore-end

We will now describe the ring subtrees of a weighted planar tree, which locally
appear as Figure~\ref{wpt-construc-fig-17}.

\begin{figure}[H]
  \centering
  \includegraphics[width=0.5\linewidth]{files/arbor_graph_ring-d80406b3ccc9a75cbacd2e8eecfa706d.pdf}
  \caption[Positive and negative ring subtrees.]{Positive and
  negative ring subtrees}
  \label{wpt-construc-fig-17}
\end{figure}

Now, resolving the plumbing for the positive subtree, we arrive at bands as in
Figure~\ref{wpt-construc-fig-12}.

\begin{figure}[H]
  \centering
  \includegraphics[width=0.5\linewidth]{files/arbor_ring-f1ab6236c3e07c2dbd1fd86733fbfa89.pdf}
  \caption[Plumbed ring bands.]{Plumbed ring bands}
  \label{wpt-construc-fig-12}
\end{figure}

Notice that the boundary of these plumbed bands has three components, as seen in
Figure~\ref{wpt-construc-fig-13}.

\begin{figure}[H]
  \centering
  \includegraphics[width=0.5\linewidth]{files/arbor_ring_no_bnd-571778a714009a521aa0c17358a4d65e.pdf}
  \caption[Ring boundary.]{Ring boundary}
  \label{wpt-construc-fig-13}
\end{figure}

With an isotopy of the tangle and inversion of the Conway circle given by the
non-plumbed square, we can arrange our plumbed bands into the standard tangle
projection seen in Figure~\ref{wpt-construc-fig-14}. This tangle
projection tells us that
the subtree in Figure~\ref{wpt-construc-fig-12} is, depending on
location of $\text{NW}$,
either the zero or infinity tangle with a ring.

\begin{figure}[H]
  \centering
  \includegraphics[width=0.3\linewidth]{files/arbor_ring_tangle-a2936a53323b2187ce9cd2a9209f32db.pdf}
  \caption[Ring Tangle.]{Ring Tangle}
  \label{wpt-construc-fig-14}
\end{figure}

%  prettier-ignore-start

\paragraph{Essential Vertex}\label{wpt-construc-sec-essential_verts}

%  prettier-ignore-end

We now classify each vertex into one of two classes, the essential vertices and
the non-essential vertices.

\begin{definition}{Bonahon and Siebenmann, Page 159
  \textbf{\citep{bonahonNewGeometricSplittings2016}}}{apn-def-2}
  We define an \textbf{essential vertex} as any vertex with valence
  (count of the number of bonds)
  greater than 3.

\end{definition}
\begin{definition}{Bonahon and Siebenmann, Page 159
  \textbf{\citep{bonahonNewGeometricSplittings2016}}}{apn-def-3}
  A vertex is called non-essential if it has valence (count of the
  number of bonds) $0,1,2$.

\end{definition}As an example, consider the vertices seen in
Figure~\ref{wpt-construc-fig-18}.

\begin{figure}[H]
  \centering
  \includegraphics[width=\linewidth]{files/arbor_ring_essential-dc5f17b0ae16077f7beced9f4b581de3.pdf}
  \caption[A weighted planar tangle tree annotated with essential
  vertices.]{A weighted planar tree annotated with essential vertices
    in orange and
  non-essential in blue}
  \label{wpt-construc-fig-18}
\end{figure}

%  prettier-ignore-start

\paragraph{Sticks of a Tree}\label{wpt-construc-sec-sticks}

%  prettier-ignore-end

The final part of the anatomy of a tree we will consider is the
\textbf{sticks} of a
tree.

\begin{definition}{Bonahon and Siebenmann, Page 159
  \textbf{\citep{bonahonNewGeometricSplittings2016}}}{wpt-construc-def-sticks-of-a-tree}
  Let $\Gamma$ be a weighted planar tree and $\LS b_i\RS$ be the set
  of essential
  vertices of $\Gamma$ including their bonds (half-edges). We call the
  $\Gamma_s=\Gamma\setminus \LS b_i\RS$ the \textbf{sticks} of
  $\Gamma$ and every
  connected component of $\Gamma_s$ a \textbf{stick}.

\end{definition}As an example, consider the tree seen in
Figure~\ref{wpt-construc-fig-18}, the sticks of
which can be seen in Figure~\ref{wpt-construc-fig-19}.

\begin{figure}[H]
  \centering
  \includegraphics[width=\linewidth]{files/arbor_ring_noessenti-ac06ddb43820e892797452240f28d7d0.pdf}
  \caption[Sticks of a tree.]{Sticks of the tree from
    Figure~\ref{wpt-construc-fig-18}, six half-open sticks and one open
  stick.}
  \label{wpt-construc-fig-19}
\end{figure}

By construction, a stick subtree of $\Gamma$ may have 0, 1, or 2 free bonds
(seen in Figure~\ref{wpt-construc-fig-sticks}). We call a stick with
0 free bonds closed, 1
free bond half-open, and 2 free bonds open. Additionally, we call a stick where
each vertex has a single weight a \textbf{proper stick}, and we call
a vertex on the
end of a stick an \textbf{end vertex}.

\begin{figure}[H]
  \centering
  \includegraphics[width=0.5\linewidth]{files/sticks_open-a37c24f6b1180f1c4b02c7cf19d6eea4.pdf}
  \caption[Closed, half-open, and an open sticks.]{From top to
    bottom, a closed, half-open, and an open stick. Each end vertex
  is colored in red.}
  \label{wpt-construc-fig-sticks}
\end{figure}

%  prettier-ignore-start

\subparagraph{Integral Tangles}\label{wpt-construc-sec-integral}

%  prettier-ignore-end

When a weighted planar tangle tree is a half-open stick containing a single
vertex with a single weight $w_0$ we call it an \textbf{integral tangle tree}.

\begin{figure}[H]
  \centering
  \includegraphics[width=\linewidth]{files/watt_integral-d0566b70563422fdde261b042662a783.pdf}
  \caption[A stick realized as a integral tangle.]{A stick realized
  as a integral tangle.}
  \label{wpt-construc-fig-integral}
\end{figure}

%  prettier-ignore-start

\subparagraph{Rational Tangles}\label{wpt-construc-sec-rational}

%  prettier-ignore-end

Bonahon and Siebenmann \citep{bonahonNewGeometricSplittings2016} give a
correspondence between stick tangle trees (stick with a single free bond) and
Conway's rational tangles \citep{conwayEnumerationKnotsLinks1970}. An
example of the
correspondence can be seen in Figure~\ref{wpt-construc-fig-rat}.

\begin{figure}[H]
  \centering
  \includegraphics[width=\linewidth]{files/watt_rational-aab54569c87acbbe1143e6b14a6cbeeb.pdf}
  \caption[A stick tangle tree realized as a rational tangle.]{A
  stick tangle tree realized as a rational tangle.}
  \label{wpt-construc-fig-rat}
\end{figure}

%  prettier-ignore-start

\subparagraph{Tree Crossing Number}\label{wpt-construc-sec-TCN}

%  prettier-ignore-end

Finally, we define the \textbf{Tree Crossing Number (TCN)} of a weighted planar
tangle tree. This corresponds to the crossing number of the tangle diagram given
by the weighted planar tangle tree.

\begin{definition}{}{apn-def-tcn}For weighted planar tangle tree
  $\Gamma$, with weights $\LS w_i\RS$. We call
  \begin{equation}
    \text{TCN}=\sum |w_i|
  \end{equation}
  the \textbf{Tree Crossing Number (TCN)}.

\end{definition}
%  prettier-ignore-start

\subsubsection{Calculus of Weighted Planar Trees}\label{calculus_on_trees}

%  prettier-ignore-end

This subsection develops a set of moves, $F^\prime_3,\ F_2,\ F_1$, and $R^\pm$,
on the weighted planar trees described in Section~\ref{subsec-wptt}. We will
restrict our discussion to a subset of the calculus of weighted planar trees. A
full description of the calculus can be found in Bonahon and Siebenmann
\citep{bonahonNewGeometricSplittings2016}. The
$F^\prime_3,\ F_2,\ F_1,\text{ and}\ R_\pm$ moves allow us to systematically
modify, without changing the topology, a weighted planar tree and form the basis
for the classification of the arborescent knots.

\paragraph{The $F^\prime_3$ Move}

The first move, and as we will see, the most important in distinguishing
tangles, is the $F_3^\prime$ move. In this move we consider the
local picture of a vertex. In the local view, isolate a single bond
(half-edge corresponding to a plumbing
square of a band), then move a weight across that bond. The impact
of this movement of a weight propagates to the descendants of the
subtree attached to the bond
(plumbing square) but leaves unchanged all other weights and bonds
(including their
attached subtrees) in the local view of the object vertex.
\newpage
\begin{definition}{Bonahon and Siebenmann, Section 12.7.3
  \textbf{\citep{bonahonNewGeometricSplittings2016}}}{wpt-moves-def-f3p-move}
  The \textbf{$F_3^\prime$ move} on a weighted arborescent tree moves
  a weight $W$ as in
  Figure~\ref{wpt-moves-fig-f3p-move-indef} and, if $W$ is odd,
  reverses the cyclic order of
  weights and bonds at all vertices of the subtree attached to the
  purple bond (half-edge) lying at odd distance
  (count of edges between two vertices) from the vertex shown. Also, when $W$ is
  odd, apply $\xi$ ($X$-axis rotation
  Figure~\ref{wpt-construc-fig-k4g}) to all free bonds
  in the subtree attached to the purple bond (half-edge) that are
  attached to a vertex at even distance from the
  vertex shown, and $\eta$ ($Y$-axis rotation
  Figure~\ref{wpt-construc-fig-k4g-y}) to those
  at odd distance. The rotations are relative to the local orientations of the
  plumbing squares on the bands corresponding to vertices at odd distance from
  the vertex carrying weight $W$.

  \begin{figure}[H]
    \centering
    \includegraphics[width=\linewidth]{files/f3p_def-fea9c634e65e4c0469220121c3d52833.pdf}
    \caption[An $F_3^\prime$ move on a subtree.]{An $F_3^\prime$ move
      on a subtree. The purple bond indicates the subtree
      impacted by the move. The blue portion of the local view
      indicates all other bonds and weights
    of the vertex.}
    \label{wpt-moves-fig-f3p-move-indef}
  \end{figure}

\end{definition}The $F_3^\prime$ move is a derivative of the more
general $F_3$ move described
in Bonahon and Siebenmann \citep{bonahonNewGeometricSplittings2016}.
\newpage
\begin{definition}{Bonahon and Siebenmann, Section 12.7.1
  \textbf{\citep{bonahonNewGeometricSplittings2016}}}{wpt-moves-def-f3-move}
  The \textbf{$F_3$ move} replaces the left side of
  Figure~\ref{wpt-moves-fig-f3-move-indef} with the right side, where
  the cyclic order of bonds and weights is reversed at all vertices
  in the subtree attached to the purple bond (half-edge) of the
  vertex shown and at odd distances from this vertex. Also, apply
  $\xi$ ($X$-axis rotation) to all free bonds in the subtree attached
  to the purple bond (half-edge) that are attached to a vertex at
  even distance from the vertex shown,
  and $\eta$ ($Y$-axis rotation) to those at odd distance. The rotations are
  relative to the orientation of the  plumbing square (Conway sphere) of the the
  band the weight is moving across, per
  Figure~\ref{wpt-construc-fig-band_orientation}.

  \begin{figure}[H]
    \centering
    \includegraphics[width=\linewidth]{files/f3_def-73cd92acd310ffe76248c72f6e024950.pdf}
    \caption[An $F_3^\prime$ move on a subtree.]{An $F_3$ move on a
      subtree. The purple bond indicates the subtree
      impacted by the move. The blue portion of the local view
      indicates all other bonds and weights
    of the vertex.}
    \label{wpt-moves-fig-f3-move-indef}
  \end{figure}

\end{definition}

\begin{note}
  $F_3^\prime$ is equivalent, in
  Figure~\ref{wpt-moves-fig-f3-move-indef}, to setting
  $X=0$, then executing $F_3$ $W$ times, decreasing $\abs{W}$ with
  each iteration.
\end{note}

We will now consider some examples of the $F_3^\prime$ move. While we explore
these examples, we view $F_3^\prime$ from the perspective of an object vertex
(object band). The object vertex may have one or more children that we will act
on with $F^\prime_3$. When we translate $F_3$ and $F_3^\prime$ into practice we
are free to operate on children as well as the parent of a vertex (band).

\subparagraph{$F_3^\prime$ on bands}

The translation of crossings across child bands models the
traditional \textbf{flype}
move, of ``Tait Flyping Conjecture'' \citep{taitKnotsIIIII1900} fame. To see the
correspondence between $F_3^\prime$ and flype we need to view the plumbed child
(or parent) band as a tangle, $T$. We can then carry out $F_3^\prime$ over this
tangle. Tracking the parts of this operation, we can see the correspondence in
Figure~\ref{uc-c-f3-e-flype_and_bnd}.
\begin{figure}[H]
  \centering
  \includegraphics[width=\linewidth]{files/bnd_f3-57da5576983f5d6c9a55f15dc96a5397.pdf}
  \caption[Flype and $F_3^\prime$.]{Flype and $F_3^\prime$, with
    orientation of the Conway sphere given by
  Section~\ref{wpt-construc-sec-wptt}.}
  \label{uc-c-f3-e-flype_and_bnd}
\end{figure}

When this is carried out, for an odd number of crossings, the child band is
inverted so it lies inside the parent band
(Figure~\ref{uc-c-f3-e-flype_bnd}). The inversion
reverses the cyclic order of the child, as described in
Definition~\ref{wpt-moves-def-f3p-move}.
Applying $F_3^\prime$ to an even number of crossings is equivalent to applying
the move on two sets of an odd number of crossings. When the first set of odd
crossings is applied, the child band is inverted; the second set inverts the
child again, leaving it where it began
(Figure~\ref{wpt-moves-fig-example-f3-even-2}).
We expand to an example where the child band has descendants, as in
Figure~\ref{uc-c-f3-e-cor}. Observe that when the $F_3^\prime$ is
applied in this case, the
child band and every band even distance from it (odd distance from the parent)
is inverted.

\begin{figure}[H]
  \centering
  \includegraphics[width=\textwidth]{files/bnd_only_f3_even_2-ece02f57e951d9e680c8d28405e22790.pdf}
  \caption[$F_3^\prime$ on the band model.]{The even case of the $F_3^\prime$
    with the purple band and any descendants of the purple band
  remaining unchanged.}
  \label{wpt-moves-fig-example-f3-even-2}
\end{figure}

\begin{figure}[H]
  \centering
  \includegraphics[width=\textwidth]{files/bnd_only_f3-f34dfee65146995d01e338de8bd5c474.pdf}
  \caption[The odd $F_3^\prime$ case on a band model realization of
    the given portion of a
  tree.]{ The odd $F_3^\prime$ case on a band model realization of
    the given portion of a
    tree, with local $X$-axis in blue and $Y$-axis in red given
    relative to the purple band.
    Yielding a $\xi$ ($X$-axis rotation) to all bands plumbed to the purple band
    or plumbed at even distance (counting plumbing squares) from the
    orange band,
    and $\eta$ ($Y$-axis rotation) to the purple band and those
    plumbed at odd distance from the orange band.
    Note that the orientations of the plumbing squares must agree before and
    after $F_3^\prime$. Following the orientations with the left hand
    (Figure~\ref{wpt-construc-fig-band_orientation})
    rule shows the orientation of the purple band reverses in $X$ and
    $Z$ in the second image due to the rotation in $Y$
  (we are no longer looking through the band in the second image). }
  \label{uc-c-f3-e-flype_bnd}
\end{figure}

\begin{figure}[H]
  \centering
  \includegraphics[width=\linewidth]{files/bands_odd_change-707797bd70b675acb7b1ee4ff8e2c6c8.pdf}
  \caption[$F_3^\prime$ when applied to a band.]{$F_3^\prime$ when
    applied to a band (gray) with a child (orange) and grandchild
    (light blue). We read the bands following the local orientation
    of the plumbing
    squares. Before the move is applied, the child (orange) band is
    traversed as;
    the blue band, a green star, a green circle, and back to the parent. After
    $F_3^\prime$ it is traversed as; a green circle, a green star,
    the blue band,
    and back to the parent. The blue band is traversed as; yellow star, yellow
  circle, and back to orange band.}
  \label{uc-c-f3-e-cor}
\end{figure}

\subparagraph{$F_3^\prime$ Examples}

Consider the weighted planar tree in
Figure~\ref{wpt-moves-fig-example_f3-even-1} and
Figure~\ref{wpt-moves-fig-example_f3-odd_1}, the left trees in each
agree in all but a
weight of a single vertex, what we will call our object vertex which is marked
in orange. The weight of this object vertex has been changed from -2 in
Figure~\ref{wpt-moves-fig-example_f3-even-1} to -3 in
Figure~\ref{wpt-moves-fig-example_f3-odd_1}.

We will first walk through
Figure~\ref{wpt-moves-fig-example_f3-even-1}. In this example,
our object weight is e ven, applying $F_3^\prime$ to the tree, the impacted
subtree (purple subtree in Figure~\ref{wpt-moves-fig-f3p-move-indef})
is unchanged except
for free bonds, which are altered as described in
Definition~\ref{wpt-moves-def-f3p-move}.

\begin{figure}[H]
  \centering
  \includegraphics[width=0.7\linewidth]{files/watt_rooted_even-4d8585086a5084d3bb46059e31a94e1b.pdf}
  \caption[$F_3^\prime$ on a weighted planar tree.]{$F_3^\prime$ on a
    weighted planar tree with even weight. The weight of the
  object weight is even, so the impacted subtree is unchaged.}
  \label{wpt-moves-fig-example_f3-even-1}
\end{figure}

In Figure~\ref{wpt-moves-fig-example_f3-odd_1}, the object weight is
odd, applying
$F_3^\prime$ the cyclic order of vertices, of the impacted subtree at an odd
distance are reversed. Additionally, all free bonds in the impacted subtree are
altered as described in Definition~\ref{wpt-moves-def-f3p-move}.

\begin{figure}[H]
  \centering
  \includegraphics[width=0.7\linewidth]{files/watt_rooted_odd-f5793e212c4d6a0fa6d460f40db1f8c4.pdf}
  \caption[$F_3^\prime$ on a weighted planar tree.]{$F_3^\prime$ on a
    weighted planar tree with odd weight clockwise. Note the
    changes in the relative positions of subtrees after the application of
  $F_3^\prime$.}
  \label{wpt-moves-fig-example_f3-odd_1}
\end{figure}

\paragraph{The $F_2$ Move}

Our second move, $F_2$, is a special application of the general $F_3$ move.

\begin{definition}{Bonahon and Siebenmann, Section 12.7.1
  \textbf{\citep{bonahonNewGeometricSplittings2016}}}{uc-c-f2-d-f2-t}
  The \textbf{$F_2$ move} on a weighted arborescent tangle tree
  reverses the cyclic order
  of bonds and weights at one vertex on the tree and at every vertex
  at even distance from it; also
  apply $\eta$ ($Y$-axis rotation) to every free bond of a vertex at
  even (or zero) distance, and
  apply $\xi$ ($X$-axis rotation) to every free bond at odd distance.
  The rotations are relative to the orientation of the plumbing square (Conway
  sphere) of the band being acted on, per Section~\ref{subsec-wptt}.

\end{definition}$F_2$ is equivalent to applying $F_3$ to a vertex by
moving a $\pm 1$ weight
around a full cycle of the children (and parent), as in
Figure~\ref{wpt-moves-fig-example_f2_cycle}. If the vertex has no
weights, the zero weight
is split into a +1 and -1, one of which completes the cycle. The +1 and
-1 then cancel, returning the vertex to zero weight. The result of carrying
out $F_2$ on a weighted planar tree can be seen in
Figure~\ref{wpt-moves-fig-example_f2}.

\begin{figure}[H]
  \centering
  \includegraphics[width=\textwidth]{files/f2_local-a580dd826a40b46351b2f9bac9e0d6dd.pdf}
  \caption[$F_3$ moving a weight in a complete cycle.]{$F_3$ moving a
  weight in a complete cycle}
  \label{wpt-moves-fig-example_f2_cycle}
\end{figure}
\begin{figure}[H]
  \centering
  \includegraphics[width=\textwidth]{files/watt_rooted-f89a1c8a169f64c614d42e3085f20ec8.pdf}
  \caption[$F_2$ on a weighted planar tree.]{$F_2$ on a weighted
    planar tree. Observe the changes to the entire tree, as
  opposed to the changes of $F_3^\prime$ which impact only a subtree.}
  \label{wpt-moves-fig-example_f2}
\end{figure}
Observe that vertices can be partitioned into two equivalence classes. Those
changed by $F_2$ applied at an even distance from the root, and those changed by
$F_2$ applied at an odd distance from the root. We write $F_2$ on the even class
as $F_{2e}$ and odd as $F_{2o}$.

\paragraph{The $F_1$ Move}

The third of the $F$ moves is the $F_1$ move, which is a repeated application of
the $F_2$ move.

\begin{definition}{Bonahon and Siebenmann, Section 12.7.1
  \textbf{\citep{bonahonNewGeometricSplittings2016}}}{uc-c-f1-d-f1-t}
  The \textbf{$F_1$ move} on a weighted arborescent tangle tree
  reverses the cyclic order
  of bonds and weights at every vertex of the graph and applies
  $\zeta$ ($z$-axis
  rotation) to every free bond. The rotations are relative to the
  orientation of the plumbing square (Conway
  sphere) of the band being acted on, per Section~\ref{subsec-wptt}.

\end{definition}To realize $F_1$ as $F_2$ moves, we successively
apply $F_{2e}$ then $F_{2o}$ to
the tree. Observe that the combination of $F_{2e}$ and $F_{2o}$ modifies the
free bonds by $\xi\eta=\zeta$.

\paragraph{The $R^\pm$ moves}

The $R$ move, or ring move is the final move we will describe on weighted planar
tangle trees and deals with the ring subtrees of a tree. The result of a ring
move on a tangle can be seen in Figure~\ref{wpt-moves-fig-example-r}.

\begin{figure}[H]
  \centering
  \includegraphics[width=\linewidth]{files/example_ring-bbe1129008e26ffeb7c9523c9fa21b23.pdf}
  \caption[$R^ -$ on on a tangle representation of a tree.]{$R^ -$ on
  on a tangle representation of a tree.}
  \label{wpt-moves-fig-example-r}
\end{figure}

\begin{definition}{Bonahon and Siebenmann, Section 12.3
  \textbf{\citep{bonahonNewGeometricSplittings2016}}}{wpt-moves-def-rm}
  The \textbf{$R^ -$} replaces the left of Figure~\ref{rmove-n-pic}
  with the right, leaving the rest of the tree unchanged.

  \begin{figure}[H]
    \centering
    \includegraphics[width=0.5\linewidth]{files/def-1f585d3bf18ce302acf77bd52e61f5f0.pdf}
    \caption[A $\LP-\RP$-ring subtree moving around a vertex.]{A
      $\LP-\RP$-ring subtree moving around a vertex. Equivalent to
    Figure~\ref{wpt-moves-fig-example-r}.}
    \label{rmove-n-pic}
  \end{figure}

\end{definition}

\begin{note}
  In Figure~\ref{wpt-moves-fig-example-r} the ring moves from the
  right to the left of the
  tangle. This corresponds to the ring subtree in
  Definition~\ref{wpt-moves-def-rm} moving from
  top to bottom of the orange portion of the tree.
\end{note}

\begin{definition}{Bonahon and Siebenmann, Section 12.3
  \textbf{\citep{bonahonNewGeometricSplittings2016}}}{wpt-moves-def-rp}
  The \textbf{$R^+$} replaces the left of Figure~\ref{rmove-n-pic}
  with the right, leaving the rest of the tree unchanged.

  \begin{figure}[H]
    \centering
    \includegraphics[width=0.5\linewidth]{files/def-e0f5401b23b1e565a07876afd1239ac4.pdf}
    \caption[A $\LP+\RP$-ring subtree moving around a vertex.]{A
    $\LP+\RP$-ring subtree moving around a vertex.}
    \label{rmove-p-pic}
  \end{figure}

\end{definition}
%  prettier-ignore-start

\subsubsection{Canonical Weighted Planar Tangle Trees
(CWPTT)}\label{sec-CWPTT-def}

%  prettier-ignore-end

Observe that the weighted planar tangle trees we have seen are badly non-unique
representatives for arborescent tangles. The first step to finding a unique
preferred representative is to put some additional conditions on a weighted
planar tree, $\Gamma$. The following conditions pare down the equivalence class
of an arborescent tangle to a more manageable level.

\begin{definition}{Bonahon and Siebenmann, Section 12.8.2
  \textbf{\citep{bonahonNewGeometricSplittings2016}}}{wpt-equi-def-abcanon}
  A weighted planar tree is called a \textbf{canonical weighted planar
  tangle tree (CWPTT)} if it has a single free bond with a label from
  $V_4$ and satisfies the following conditions:

  \begin{itemize}
    \item \textbf{Weight Condition (W)} At each vertex of $\Gamma$, at
      most one weight is
      non-zero.

    \item \textbf{Stick Conditions}

      \begin{itemize}
        \item[\textbf{(S.1)}] On any stick the weights of the
          vertices are non-zero
          except for end vertices that have a bond free in $\Gamma$
          and for the case $\Gamma$ is
          Figure~\ref{wpt-construc-fig-stick_cond-1} or
          Figure~\ref{wpt-construc-fig-stick_cond-2}.
        \item[\textbf{(S.2)}] The non-zero weights along any stick are of
          alternating sign.
        \item[\textbf{(S.3)}] No end vertex of a stick has weight $\pm
          1$ unless it
          has a bond free in $\Gamma$.
      \end{itemize}

      \begin{figure}[H]
        \centering
        \begin{subfigure}[b]{0.45\textwidth}
          \centering
          \includegraphics[width=0.3\textwidth]{files/0-38f5b78e700440f6bd95db3013659000.pdf}
          \caption[The zero tangle. Where the indicated vertex is
          $v_0$.]{The zero tangle.}
          \label{wpt-construc-fig-stick_cond-1}
        \end{subfigure}
        ~
        \begin{subfigure}[b]{0.45\textwidth}
          \centering
          \includegraphics[width=0.3\textwidth]{files/00-ff78b5394e9bf86a9ab4f9dd4fcec2d4.pdf}
          \caption[The infinity tangle.]{The infinity tangle. Where
          the indicated vertex is $v_0$ and $v_1$.}
          \label{wpt-construc-fig-stick_cond-2}
        \end{subfigure}
        \caption{Where the given sticks are the entire tree $\Gamma$}
        \label{wpt-construc-fig-stick_cond}
      \end{figure}
    \item One of:

      \begin{itemize}
        \item \textbf{Positivity Condition (P)} Except for those with a
          free bond, there are no sticks in $\Gamma$ of the forms
          Figure~\ref{wpt-construc-fig-positivity_cond-1} or
          Figure~\ref{wpt-construc-fig-positivity_cond-2}.

          \begin{figure}[H]
            \centering
            \begin{subfigure}[b]{0.45\textwidth}
              \centering
              \includegraphics[width=0.3\textwidth]{files/am2-aaefc226f30ccf19cdbae75e67deae18.pdf}
              \caption[The -2 integral tangle.]{The -2 integral
              tangle. Where the indicated vertex is $v_i$ with $i\neq 0$}
              \label{wpt-construc-fig-positivity_cond-1}
            \end{subfigure}
            ~
            \begin{subfigure}[b]{0.45\textwidth}
              \centering
              \includegraphics[width=0.3\textwidth]{files/am2a-16bd870e6cc41ef82a45bae5879e4226.pdf}
              \caption[A fully open stick with -2 crossings.]{A fully
                open stick with -2 crossings. Where the indicated
              vertex is $v_i$ with $i\neq 0$}
              \label{wpt-construc-fig-positivity_cond-2}
            \end{subfigure}
            \caption[Positivity condition.]{}
            \label{wpt-construc-fig-positivity_cond}
          \end{figure}
        \item \textbf{Negativity Condition (N)} Except for those with a
          free bond, there are no sticks in $\Gamma$ of the forms
          Figure~\ref{wpt-construc-fig-negativity_cond-1} or
          Figure~\ref{wpt-construc-fig-negativity_cond-2}.

          \begin{figure}[H]
            \centering
            \begin{subfigure}[b]{0.45\textwidth}
              \centering
              \includegraphics[width=0.3\textwidth]{files/a2-2b741f8ba3f1cb8d8849231ce6357e1b.pdf}
              \caption[The 2 integral tangle.]{The 2 integral tangle.
              Where the indicated vertex is $v_i$ with $i\neq 0$}
              \label{wpt-construc-fig-negativity_cond-1}
            \end{subfigure}
            ~
            \begin{subfigure}[b]{0.45\textwidth}
              \centering
              \includegraphics[width=0.3\textwidth]{files/a2a-ec06a2a1420a8190c1216d90a131ca77.pdf}
              \caption[A fully open stick with 2 crossings.]{A fully
                open stick with 2 crossings.W here the indicated vertex
              is $v_i$ with $i\neq 0$}
              \label{wpt-construc-fig-negativity_cond-2}
            \end{subfigure}
            \caption[Negativity condition.]{}
            \label{wpt-construc-fig-negativity_cond}
          \end{figure}
      \end{itemize}
  \end{itemize}

\end{definition}

\begin{note}
  The set of CWPTT for an arborescent tangle can be large.
\end{note}
\begin{note}
  The positivity and negativity conditions are a consequence of the behavior of
  two crossing tangles seen in Figure~\ref{minimal-fig-nonmin}. We
  will adopt the
  $\LP+\RP$ as our prefered form.
\end{note}
Bonahon and Siebenmann show in Corollary~\ref{wpt-equi-lemma-exist}
that these conditions are
sufficient to realize every arborescent tangle. In fact, every weighted planar
tangle tree can be turned into a CWPTT by a series of moves in an extended
calculus on weighted planar trees
\citep{bonahonNewGeometricSplittings2016}. We call
this process \textbf{canonization} of a weighted planar tangle tree.

\begin{corollary}{Existance of CWPTT, Bonahon and Siebenmann
    Corollary 12.20
  \textbf{\citep{bonahonNewGeometricSplittings2016}}}{wpt-equi-lemma-exist}
  Every arborescent tangle is obtained by plumbing operations from
  arborescent tangles associated to positively (or negatively)
  canonical weighted planar trees (with labels in $V_4$ at free bonds).

\end{corollary}We note some consequences of the positivity and
negativity condition. First, a
positive CWPTT ($\LP +\RP$-CWPTT) can be transformed by a sequence of moves in
the extended calculus of weighted planar trees into a negative CWPTT
($\LP -\RP$-CWPTT). Similarly, a negative CWPTT can be transformed into a
positive CWPTT. Second, we note that a CWPTT, with no modification, can be both
positive and negative. We will refer to these trees as \textbf{neutral} trees.

Bonahon and Siebenmann give a classification of arborescent tangles via moves on
CWPTT.

\begin{theorem}{Classification Theorem for Canonical Weighted Planar
    Tangle Trees, Bonahon and Siebenmann, Theorem 12.21
  \textbf{\citep{bonahonNewGeometricSplittings2016}}}{wpt-equi-thm-classi}
  Consider two positive (or negative) CWPTT
  $\Gamma^{\,}$ and $\Gamma^\prime$, with free bonds labeled by
  elements of $V_4$. Plumbing according to $\Gamma$ and $\Gamma^\prime$
  gives isomorphic arborescent tangles if and only if $\Gamma$ and
  $\Gamma^\prime$ can be deduced from each other by a sequence of moves
  ($F_1$), ($F_2$), ($F_3^\prime$), and the modified ring moves $\LP\pm R\RP$.

\end{theorem}Further, Bonahon and Siebenmann describe an algorithm
for producing these
sequences of moves. This algorithm will be useful to us in
Section~\ref{sec-rlitt-generation}.

\begin{theorem}{Bonahon and Siebenmann, Theorem 12.19
  \textbf{\citep{bonahonNewGeometricSplittings2016}}}{wpt-equi-cor-algo}
  There exists an effective algorithm which, for any weighted planar
  tree $\Gamma$
  with free bonds labeled by elements of $V_4$, alters $\Gamma$ by a sequence of
  moves of the calculus of arborescent tangles to produce its collection of
  positively (or negatively) canonical weighted planar trees.

\end{theorem}\paragraph{Canonical Vertex}

The conditions for a weighted planar tree to be a CWPTT are phrased for the
global context of a weighted planar tangle tree. We now recontextualize those
conditions for a local view, a single vertex of the tree.

\begin{definition}{}{vertex-canon-def}
  A vertex $v_i$ of a weighted planar tangle tree $\Gamma$ with a single free
  bond labeled from $V_4$ is said to be \textbf{$\LP+\RP$-canonical}
  if $v_i$ has at
  most one non-zero weight $w_i$, and $i$ is zero (the root) with no
  conditions or
  $i$ is not zero with the following conditions satisfied:

  \begin{enumerate}[I.]

    \item {If the valence of $v_i$ is 1 and all of:
        \begin{itemize}
          \item{Stick Condition:
              \begin{enumerate}[1.]
                \item {$w_i\neq 0$ unless $i=1$ and $w_0=0$ (the
                  weight of the root)}
                \item {If the valence of $v_{i -1}$ (the parent) is 2
                    then $\text{sign}\LP w_i\RP\neq\text{sign}\LP w_{i
                  -1}\RP$ unless $i=1$ and $w_0=0$}
                \item {$w_i\neq \pm 1$}
              \end{enumerate}
            }
          \item {Positivity Condition:
            \begin{enumerate}[1)]
              \item{If the valence of $v_{i -1}$ (the parent) is
                greater than 2 then $w_i\neq -2$}
            \end{enumerate}
          }
      \end{itemize}
    }

  \item {If the valence of $v_i$ is 2 and all of:
      \begin{enumerate}
        \item{Stick Condition:
            \begin{enumerate}[1.]
              \item {$w_i\neq 0$}
              \item{
                \begin{enumerate}[i)]
                  \item{ If valence of the child is 2 then
                      $\text{sign}\LP w_i\RP\neq\text{sign}\LP
                    w_{i+1}\RP$ (the child)}
                  \item {If valence of the parent is 2 then
                      $\text{sign}\LP w_i\RP\neq\text{sign}\LP w_{i
                    -1}\RP$ (the parent)}
                \end{enumerate}
              }
            \item {If valence of the parent or valence of the child
              is greater than 2 (is essential) then $w_i\neq \pm1$}
          \end{enumerate}
        }
      \item {Positivity Condition:
          \begin{enumerate}[1.]
            \item {If valence of the parent and valence of the child
              is greater than 2 (both are essential) then $w_i\neq \m2$}
          \end{enumerate}
        }
    \end{enumerate}
  }
\end{enumerate}

\end{definition}From this definition, we now show that these
conditions are identical to those
in the global context of Definition~\ref{wpt-equi-def-abcanon}.

\begin{theorem}{}{vertex-and-cannon}$\Gamma$ is a $\LP+\RP$-CWPTT if
and only if all the vertices of $\Gamma$ are
$\LP+\RP$-canonical.

\end{theorem}
\begin{proof}
Checking each canonicity condition locally shows both directions.
\end{proof}
%  prettier-ignore-start

The definition and proof for $\LP -\RP$-canonical vertices are identical.
Similarly to the canonical tree case, we define a third positivity class for a
vertex, the \textbf{neutral vertex}, a vertex that is both $\LP
-\RP$-canonical and
$\LP+\RP$-canonical.

\subsubsection{Minimalization of CWPTT}\label{sec-minimalization}

%  prettier-ignore-end

\paragraph{CWPTT are Not Minimal}

A common measure for the complexity of knots and their relatives is the
\textbf{minimal crossing number}. That being the least number of
crossings needed to
realize the object in a diagram, we call that diagram the
\textbf{minimal diagram}.
It is natural to ask if our CWPTT are minimal representatives among either all
representations or arborescent representations. A quick analysis of the process
of canonization demonstrates that CWPTT are unfortunately far from minimal even
among arborescent representatives. An example of canonizing a tangle, making
that tangle non-minimal, is seen in Figure~\ref{minimal-fig-nonmin_min}.

\begin{figure}[H]
\centering
\begin{subfigure}[b]{\textwidth}
\centering
\includegraphics[width=.6\textwidth]{files/non-minimal_minimali-5d080586fc68b86cb67ded41def430c0.pdf}
\caption[A minimal presentation of a arborescent tangle.]{A minimal
  presentation of a arborescent tangle in both its orthogonal projection
as well as its weighted planar tree.}
\label{minimal-fig-nonmin_min}
\end{subfigure}
\newline
\centering
\begin{subfigure}[b]{\textwidth}
\centering
\includegraphics[width=.6\textwidth]{files/non-minimal-b7d96efc2cab4d59d3fdbe8e4e6fd919.pdf}
\caption[A non-minimal presentation of the same arborescent
tangle.]{A non-minimal presentation of the same arborescent tangle as
  Figure~\ref{minimal-fig-nonmin_min} in both its orthogonal projection
as well as its CWPTT.}
\label{minimal-fig-nonmin_non}
\end{subfigure}
\caption[Minimal and non-minimal trees.]{}
\label{minimal-fig-nonmin}
\end{figure}\paragraph{Canonization Can Increase Complexity}

As we have seen, a CWPTT often does not realize a minimal crossing
representative for an arborescent tangle. Since minimal crossing number is such
a common measure for complexity, we should understand how canonization impacts
the crossing number complexity of a CWPTT. We will accomplish this by
identifying a (non-unique) minimal arborescent representative for each tangle.
That is, a weighted planar tangle tree with minimal TCN among all weighted
planar tangle trees in its equivalence class. To begin we expand our
understanding of the moves in the calculus of arborescent tangles to those that
alter weights arithmetically. These moves are related to the arithmetic
operations on continued fractions \citep{bonahonNewGeometricSplittings2016}.

\begin{definition}{Bonahon and Siebenmann, Section 12.3
\textbf{\citep{bonahonNewGeometricSplittings2016}}}{minimal-def-arithmetic}
When carrying out the following \textbf{arithmetic moves} the
relative positions of weights are
critical to the invariance of the underlying knot pair.

\begin{itemize}
\item (0.1) The \textbf{0.1 move} replaces the left side with the
  right side of Figure~\ref{minimal-fig-move01}.

  \begin{figure}[H]
    \centering
    \includegraphics[width=0.625\linewidth]{files/def-06392bbc82305bf238bf19fae97e05ba.pdf}
    \caption[Move 0.1 on a weighted planar tree.]{Move 0.1 on a
    weighted planar tree.}
    \label{minimal-fig-move01}
  \end{figure}

\item (0.2) The \textbf{0.2 move} replaces the left side with the
  right side of Figure~\ref{minimal-fig-move02}. Additionally, the
  cyclic order of all descendants in the purple subtree is reversed
  and $\zeta$ ($Z$-axis rotation
  Figure~\ref{wpt-construc-fig-k4g-z}). The vertices $a$ and $b$ need
  not be valence two, either or both may have a valence greater than two.

  \begin{figure}[H]
    \centering
    \includegraphics[width=0.625\linewidth]{files/def-c1d1c36f8c5b6b3cf209c56ae164e0e1.pdf}
    \caption[Move 0.2 on a weighted planar tree.]{Move 0.2 on a
    weighted planar tree.}
    \label{minimal-fig-move02}
  \end{figure}

\item (1.1) The \textbf{1.1 move} replaces the left side with the
  right side of Figure~\ref{minimal-fig-move11}.

  \begin{figure}[H]
    \centering
    \includegraphics[width=0.625\linewidth]{files/def-c88ffe87967c699fe0df6b57f440c690.pdf}
    \caption[Move 1.1 on a weighted planar tree.]{Move 1.1 on a
    weighted planar tree.}
    \label{minimal-fig-move11}
  \end{figure}

\item (1.2) The \textbf{1.2 move} replaces the left side with the
  right side of Figure~\ref{minimal-fig-move12}. The vertices $a$ and
  $b$ need not be valence two, either or both may have a valence
  greater than two.

  \begin{figure}[H]
    \centering
    \includegraphics[width=0.625\linewidth]{files/def-7fcc9ebbcf1bc20bcad55cabc45a67d1.pdf}
    \caption[Move 1.2 on a weighted planar tree.]{Move 1.2 on a
    weighted planar tree.}
    \label{minimal-fig-move12}
  \end{figure}

\item (2.1) Replace the left side with the right side of
  Figure~\ref{minimal-fig-move21}

  \begin{figure}[H]
    \centering
    \includegraphics[width=0.625\linewidth]{files/def-0d1c913041ef32fcc5f274b63cf2c009.pdf}
    \caption[Move 2.1 on a weighted planar tree.]{Move 2.1 on a
    weighted planar tree.}
    \label{minimal-fig-move21}
  \end{figure}

\item (2.2) Replace the left side with the right side of
  Figure~\ref{minimal-fig-move22}. The vertices $a$ and $b$ need not
  be valence two, either or both may have a valence greater than two.

  \begin{figure}[H]
    \centering
    \includegraphics[width=0.625\linewidth]{files/def-53b898a2529b73e08acc755c45e36375.pdf}
    \caption[Move 2.2 on a weighted planar tree.]{Move 2.2 on a
    weighted planar tree.}
    \label{minimal-fig-move22}
  \end{figure}
\end{itemize}

\end{definition}
\begin{note}
The 2.1 and 2.2 moves are what allow us to pass between the $\LP+\RP$ and
$\LP -\RP$ canonical classes of trees.
\end{note}

From here we will show that canonization of minimal trees increases TCN
complexity in a controlled manner.

\begin{theorem}{}{minimal-thm-minimal}A minimal tree canonizes to a
$\LP+\RP$-CWPTT, with only the moves 1.2, 2.1,
and 2.2 increasing TCN.

\end{theorem}
\begin{proof}Let $\Gamma$ be a minimal TCN arborescent representative
of its equivalence
class. Starting with the weight condition
(W) maximally apply $F_3$ consolidating the weights of each vertex.
$F_3$ may need to be reapplied after application of arithmetic moves
and does not impact TCN.

Next we handle the stick condition starting with condition S.2 concerning
sticks having non-zero weights. Maximally apply
to $\Gamma$ moves 0.1, and 0.2; this removes zero weights of
sticks. In the general canonization process S.1, the alternating
stick condition, is handled by application
of move 1.2. However, on a pair of vertices violating condition
S.1, move 1.2
decreases the TCN. This means by minimality of $\Gamma$, condition
S.1 must already be
satisfied. Finally to obtain condition S.3, we apply moves 1.1 and
1.2, where move 1.2 may increase crossing number by 1.

The last condition to enforce is P, the positvity condition, which is
done by modifying $\Gamma$ by application of moves 2.1
and 2.2. Similarly to our S.1 case, $\Gamma$ is minimal, 2.1,
and 2.2 cannot decrease TCN. However, 2.1 and 2.2 may increase TCN
by 1 and 2 respectively.

\end{proof}

We note that
Theorem~\ref{minimal-thm-minimal} indicates that
any opportunities to decrease TCN in a CWPTT are found on the end
vertices of sticks adjacent to
an essential vertex. Particularly, at least one weight
used in the execution of 1.2, 2.1, and 2.2 must be carried by an essential
vertex. Reversing the sequence of moves in
Theorem~\ref{minimal-thm-minimal} tells us that a
minimal tree can be constructed from a CWPTT by application of TCN
decreasing moves 1.2,
2.1, and 2.2. It is important to note that this does not guarantee that
every set of applications of the 1.2, 2.1, and 2.2 moves to a CWPTT
minimizes the TCN, only the existence of a path to a minimal tree.

\begin{note}
The reversed sequence of moves in Theorem~\ref{minimal-thm-minimal}
taking us from a CWPTT to a minimal weighted planar tangle tree does
not need to increase TCN to obtain minimality.
\end{note}

\paragraph{Bounding complexity}

We now introduce a bound on complexity between a CWPTT and a minimal
representation of that tree.  We build the bound by identifying the
maximum number of subtrees of a CWPTT which admit
a 1.2, 2.1, or 2.2 move. We begin by identifying the smallest TCN of a
subtree that admits each move. These minimal subtrees follow directly from
Definition~\ref{minimal-def-arithmetic} and our essential vertex
requirement. The smallest, by TCN, canonical subtree admitting move 1.2 is
7, as in Figure~\ref{minimal-fig-minimize_move12}.
For move 2.1 the smallest TCN for a canonical subtree is 5, as in
Figure~\ref{minimal-fig-minimize_move21}, and for 2.2 the smallest TCN is
10, as in Figure~\ref{minimal-fig-minimize_move22}. Combining these
smallest TCN subtrees with the how each move decreases TCN gives the
bound in Equation~(\ref{bound-arbor-cwptt}), where $\Gamma_m$ is a
minimal representative for the
tangle class of $\Gamma$.

\begin{equation}
\label{bound-arbor-cwptt}
\begin{aligned}
\text{TCN}\LP\Gamma\RP-\text{TCN}\LP\Gamma_m\RP&\leq
1\cdot\left\lfloor\frac{\text{TCN}\LP\Gamma\RP}{7}\right\rfloor+1\cdot\left\lfloor\frac{\text{TCN}\LP\Gamma\RP}{5}\right\rfloor+2\cdot\left\lfloor\frac{\text{TCN}\LP\Gamma\RP}{10}\right\rfloor\\
&\leq
\frac{\text{TCN}\LP\Gamma\RP}{7}+\frac{\text{TCN}\LP\Gamma\RP}{5}+2\cdot\frac{\text{TCN}\LP\Gamma\RP}{10}\\
&=\frac{\text{TCN}\LP\Gamma\RP\cdot 19}{35}\\
\end{aligned}
\end{equation}

\begin{figure}[H]
\centering
\begin{subfigure}[b]{.3\textwidth}
\centering
\includegraphics[width=\textwidth]{files/minimal-382709d139690600e8a2d4027e6ae77c.pdf}
\caption[Admits move 1.2.]{Admits move 1.2}
\label{minimal-fig-minimize_move12}
\end{subfigure}
~
\begin{subfigure}[b]{.2\textwidth}
\centering
\includegraphics[width=\textwidth]{files/minimal-8a6f09e8a725817c264714a97dc2e1f8.pdf}
\caption[Admits move 2.1]{Admits move 2.1}
\label{minimal-fig-minimize_move21}
\end{subfigure}
~
\begin{subfigure}[b]{.3\textwidth}
\centering
\includegraphics[width=\textwidth]{files/minimal-d307983510930002fc016efa47e0e062.pdf}
\caption[Admits move 2.2]{Admits move 2.2}
\label{minimal-fig-minimize_move22}
\end{subfigure}

\label{minimal-cont-move-subtrees}
\caption[Subtrees for minimalization moves.]{Examples of
canoncial subtrees of a $\LP+\RP$-CWPTT which admit the given moves.
Note: These are subtrees admiting the moves, but not the only subtrees admiting
the moves. }
\end{figure}
%  prettier-ignore-start

\subsection{Right Leaning Identity CWPTT}\label{subsec-rlitt}

%  prettier-ignore-end

The CWPTT are sufficient for distinguishing any two arborescent tangles via
moves on their trees. Unfortunately, the equivalence class of CWPTT is still too
large for computational enumeration to be feasible. The time required for
pairwise comparisons grows badly exponentially. Luckily, from the class of CWPTT
for an arborescent tangle, we can select a unique preferred form that allows for
efficient direct enumeration by computer. To achieve this we will define two
additional conditions for CWPTT, first the right leaning condition, and second,
the identity condition. We call these preferred CWPTT \textbf{Right
Leaning Identity
Canonical Weighted Planar Tangle Trees (RLITT)}.

\subsubsection{Existence of Right Leaning CWPTT}

We start our construction of RLITT by defining what conditions make a CWPTT a
right leaning CWPTT.

\begin{definition}{}{rli-const-def-rl}A CWPTT is called \textbf{right
leaning} if all weights are in the highest indexed
region (as in Section~\ref{indexing-rpt}) of each vertex.
Additionally, any ring subtrees
that are children of a vertex are the highest indexed children of that vertex.

\end{definition}Our next step is to show that every arborescent
tangle has a right leaning
representative.

\begin{theorem}{}{rli-const-thm-rl-exists}Every arborescent tangle
has a right leaning CWPTT representative.

\end{theorem}
\begin{proof}Let $\Gamma$ be a CWPTT representative for a tangle $T$.
If every weight
$w_i$ of $\Gamma$ is in the highest indexed region of $\Gamma_{w_i}$, we
are done. Otherwise, we will follow a similar algorithm to that outlined by
Bonahon and Siebenmann \citep{bonahonNewGeometricSplittings2016} for
distinguishing
CWPTT. Let $w_i$ be the weight for the lowest indexed vertex $v_i$ not in its
highest indexed region of $\Gamma_{v_i}$. With move $F_3^\prime$ shift $w_i$ so
that it lies in the highest indexed region. Further, choose to shift $w_i$
anti-clockwise, as this ensures that $v_j$ with $j<i$ are unchanged
when $w_i$ is
odd. Additionally, with ring moves position the ring subtrees in the right most
region before the weight. We repeat this process for any $v_k$ with
$i<k$ where the weight $w_k$ not
in the highest indexed region. Since $\Gamma$ has finite vertices, the
algorithm terminates with a $\Gamma$ transformed into a right leaning tree
completing the proof.

\end{proof}
\subsubsection{Existence of Identity CWPTT}

Our second step in the construction of RLITT is to define what conditions make a
CWPTT an identity CWPTT.

\begin{definition}{}{rli-const-def-identity}A CWPTT is called an
\textbf{identity tree} if its free bond is marked by
$\iota\in V_4$.

\end{definition}Again, we must show that every arborescent tangle has
an identity
representative.

\begin{theorem}{}{rli-const-thm-ident_exists}Every arborescent tangle
has an identity CWPTT representative.

\end{theorem}
\begin{proof}Let $\Gamma$ be a CWPTT representative for a tangle $T$.
If the label
$\alpha$ for the free bond of $\Gamma$ is $\iota$ we are done. Otherwise,
we fall into one of three cases:

\begin{itemize}
\item $\alpha=\zeta$: In the case $\alpha$ is $\zeta$ we apply move $F_1$. This
  modifies $\alpha$ by $\zeta$ yielding $\alpha\zeta=\zeta\zeta=\iota$.
\item $\alpha=\eta$: In the case $\alpha$ is $\eta$ we apply move $F_{2e}$. This
  modifies $\alpha$ by $\eta$ yielding $\alpha\eta=\eta\eta=\iota$.
\item $\alpha=\xi$: In the case $\alpha$ is $\xi$ we apply move $F_{2o}$. This
  modifies $\alpha$ by $\xi$ yielding $\alpha\xi=\xi\xi=\iota$.
\end{itemize}

This transforms $\Gamma$ into an identity tree completing the proof.

\end{proof}
\subsubsection{Existence of Right Leaning Identity CWPTT (RLITT)}

What we have shown is that every arborescent tangle has at least one right
leaning CWPTT and at least one identity CWPTT representative. Combining these
two ideas, we will show that every arborescent tangle has at least one CWPTT
that is right leaning and identity, we call such a CWPTT a RLITT.

\begin{definition}{}{rli-const-def-rlident}A CWPTT is called a
\textbf{right leaning identity tangle tree (RLITT)} if it's a
right leaning and identity tree.

\end{definition}
\begin{theorem}{}{rli-const-thm-rightident_exists}Every CWPTT has a
right leaning identity representative.

\end{theorem}
\begin{proof}Let $\Gamma$ be an identity CWPTT representative for a tangle $T$.
Applying the algorithm described in the proof of
Theorem~\ref{rli-const-thm-rl-exists} transforms $\Gamma$ into a right leaning
tree. Our requirement that $F_3^\prime$ be anti-clockwise ensures that the
resulting tree retains label $\iota$. This shows that $\Gamma$ can be
represented as a right leaning identity CWPTT.

\end{proof}
\subsubsection{Uniqueness of Right Leaning Identity CWPTT}

Our final step at identifying a preferred representative CWPTT of an arborescent
tangle is to show that a $\LP +\RP$-RLITT is unique in the set of CWPTT
representing an arborescent tangle. We will utilize the key proposition from
Bonahon and Siebenmann \citep{bonahonNewGeometricSplittings2016} a
consequence of
Theorem~\ref{wpt-equi-thm-classi} and Theorem~\ref{wpt-equi-cor-algo}.

\begin{proposition}{Bonahon and Siebenmann, Proposition 12.22
\textbf{\citep{bonahonNewGeometricSplittings2016}}}{rli-const-prop-not_iso}
Let $\Gamma$ and $\Gamma^\prime$ be $\LP +\RP$-CWPTT tangle trees
with isomorphic
underlying abstract trees. Further, let $\varphi$ be a sequence of moves of the
calculus of arborescent tangles ($F_1$, $F_2$, $F_3^\prime$, and the
modified ring moves $\LP\pm R\RP$). Assume that there is an $i > 0$ such that
$\varphi$ respects the cyclic orders of weight and bonds at each vertex $v_j$
with $j < i$, and that the labels of the free bond $\alpha$ in
$\Gamma$ and of $\varphi\LP \alpha\RP$ in $\Gamma^\prime$ are identical. Assume
moreover that one of:

\begin{enumerate}
\item $\varphi$ reverses the cyclic order of bonds at $v_i$
\item $\varphi$ does not respect the label in $V_4$ of some free bond of a
  vertex $v_j$ with $1 \leq j < i$.
\end{enumerate}

Then $\varphi$ is not a sequence of moves of the calculus of arborescent tangles
taking $\Gamma^\prime$ to $\Gamma$.

\end{proposition}
\begin{theorem}{}{rli-const-thm-rlitt_unique}The $\LP +\RP$-RLITT
representative is unique in the class of CWPTT.

\end{theorem}
\begin{proof}Let $\Gamma$ and $\Gamma^\prime$ be two $\LP +\RP$-RLITT
representatives for an arborescent
tangle $T$. Assume for the sake of contradiction that
$\Gamma\neq \Gamma^\prime$, meaning $T$ has two distinct $\LP +\RP$-RLITT. The
classification result in Theorem~\ref{wpt-equi-thm-classi} and algorithm given
by Theorem~\ref{wpt-equi-cor-algo} we produce a sequence of moves in the
calculus of arborescent tangles that takes $\Gamma^\prime$ to $\Gamma$. Let
$\varphi$ be such a sequence. By construction the labels in $V_4$ of $\Gamma$
and $\Gamma^\prime$ agree. Now, since $\Gamma\neq \Gamma^\prime$ there must be a
first, in the total order, vertex $v_i$ where $\Gamma$ and $\Gamma^\prime$
disagree. As $\LP +\RP$-RLITT the location of weights for $v_i$ in $\Gamma$ and
$\Gamma^\prime$ must appear in the same region. This requires that the
disagreement at $v_i$ must be in cyclic order of its children. We find ourselves
in the first case of Proposition~\ref{rli-const-prop-not_iso}, making
$\varphi$ not a
sequence of moves of the calculus of arborescent tangles taking $\Gamma^\prime$
to $\Gamma$.

\end{proof}
%  prettier-ignore-start

\subsection{Computational Methods}\label{subsec-computation}

%  prettier-ignore-end

%  prettier-ignore-start

\subsubsection{An Encoding Strategy for Arborescent Knots and
Tangles}\label{sec-arborescent-linear}

%  prettier-ignore-end

The various flavors of weighted planar trees we have seen thus far are a useful
tool for manipulation of arborescent tangles by humans or machines.
Unfortunately, the tree structure is quite difficult to store directly in a
computer database. We will rectify this by introducing a linearization strategy
for weighted planar trees. This linearization strategy is designed to encode not
only CWPTT but arbitrary weighted planar tangle trees. If a weighted planar
tangle tree has more than one free bond we list the label as a subtree. We will
omit this from our algorithm description as we are primarily concerned with
RLITT.

We will descend the tree following the indexing of the total order
(Section~\ref{indexing-rpt}), where the total ordering is an ideal
ordering, specifically the
depth first ordering. As we descend the tree, we annotate the
linearization with two
sets of delimiters. Each delimiter communicates extra information about the type
of subtree it is delimiting, the two sets of delimiters are as follows:

\begin{itemize}
\item $\LB\ \RB$: Corresponds to a half-open proper stick and is
interpreted as a
twist vector for a rational tangle
\cite{kauffmanClassificationRationalKnots2002}.
\item $\LP\ \RP$: Corresponds to a vertex not on a half-open stick.
\end{itemize}

We will now walk through an example of the linearization algorithm. Let $\Gamma$
be a weighted planar tangle tree seen in Figure~\ref{wpt-rli-fig-23}.
As we walk the tree,
the vertex currently being linearized will be called the \textbf{object vertex}.

\paragraph{Tangle Linearization Example}

We begin by adding the $V_4$ label for our tangle to the linearization. We then
start the following algorithm with the root as the object vertex.

For the object vertex, we add a `$\LP\ \right.$' delimiter to our linearization.
Adding to our linearization the weights and children of the object vertex in an
anti-clockwise order. When a child bond is encountered, we descend to that
child. When we descend, we have two cases to consider, the child is a half-open
proper stick or otherwise.

\paragraph{Case 1: The Child Is The Root Of A Half-Open And Proper Stick}

When the child is the root of a is proper and half-open (contains a leaf
vertex), we append that stick as the twist vector
(Definition~\ref{rational-def-twistvector}) for the corresponding rational
tangle. Let the stick consist of the vertices, $v_i\cdots v_{i+k}$, and weights,
$w_i\cdots w_{i+k}$. We delimit the stick with $\LB\ \RB$, with each weight
separated by a space, and the leaf weight as the left most entry of the twist
vector. Further, every other entry is multiplied by -1, forcing the sign of
all entries to match, as in (\ref{linear-math-tv}).

\begin{equation}
\label{linear-math-tv}
\LB w_{i+k}\ \m w_{i+k-1}\cdots\ \m w_{i-1}\ w_{i}\RB
\end{equation}

\paragraph{Case 2: The Child Is Essential, On A Open Stick, Or On A
Non-Proper Stick}

In this case, we restart the algorithm from the beginning with the current
vertex as the object vertex.

When we have exhausted the children for the object vertex, we close our
linearization for that vertex with the delimiter `$\LN\ \RP$'. We then return to
the parent linearization, repeating until all vertices have been exhausted. An
example of a tree encoded with this strategy can be seen in
Figure~\ref{wpt-rli-fig-23}.

\begin{figure}[H]
\centering
\includegraphics[width=\linewidth]{files/watt_walk_tangle-18fc70625153f60d62d4bd17ca1eac74.pdf}
\caption[Encoded tree subtrees are indicated by color.]{Encoded tree
subtrees are indicated by color. Encoding follows the indicated path.}
\label{wpt-rli-fig-23}
\end{figure}

%  prettier-ignore-start

\subsubsection{Generation of Right Leaning Identity Weighted Planar
Tangle Trees}\label{sec-rlitt-generation}

%  prettier-ignore-end

\paragraph{Generation of Rooted Plane Tree}

Just as rooted plane trees serve as the scaffolding we built WPTT on, a rooted
plane tree algorithm will serve as the backbone for our RLITT generation
algorithm. We will now give a brief description of the generation algorithm for
rooted plane trees given by Nakano
\citep{nakanoEfficientGenerationPlane2002}. We
begin by defining the \textbf{rightmost path} of a rooted plane tree $\Gamma$.

\begin{definition}{Nakano, Section 2
\textbf{\citep{nakanoEfficientGenerationPlane2002}}}{}
Let $v_0$ be the root and $v_i$ be the highest indexed leaf (vertex of valence
$\leq 1$) of a rooted plane tree $\Gamma$. The unique path
$\LP v_0,\,\cdots,\,v_i \RP$, in the standard graph theoretic sense, is called
the \textbf{rightmost path} of $\Gamma$.

\end{definition}Next, we define a grafting operation on rooted plane
trees $\Gamma$ and
$\Gamma^\prime$.

\begin{definition}{}{rli-gen-def-grafting_op}Let
$\mathcal{T}^{p}_{n}$ be the set of rooted plane trees on $n$ vertices,
$\Gamma_r\in \mathcal{T}^{p}_{n}$ , and $\Gamma_s\in \mathcal{T}^{p}_{m}$.
Define the \textbf{grafting operation} as follows.

\begin{equation}
\begin{aligned}
  \star_i:\mathcal{T}^{rp}_{n}\times\mathcal{T}^{rp}_{m}&\to\mathcal{T}^{rp}_{n+m}\\
  \Gamma_r\times\Gamma_s&\mapsto\Gamma_r\star_i\Gamma_s
\end{aligned}
\end{equation}

At the vertex $v_i$ of $\Gamma_r$, introduce an edge to the root of $\Gamma_s$.
Now, adjust the indexing of a vertex $s_k$ of $\Gamma_s$ as $v_{n+k}$, placing
$\Gamma_s$ as the rightmost child of $v_i$. This results in a rooted plane tree
$\Gamma\in \mathcal{T}^{rp}_{n+m}$.

When grafting at the root $v_0$ we omit the index label in the grafting
operation, that is, $\star_0$ is written simply as $\star$. We call
$\Gamma_r$ the
\textbf{rootstock} and $\Gamma_s$ the \textbf{scion}.

\end{definition}
\begin{figure}[H]
\centering
\includegraphics[width=0.6\linewidth]{files/arbor_graph_grafting-61926ec067da7853030611f875cb397e.pdf}
\caption[Grafting a scion.]{Grafting a scion $\Gamma_s$ to a
rootstock $\Gamma_r$ with $\Gamma_r\star_3\Gamma_s$}
\label{rli-gen-fig-scion_grafting}
\end{figure}
Now we are prepared to give the algorithm to generate all rooted plane trees of
a given size that are created from $\Gamma$ by a sequence of $\star_\ast$
operations on the rightmost path of $\Gamma$.
\newpage
\begin{remark}{Find all rooted plane trees of a given size created
from \textbf{$\Gamma$
\citep{nakanoEfficientGenerationPlane2002}}}{find-all-related-trees}
\textbf{Input}

\begin{itemize}
\item A tree $\Gamma\in \mathcal{T}^{rp}_{i}$, $i<n$
\item A target size $n\in T$
\end{itemize}

\textbf{Output}

\begin{itemize}
\item All collections of trees $\Gamma\in \mathcal{T}^{rp}_{n}$
  created from $\Gamma$
\end{itemize}

\textbf{Routine}

\begin{enumerate}
\item Set $P$ to the rightmost path of $\Gamma$
\item Set $\Gamma_1$ as the single vertex rooted plane tree
\item If the number of vertices of $\Gamma$ is $n$ exit the algorithm
\item For each vertex $v_i$ in $P$
  \begin{enumerate}
    \item Graft $\Gamma_1$ onto $\Gamma$ as $\Gamma\star_i
      \Gamma_1=\Gamma^\prime$
    \item Start a new instance of Algorithm~\ref{find-all-related-trees} with
      $\Gamma=\Gamma^\prime$ and size=size
  \end{enumerate}
\end{enumerate}

\end{remark}This algorithm can be extended to an algorithm that finds
all rooted plane trees
up to a given size as follows.

\begin{remark}{Efficient generation of rooted plane trees
\textbf{\citep{nakanoEfficientGenerationPlane2002}}}{find-efficient-trees}
\textbf{Input}

\begin{itemize}
\item A target number of vertices of a tree $n\in \Z$
\end{itemize}

\textbf{Output}

\begin{itemize}
\item All trees $\Gamma\in \mathcal{T}^{rp}_{n}$
\end{itemize}

\textbf{Routine}

\begin{enumerate}
\item Set $\Gamma_1$ as the single vertex rooted plane tree
\item Execute Algorithm~\ref{find-all-related-trees} with
  $\Gamma=\Gamma_1$ and $n=n$
\end{enumerate}

\end{remark}
\newpage
\paragraph{Modification for RLITT}

The algorithm described above serves as the inspiration for the algorithm we
will build now for the enumeration of the arborescent tangles. Building this
algorithm begins with modifying the grafting $\star_i$ operation to operate on
weighted planar tangle trees as follows.

\begin{definition}{}{rli-gen-def-grafting_op-wpt}The \textbf{grafting
operation} $\star_i$ for weighted planar tangle trees, we require
that the free bond of the scion be grafted to $v_i$. We also specify that the
scion be grafted so that the rightmost weight of $v_i$ remains to the right of
the scion after grafting; this can be seen in
Figure~\ref{rli-gen-fig-scion_grafting_wth_weight}.

\end{definition}
\begin{figure}[H]
\centering
\begin{subfigure}[b]{\textwidth}
\centering
\includegraphics[width=0.6\textwidth]{files/awptt_graph_pregraft-28141689ce1fed14e0a3223ce9a0eede.pdf}
\caption[A WPTT rootstock and RLITT scion grafted.]{A rootstock
  $\Gamma_r=\iota\LP \LP2\LP2\LB3\RB2\LB -3\RB3\RP \LB3\RB4\RP 4\RP$
  in grey and scion $\Gamma_s=\iota\LB 10\ 9 \RB$ in orange. Each vertex is
labeled with its index in the order on $\Gamma$.}
\label{rli-gen-fig-scion_grafting_wth_weight_1}
\end{subfigure}
\newline
\begin{subfigure}[b]{\textwidth}
\centering
\includegraphics[width=0.6\textwidth]{files/awptt_graph_grafted-e265b8e668aca3adac89f27cd9529cd2.pdf}
\caption[Grafting at $v_2$.]{Grafting at $v_2$ yields
  $\Gamma_r\star_2\Gamma_s=\iota\LP \LP2\LP2\LB3\RB2\LB -3\RB\LB 10\ 9
\RB 3\RP \LB3\RB4\RP 4\RP$}
\label{rli-gen-fig-scion_grafting_wth_weight_2}
\end{subfigure}
\caption[Grafting trees.]{}
\label{rli-gen-fig-scion_grafting_wth_weight}
\end{figure}Just as we have adjusted the grafting operator, we must
adjust the Nakano
algorithm so it is aware of the extra data in an RLITT. The initial reaction to
this problem may be to simply annotate the scion of the grafting operation with
the weights necessary to reach a target TCN. Unfortunately, this method quickly
runs into issues generating even just the Montesinos tangles, a smaller class of
tangle described by \citep{bonahonNewGeometricSplittings2016}. We must make a
slightly more radical change to the Nakano algorithm, that being, grafting an
entire RLITT to only the root $v_0$ of the rootstock. We will now show that a
list of integral tangles (single vertex RLITT
Section~\ref{subsec-integral_tangle})
combined with grafting at the root generates all $\LP+\RP$-RLITT. To start, we
will prove that each $\LP+\RP$-RLITT is integral or the result of grafting
weighted planar tangle trees at the root.

\begin{theorem}{}{rli-gen-const-thm-acnm1toacn}Every $\Gamma$
$\LP+\RP$-RLITT of TCN $n$ is one of two forms:

\begin{enumerate}
\item $\Gamma$ is a single vertex with weight $\pm n$.
\item $\Gamma$ is the result of grafting at the root of some
  rootstock $\Gamma_r$
  and $\LP+\RP$-RLITT scion $\Gamma_s$ where:
  \begin{enumerate}
    \item In $\Gamma_r$, $v_0$ is valence two, and $v_1$ is canonical
      except for violating the stick condition by
      $\text{Sign}\LP v_0\RP=\text{Sign}\LP v_1\RP$. Each vertex in
      $\LS v_i\RS_{i=2}^n$ of $\Gamma_r$ is $\LP +\RP$-canonical.
    \item $\Gamma_r$ is $\LP+\RP$-RLITT
  \end{enumerate}
\end{enumerate}

\end{theorem}
\begin{proof}Let $\Gamma$ be a $\LP +\RP$-RLITT, we have three cases
based on the valence of
$v_0\in \Gamma$.

Valence of $v_0$ is:

\begin{enumerate}
\item \textbf{One:} $\Gamma$ is integral
  (Section~\ref{subsec-integral_tangle}), we fall into
  the first condition.

\item \textbf{Two:} $\Gamma$ is a stick at the root, we let
  $\Gamma_r=\iota\LB w_0\RB$
  and $\Gamma_s=\iota\LP\alpha w_1\RP$, where $\alpha$ is the remaining
  vertices, weights, and bonds of $\Gamma$. $\Gamma_r$ is integral, and as such
  is $\LP+\RP$-RLITT. Now to show that $\Gamma_s$ is $\LP+\RP$-RLITT. Since
  $\Gamma$ is $\LP+\RP$-RLITT, each $v_i$ in $\Gamma$ is canonical. Each vertex
  in $\Gamma_s$ is also in $\Gamma$, so $\Gamma_s$ has only canonical vertices
  and by Theorem~\ref{vertex-and-cannon} is $\LP+\RP$-canonical.
  Finally, since locations of
  weights are unchanged and grafting requires the scion to have identity label
  $\Gamma_s$ is $\LP+\RP$-RLITT.

\item \textbf{Three:} The root of $\Gamma$ has two children, we let
  $\Gamma_s$ be the tree
  with the right child of $v_0$ as a root and
  $\Gamma_r=\iota\LP \alpha w_0\RP$, where $\alpha$ is the remaining vertices,
  weights, and bonds of $\Gamma$. Showing $\Gamma_s$ is $\LP+\RP$-RLITT follows
  identically as it did in the first form.

  We now show $\Gamma_r$ is $\LP+\RP$-RLITT. First, given that $\Gamma$ is
  $\LP+\RP$-RLITT, each $v_i$ in $\Gamma$ is canonical, consequently each
  vertex that is unchanged in $\Gamma_r$ ($\LS v_i\RS_{i=2}^n$) is canonical.
  The vertices that differ in $\Gamma_r$ are $v_0$ and $v_1$, $v_0$ remains
  the root of $\Gamma_r$ so is canonical. Showing the canonicity of $v_1$
  depends on the valence of $v_1$. When the valence is 0 or is greater
  than 2, $\Gamma_r$ is $\LP+\RP$-RLITT by the same argument as $\Gamma_s$.
  When $v_1$ is valence 1 or 2, $\Gamma_r$ begins with a stick, and $v_1$
  continues to satisfy the weight, positivity, and the $\pm 1$ and 0 portion
  of the stick condition. For the sign condition, when the signs agree, we are
  in form 2.1 of this theorem, and when they disagree, we are in form 2.2.
  \newpage
\item \textbf{Greater than Three:} The final case is when the valence
  of $v_0$ is greater
  than three. We let $\Gamma_s$ be the tree with the right child of $v_0$ as a
  root and $\Gamma_r=\iota\LP \alpha w_0\RP$, where $\alpha$ is the remaining
  vertices, weights, and bonds of $\Gamma$. Both $\Gamma_r$ and $\Gamma_s$
  follow the same argument used for $\Gamma_s$ of the valence 1 case.
\end{enumerate}

\end{proof}An identical theorem can be phrased for the $\LP
-\RP$-RLITT case. With this
result, we can start our construction of a grafting algorithm for arborescent
tangles.

\begin{remark}{Find weighted planar trees by grafting RLITT scions to
the root of RLITT rootstocks}{find-grafted-trees}\textbf{Input}

\begin{itemize}
\item A collection of of RLITT scions $T_s$
\item A collection of RLITT rootstocks $T_r$
\end{itemize}

\textbf{Output}

\begin{itemize}
\item A collection of weighted planar trees
\end{itemize}

\textbf{Routine}

\begin{enumerate}
\item for each combination of $\Gamma_r\in T_r$ and $\Gamma_s \in T_s$
  \begin{enumerate}
    \item Compute $\Gamma_r\star \Gamma_s$
  \end{enumerate}
\end{enumerate}

\end{remark}Executing the algorithm with $\Gamma_r=\LS \iota\LB
1\RB\RS$ as the rootstock
and $\Gamma_s=\LS \iota\LB 2\ 0\RB\RS$ produces the tangle
$\iota\LB 2\ 0\ 1\RB$. Unfortunately, this resultant tangle violates the stick
condition and hence is not canonical. The remainder of this
subsection will refine
the grafting algorithm to satisfy each of the RLITT conditions.

\subparagraph{Weight Condition}

The simplest condition to verify is the weight condition. By construction,
grafting the rootstock and scion introduces no additional weights at the
grafting vertex. Meaning, the weight condition is satisfied with no adjustment
to the algorithm.

\subparagraph{Identity Condition}

The next RLITT condition to address is the identity condition. We note that the
$\star_i$ operation does not modify the $V_4$ label of the rootstock. This
observation means if the rootstock is identity, the grafted tree will also be
identity.

%  prettier-ignore-start

\subparagraph{Stick Condition}\label{rli-gen-sec-stick-con}

%  prettier-ignore-end

To start with the stick condition we will prove that if grafting produces a
non-canonical tree the non-canonical vertex must be adjacent to the root.

\begin{theorem}{}{only-the-root-matters}For $\Gamma_r$ a $\LP+\RP
-$RLITT and $\Gamma_s$ a $\LP+\RP -$RLITT scion,
the result of $\Gamma=\Gamma_r\star\Gamma_s$ is canonical, if all
$v_i$ at distance 1 or less from the root are canonical.

\end{theorem}
\begin{proof}We need to show that each vertex $v_i$ at a distance
greater than one from the
root of $\Gamma=\Gamma_r\star\Gamma_s$ is canonical. The vertex $v_i$ is also a
vertex of either $\Gamma_r$ or $\Gamma_s$. If the vertex is in $\Gamma_r$ then
$v_i$ has a parent in $\Gamma_r$ and if the valence of $v_i$ is 2 or more
$v_i$ also has children in $\Gamma_r$. The parent and children in $\Gamma$ are
the same as the parent and children in $\Gamma_r$. Since $\Gamma_r$ is RLITT
$v_i$ is canonical in $\Gamma_r$ and since it shares a parent and children in
$\Gamma$ it is also canonical in $\Gamma$. Similarly, for $v_i$ in the scion,
showing canonicity of grafting is dependent only on the canonicity of the
vertices at a distance up to one from the root.

\end{proof}The $\LP -\RP$-RLITT case is shown identically, covering
the majority of
possibilities. However, we need to take special care when grafting a
non-negative to a non-positive tree (or the reverse). Before we address that
case we define a restricted class of scions that, after grafting, satisfy the
nonzero portion of the stick condition.

\begin{definition}{}{}A $\LP+\RP -$RLITT (respectively $\LP+\RP
-$RLITT) $\Gamma$ with root weight $w_0$
is called a \textbf{good scion} when either:

\begin{enumerate}
\item $w_0\neq0$
\item $w_0=0$ and the valence of $v_0$ is greater than 2 (essential)
\end{enumerate}

\end{definition}
\begin{theorem}{}{positive-and-negative-dont-mix}For $\Gamma_r$ a
non-negative $\LP+\RP -$RLITT, and $\Gamma_s$ a good
non-positive $\LP -\RP -$RLITT scion, the result of
$\Gamma=\Gamma_r\star\Gamma_s$ is non-canonical.

\end{theorem}
\begin{proof}$\Gamma_r$ is a non-neutral $\LP+\RP -$RLITT, so it has
a non-root vertex $v_{i}$,
which is a stick of the form Figure~\ref{wpt-construc-fig-positivity_cond-1} or
Figure~\ref{wpt-construc-fig-positivity_cond-2}. Similarly,
$\Gamma_s$ is a non-neutral
$\LP+\RP -$RLITT, so it has a non-root vertex $v_{j}$, which is a stick of the
form Figure~\ref{wpt-construc-fig-negativity_cond-1} or
Figure~\ref{wpt-construc-fig-negativity_cond-2}.
Since, $v_i$ and $v_j$ are not at the root, they remain sticks of the form
Figure~\ref{wpt-construc-fig-positivity_cond-1},
Figure~\ref{wpt-construc-fig-positivity_cond-2},
Figure~\ref{wpt-construc-fig-negativity_cond-1}, or
Figure~\ref{wpt-construc-fig-negativity_cond-2}
after grafting. Making $\Gamma$ neither $\LP+\RP -$RLITT nor $\LP
-\RP -$RLITT, as
we desired.

\end{proof}

\newpage
\begin{remark}{Find weighted planar trees by grafting RLITT good
scions to the root of RLITT rootstocks}{find-grafted-good-trees}\textbf{Input}

\begin{itemize}
\item A collection of RLITT good scions $T_s$
\item A collection of RLITT rootstocks $T_r$
\end{itemize}

\textbf{Output}

\begin{itemize}
\item A collection of weighted planar trees (still not guaranteed to be RLITT)
\end{itemize}

\textbf{Routine}

\begin{enumerate}
\item for each combination of $\Gamma_s\in T_s$ and $\Gamma_r \in T_r$
  \begin{enumerate}
    \item Compute $\Gamma = \Gamma_r\star \Gamma_s$
    \item for each vertex $v_i$ at distance 1 from the root of $\Gamma$
      \begin{enumerate}
        \item Continue to the next iteration of the outer loop if
          $v_i$ fails to satisfy the stick condition
      \end{enumerate}

    \item Report $\Gamma$
  \end{enumerate}
\end{enumerate}

\end{remark}
%  prettier-ignore-start

\subparagraph{Positivity/Negativity Condition}\label{rli-gen-sec-pm-con}

%  prettier-ignore-end

Our approach to the positivity and negativity condition follows our approach to
the stick condition. We will leverage
Theorem~\ref{only-the-root-matters} to add a check for
positvity and negativity in our algorithm.

\begin{remark}{Find weighted planar trees by grafting
\textbf{$\LP+\RP$-RLITT good scions to the root of $\LP+\RP$-RLITT
rootstocks}}{find-grafted-good-trees-p}
\textbf{Input}

\begin{itemize}
\item A collection of $\LP+\RP$-RLITT good scions $T_s$
\item A collection of $\LP+\RP$-RLITT rootstocks $T_r$
\end{itemize}

\textbf{Output}

\begin{itemize}
\item A collection of weighted planar trees (still not guaranteed to be RLITT)
\end{itemize}

\textbf{Routine}

\begin{enumerate}
\item for each combination of $\Gamma_s\in T_s$ and $\Gamma_r \in T_r$
  \begin{enumerate}
    \item Compute $\Gamma = \Gamma_r\star \Gamma_s$
    \item for each vertex $v_i$ at distance 1 from the root of $\Gamma$
      \begin{enumerate}
        \item Continue to the next iteration of the outer loop if
          $v_i$ fails to satisfy the stick condition
        \item Continue to the next iteration of the outer loop if
          $v_i$ fails to satisfy the positivity condition
      \end{enumerate}

    \item Report $\Gamma$
  \end{enumerate}
\end{enumerate}

\end{remark}
\begin{remark}{Find weighted planar trees by grafting \textbf{$\LP
  -\RP$-RLITT good scions to the root of $\LP -\RP$-RLITT
rootstocks}}{find-grafted-good-trees-n}
\textbf{Input}

\begin{itemize}
\item A collection of $\LP -\RP$-RLITT good scions $T_s$
\item A collection of $\LP -\RP$-RLITT rootstocks $T_r$
\end{itemize}

\textbf{Output}

\begin{itemize}
\item A collection of weighted planar trees (still not guaranteed to be RLITT)
\end{itemize}

\textbf{Routine}

\begin{enumerate}
\item for each combination of $\Gamma_s\in T_s$ and $\Gamma_r \in T_r$
  \begin{enumerate}
    \item Compute $\Gamma = \Gamma_r\star \Gamma_s$
    \item for each vertex $v_i$ at distance 1 from the root of $\Gamma$
      \begin{enumerate}
        \item Continue to the next iteration of the outer loop if
          $v_i$ fails to satisfy the stick condition
        \item Continue to the next iteration of the outer loop if
          $v_i$ fails to satisfy the negativity condition
      \end{enumerate}

    \item Report $\Gamma$
  \end{enumerate}
\end{enumerate}

\end{remark}\subparagraph{Right Leaning Condition}

Satisfying the right leaning condition is a consequence of the modified
$\star_i$ operation. Our definition of the $\star_i$ operation grafts the scion
in such a way that weights at $v_i$ are always right of the scion. To fully
satisfy the right leaning condition, we need to ensure that any stick subtrees
of are in the right most postions. This is accomplished with a slight
modification of our grafting algorithms.

\begin{remark}{Find weighted planar trees by grafting
\textbf{$\LP+\RP$-RLITT good scions to the root of $\LP+\RP$-RLITT
rootstocks}}{find-grafted-good-trees-p-r}
\textbf{Input}

\begin{itemize}
\item A collection of $\LP+\RP$-RLITT good scions $T_s$
\item A collection of $\LP+\RP$-RLITT rootstocks $T_r$
\end{itemize}

\textbf{Output}

\begin{itemize}
\item A collection of RLITT
\end{itemize}

\textbf{Routine}

\begin{enumerate}
\item for each combination of $\Gamma_s\in T_s$ and $\Gamma_r \in T_r$
  \begin{enumerate}
    \item Compute $\Gamma = \Gamma_r\star \Gamma_s$
    \item Shift ring subtrees of the root of $\Gamma$ to the right
    \item for each vertex $v_i$ at distance 1 from the root of $\Gamma$
      \begin{enumerate}
        \item Continue to the next iteration of the outer loop if
          $v_i$ fails to satisfy the stick condition
        \item Continue to the next iteration of the outer loop if
          $v_i$ fails to satisfy the positivity condition
      \end{enumerate}

    \item Report $\Gamma$
  \end{enumerate}
\end{enumerate}

\end{remark}
\begin{remark}{Find weighted planar trees by grafting \textbf{$\LP
  -\RP$-RLITT good scions to the root of $\LP -\RP$-RLITT
rootstocks}}{find-grafted-good-trees-n-r}
\textbf{Input}

\begin{itemize}
\item A collection of $\LP -\RP$-RLITT good scions $T_s$
\item A collection of $\LP -\RP$-RLITT rootstocks $T_r$
\end{itemize}

\textbf{Output}

\begin{itemize}
\item A collection of RLITT
\end{itemize}

\textbf{Routine}

\begin{enumerate}
\item for each combination of $\Gamma_s\in T_s$ and $\Gamma_r \in T_r$
  \begin{enumerate}
    \item Compute $\Gamma = \Gamma_r\star \Gamma_s$
    \item Shift ring subtrees of the root of $\Gamma$ to the right
    \item for each vertex $v_i$ at distance 1 from the root of $\Gamma$
      \begin{enumerate}
        \item Continue to the next iteration of the outer loop if
          $v_i$ fails to satisfy the stick condition
        \item Continue to the next iteration of the outer loop if
          $v_i$ fails to satisfy the negativity condition
      \end{enumerate}

    \item Report $\Gamma$
  \end{enumerate}
\end{enumerate}

\end{remark}\paragraph{Full Generation Algorithm}

The algorithm we have developed so far generates new RLITT from two restricted
collections of trees. Unfortunately, it doesn't yet tell us how to select the
collections that guarantee the generation of all arborescent tangles up to a
target TCN are represented. The final step in the generation scheme is
describing how to build these collections. With computer enumeration in mind, we
would like for our strategy to be easily split into jobs that can be run in
parallel.

We observe that when grafting $\Gamma_r\star_i\Gamma_s=\Gamma$, the TCN $r$ of
$\Gamma_r$ and the TCN $s$ of $\Gamma_s$, sum to the TCN of $\Gamma$. This
observation is the key underpinning of the strategy we use to define discrete
generation jobs. For a target TCN we have the integer pairs seen in
(\ref{rli-gen-eq-buckets}) that sum to the target TCN. Each of these
pairs defines two classes, determined by TCN, of RLITT that can be grafted.

\begin{equation}
\label{rli-gen-eq-buckets}
\begin{aligned}
(0&,\text{TCN})\\
( 1&,\text{TCN}-1)\\
&\vdots\\
(\text{TCN}-1&,1)\\
(\text{TCN}&,0)
\end{aligned}
\end{equation}

The next question we should ask is if we can simplify the list at all. First, as
we saw in the discussion of the stick condition
Section~\ref{rli-gen-sec-stick-con}, we
need our scions to be good. This means we cannot have 1 or 0 in the second
position of the pair, excluding $(\text{TCN} -1,1)$ and
$(\text{TCN},0)$ from the
list. Second, the pair $(0,\text{TCN})$ will be excluded from our list, since
the zero-crossing tangle $\iota[0\ 0]$ can't serve as rootstock; grafting any
scion would violate the stick condition. We will recover tangles with root
weight 0 with a post-processing step.

The following is the recursive algorithm used to take us from the set of RLITT
with $\text{TCN} -1$ to the set of RLITT of the target TCN.

\begin{remark}{Find RLITT of given TCN from all RLITT of
TCN-1}{find-rlitt-from-acnm1toacn}\textbf{Input}

\begin{itemize}
\item A target TCN
\item All RILTT up to TCN-1
\end{itemize}

\textbf{Output}

\begin{itemize}
\item A set $T$ of all RLITT of TCN
\end{itemize}

\textbf{Routine}

\begin{enumerate}
\item Set $i=1$
\item Set $N=\text{TCN} -1$
\item Set $T$ to the set $\LS \iota[\text{TCN}], \iota[0\ \text{TCN}]\RS$
\item For each pair $(i,N -i)$
  \begin{enumerate}
    \item Set $T_{s+}$ to be the set of $+$-RLITT good scions with TCN
      equal to $\text{N} -i$
    \item Set $T_{s -}$ to be the set of $( -)$-RLITT good scions with TCN
      equal to $\text{N} -i$
    \item Set $T_{r+}$ to be the set of $(+)$-RLITT TCN equal to $i$
    \item Set $T_{r -}$ to be the set of $( -)$-RLITT TCN equal to $i$
    \item Execute Algorithm~\ref{find-grafted-good-trees-n-r} with
      input $T_{r -}$ and $T_{s -}$
    \item Add the results to $T$
    \item Execute Algorithm~\ref{find-grafted-good-trees-p-r} with
      input $T_{r+}$ and $T_{s+}$
    \item Add the results to $T$
  \end{enumerate}

\item For every RLITT $\Gamma$ in $T$
  \begin{enumerate}
    \item Continue to the next iteration of the loop if root of
      $\Gamma$ is valence two with weight zero
    \item Compute $\iota[0]\star\Gamma$ and add to T
  \end{enumerate}
\end{enumerate}

\end{remark}
\begin{remark}{Find RLITT up to a given TCN}{find-rlitt-up-to-acn}\textbf{Input}

\begin{itemize}
\item A target TCN
\end{itemize}

\textbf{Output}

\begin{itemize}
\item A set $T$ of all RLITT up to TCN
\end{itemize}

\textbf{Routine}

\begin{enumerate}
\item Set $T$ to be the set $\LS
  \iota[0],\ \iota[0\ 0],\ \iota[1],\ \iota[ -1],\ \iota[2],\ \iota[
  -2],\,\ \iota[2\ 0],\ \iota[ -2\ 0]\RS$
\item for i from 3 to TCN
  \begin{enumerate}
    \item Execute Algorithm~\ref{find-rlitt-from-acnm1toacn} with
      input TCN $i$ and RLITT set
      $T$.
    \item Add the results to $T$
  \end{enumerate}
\end{enumerate}

\end{remark}It is important to note that the output of
Algorithm~\ref{find-rlitt-up-to-acn} includes
duplicates in the form of $\LP+\RP$-RLITT and $\LP -\RP$-RLITT pairs. To
deduplicate our list so it contains only topologically unique objects, we select
from the list the collection of $\LP+\RP$-RLITT.


\chapter{Software and its Engineering}\label{ch-software}

We now shift from the mathematical world to the world of computers and software.
We start with a brief overview of the goals of product management and software
engineering (Section~\ref{sec-product-vs-project}). This includes the
development of model
software engineering processes tailored for use in professional and
undergraduate mathematics research. Next, we use the engineering concepts of
Section~\ref{sec-product-vs-project} to compare and contrast a
collection of tools currently
available for use in knot theory research
(Section~\ref{sec-surveyoftools}). Using the
analysis of current tools, and our software process we define a set of
requirements and a system design for a new general purpose knot theory software
toolbox (Section~\ref{sec-archofktst}). Finally, using this system
design, and following our
process, we present documentation for the software units that satisfy the
tabulation theory of Chapter~\ref{ch-tabulation}.

\section{Basics of Product Management and Software Engineering}

%  prettier-ignore-start

\subsection{Project vs. Product}\label{sec-product-vs-project}

%  prettier-ignore-end

One of the most pervasive ideas held by amateur software developers is that
documentation means simply commenting your code. This idea and its consequences
are one of the fundamental differentiators between a piece of software being a
project or becoming a product. A project is short-lived with no consideration of
long-term reuse or usage by a community. Much preferred is the creation of
products that are long-term, reusable, extensible, and portable investments of
time and energy (effort).

The creation of a product requires planning and documentation of a system far
beyond simple code comments. When developing a product, we must include
thoughtful and deliberate consideration of the goals and how we will achieve
them. For example, identifying where effort invested now to create a robust
system can save effort later, and enumerating and analyzing risk to the product
can reduce the chance of total product failure.

The creation of a software product has two phases, a generic (not software
specific) high-level product management/development phase followed by a detailed
software engineering phase. Having this two-pass approach allows for the product
to be well and thoroughly defined and assessed before any technical engineering
work is started, reducing the risk of a product being intractable based on the
product constraints. The remainder of this chapter will give context to product
management and software engineering practices, with preparing undergraduates for
research as the goal. We will start with a design for a course on teaching
project management to undergraduate researchers. We then develop a software
engineering process for use in mathematics research.

%  prettier-ignore-start

\subsection{Instruction of Product Management}\label{sec-product-management}

%  prettier-ignore-end

Undergraduate researchers, in most academic disciplines, have little exposure to
systematic product management. Outside a structured classroom setting, an
undergraduate researcher may have never completed a project, let alone worked on
a product. Giving undergraduate researchers a minimal set of product management
tools increases their ability to estimate and bring together successful
products.

In this section we describe a course design for a basic undergraduate course in
product management. The perspective of the course design is not software
focused, but is intended for general use across disciplines. In six weeks of
instruction, this course introduces students to the key concepts in product
management needed to ideate and systematically complete complex products. The
design utilizes the ``Backward Design'' of Wiggins and McTighe
\citep{wigginsUnderstandingDesign2008}. Many of the ideas come from Pressman and
Maxim \citep{pressmanSoftwareEngineeringPractitioners2015a} but have
been massaged to
be less software focused. Full course design and template documents are found
published on GitHub \citep{joestarrJoecstarrMfaCoBPMV1002025}.

\subsubsection{Stage 1 - Desired Results}

\paragraph{Established Goals}

Product management is an important part of being successful in academia as well
as industry. To take a product from ideation to a publishable product can be
easy when everything goes right. However, while completing a product, issues
often arise that impact the ability to meet product goals. This course will
introduce product management concepts that will enable practitioners to mitigate
these hidden impacts.

The overarching theme for the course is:

\begin{quote}
  Be kind to future you.
\end{quote}

\paragraph{Transfer}

\textit{Students will be able to independently use their learning to\dots}

Students should leave this course knowing that there are tools to reach for when
planning future products. Additionally, students should have a surface
understanding of at least one specific tool to manage each section of the
product life cycle.

\paragraph{Meaning}

\subparagraph{Understandings}

\textit{Students will understand that\dots}

\begin{itemize}
  \item Product creation is an accomplishable endeavor.
  \item Products have long lives, but projects die fast.
  \item ``Future you'' is a person that you need to collaborate with.
  \item Product documentation makes life easier.
  \item Stakeholder agreement is essential.
  \item Risk management from the ``Indiana Jones School of Risk
    Management'' - Rob
    Thomsett \citep{thomsettIndianaJonesSchool1992} is bad.
  \item Products fall behind ``One day at a time'' - Fred Brooks
    \citep{brooksMythicalManmonthEssays2013}
\end{itemize}

\subparagraph{Essential Questions}

\begin{itemize}
  \item How do you ideate for a product?
  \item How do you design a product plan?
  \item How do you evaluate risk?
  \item How do you create tasks for a product?
  \item How do you create a product schedule?
\end{itemize}

\paragraph{Acquisition}

\subparagraph{Students will know\dots}

\begin{itemize}
  \item How to use Crazy8s for ideation.
  \item How a basic product plan is used.
  \item How a risk management plan is used.
  \item How a product is scheduled.
\end{itemize}

\subparagraph{Students will be skilled at\dots}

Students will be able to utilize product management tools in practice.

\subsubsection{Stage 2 - Evidence and Assessment}

\paragraph{Evaluative Criteria}

\begin{itemize}
  \item Students have a well-formed product plan.
  \item Students have a well-formed risk management plan.
  \item Students have a well-formed product schedule.
  \item Students complete a product.
\end{itemize}

\paragraph{Assessment Evidence}

\subparagraph{Performance Task(s):}

Students will complete a collection of product management documents for their
product. The course will culminate with a 1:1 retrospective meeting where the
student will present their completed material and reflect on successes and
failures in their planning.

\subparagraph{Other Evidence:}

Students will have a touchpoint meeting every other week, with meeting minutes
summarized in a canvas reflection.

\subsubsection{Stage 3 - Learning Plan: \textit{Summary of Key
Learning Events and Instruction}}

The course will contain five to six weeks of instruction followed by
{\textasciitilde}10 weeks
of lightly supervised working time. The following is a high-level overview of
the course schedule:

\begin{enumerate}
  \item Week 1 and 2:
    \begin{enumerate}
      \item Product ideation
    \end{enumerate}

  \item Week 2:
    \begin{enumerate}
      \item A product plan
      \item Overview
      \item Assessment and control
    \end{enumerate}

  \item Week 3:
    \begin{enumerate}
      \item Requirements design
    \end{enumerate}

  \item Week 4:
    \begin{enumerate}
      \item Risk management
    \end{enumerate}

  \item Week 5:
    \begin{enumerate}
      \item Product schedule
    \end{enumerate}

  \item Week 7-15: Individual work
  \item Week 16: Retrospective meetings
\end{enumerate}

%  prettier-ignore-start

\subsection{Software Engineering and Life Cycle}\label{sec-life-cycle}

%  prettier-ignore-end

The second stage of software product development is the software engineering
process. Just as in the product management section
(Section~\ref{sec-product-management}), we
are approaching the software engineering process from an undergraduate training
perspective. However, unlike our discussion of general product management, we
assume some prior knowledge of software practices. We assume familiarity with
programming, types of programming languages, and the basic structure of a
program. If the reader feels unprepared, they may find it useful to complete one
of the many free asynchronous online courses offered by major
universities\footnote{At the time of writing ``Python for Everybody''
  by the University of Michigan
is a great choice.} and browse a standard intoduction to computation text such
as ``Introduction to the theory of computation'' by
Sipser\citep{sipserIntroductionTheoryComputation2013}. With this in
mind, we will
focus on engineering processes needed for our tangle tabulation use case,
omitting discussion of the practice of programming itself.

We will first define what steps we will take as part of our model process and
any quality gates\footnote{A collection of quality goals that need to
  be satisfied to call a step
``complete''.} to be satisfied before moving between those steps. We
call this collection of steps and transitions a \textbf{software life
cycle}. Rather
than reinventing the wheel, we will build on top of an existing life cycle
model. There are several existing models available to us, the most common
processes used in industry are agile models such as extreme programming
\citep{beckExtremeProgrammingExplained2012}
(Figure~\ref{fig-extreme-model}) and scrum
\citep{nonakaNewNewProduct1986} (Figure~\ref{fig-scrum-model}). Less
common in industry but
historically relevant are linear models such as the
waterfall \citep{beningtonProductionLargeComputer1983}
(Figure~\ref{fig-waterfall-model}) or the V
model \citep{forsbergRelationshipSystemEngineering1991}
(Figure~\ref{fig-v-model}).

\begin{figure}[H]
  \centering
  \begin{subfigure}[b]{\textwidth}
    \centering
    \includegraphics[width=0.5\textwidth]{files/extreme_programming-8468ea3e3b54f36cfb2d148b529104bc.pdf}
    \caption[The extreme programming agile life cycle.]{The extreme
    programing agile life cycle\citep{donwellsExtremeProgrammingsvgCC2010}.}
    \label{fig-extreme-model}
  \end{subfigure}
  \newline
  \centering
  \begin{subfigure}[b]{\textwidth}
    \centering
    \includegraphics[width=0.8\textwidth]{files/scrum_process-f682cc7dd6febc3e27c3f1f4d4f24b68.pdf}
    \caption[The scrum agile life cycle.]{The scrum agile life cycle
    \citep{lakeworksScrumProjectManagement2008}.}
    \label{fig-scrum-model}
  \end{subfigure}
  \caption[Life cycle diagrams of two agile process
  models.]{Life cycle diagrams of two agile process models.}
\end{figure}

\begin{figure}[H]
  \centering
  \begin{subfigure}[b]{\textwidth}
    \centering
    \includegraphics[width=.6\textwidth]{files/waterfall_process.dr-3297c507a9048d1c8c418fdc55ed172f.pdf}
    \caption[The waterfall process model.]{The waterfall process
    model. Note that the model allows no back tracking.}
    \label{fig-waterfall-model}
  \end{subfigure}
  \newline
  \centering
  \begin{subfigure}[b]{\textwidth}
    \centering
    \includegraphics[width=.7\textwidth]{files/v_process.drawio-44bb05aab7fdb452f2c04919e16b17da.pdf}
    \caption[The V process model.]{The V process model.}
    \label{fig-v-model}
  \end{subfigure}
  \caption[Life cycle phase progression models of two two linear process
  models.]{Life cycle phase progression models of two two linear
  process models.}
\end{figure}Agile processes have become ubiquitous in industry as
they allow for tight
feedback loops which ensures the product meets the needs of stakeholders,
reducing the risk of misunderstandings. In a research context, by the time
product software is being written, requirements and expectations for the
software are necessarily fully understood and well-defined. Consequently, and
contrary to industry trends, linear models are the most appropriate for research
contexts. While researchers can be expected to define well-considered
requirements, as amateur software engineers, it is rare that the design and
programming techniques are mature enough to support the strict progression of a
waterfall process. As such, a V model where downstream phases feedback into
previous phases is ideal. For our model process, however, we will make a single
change, disallowing feedback caused by down stream phases to change
requirements, as seen in Figure~\ref{se-fig-modifiedv}.
\begin{figure}[H]
  \begin{subfigure}[b]{.35\textwidth}
    \centering
    \includegraphics[width=\textwidth]{files/v_mod_process-380fbdc60ed09262d1d4071193280a64.pdf}
    \caption[The modified V process model.]{The modified V process model.}
    \label{se-fig-modifiedmultiv}
  \end{subfigure}
  ~
  \begin{subfigure}[b]{.6\textwidth}
    \centering
    \includegraphics[width=\textwidth]{files/v_multi_process-5a99d5c6ced50541d38c398c1f51bd68.pdf}
    \caption[The parallel V process model.]{The parallel V process model.}
    \label{se-fig-modifiedv}
  \end{subfigure}
  \caption[V process models.]{}
\end{figure}

Often software products are developed by teams of engineers. In these team
environments, it is important that the software process be easily
parallelizable.
Our modified V model can be parallelized as in
Figure~\ref{se-fig-modifiedmultiv}. Allowing
the process to be utilized by individual researchers at primarily undergraduate
institutions or large REU\footnote{Research Experiences for
  Undergraduates, a program funded by the National
Science Foundation (NSF).} projects.

The remainder of this section describes each phase of our modified V model. In
each subsection, we will describe the activities that should be carried out
during that phase, as well as an overview of what, if any, diagrams we should
expect to be created. The diagrams we will discuss for each phase are
simplifications of standard UML \citep{UnifiedModelingLanguage2017} diagrams.
Throughout the phases, we will use an implementation of the game of
``Go Fish'' as
an example software product.

\begin{note}
  Since the rules of ``Go Fish'' are highly non-standard, we will use the
  rules defined by Parlett \citep{parlettPenguinBookCard2009} as a common base.
\end{note}

%  prettier-ignore-start

\subsubsection{Requirements}\label{subsec-requirements}

%  prettier-ignore-end

Requirements define the conditions and behaviors that are expected out of a
system. Writing specific and non-ambiguous requirements is a surprisingly
difficult task, for example, when writing a set of requirements for
``Go Fish'' we
may define a requirement such as Example~\ref{se-fig-bad_req}.

\begin{example}{A requirement for a fishing action}{se-fig-bad_req}
  \begin{requirement}{ Player Goes Fishing}{}
    At the beginning of the active player's turn that player shall request a
    card from any other player.
  \end{requirement}

\end{example}

\begin{note}
  Observe the indicative mood used in Example~\ref{se-fig-bad_req}.
  The indicative mood, as in
  the use of ``shall,'' helps the designer reduce ambiguity by
  sharpening precision.
\end{note}

On its face, Example~\ref{se-fig-bad_req} seems to be a perfectly
good requirement phrasing
the ``fishing'' phase of a turn. However, if you put yourself in the shoes of a
person who has never played ``Go Fish,'' you might be confused by how
to ask for a
card. Can the active player ask for a 10, or should they ask specifically for
a $10\heartsuit$? Fixing this ambiguity in
Example~\ref{se-fig-bad_req} can by done by making
the requirement more precises Example~\ref{se-fig-bad_req_fixed}.

\begin{example}{An updated requirement for a fishing
  action}{se-fig-bad_req_fixed}
  \begin{requirement}{ Player Goes Fishing}{}
    At the beginning of the active player's turn that player shall request a
    card, by rank and suit, from any other player.
  \end{requirement}

\end{example}In a research context, the phrasing of a requirement, as in
Example~\ref{se-fig-bad_req_fixed} is often redundant. Most pieces of
software in a rigorous
mathematical context will have backing from theorems and definitions that
unambiguously define what the software should do. This means we must change how
we think about requirements in the research setting. Instead of requirements of
the style of Example~\ref{se-fig-bad_req_fixed} we phrase
requirements as \textbf{use cases}.

\begin{definition}{Paraphrasing Pressman and Maxim, Page 149
  \textbf{\citep{pressmanSoftwareEngineeringPractitioners2015a}}}{}
  A \textbf{use case} tells a stylized story about how an end user
  (playing one of a number
  of possible roles) interacts with the system under a specific set of
  circumstances. The story may be narrative text, an outline of tasks or
  interactions, a template-based description, or a diagrammatic representation.

\end{definition}In this context, we may rephrase
Example~\ref{se-fig-bad_req_fixed} as a use case, such as
Example~\ref{se-fig-use_case}.

\begin{example}{A usecase for a fishing action}{se-fig-use_case}
  \begin{usecase}{ Player Goes Fishing}{}
    A player asks another player for a specific card (rank and suit).
  \end{usecase}

\end{example}\paragraph{Use Case Diagram}

One popular way to phrase and visualize a collection of use cases is a use case
diagram \citep{UnifiedModelingLanguage2017}. A use case diagram shows
the connections
between actors (Player and Dealer in Figure~\ref{se-ex-usecase}) and
use cases (oval cells
in Figure~\ref{se-ex-usecase}). When an actor is connected to a use
case, we interpret that
connection as the user being able to initiate (kicking off the story the use
case tells) that use case. In Figure~\ref{se-ex-usecase} we have a
connection between use
cases, the ``include'' connection, this connection models a use case initiating
another use case. The ``include'' relationship is useful for generalizing
behavior, in Figure~\ref{se-ex-usecase} we see the \texttt{matching}
use case included in both the
\texttt{fishing} and \texttt{drawing} use cases.

\begin{figure}[H]
  \centering
  \includegraphics[width=\linewidth]{files/mermaid-9f149812-7c2265bd36529cda5fe32ef9ffe6a5b9.png}
  \caption[A use case diagram for Go Fish.]{A use case diagram for Go Fish.}
  \label{se-ex-usecase}
\end{figure}
%  prettier-ignore-start

\subsubsection{Software Design}\label{subsec-softearedesign}

%  prettier-ignore-end

After expectations of a system are set in the requirements phase, we can
decompose the problem into a software design. Pressman and Maxim
\citep{pressmanSoftwareEngineeringPractitioners2015a} outline eight principles
(Definition~\ref{se-def-8core}) that guide this process. This problem
decomposition tells us how
to break the software into discrete pieces of functionality called
\textbf{units}.
Depending on the team, their needs, and the technologies they are using, a unit
can be sized anywhere from a single function to a collection of files.
\newpage
\begin{definition}{Paraphrasing Pressman and Maxim, Section 7.2.2
  \textbf{\citep{pressmanSoftwareEngineeringPractitioners2015a}}}{se-def-8core}
  \begin{enumerate}
    \item Divide and conquer: You should break a hard problem into
      smaller solvable
      problems where possible.
    \item Understand the use of abstraction: Solving your problem is
      good, solving a
      more general version of your problem is better. Write software
      that hits the
      ``sweet spot'' of abstraction. Software that is not too general
      that it's hard
      to use for your specifc needs, and not to specialized that you
      can't use it
      again for a similar problem.
    \item Strive for consistency: It's easier to use/build intuition
      when choices are
      consistent. When you open a textbook, why is it easy to find the index?
      Because they are consistently in the same location.
    \item Focus on the transfer of information: Software, at its most
      basic, is about
      manipulating data. Knowing where that data moves and how it's consumed
      is key to understanding and contextualizing a problem.
    \item Build software that exhibits effective modularity: When
      decomposing a problem
      as in (1), the smaller problems should have low coupling (see
      Definition~\ref{se-def-coup}) and
      high cohesion (see Definition~\ref{se-def-coh}).
    \item Look for patterns: Look for ways common design patterns
      (see Definition~\ref{se-def-designpat}) can
      be used to solve the problem.
    \item When possible, represent the problem and its solution from a number of
      different perspectives: It's often the case that the first
      solution (obvious
      solution) is not the best/ideal solution. Approaching a problem from many
      perspectives helps identify alternative solutions.
    \item Remember that someone will maintain the software: ``Be kind
      to future you.''
      Spend an hour now to save days later.
  \end{enumerate}

\end{definition}
\begin{definition}{Paraphrasing Pressman and Maxim, Section 14.2.4
  \textbf{\citep{pressmanSoftwareEngineeringPractitioners2015a}}}{se-def-coup}
  \textbf{Coupling} is a qualitative measure of the degree to which units are
  connected to one another. As units become more interdependent, coupling
  increases.

\end{definition}
\begin{definition}{Paraphrasing Pressman and Maxim, Section 14.2.3
  \textbf{\citep{pressmanSoftwareEngineeringPractitioners2015a}}}{se-def-coh}
  Within the context of unit-level design for systems,
  \textbf{cohesion} implies that a
  unit encapsulates only attributes and operations that are closely
  related to one another and to the unit itself.

\end{definition}
\begin{definition}{Paraphrasing Pressman and Maxim, Page 349
  \textbf{\citep{pressmanSoftwareEngineeringPractitioners2015a}}}{se-def-designpat}
  A \textbf{design pattern} is an abstraction that prescribes a
  design solution to a
  specific, well-bounded design problem. A design pattern saves you from
  ``reinventing the wheel,'' or worse, inventing a ``new wheel'' that
  is slightly out
  of round, too small for its intended use, and too narrow for the
  ground it will
  roll over.

  As an example we give a common design pattern the \textbf{Iterator
  Pattern} From Gamma \citep{gammaDesignPatternsElements1995}:

  \textbf{Intent}

  Provide a way to access the elements of an aggregate object
  sequentially without
  exposing its underlying representation.

  \textbf{Motivation}

  An aggregate object such as a list should give you a way to access
  its elements
  without exposing its internal structure. Moreover, you might want to traverse
  the list in different ways, depending on what you want to accomplish. But you
  probably don't want to bloat the List interface with operations for different
  traversals, even if you could anticipate the ones you will need.
  You might also
  need to have more than one traversal pending on the same list.

  The Iterator pattern lets you do all this. The key idea in this pattern is to
  take the responsibility for access and traversal out of the list
  object and put
  it into an iterator object. The Iterator class defines an interface for
  accessing the list's elements. An iterator object is responsible for keeping
  track of the current element; that is, it knows which elements have been
  traversed already.

  \textbf{Applicability}

  Use the Iterator pattern:

  \begin{itemize}
    \item to access an aggregate object's contents without exposing its internal
      representation.
    \item to support multiple traversals of aggregate objects.
    \item to provide a uniform interface for traversing different
      aggregate structures
      (that is, to support polymorphic iteration).
  \end{itemize}

\end{definition}\paragraph{Diagrams}

\subparagraph{Block Diagram}

To diagrammatically represent a software design, we utilize a modified block
diagram. A block diagram gives a high-level description of the discrete units of
a software design and how those units relate to each other. The units are given
by blocks containing a descriptive title. Units that satisfy similar use cases
and may be abstractable to a common design are grouped together in container
blocks (\texttt{User Interface} in Figure~\ref{se-ex-block}).
Connections between blocks are
recorded with decorated arrows, the decorations indicate the multiplicity of the
relationship between components, such as \texttt{1} for a one-to-one,
\texttt{1..*} for a 1 to
many, or \texttt{*..1} for a many to 1 mapping.

\begin{figure}[H]
  \centering
  \includegraphics[width=\linewidth]{files/mermaid-9be24c8a-b6a23769af2e4792ccf4a75e5020712d.png}
  \caption[A block diagram for Go Fish.]{A block diagram for Go Fish.}
  \label{se-ex-block}
\end{figure}
\subparagraph{Sequence Diagram}

When the software design must account for communication between units, a
sequence diagram may be used. A sequence diagram records the actions and data
transfers that actors (units) take during a use case. A sequence diagram encodes
time on the vertical axis, with actors as vertical lanes. Interactions between
actors over time are indicated by annotated arrows between lanes. Arrow
annotations are a description of the interaction taking place. Conditional
sequences are indicated by boxes delimiting the possible sequences of the
interaction (\texttt{is negative} and \texttt{is positive}
subsequences in Figure~\ref{se-ex-seq}).

\begin{figure}[H]
  \centering
  \includegraphics[height=0.4\textheight]{files/mermaid-65edd29b-3f9def422ece398989b9a45e6815dbcb.png}
  \caption[A sequence diagram for Go Fish turn.]{A sequence diagram
  for Go Fish turn.}\label{se-ex-seq}
\end{figure}
%  prettier-ignore-start

\subsubsection{Unit Design}\label{sec-product-unit-design}

%  prettier-ignore-end

As we proceed down the left side of the V model, we narrow our focus,
progressively becoming less abstract and more concrete in our representation of
our software. The software design gives us a description of what units we will
need, but does not give an actionable description of those units. A unit
description gives that actionable description, a language agnostic but directly
implementable (programmable) description for what a unit contains and how a unit
works.

A description for a unit contains the data members (variables) and the
functional members a unit contains. Each of these items is described in two
ways, first a plain English description, and second included diagrammatically in
a class diagram \citep{UnifiedModelingLanguage2017} for the unit. The
plain English
description for each member describes what that member is intended to be or do.
In the case of a functional member, along with an English description, a
diagrammatic definition as a state
machine\citep{UnifiedModelingLanguage2017} should
also be given.

As with the special language/mood used when designing requirements
(Section~\ref{subsec-requirements}), documentation at the unit level
and below have a
particular frugal and direct style. When this style is first encountered it can
be jarring, and the writing appear shockingly poor. However, just as
the use of the
indicative mood in requirements encourages precision, the style of documentation
at this level serves a purpose. Documentation at this level is not intended to
tell a story or describe a problem, but to directly define a computational unit
one layer of abstraction above code. Simple direct language here encourages the
clear communication of expectations and intents that translate to code.
Complicated ideas, at this level, are easier to express (and consume) when
presented diagrammatically. Our English descriptions serve simply to supplement
the diagrams.

\begin{note}
  Writing long, elaborate explanations at the unit level should prompt you to
  engage in analysis of your design. The need for explanations like this
  often indicates you are missing something at the software design
  level, look for
  missing abstraction or a place to divide and conquer.
\end{note}

\paragraph{Diagrams}

\subparagraph{Class Diagram}

$\,$

A class diagram \citep{UnifiedModelingLanguage2017}, containing a collection of
blocks and their relationships, similar to a block diagram. However, in a class
diagram, the blocks are functional subunits of a unit. Each block contains a set
of data members followed by a set of functional members. In an object aware
language, such as Python or Java, a block might directly correspond to a class
in the language sense. In an imperative language, such as C, the blocks may
correspond to one or more \texttt{.c} or \texttt{.h} files.

Each member of the class is decorated to indicate public or private visibility,
meaning, visibility to the world outside the unit. A $+$ is used to indicate
public visibility, meaning the member can be seen and used by other classes, and
a $-$ to indicate private visibility, meaning the member can only be seen from
inside the class. When referencing other units in the system, the external units
are truncated to an empty class (as in the \texttt{Card} class in
Figure~\ref{se-cd-fig-agg}). One or
more optional decorators can be added to a class to further contextualize the
class. The decorators that we allow in a class diagram are:

\begin{enumerate}
  \item Enumeration: Indicating the class is an enumeration.
  \item Interface: Indicating the class is an interface. This is usually used to
    simplify the diagram when common collections of data/functions need to be
    repeated.
  \item External: Indicating the class is defined outside the current unit.
\end{enumerate}

The relationships between classes are described by arrows between those classes.
Each type of arrow used in a class diagram defines a slightly different type of
relationship. The arrows that we will allow in a class diagram are the
aggregation (Figure~\ref{se-cd-fig-agg}) and the realization
(Figure~\ref{se-cd-fig-real}). The
aggregation connection describes the relationship when one class uses
(aggregates) the connected class.

\begin{figure}[h]
  \centering
  \begin{subfigure}[c]{0.45\textwidth}
    \centering
    \includegraphics[width=0.2\textwidth]{files/mermaid-1e7aaf39-93e3af7002f2c2a71b2599db1135755f.png}
    \caption[An aggregation connection. ]{An aggregation
    connection.\\Class A aggregates Class B.}
    \label{se-cd-fig-agg}
  \end{subfigure}
  \quad
  \quad
  \quad
  \centering
  \begin{subfigure}[c]{0.45\textwidth}
    \centering
    \includegraphics[width=0.2\textwidth]{files/mermaid-03ad04eb-9aab69420f4255ea41d0c4db56d31209.png}
    \caption[A realization connection.]{A realization
    connection.\\Class A realizes Class B.}
    \label{se-cd-fig-real}
  \end{subfigure}
  \caption[Types of class diagram connections.]{}
\end{figure}

In Figure~\ref{se-ex-class}, a class diagram for a Go Fish
player, we see the \texttt{Player} class using (aggregating) the
\texttt{Hand} class. In
Figure~\ref{se-cd-fig-real} we see the realization connection, which
describes the
relationship when a class implements an interface. An example can be seen in
Figure~\ref{se-ex-class} with the \texttt{Hand} and
\texttt{Book}\footnote{A book is the matched sets of cards that are
  counted at the end of Go Fish
to determine the winner.} classes realizing the
\texttt{CardCollection} interface. Both the \texttt{Hand} and
\texttt{Book} classes are collections
of multiple cards and will need common data, such as a function to print all
cards in the collection. This common (expected) data is modeled as an interface
that can be reused without needing to be rewritten.
\newpage
\begin{note}
  In some object-oriented languages, this relationship may be defined by an
  interface or with inheritance. However, in languages that are not
  object oriented
  such as C, we instead define this common set of data as an abstract interface
  for design purposes that we must implement in each component that realizes the
  interface.
\end{note}

\begin{figure}[H]
  \centering
  \includegraphics[height=0.7\textheight]{files/mermaid-9b06e87f-775f95fc702d1a5535ed3b0f46fad1e5.png}
  \caption[A class diagram for a Go Fish player.]{A class diagram for
  a Go Fish player.}\label{se-ex-class}
\end{figure}
\newpage
\subparagraph{State Machine}

A state machine \citep{UnifiedModelingLanguage2017} diagram describes
the collection
of states and transitions that a function can move between. When defining a
state machine, we start with two special states. The first special state is the
start state, indicated by a filled in circle, and the second special state is
the end state, indicated by a filled in circle surrounded by a ring. Other
states are recorded by a box with text describing the state. Decision points are
documented with a diamond, with each path of the condition decorated with text.

\begin{figure}[H]
  \centering
  \includegraphics[height=0.5\textheight]{files/mermaid-4ae3bf05-a0b4390a5e0e6be01a54489d5b62bb22.png}
  \caption[A state machine diagram for a Go Fish turn.]{A state
  machine diagram for a Go Fish turn.}\label{se-ex-sm}
\end{figure}
\subsubsection{Implementation}

Once the design of a unit is complete, we are ready to implement (program) that
unit. There are no special activities in the implementation phase outside of
programming effort. It is important to strictly follow the unit and system
design in this phase. If deficiencies in design are found, the V model allows
for that feedback to be pushed to earlier phases and addressed by flowing back
through the V model. Good practice in the programming stage is to add a comment
describing each unit implementation as well as each data an functional memeber.
The goal of these comments is to briefly describe the code for future readers,
but they cannot replace the written documentation from previous sections.
Additional good practices are to consistently format the code with a common
style, utilize good data hygiene with a version control system, and favor simple
implementations over ``clever''.

\subsubsection{Unit Testing}

With implementation complete, we move to the right side of the V, which consists
of the verification activities. The first verification phase is the isolated
verification of each unit, called \textbf{unit testing}. In this
phase, we design and
carry out tests of the units we have implemented. When designing tests, it is
important to draw from the unit design directly, instead of designing tests
against the actual implemented code. Testing the code against the design ensures
that we are not drawing the target around the arrow. Tests, particularly unit
tests, can and should, be separated into two classes of tests. The first class
is those on the ``happy path'', validating that ``good'' well-formed
input generates
expected output, see Example~\ref{se-ex-happyut}. The second class
are those on the ``unhappy
path'', validating that ``bad'' or malformed input is handled as expected, see
Example~\ref{se-ex-unhappyut}. Each public, accessible to external
units, interface of a unit
should have at least one unit test. If validation in this, or the following
phases, finds deficiencies in the design, the V model allows for that feedback
to be pushed to earlier phases and addressed by flowing back through the V
model.

\paragraph{Test Cards}

As we walk up the right side of the V, instead of diagrams we utilize test
cards. A test card is used to define a set of requirements that a test will be
implemented against. Each test card has four fields with content as follows:

\begin{itemize}
  \item \textbf{Test Name}: The unique name for the test. This name
    is used in test
    reports to identify what has failed.

  \item \textbf{Description}: The description of what the test is
    validating and how that
    validation works. This section may include text and diagrams.

  \item \textbf{Inputs}: A set of inputs to feed the unit.

  \item \textbf{Outputs}: The outputs that are expected when the
    inputs are fed to the
    unit.
\end{itemize}

The following are two examples of unit test cards for the Go Fish product. One
test card is on the happy path, and one is on the unhappy path.

\begin{example}{A happy path test card for a player turn in Go
  Fish}{se-ex-happyut}
  \begin{testcard}{ Request A Card, Reponse Is Positive}{}
    \textbf{Description}: Active player requests a card from a target
    player. The target
    player has the requested card and produces it.

    \textbf{Inputs}:

    \begin{itemize}
      \item A valid requested card
      \item A valid target player with requested card in hand
    \end{itemize}

    \textbf{Outputs}:

    \begin{itemize}
      \item Active player puts the received card in hand
    \end{itemize}
  \end{testcard}

\end{example}

\begin{example}{A unhappy path test card for a player turn in Go
  Fish}{se-ex-unhappyut}
  \begin{testcard}{ Illegally Request A Card}{}
    \textbf{Description}: Active player requests a card that is not
    in their hand.

    \textbf{Inputs}:

    \begin{itemize}
      \item An invalid requested card
    \end{itemize}

    \textbf{Outputs}:

    \begin{itemize}
      \item Active player is notified the request failed
    \end{itemize}
  \end{testcard}

\end{example}
\subsubsection{Integration Testing}

While working in software, it is rare for a system to have a single unit or a
collection of units that are completely uncoupled
(Definition~\ref{se-def-coup}). When we do
have units that depend on each other, we call the process of sticking
those units
together, \textbf{integration}. We verified during the unit tests
that implemented
units work individually as expected. In the integration testing phase, we verify
that implemented units work together as expected. Tests in this phase should be
designed against the artifacts from the software design phase
(Section~\ref{subsec-softearedesign}) as well as the use cases
defined in the requirements
phase (Section~\ref{subsec-requirements}).

\subsubsection{Acceptance Testing}

When we have completed unit and integration testing, we have verified that the
software, based on the design, doesn't unexpectedly break and satisfies written
requirements (use cases). However, we have not yet verified that the system
satisfies what all stakeholders wanted. In our Go Fish example, this phase may
include a presentation to a customer. This is the point where a customer may
find that our interpretations of Example~\ref{se-fig-bad_req}
disagree, and we update to
Example~\ref{se-fig-bad_req_fixed}.

%  prettier-ignore-start

\section{Survey of Knot Theory Software Tools}\label{sec-surveyoftools}

%  prettier-ignore-end

Having now established a model process for developing software we can start our
efforts in developing a design for a general knot theory software toolbox. To
begin, we explore a collection of tools currently in use in the computational
knot theory space. Considering the strengths and shortcomings of these tools
helps inform the requirements and possible use cases for our general knot theory
software toolbox.

\paragraph{KnotPlot}

KnotPlot\citep{schareinInteractiveTopologicalDrawing1998} is a closed
source\footnote{Closed source software is a software product with no
published source code.}
knot computation tool primarily and widely used for diagramming knots, this
includes several diagrams in this thesis. Knotplot was developed as a portion of
Dr. Robert Scharein's PhD thesis
\citep{schareinInteractiveTopologicalDrawing1998},
he is also the primary maintainer. The primary interface for KnotPlot is by
interacting with a GUI\footnote{A graphical user interface, such as a
  Windows or the screen on a copy
machine.}. This makes the onboarding process for
non-technical users, including undergraduate researchers, straightforward and
the learning curve shallow. Knotplot includes many export options and
specialized research computational tools, some undocumented. While support is
readily available from Dr. Scharein this can make power use of KnotPlot
difficult. KnotPlot has recently added a programmatic interface allowing general
scripting in the programming language Lua. This scripting interface allows for
custom diagramming and custom components. These custom scripts are limited
however by the hooks available in KnotPlot. It is not always clear what calls to
make into KnotPlot to accomplish your goals. Additionally, since the Lua
interface is compiled into KnotPlot and isn't a language widely used in
research, finding supported external libraries can be difficult. KnotPlot is an
unparalleled knot diagramming tool, but the environment becomes a walled garden
that can be difficult to build into a tool chain.

\paragraph{Mathematica Libraries}

%  (Mathematica https://katlas.org/wiki/Printable_Manual)

KnotTheory \citep{bar-natanKnotTheory} and LinKnot
\citep{jablanLinKnotKnotTheory2007} are
collections of knot theory tools and datasets that are used with the Wolfram
Mathematica system. Both KnotTheory and LinKnot are open
source\footnote{Open-source software is a software product with
  publicly published source
code.} and
available for download. Mathematica is a natural environment for mathematics
research, as the language syntax is similar to written theoretical mathematics.
This similarity of appearance lowers the learning curve for use of the tooling.
Mathematica is also commonly used in undergraduate calculus sequences, making
the onboarding of undergraduate researchers straight forward. The interfaces
defined by both KnotTheory and LinKnot are natural in the Mathematica context,
further shallowing the learning curve for those already familiar with
Mathematica. Additionally, both collections are well documented by the website
``knot atlas'' \citep{bar-natanKnotTheory} and the textbook ``LinKnot''
\citep{jablanLinKnotKnotTheory2007} for KnotTheory and LinKnot, respectively.
Unfortunately, neither collection is under active development or maintenance,
with the last published versions from 2016 for KnotTheory and 2011 for
LinKnot. Additionally, neither collection is released under version control, has
a bug log, or has a published test suite. Missing the traceability implicit in
these artifacts makes program correctness something that must be considered when
using the collections.

%  prettier-ignore-start

\paragraph{SnapPy/SnapPea}\label{sec-surveyoftools-snap}

%  prettier-ignore-end

SnapPy \citep{SnapPy} is a Python wrapper for SnapPea, which is a
collection of C
libraries. SnapPy and SnapPea are open source and published in the same GitHub
repository which includes bug reporting and test suites. SnapPy's Python
bindings allow researchers to leverage the massive Python ecosystem, allowing
for wide and varied usage. SnapPy can be utilized in anything from a simple
command line tool, to a GUI tool to draw knots, to a webapp doing knot
computations. This allows SnapPy to be tailored to the needs of individual
researchers, allowing for simple subsets of functionality to be presented to
undergraduate researchers. Each of these uses benefits from the decoupling of
SnapPy and the core logic in SnapPea, allowing for fast execution of the SnapPea
C code, but simple supportable Python bindings. The SnapPea layer, however, is
designed on an ad hoc basis, meaning each file/unit stands alone with little
structural overlap. If a developer wants to reuse the SnapPea C layer directly,
they must reverse engineer the structures they wish to reuse.

%  prettier-ignore-start

\section{Architecture of A Knot Theory Software Toolbox}\label{sec-archofktst}

%  prettier-ignore-end

In Section~\ref{sec-surveyoftools} we discussed prior art in the knot
computation space, in
this section we utilize that analysis to design a software architecture for a
general knot theory software toolbox. To complete this design, we will execute
the first two phases of the modified V model we developed in
Section~\ref{sec-life-cycle}.

\subsection{Requirements}

In this section we carry out the requirements phase of the modified V model. We
will create a set of requirements and use cases that model the expectations we
have for a general knot theory software toolbox. First, we would like our system
to be easy to use. As we saw, easy to use can encompass various possibilities.
We can capture all of these possibilities by decoupling the theoretical
functionality from the user interfaces, and instead we will implement specific
interfaces for specific users.

\begin{requirement}{ User Interface Goals}{}
  The system shall not couple functionality to user interface.
\end{requirement}

This design goal allows the interface to be a Jupyter notebook during
undergraduate knot theory class, a Mathematica library for research, or a tool
run on a university cluster for high-performance needs. With any possible target
environment as a goal, the system must not be coupled to a particular platform
(Windows, Linux, etc.), informing our second requirement.

\begin{requirement}{ Portability Goals}{}
  The system shall be platform (OS, language, toolchain, I/O (Input
  and Output)) agnostic.
\end{requirement}

The tools we saw in Section~\ref{sec-surveyoftools} are all what we
may call monolithic,
meaning, from a development perspective, if the tool is to be used as a part in
a new system, the whole tool must be included. In general, it is preferable to
include only the functionality a system actually requires. For example, consider
a system is being built to teach a seminar on constructing the Jones polynomial.
That system will need to include a Jones polynomial component. Needing to pull
in at the same time a hyperbolic volume component that serves no purpose,
needlessly increasing the complexity of the system. These extensibility goals
inform an encapsulation design goal.

\begin{requirement}{ Encapsulation Goals}{}
  System use cases shall be encapsulated into feature components.
\end{requirement}

Encapsulating each feature allows for the system components to be used by
developers to build projects. However, as we saw in our discussion of SnapPy
(Section~\ref{sec-surveyoftools-snap}), encapsulation itself does not
remove all difficulty
in reuse. We can further lower the difficulty of reuse by increasing commonality
between components without coupling the components. This is accomplished by
ensuring that every system component adheres to a set of common design patterns.

\begin{requirement}{ Pattern Goals}{}
  System use cases shall adhere to a set of design patterns.
\end{requirement}

Finally, we can further reduce the overhead of reuse we enforce a common
development process on system components.

\begin{requirement}{ Documentation Goals}{}
  System use cases shall be documented and planned as outlined by the modified V
  model (Section~\ref{sec-life-cycle}).
\end{requirement}

We now develop a collection of use cases that address the high-level behaviors
expected by the general knot theory software toolbox.

\begin{usecase}{ Use The Software}{}
  A user interacts with the system.
\end{usecase}

\begin{usecase}{ Data Is Manipulated}{}
  Data in the system is manipulated into somthing different.
\end{usecase}

\begin{usecase}{ Data Is Created}{}
  New data in the system is created.
\end{usecase}

\begin{usecase}{ Data Is Written}{}
  Data is written down for future use.
\end{usecase}

\begin{figure}[H]
  \centering
  \includegraphics[width=\linewidth]{files/mermaid-b6bfc2a4-6a25e6f7cf9bed78913e1a8e65a9cbb2.png}
  \caption{A use case diagram for the listed use cases.}
\end{figure}

%  prettier-ignore-start

\subsection{Software Design}\label{sec-system_design}

%  prettier-ignore-end

With a set of requirements for our design, we can now describe a software
design. Each requirement can be partitioned into one of two classes of
requirements, functional or non-functional. Where, a functional requirement can
impact the implementation of code and a non-functional requirement cannot. We
start by addressing a software design for the non-functional requirements:

\begin{itemize}
  \item User Interface Goals
  \item Portability Goals
  \item Documentation Goals
\end{itemize}

The user interface goal is satisfied by simply excluding any user interface
design from the software. The portability goal tells us that we need to pick a
technology stack that is supported on the maximal number of platforms. The clear
choice is to implement the software in the C language. The C language is widely
used, and a C compiler exists to target just about any platform. The following
is a selection of C compilers and tool chains and their targets:

\begin{itemize}
  \item Cython\citep{behnel2011cython}: ``Cython is a Python compiler
    that makes writing C
    extensions for Python as easy as Python itself. Cython is based on Pyrex,
    but supports more cutting edge functionality and optimizations.'' - Cython
    Documentation
  \item GNU\citep{GCCGNUCompiler2025}: A C compiler that can has
    around 200 targets
    including
    \begin{itemize}
      \item x86\_64
      \item ARM
      \item Motorola 68000
      \item PowerPC
    \end{itemize}

  \item
    Emscripten\citep{zakaiEmscriptenLLVMtoJavaScriptCompiler2011}: Compiles C to
    WebAssembly\citep{WebAssemblyCoreSpecification2} allowing for C
    code to be used
    in web systems.
  \item Embedded compilers: Various compilers for esoteric embedded
    systems.\footnote{An embedded system is a small, usually low
      power, computer (microcontroller)
      that is built into a product. For example, the power seat of
      your car is an
    embedded system.}
\end{itemize}

Selecting C for an implementation language has some risks, primarily caused by
the low-level\footnote{A low-level language such as C or Rust
  compiles to machine code that runs
  directly on the hardware. A high-level language is one that abstracts
functionality away from hardware.} nature of C. As a low-level
language, C has no
built-in garbage collection mechanism, that is, memory can be allocated but is
not unallocated automatically. To mitigate this risk in C development we
disallow memory allocation in our components. Any memory allocation must happen
in the user interface layer, and then buffers with known sizes are passed to the
components.

To address the two remaining functional requirements, encapsulation goals and
pattern goals, we define a collection of generic component interfaces that cover
common use cases we described in the previous section. We also model these
interfaces as a block diagram demonstrating the relationship between the
interfaces.

\begin{itemize}
  \item Runner: The runner interface serves as the stand-in for user interfaces.
    Since we are decoupling the interface from the functionality, there is no
    further design consideration for the runner interface.
  \item Runnables: Runnables serve as the functionality that is
    called by a runner.
    \begin{itemize}
      \item Computation: The computation interface is a runnable that
        operates in a
        one call, one return program flow (one thing in one thing out). Example:
        Compute Jones Polynomial and Compute Writhe.
      \item Generator: The generation interface is a runnable that
        operates in a one
        call, multiple return flow (one thing in many things out). Example:
        generate rational tangles, generate Montesinos tangles, and generate
        arborescent tangles.
    \end{itemize}

  \item Data Wranglers: Data wranglers serve as a non-user facing layer used by
    runnables to handle data.
    \begin{itemize}
      \item Notation: The notation interface defines data structures used to
        represent knot data. Additionally, the interface describes the
        translation between string representations and data structures. Example:
        Conway notation, algebraic tangle tree notation, and weighted planar
        tangle tree notation.
      \item Storage: The storage interface serves the need for
        components to read
        and write from external systems. Examples: the command line or a
        database.
    \end{itemize}
\end{itemize}

\begin{figure}[H]
  \centering
  \includegraphics[width=\linewidth]{files/mermaid-606404c6-6d0191c2858c7930f917b0f7b763e040.png}
  \caption{A block diagram of the architecture of a knot theory
  software toolbox}
\end{figure}

\subsection{Unit Design for Generic Interfaces}

Based on the system design, we will now give unit designs for each of our
generic interfaces. Each interface has a brief description of the component, a
class diagram for the component, and a brief description of each data member and
function.

%  prettier-ignore-start
\subsubsection{Generator Interface}\label{sec-interfaces-generator}

%  prettier-ignore-end

The generator interface defines the general form for a component used to perform
a knot operation. Then when the generator component is invoked, it produces more
than a single output. The generator component does not allocate memory, it must
be configured with sufficient buffer space to successfully execute.
\paragraph{Class Diagram}

$\,$

\begin{figure}[H]
  \centering
  \includegraphics[height=0.3\textheight]{files/mermaid-5a8e90a2-8a299f21031f3a7df7deb9dc167483dd.png}
\end{figure}

\paragraph{Functionality}

\subparagraph{Public Structures}

\subparagraph{Generator Configuration Structure}

The generator configuration structure defines the collection of data the
component needs for a single run. Setting a configuration should be considered
equivalent to instantiating a class in a high-level language. However, in this
case, there is only ever a single active instance of the class.

\subparagraph{Public Functions}

\subparagraph{Configuration Function}

The function will take a configuration as input and set the local configuration
instance to that input. The function returns a flag indicating whether the
function was successful. This function can be considered analogous to
the \texttt{init}
function of a class in a high-level language.
\newpage
\subparagraph{Generate Function}

When this function is invoked, the generation process begins. The actual
internal functionality is specific to the specific generator. The function
returns a flag indicating whether the function was successful.

The flow for a generator is modeled by the following state machine:

\begin{figure}[H]
  \centering
  \includegraphics[height=0.4\textheight]{files/mermaid-18bba444-2a831e2dcdf1f4d7d6b872c9fe1e1af5.png}
\end{figure}

%  prettier-ignore-start

\newpage
\subsubsection{Computation Interface}\label{sec-interfaces-computation}

%  prettier-ignore-end

The computation interface defines the general form for a component used to
perform a knot operation. When the computation component is invoked, it produces
a single output. The computation component does not allocate memory, it must be
configured with sufficient buffer space to successfully execute.

\paragraph{Class Diagram}

$\,$

\begin{figure}[H]
  \centering
  \includegraphics[height=0.4\textheight]{files/mermaid-8aaebd29-c1f3ce664183cdcf75ffaa94ffab8126.png}
\end{figure}

\paragraph{Functionality}

\subparagraph{Public Structures}

\subparagraph{Computation Configuration Structure}

The computation configuration structure defines the collection of data the
component needs for a single run. Setting a configurationshould be considered
equivalent to instantiating a class in a high-level language. However, in this
case, there is only ever a single active instance of the class.

\subparagraph{Computation Result Structure}

The computation result structure defines the collection of data the component
will produce in a single run. This is used as an alternative to the write
interface, allowing the component to be used internally in other computation or
generator components.
\newpage
\subparagraph{Public Functions}

\subparagraph{Configuration Function}

The function will take a configuration as input and set the local configuration
instance to that input. The function returns a flag indicating whether the
function was successful. This function can be considered analogous to
the \texttt{init}
function of a class in a high-level language.

\subparagraph{Compute Function}

When this function is invoked, the computation process begins. The actual
internal functionality is specific to the specific computation. The function
returns a flag indicating whether the function was successful.

The flow for a computation is modeled by the following state machine:

\begin{figure}[H]
  \centering
  \includegraphics[width=0.47\linewidth]{files/mermaid-6a25d18a-8170bf662108fc27b7e4b0657d910a16.png}
\end{figure}

\subparagraph{Result Function}

When this function is invoked, the result of the computation process is
reported. The actual internal functionality is specific to the specific
computation.

%  prettier-ignore-start
\newpage
\subsubsection{Notation Interface}\label{sec-interfaces-notation}

%  prettier-ignore-end

The notation interface defines the general form for a component used to store
knot notational data. This includes a computational data structure for a knot
notation and the functions required for translation into and out of string
representations of that notation.

\paragraph{Class Diagram}

$\,$

\begin{figure}[H]
  \centering
  \includegraphics[height=0.25\textheight]{files/mermaid-2994999e-50bad96723d99e60bd6c0c09d441efc6.png}
\end{figure}

\paragraph{Functionality}

\subparagraph{Public Structures}

\subparagraph{Notation Data Structure}

This is the primary data structure for a notation component. This data structure
defines and stores the computational representation of a knot notation.

\subparagraph{Public Functions}

\subparagraph{Encode Function}

The encode function takes in the string representation of a knot notation,
processes the string, and stores the computational representation into a
notation data structure.

\subparagraph{Decode Function}

The decode function takes in a computational representation of a notation data
structure and processes it into a string representation of a knot notation.

%  prettier-ignore-start
\newpage
\subsubsection{Storage Interface}\label{sec-interfaces-storage}

%  prettier-ignore-end

The storage interface defines the general form for a component that reads and
writes key value pairs. These functions will generally be defined by the user
interface layer and passed to components as function pointers. Data is assumed
to be formatted in a key value store with two layers. The outermost layer is
indexed by a value called a \textbf{key}. The value for the key is
another collection
of key value pairs. At this level, we call the key value is called an
\textbf{index}
and value entry \textbf{value}.

\begin{example}{A JSON data store with the key:index:value structure.}{}
  \begin{center}
\begin{lstlisting}[language=json,numbers=none]
{
    "key": {
        "index": "value"
    },
    "[1 1 1]": {
        "crossing_number": "3",
        "is_rational": "true"
        }
}
\end{lstlisting}
  \end{center}
\end{example}

\paragraph{Class Diagram}

$\,$

\begin{figure}[H]
  \centering
  \includegraphics[height=0.2\textheight]{files/mermaid-6edd1b25-eb992c6387dbe86df4026001d05a54ba.png}
\end{figure}

\paragraph{Functionality}

\subparagraph{Public Functions}

\subparagraph{Read Function}

The read function takes in a key index pair, then reads the value at key:index
from the data store and returns the value.

\subparagraph{Write Function}

The write function takes in a key, index, and value tuple and writes the value
to the data store at key:index.

%  prettier-ignore-start
\newpage
\section{Unit Design for Tangle Tabulation}\label{sec-unitdesign}

%  prettier-ignore-end

The unit descriptions in this section are living documents with
latest version as well as their associated implementations available on
GitHub~\citep{Starr_The_Tanglenomicon_Core_2025}.


\chapter{Future Directions and Undergraduate
Research}\label{ch-future-directions}

In this chapter, we describe the future research directions for the tabulation
of tangles. The future directions take two forms. First, the direct next steps
to the work of this thesis
(Section~\ref{sec-future_work-continued_tabulation}). Second, an
undergraduate research research experience program with a collection
of undergraduate
problems (Section~\ref{sec-future_work-tabulation}).

%  prettier-ignore-start

\section{Continued Tabulation}\label{sec-future_work-continued_tabulation}

%  prettier-ignore-end

In this section, we describe the next steps for what has been discussed in
Chapter~\ref{ch-tabulation}. The first item to tackle in our future tabulation
work is the reconciliation of our table of arborescent tangles with
the algebraic table produced by Gren, Sulkowska, and,
Gabrov\v{s}ek \citep{gren2025classificationalgebraictangles}. Following this our
efforts take two forms, first a collection of minimalization problems for
the data we have generated. Second, is the expansion of this work
to cover the complete set of two string tangles, the arborescent and
the polygonal.

\subsection{Minimalization of Arborescent Tangles}

As we discussed in \nf{sec-minimalization}, that RLITT are often non-minimal
arborescent representatives of a tangle. In fact, minimal arborescent
representations may not be minimal tangle representations. Conway gives an
example \citep{conwayEnumerationKnotsLinks1970} where an arborescent
(algebraic) knot
is transformed into a minimal polygonal knot (\nf{subsubsec-opo-insert}). In
Figure~\ref{fig-6starstar_nonminimal} we rephrase Conway's example for tangles.

\begin{figure}[H]
  \centering
  \includegraphics[width=\linewidth]{files/fig-6starstar_nonmin-065532cac05a0e9000a7c014da0e0610.pdf}
  \caption[An arborescent tangle being turned into a polygonal
  tangle.]{An arborescent tangle being turned into a polygonal tangle
    via a sequence of
  interpolated Reidemister moves.}
  \label{fig-6starstar_nonminimal}
\end{figure}

This leads to two items that must be addressed to ensure the list of tangles
contains minimal diagrams.

\subsubsection{Identification of Minimal Arborescent Tangle Representations}

The first item, and most straightforward to address, is the identification of a
minimal arborescent representative of a given RLITT $\Gamma$. This requires the
identification of the weighted planar tree related to $\Gamma$ whose TCN is
minimal. We saw in \nf{sec-minimalization} the ways canonization can increase
complexity in a weighted planar tangle tree. To take $\Gamma$ to its minimal
form, we will need to develop the theory for and an implementation of an
efficient algorithm to systematically de-canonize $\Gamma$ into its minimal
arborescent form.

\subsubsection{Identification of Minimal Representations for
Arborescent Tangles}

Second, and more challenging, is the identification of
the minimal representative of a given arborescent tangle. That is, identifying
the minimal representative, arborescent or otherwise as in
Figure~\ref{fig-6starstar_nonminimal}. This task requires the development of a
classification of the subtrees of a weighted planar tree that correspond to
moves of the type similar to that seen in
Figure~\ref{fig-6starstar_nonminimal}. The
complexity of this task is compounded by the fact that the family of polygon
graphs allowing these types of moves is infinite (easily shown via an
induction). Further, the moves that enable arborescent tangles that are
minimally polygonal are not limited to the moves on the marked $6^{**}$
(Figure~\ref{fig-6starstar_nonminimal}). We can see a second class in
Figure~\ref{fig-other_nonminimal}.

\begin{figure}[H]
  \centering
  \includegraphics[width=\linewidth]{files/fig-other_nonminimal-665b635bba53dc590e590ae2271ee742.pdf}
  \caption[An arborescent tangle turned into a polygonal tangle.]{An
  arborescent tangle turned into a polygonal tangle.}
  \label{fig-other_nonminimal}
\end{figure}

\subsection{Polygonal Tangles}

We now discuss the expansion of the tangle tables to include all polygonal
tangles up to a target crossing number. Expanding tangle tables to include the
polygonal tangles is useful as at high crossing numbers, the polygonal tangles
dominate the arborescent tangles. Unfortunately, for the polygonal case, we lack
a general classification result. Meaning, as it stands, we have no theoretical
mechanism for direct generation of unique representatives for a polygonal tangle
as we have done with the classes of tangle in this thesis. The development of a
general classification for polygonal tangles is difficult, at least as difficult
as a general classification of knots. With this in mind, we will discuss two
possible directions for expanding the polygonal tangles.

\subsubsection{Ad Hoc Classification of Constellations}

In \nf{subsubsec-opo-insert}, we discussed constellations
\citep{connollyClassificationTabulation2string2021} used for
generation of polygonal
tangles via insertion. The crossing number of a polygonal tangle is bounded
below by the vertex count of its constellation. So the number of constellations
represented at reasonable crossing numbers is small. Additionally, when the
difference between crossing number and vertex count is low, many of the inserted
tangles will have low crossing numbers, again bounding complexity. In his thesis
work, Connolly \citep{connollyClassificationTabulation2string2021}
enumerates the ten
smallest constellations, those with ten or fewer vertices.

\begin{note}
  It's worth noting that expanding the table of constellations for a
  polygon graph is computationally
  hard, shown NP-Complete by
  Cook\citep{cookComplexityTheoremprovingProcedures1971}.
\end{note}

With low vertex count polygons, and at low crossing number, developing an ad hoc
classification result for each constellation may be a fruitful approach. For
example, consider the constellation seen in
Figure~\ref{fig-6starstar_const}, we can
enumerate the possible crossing numbers and locations for tangles to be
inserted, Table~\ref{sec-fw-adhoc-tab-to7}.

\begin{figure}[H]
  \centering
  \includegraphics[width=0.5\linewidth]{files/fig-6starstar_const-5b51b8bba46d02ef6ff52c23ba10c900.pdf}
  \caption[The unique constellation for $6^{\ast\ast}$.]{The unique
  constellation for $6^{\ast\ast}$}
  \label{fig-6starstar_const}
\end{figure}

We see that up to 7 crossings this constellation only admits rational
insertions, at 8 crossings we see our first Montesinos. Completing an analysis
of the possibilities for insertion may reveal patterns that allow a
classification of this $6^{**}$ constellation.
% Even partial results in this
% arena may yield more efficient heuristics when utilizing the methodology in
% \nf{subsubsec-fw-brute}.

\begin{figure}
  \centering
  \caption[Possible insertions of $6^{\ast\ast}$ by crossing
  number.]{Possible insertions of $6^{\ast\ast}$ by crossing number.}
  \label{sec-fw-adhoc-tab-to7}
  \begin{tabular}{p{\dimexpr 0.500\linewidth-2\tabcolsep}p{\dimexpr
    0.500\linewidth-2\tabcolsep}}
    \toprule
    Crossing Number & Possible Crossing numbers for insertion \\
    \hline
    5 & $\ast.1.1.1.1.1$ \\
    6 & \(\displaystyle
      \begin{aligned}\ast.2.1.1.1.1,\ast.1.2.1.1.1\\\ast.1.1.2.1.1,\ast.1.1.1.2.1,\\\ast.1.1.1.1.2
    \end{aligned} \) \\
    7 & \(\displaystyle
      \begin{aligned}\ast.1.1.1.2.2,*.1.1.2.1.2\\\ast.1.1.2.2.1,*.1.2.1.1.2,\\\ast.1.2.1.2.1,*.1.2.2.1.1,\\\ast.2.1.1.1.2,*.2.1.1.2.1,\\\ast.2.1.2.1.1,*.2.2.1.1.1,\\\ast.3.1.1.1.1,*.1.3.1.1.1,\\\ast.1.1.3.1.1,*.1.1.1.3.1,\\\ast.1.1.1.1.3
    \end{aligned} \) \\
    \bottomrule
  \end{tabular}
\end{figure}

%  prettier-ignore-start

% \subsubsection{Brute Force Tabulation}\label{subsubsec-fw-brute}

% %  prettier-ignore-end

% Without a full or partial classification of the polygonal tangles,
% we must take
% an alternative approach to what we have seen in this thesis. That
% alternative is
% the brute force, two pass enumeration strategy used in previous
% knot tabulation
% efforts \citep{dowkerClassificationKnotProjections1983,
% hosteFirst1701936Knots1998, burtonNext350Million2020} and outlined
% in Section~\ref{sec-history-of-tabulation}. The key
% difficulty in this methodology is the selection of invariants that combine to
% uniquely identify tangles. This difficulty is due to the fast growth rate for
% the count of tangles of a given crossing number. This growth rate means any
% invariant that is selected must have an efficient computation
% strategy. If, for
% example, as Burton did \citep{burtonNext350Million2020}, we select
% hyperbolic volume,
% we may be able to distinguish a large portion of tangles, however computation
% for crossing numbers as low as 10 will be intractable due to the
% per tangle time
% to compute the volume. This says nothing about the raw storage needed to hold
% the computed and partial data. A seemingly better choice are invariants
% (Section~\ref{subsec-invariant}) which have polynomial time
% computations, such as those
% introduced by van der Veen and Bar-Natan (to be published
% \citep{vanderveenKnotInvariantsFinite}). While weaker than
% hyperbolic volume, these
% invariants are stronger than the polynomial invariants, and faster
% than both to
% compute. Statistics for polynomial invariants can be found in Maguire's thesis
% work \citep{maguireKhovanovHomologyLegendrian}.

%  prettier-ignore-start

\section{Tabulation as Undergraduate Research}\label{sec-future_work-tabulation}

%  prettier-ignore-end

\subsection{A Research Experience Program for Undergraduates}

The accessibility of knot theory was discussed in
Section~\ref{sec-intro-intuit_knot_theory}.
This section elaborates on how that accessibility can be leveraged to engage
undergraduates in research. Throughout this thesis, we have investigated and
observed the depth and complexity of tabulation. We have seen how easily
portions of complex tabulation problems can be ``peeled off'' and decomposed as
self-contained problems. Additionally, we discussed product management training
(Section~\ref{sec-product-management}) and developed a software
enginering life cycle
(Section~\ref{sec-life-cycle}) for use in organizing undergraduate
research. These self
contained problems combined with our processes, produce a research experience
program ideal for undergraduates. The research experience program can be
enhanced by sequencing problems with a gradual release of responsibility model
as described by Fisher and Frey \citep{fisherBetterLearningStructured2013}.

We now outline a multi-semester program for training of undergraduates,
beginning when those students have only lower division (college algebra level)
maturity. To begin, we engage in directed reading with low level accessible
texts such as this thesis or The Knot Book: An Elementary Introduction to the
Mathematical Theory of Knots by Adams
\citep{adamsKnotBookElementary2004}. When the
student has gained a basic understanding of knots, a structured play problem
(potentially non-original work) can be introduced. Fitting the need here are
problems such as: the sculpting and 3D printing of stick
knots\footnote{A stick knot is a knot made of straight lines of a
unit length.} in a
program such as Blender or OpenSCAD \nf{sec-proj-sticks}, the creation of knot
mosaics \nf{sec-proj-mosaic}, or quilting Celtic knots
\nf{sec-proj-quilt}. The goal of
such an activity is to build wonder and cultivate confidence in the students
investigation skills.

As the student matures, their investigation skills improve, and their
knowledgebase deepens, they can be presented with more complex reading and novel
research problems. Depending on the student's interest, we present additional
but more advanced readings such as LinKnot
\citep{jablanLinKnotKnotTheory2007} or
accessible research papers such as Burton
\citep{burtonNext350Million2020}. We should
encourage freedom in these readings, allowing students to select sections and
formulate questions of their own. At this level, problems should still be well
structured having a clear path from start to finish but requiring the filling in
of gaps with original work. We should consider problems such as the translation
of notations as in \nf{sec-proj-notations}, or the computation of a
well understood
invariant, as in \nf{sec-projs-invariants}.

The program culminates with a mature undergraduate researcher ready to tackle
complex problems. At this point we expect the student to have mature reading and
reasoning skills, but perhaps lack skills such as literature review. Support for
reading at this stage should be focused on assisting students in finding answers
rather than answers being provided. Students may be prepared to formulate a
research question of their own and this should be encouraged; however presenting
students with ideas to build on or select from is beneficial. An ideal problem
here should fit student's interests and have a clear goal but perhaps no clear
starting point, for example the random tangle sampling seen in
\nf{sec-proj-rand}.

\subsection{Infrastructure of a Tabulation Program}

One key issue that must be addressed in a tabulation research program is
computational needs. Computing on knots and tangles is not necessarily a
computationally challenging task. Many problems are simple to describe
computationally and have efficient implementations. The primary challenge for
tabulation research stems from the raw scale of the dataset, as both the knot
and tangle datasets grow exponentially. This exponential growth of the data
gives us two primary areas of concern, time to compute and space to store. Even
if a problem has a nice constant time or linear time solution, doing a
computation on every knot or tangle in our dataset turns the problem into an
exponential one, a computational ``death by a thousand cuts''.

Generally when research questions run up against computational time constraints,
solutions take one of two forms, vertical scaling or horizontal scaling. We will
explain the two by analogy. Imagine we are trying to move a large boulder with a
bulldozer that is just not powerful enough for the job. We can trade in our
bulldozer for a bigger more powerful bulldozer that will push the rock without a
problem, this would be vertical scaling. Alternatively, we can blow up the
boulder and trade in our bulldozer for multiple smaller bulldozers, each able to
simultaneously move bits of the boulder, this would be horizontal scaling. In
our tabulation case, we could feasibly use either solution. If we vertically
scale, we could complete each computation faster. This makes sense for some
computations where each computation is slow such as hyperbolic volume. For other
computations, like the grafting seen in \nf{sec-rlitt-generation},
each computation
requires such little computational effort we would quickly run up against data
retrieval bottlenecks. In cases like this, distributing the effort horizontally
means, with some infrastructure effort, on aggregate we have less idle time in
the system. Another key feature of horizontal scaling in our case is the common
availability of clusters, at primarily undergraduate institutions. These
primarily undergraduate institutions often have no access to the large super
computers available at large institutions.

The planning and design of infrastructure leads us to address the second
challenge, space to store data. We will touch on two points: first the actual
storage of the data, and second, accessing and adding to the dataset. As we have
discussed, knot and tangle data grows exponentially, as expected, the space
needed to store that data also grows exponentially. As a benchmark, we use
arborescent tangles and the space required to store them in their linearized
form (\nf{sec-arborescent-linear}).

\bigskip\noindent
\begin{tabular}{p{\dimexpr 0.333\linewidth-2\tabcolsep}p{\dimexpr
  0.333\linewidth-2\tabcolsep}p{\dimexpr 0.333\linewidth-2\tabcolsep}}
  \toprule
  Tree Crossing Number & Projected Total number of tangles up to TCN
  & Projected Total size of notations up to TCN \\
  \hline
  20 &
  20178846455.0426 &
  744.93 GB\\
  21&
  77404113447.2751&
  3.02 TB\\
  22&
  296920571662.606&
  12.24 TB\\
  23&
  1138987289416.26&
  49.59 TB\\
  24&
  4369161597793.56&
  200.98 TB\\
  25&
  16760135017593&
  814.55 TB\\
  26&
  64292004387526.9&
  3.30 PB\\
  27&
  246624621285968&
  13.38 PB\\
  28&
  946053943972148&
  54.23 PB\\
  29&
  3629070212865634&
  219.77 PB\\

  \bottomrule
\end{tabular}

\bigskip$\,$ As we can see, the space required for storing tangles
quickly becomes
large. For perspective, a basic storage solution holds at least two copies of
the data, meaning to store arborescent tangles up to 21 crossings, we need
$3\times4\text{TB}=12\text{TB}$ of disk space (using 4TB as it's a common disk
size). More robust would be a solution that allows for two drive failures such
as RAIDZ2, in this case, we require $\LP2+3\RP\times4\text{TB}=20\text{TB}$. At
approximately $\$15$ per TB, putting us at around $\$200$ for the basic solution
and $\$300$ for the robust solution. It's important to remember this is the
required space and cost to store only a sequential list of notations for tangles
as in Example~\ref{sequential-list} making retrieval of a specific
tangle difficult.

\begin{example}{}{sequential-list}
  $$\iota\LP\LP\LB 2\RB \LB \m 3\RB \RP1\RP\iota\LP\LP\LB 2\RB \LB
  2\RB \m 1\RP1\RP\iota\LP\LP\LB 2\RB \LB 2\RB 1\RP1\RP\cdots$$
  A sequential list of tangles.
\end{example}

To expand the list, we require infrastructure that allows for random access, the
ability to search the data, and the ability to add additional data. This will
require that the data be stored in a database system. Unfortunately, a database
does come with a downside, it increases the storage requirements for the data
with mandatory overhead. With an effective data model and some consideration of
what should be stored and what should be computed on demand, we can mitigate
some of this overhead. However, even at low crossing number $\leq 12$, we will
quickly run into storage issues as we complete undergraduate computation
problems.

We will now discuss the selection and design of a database for a table of
tangles. Our data will be used by undergraduates, so an ideal database system is
one with a data model that has a shallow learning curve. Additionally, our data
is largely non-relational, meaning we don't need a database system geared to
relational data. These two items make a noSQL database system ideal. Just as we
had options for vertical scaling and horizontal scaling for carrying out the
computations we have the same two options for serving our database. However, in
the service case our choice is significantly more clear. Based on the size of
our data if we select vertical scaling our cost for a server will be in the
$\$10k -\$20k$ (not including storage cost) range and we may still end up with
bottlenecks. Therefore, for our needs, horizontal scaling is ideal. In this case
we can use a few low-cost cloud servers to store portions of the data, with a
coordinator balancing load across the system. If we encounter a bottleneck,
instead of buying a whole new expensive server, we simply add another low cost
server to our system. This horizontal scaling concept is called sharding, a
feature of MongoDB an ideal choice for our needs.

%  prettier-ignore-start

\subsection{Selection of Undergraduate Projects}\label{sec-selection_projects}

%  prettier-ignore-end

In this section we will provide a curated collection of undergraduate research
problem statements. We will also give a brief outline for each, contextualizing
the problem and describing what phase of the research experience program the
problem may be appropriate for:

\begin{enumerate}
  \item \textbf{Lower Division Student:} A lower division student is
    a student with little
    to no research or abstract math experience. A student at this
    level should be
    expected to have completed a college algebra course and started a calculus
    sequence. For students with a computational background we should expect the
    student to have started an introduction to programming course.
  \item \textbf{Intermediate Student:} An intermediate student is a
    student who has some
    exposure to abstract math. This could take the form of solving a lower
    division problem. These students should be well into a calculus sequence,
    having completed calculus II (advanced integration) or calculus III (vector
    calculus). For students with a computational background we should expect the
    student to have completed an introduction to programming sequence
    and started
    a course on algorithms and data structures.
  \item \textbf{Upper Division Student:} An upper division student is
    a student who can be
    expected to work semi-independently. They have solved one or more
    intermediate student problems, have completed the standard
    calculus sequence,
    and have begun abstract math courses. For students with a computational
    background we should expect the student to have completed a discrete methods
    course, ideally covering computational complexity theory.
\end{enumerate}

The remainder of the section gives statements for problems appropriate for
undergraduate research. The problems in the list fall into five types:
visualizations (\nf{sec-proj-visual}), invariants
(\nf{sec-projs-invariants}), notations
(\nf{sec-proj-notations}), generation (\nf{sec-proj-gen}), and
potpourri\footnote{Definition: A miscellaneous collection}
(\nf{sec-proj-pot}).

%  prettier-ignore-start

\subsubsection{Visualization}\label{sec-proj-visual}

%  prettier-ignore-end

Visualzation and spatial reasoning is a critically important for work in knot
theory. Problems of the visualization type develop specific visualizations or
general visualization tools for knots and tangles.

%  prettier-ignore-start

\paragraph{Create Knot Mosaics: Lower Division Student}\label{sec-proj-mosaic}

%  prettier-ignore-end

\subparagraph{Problem Statement}

Create a knot mosaic that has a particular property.

\subparagraph{Brief}

Knot mosaics are a simple method for creating knots from a collection of tiles.
Creating mosaics with a particular property, a specific writhe, for example, is
a fun and engaging activity where abstraction of a concept can be explored.
Modifying the tile set can add additional complexity to the task.

%  prettier-ignore-start

\paragraph{Create Stick Knots: Lower Division Student}\label{sec-proj-sticks}

%  prettier-ignore-end

\subparagraph{Problem Statement}

Create stick knots with desired properties.

\subparagraph{Brief}

Stick knots are knots built from a collection of unit sticks. Creating a
physical model by hand or with computer design and 3D printing develops spatial
reasoning skills needed for work in higher knot theory.

%  prettier-ignore-start

\paragraph{Creating Celtic Knots: Lower Division Student}\label{sec-proj-quilt}

%  prettier-ignore-end

\subparagraph{Problem Statement}

Create Celtic knots with desired properties.

\subparagraph{Brief}

Celtic knots are common artistic knots. Exploring the creation of a unique
ruleset for creating Celtic knots is an opportunity to develop a unique
understanding of the diagrammatic nuances of knot theory.

%  prettier-ignore-start

\paragraph{Compute Diagram for General Notations: Intermediate
Student}\label{sec-proj-diagram_att}

%  prettier-ignore-end

\subparagraph{Problem Statement}

Create an interface for plotting knots in an arbitrary notation in KnotPlot.

\subparagraph{Brief}

An aspect of knot theory that makes it among the most accessible higher math
domains is the ability for anyone to draw pictures of knots. A continuing theme
of this thesis is that computations/operations are easy by hand, with the
qualifier, ``up to reasonable crossing number''. This carries through with the
drawing of the diagrams. A common drawing tool in knot theory is KnotPlot
\citep{schareinInteractiveTopologicalDrawing1998}, unfortunatly KnotPlot has no
interface for drawing knots in ``arbitrary'' notations. The tool however, does
have a Lua scripting interface in which an arbitrary notation decoder can be
designed.
\newpage
\paragraph{Create VR Band Plumbing Visualizer: Upper Division Student}

\subparagraph{Problem Statement}

Create a 3D VR visualizer for the band plumbing construction for arborescent
knots.

\subparagraph{Brief}

The plumbing construction for arborescent knots and tangles is easiest
visualized in 3D. The ideal for visualizing these objects is in VR as this is
reduces the need for spatial reasoning. While the theory exists for creating the
objects, the linear algebra required makes this an upper division problem.

%  prettier-ignore-start

\subsubsection{Invariants}\label{sec-projs-invariants}

%  prettier-ignore-end

One way to build conjecture is by the analysis of patterns in data, these
conjectures often lead to the development of new theory. Problems in this
section create the collections of data that can be used for developing those
conjectures and theory.

%  prettier-ignore-start

\paragraph{Compute Polynomial From A Tangle Notation: Intermediate
Student or Upper Division Student}\label{sec-proj-homflypt}

%  prettier-ignore-end

\subparagraph{Problem Statement}

Develop the theory needed for efficiently computing polynomials of tangles

\subparagraph{Brief}

One of the most important advancements in knot theory was the discovery of knot
polynomials as a class of knot invariants. As an example, one of the most
powerful of these polynomials is the HOMFLYPT polynomial
\citep{freydNewPolynomialInvariant1985} constructed from the skein
relations equation
Figure~\ref{fig-future_work-skein_homfly}.

\begin{equation}
  \begin{aligned}
    P\LP\text{unknot}\RP&=1\\
    \ell P\left(L_{+}\right)+\ell^{-1} P\left(L_{-}\right)+m
    P\left(L_0\right)&=0\\
  \end{aligned}
\end{equation}

Conveniently, the data needed to apply the skein relations is precisely the data
encoded by RLITT, relative crossing data.

\begin{figure}[H]
  \centering
  \includegraphics[width=0.5\linewidth]{files/Skein_HOMFLY-6ebea249863dfdbae5e3432af0414ba9.pdf}
  \caption[The skein relation for the HOMFLYPT polynomial.]{The skein
    relation for the HOMFLYPT polynomial.  (Public domain, via
    Wikimedia Commons\citep{pbroks13SkeinHOMFLYPublic2008})
    % Similar to that seen for the Kauffman bracket Section~\ref{subsec-kauff}.
  }
  \label{fig-future_work-skein_homfly}
\end{figure}

Depending on the polynomial selected, the problem is appropriate for
intermediate students or upper division students. When the polynomial has a
developed tangle theory, the solution will have a well defined start and end
point and is appropriate for intermediate students. Otherwise, the full tangle
theory must be developed. This requires experience with the development of
original abstract theory, making the problem appropriate for upper division
students.

\paragraph{Compute Finite Type Invariant From A Tangle Notation:
Intermediate Student or Upper Division Student}

\subparagraph{Problem Statement}

Develop the theory needed for efficiently computing a finite type invariant of
tangles

\subparagraph{Brief}

Similar to the computation of polynomial invariants the computation of finite
type invariants expands our table with data useful for binning future tangles
and knots. Depending on the invariant selected, the problem is appropriate for
intermediate or upper division students. When the invariant has a developed
tangle theory the solution will have a well defined start and end point and be
appropriate for intermediate students. Otherwise, the full tangle theory must be
developed. This requires experience with the development of original abstract
theory, making the problem appropriate for upper division students.

%  prettier-ignore-start

\subsubsection{Notations}\label{sec-proj-notations}

%  prettier-ignore-end

There are many ways to encode the data of a knot, each with advantages and
disadvantages. Throughout this thesis, the primary target notation was the RLITT
Section~\ref{subsec-rlitt}. This subsection discusses several useful
notations where a
computational tool translating from and to RLITT is desired. Since the source
and destination notation in each problem are well understood each is appropriate
for intermediate students.

%  prettier-ignore-start

\paragraph{Notation Description for Extended Gauss Notation:
Intermediate Student}\label{sec-proj-note_gauss}

%  prettier-ignore-end

\subparagraph{Problem Statement}

Develop the theory translating RLITT to extended Gauss notation. Additionally,
develop the software needed for storing and translating per theory.

%  prettier-ignore-start

\paragraph{Notation Description For Planar Diagram (PD) Notation:
Intermediate Student}\label{sec-proj-note_pd}

%  prettier-ignore-end

\subparagraph{Problem Statement}

Develop the theory translating RLITT to PD notation. Additionally, develop the
software required for storing and translating per theory.

%  prettier-ignore-start

\paragraph{Notation Description For DT Notation: Intermediate
Student}\label{sec-proj-note_dt}

%  prettier-ignore-end

\subparagraph{Problem Statement}

Develop the theory translating RLITT to DT notation. Additionally, develop the
software needed for storing and translating per theory.

\subsubsection{Generation}\label{sec-proj-gen}

%  prettier-ignore-end

This section expands the census of tangles to more abstract classes. These
expanded lists increase accessibility of complex objects allowing for the
creation of new theory.

\paragraph{Create a table of virtual tangles: Upper Division Student}

\subparagraph{Problem Statement}

Create the theory needed to construct a table of virtual tangles.

\subparagraph{Brief}

Virtual knots, developed by Kauffman
\citep{kauffmanVirtualKnotTheory1999}, are an
extension of the knot concept where a knot shadow need not be planar. Some work
has been done on classifying the virtual tangles by Mellor and Nevin
\citep{mellorVirtualRationalTangles2020}. The full tangle theory must
be developed to
solve this problem. This requires experience with the development of original
abstract theory, making the problem appropriate for upper division students

\paragraph{Create a table of $n$ string tangles: Upper Division Student}

\subparagraph{Problem Statement}

Create the theory needed to construct a table of $n$ string tangles.

\subparagraph{Brief}

The tangles we have worked with in this thesis are the two string tangles, those
with four fixed points on the Conway circle. A natural extension to this concept
is the $n$ string tangle. Recently, Kwon
\citep{kwonClassificationRational3tangles2025}, the 3 strings
rational tangles, have
been classified. For the remaining cases, the full tangle theory must be
developed to solve the problem, this requires experience with the development of
original abstract theory, making the problem appropriate for upper division
students

%  prettier-ignore-start

\subsubsection{Potpourri}\label{sec-proj-pot}

%  prettier-ignore-end

Problems in this section are those that do not fit into other classes of
problems.

%  prettier-ignore-start

\paragraph{Random Tangle Sampling: Upper Division Student}\label{sec-proj-rand}

%  prettier-ignore-end

\subparagraph{Problem Statement}

Create the theory and software to select from a collection or generate a tangle
at random with an understood distribution.

\subparagraph{Brief}

In the introduction chapter Section~\ref{sec-intro-applications},
applications of knots to the
hard sciences were discussed. When working in the hard sciences, being able to
sample with an understood distribution from a collection is important. Similarly
to the previous notational projects, describing and implementing random sampling
methodologies is a class of extremely useful tabulation projects. The full
tangle theory must be developed to solve this problem. This requires experience
with the development of original abstract theory, making the problem appropriate
for upper division students
\newpage
\paragraph{Develop a tangle analogue for petal knots: Upper Division Student}

\subparagraph{Problem Statement}

Create the theory needed for a well defined tangle analogue of the petal
knots\citep{adamsKnotProjectionsSingle2015}.

\subparagraph{Brief}

Petal knots, first developed by Adams\citep{adamsKnotProjectionsSingle2015}, are
knots in which all crossings are colinear in the orthogonal projection, an
``ubercrossing''. Converting these objects to a braid is
straightforward, however
less obvious is converting to a two string tangle. Identifying a tangle analogue
for the petal knots may allow for computation of a whole new family of tangle
data. The full tangle theory must be developed to solve this problem This
requires experience with the development of original abstract theory, making the
problem appropriate for upper division students


\chapter{Census of RLITT up to TCN 9}

A full table of RLITT up to TCN 18 is available as supplimental material of this
work.
A full table of RLITT up to TCN 16 is available on GitHub
\citep{Starr_The_Tanglenomicon_A_2025}.
%
\bigskip
% \appendix

% \backmatter

\raggedright
\printbibliography
$\,$
\newpage
$\,$
\newpage
$\,$
\newpage
$\,$
\newpage
\end{document}
