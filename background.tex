\chapter{Preliminaries}

%  prettier-ignore-start

\section{History of Knot Tabulation}\label{sec-history-of-tabulation}

%  prettier-ignore-end

Mathematical interest in knots was brought to the forefront of mathematics and
physics during the 1860s by Lord Kelvin. A central area of research in physics
during the 1860s was the investigation of the fundamental building blocks of
matter: the atoms. Lord Kelvin hypothesized that atoms were knotted vortices in
the aether \citep{thomsonVortexAtoms1867}. With this hypothesis, a
natural next step
is the creation of a table of the elements which, by hypothesis, was a table of
knots.

\subsection{Knot Tabulation by Hand}

With Lord Kelvin's vortex hypothesis as the driving force for knot tabulation,
the first knot table was produced, via hand computation, by P.G. Tait. Tait's
first table, completed in the 1860s, contained prime knots (described in
\nf{subsec-prime_knot}) up to seven crossings
\citep{taitFirstSevenOrders1884} (described in \nf{subsec-knot_def}), a table
of the first seven knots can be seen in Figure~\ref{fig-hist-7table}

\begin{figure}[H]
  \centering
  \includegraphics[height=0.4\textheight]{files/knots_to_7-f13b9b5556c173ab3d955b74f82b2e53.pdf}
  \caption[A table of the first seven prime knots.]{A table of the
  first seven prime knots. \citep{schareinInteractiveTopologicalDrawing1998}}
  \label{fig-hist-7table}
\end{figure}

With a table of seven crossing knots complete, Tait's work in knot tabulation
continued, alongside Kirkman and Little
\citep{taitTenfoldKnottiness1885,
kirkmanEnumerationDescriptionConstruction1885, littleKnotsCensusOrder1885},
for the next 25 years. The trio ultimately constructed a complete list of prime
knots up to ten crossings (250 knots + 1 repeat). When the knot
theoretic machinery
available at the time is considered, the completion of these tables with such
high accuracy was a Herculean task. The tables contained a single error, two
equivalent (described in \nf{subsec-knot_equivalence}) ten crossing knots
(Figure~\ref{fig-hist-perko}), identified in 1974 by Perko, an
amateur mathematician
\citep{perkoClassificationKnots1974}.

\begin{figure}[H]
  \centering
  \includegraphics[width=0.5\linewidth]{files/perko_pair-f143cdf72c123a1835bc5e48f71d8245.pdf}
  \caption[The Perko pair.]{The Perko pair, a pair of equivalent ten
    crossing knots appearing as an
    error in early knot tables
    \citep{schareinInteractiveTopologicalDrawing1998,
  perkoClassificationKnots1974}.}
  \label{fig-hist-perko}
\end{figure}

After the completion of the ten crossing tables, efforts in knot tabulation
stagnated, with few concerted efforts and little progress being made in
expanding the tables. The next researcher to take up the tabulation torch was
Conway in the 1960s \citep{conwayEnumerationKnotsLinks1970}. Conway,
in ``a few hours''
\citep{conwayEnumerationKnotsLinks1970}, tabulated knots to eleven
crossings, with
only four omissions\citep{caudron1982classification}. Conway's work
continued by hand
computation, but employed a novel approach to tabulation. He described
decompositions of knots into building blocks, which he called
tangles\citep{conwayEnumerationKnotsLinks1970}. Conway paired this
with a calculus to
glue the blocks together. Under Conway's tangle calculus, the combinatorial work
of knot tabulation became a game of building from simple to complex. Inspired by
Conway's strategies, a second effort to enumerate eleven crossing knots was
carried out by Caudron \citep{caudron1982classification}, verifying
Conway's findings
and rectifying the four omissions. Caudron's confirmation of the eleven crossing
tables marked the final chapter in the hand computation era of knot tabulation.

\subsection{Knot Tabulation by Computer}

With advancements in manufacturing in the early 1980s, electronic computers
became closer to commodity products. This allowed for researchers of diverse
backgrounds and interests to take their crack at time-consuming computational
tasks. One of these computational tasks was the construction of knot tables. The
first to construct a knot table by computer were Dowker and Thistlethwaite in
1983 \citep{dowkerClassificationKnotProjections1983}, who produced a
table of all
prime knots to thirteen crossings. The pair implemented a novel two pass
approach that has served as an outline for all major efforts that have followed.
The process begins with a first pass to enumerate all possible knot diagrams
(described in \nf{subsec-knot_def}). This is followed by a second pass where
``sufficient invariants to distinguish them (knots) from each other''
\citep{dowkerClassificationKnotProjections1983} are computed. This effectively
assigns knots to equivalence classes (bins \footnote{In software
    development, the mathematical concept of an equivalence class is
often called a ``bin''.}), hence finding and removing
all duplicate entries from the list (deduping \footnote{Short for
    deduplicating, meaning the removal of duplicate entries from the
list.}).

The next table produced by computer, knots up to sixteen crossing, was given by
Hoste, Thistlethwaite, and Weeks \citep{hosteFirst1701936Knots1998}
in 1998. Their
process deviates only slightly from the earlier approach of Dowker and
Thistlethwaite, by leaning on heuristics\footnote{When discussing
  algorithms, a heuristic describes a special case that, when
seen, short circuits the algorithm, reducing unnecessary work.} to
limit the duplicates
found in their first pass. This preprocessing in the first pass allowed Hoste,
Thistlethwaite, and Weeks to compute their table in one to two weeks of wall
time\footnote{The real effective time of a clock on a wall, this is
  different from CPU
time which is a relative measure of time.}.

The most recent computational effort was carried out by Burton in 2020
\citep{burtonNext350Million2020}, finding knots to nineteen crossings. Burton's
program followed closely to the two pass processes, with further heuristic work
to preprocess in the first pass and heavily relying on the hyperbolic volume
invariant for the second pass. The computation of the nineteen crossing table
required months of wall time on a cluster\footnote{An aggregation of
  many computers, each with many computational cores. We'll
  see later in this thesis that tabulation is massively
parallelizable.}, serving as an important
signpost for the time requirement problems of knot tabulation.

\subsection{Tangle Tabulation}

Conway's tangle construction allowed him to quickly and effectively tabulate
knots by hand. These building blocks of knots are interesting mathematical
objects in their own right, however tabulation efforts for tangles have been
sparse. The importance of the creation of a large table of tangles has been
called:

\begin{quote}
  ``The Most Important Missing Infrastructure Project in Knot
  Theory'' - Bar-Natan
  \citep{bar-natanMostImportantMissing}
\end{quote}

As it stands, tables of tangles have been generated by hand up to seven
crossings by Kanenobu, Saito, and Satoh
\citep{kanenobuTanglesSevenCrossings2003a}.
Computer driven efforts have been undertaken by several members of the
University of Iowa Applied Topology (UIAT) group namely Conolly
\citep{connollyClassificationTabulation2string2021}, Bryhtan
\citep{bryhtanTabulating2stringTangles2024}, and this thesis.
Separately, a table of
algebraic tangles has been produced by Gren, Sulkowska, and,
Gabrov\v{s}ek \citep{gren2025classificationalgebraictangles}. Their tabulation
is built on a binary operation tree based on Conway's
\citep{conwayEnumerationKnotsLinks1970} tangle calculus. Similar binary
operation trees can be seen in Conolly
\citep{connollyClassificationTabulation2string2021}, Caudron
\citep{caudron1982classification}, and discussed in
Section~\ref{sec-monttang}.

\section{Foundations of Knots}

We now begin a more formal description of the foundations of the theory of
knots. Our treatment will begin with the general definition of knots, as well as
similar knot-like objects. Next, we will discuss ways in which two knots can be
considered equivalent. With this, we'll give an example of a common invariant
for knots. Finally, we conclude with descriptions of the notational strategy
that inspired the rest of this thesis, the Conway notation.

%  prettier-ignore-start

\subsection{Definition of a Knot}\label{subsec-knot_def}

%  prettier-ignore-end

As with anything, we must start with a definition, here we give one for a
\textbf{proper knot}.

\begin{definition}{From Jablan and Sazdanovi{\'c}, Definition 1.17
  \textbf{\citep{jablanLinKnotKnotTheory2007}}}{def-knot}
  A \textbf{proper knot} is a smooth embedding of a circle $S^1$ into Euclidean
  3-dimensional space $\R^3$ (or the 3-dimensional sphere
  $S^3$).

\end{definition}With some consideration, we can see this definition
aligns with our intuitive
description of a knot given in
Section~\ref{sec-intro-intuit_knot_theory}. We note that this
definition gives two choices for ambient space, for this thesis we will prefer
$S^3$ as an ambient space, the preference will become clear in later sections. A
natural extension of the concept of a knot is to allow for more than a single
$S^1$ component to be embedded into the ambient space. This allowance gives us
the concept of a \textbf{$c$ component link}.

\begin{definition}{From Jablan and Sazdanovi{\'c}, Definition 1.17
  \textbf{\citep{jablanLinKnotKnotTheory2007}}}{def-link}
  A \textbf{$c$ component link} is a smooth embedding of $c$ disjoint
  copies of a circle
  $S^1$ into $\R^3$ (or $S^3$), where the embeddings of circles $S_i^1$ are its
  components ($i = 1, 2,\dots, c$).

\end{definition}
\begin{convention}
  For convenience, and brevity, in the remainder of this thesis we
  will adopt the
  following convention. The term \textbf{knot} refers to the collection of
  all proper knots and
  all $c$ component links. If we find the need to exclude the $c$
  component links
  from consideration, we will use the term proper knot.
\end{convention}

Playing with this three-dimensional construction for knots, it will quickly
become apparent that three-dimensional knots are unwieldy to work with. To
simplify our work, we will now build a two-dimensionally encoded model for
knots, a \textbf{knot diagram}. We start by taking a knot $K\subset
S^3$. We then
select an $S^2$ such that $K$ lies fully in the interior. Now, for any plane
that lies tangent to $S^2$, we take an orthogonal projection of the knot onto
the plane. We require that the projection have no \textbf{degenerate crossings},
intersections of the knot projection where more than two points are collinear or
where the crossing is not transverse.
We call this projection a \textbf{knot shadow}, an example can be seen in
Figure~\ref{fig-knot_def-shadow}. A knot shadow is interpreted as a planar
graph\footnote{A graph in the mathematical sense, is a set of
  vertices combined with a set of
  relationships between those vertices called edges. A planar graph is a graph
that when drawn in the plane has no overlapping edges.}, with points
where strands overlap (are collinear in the
projection) as vertices, and edges of the shadow as the strands between the
overlaps.

\begin{figure}[H]
  \centering
  \includegraphics[width=0.5\linewidth]{files/knot_shadow-bca368cd9027f5ce97cd68c0bb57f1be.pdf}
  \caption[A schematic diagram demonstrating a knot and its
  shadow.]{A schematic diagram demonstrating a knot (orange) and its
    shadow (grey).
    Imagine a light shining from above the knot onto a piece of paper. The knot
  shadow is the shadow cast on the paper.}
  \label{fig-knot_def-shadow}
\end{figure}

Taking only the shadow of a knot we lose some data that is intuitively
important, the crossings of a knot (the relative distance of collinear points).
To recover this data in a diagram, we split the edges of the shadow where the
strand closer to the projection plane appears to travel under the edge
corresponding to the strand further from the plane. We call the edge that is
split the \textbf{under strand}, and the non-split strand the
\textbf{over strand}. These
augmented knot shadows are called knot diagrams and will serve as our primary
schematic model for knots throughout this thesis. We call the count of crossings
in a knot diagram the \textbf{crossing number} of the knot diagram.

We finish with naming a knot with particular significance, the knot with no
crossings in its diagram is called the \textbf{unknot}.

%  prettier-ignore-start

\subsection{Knot Equivalence}\label{subsec-knot_equivalence}

%  prettier-ignore-end

Armed with the formal definition of a knot, we can make our first progress in
answering the overarching question from
Section~\ref{sec-intro-intuit_knot_theory}.

\begin{quote}
  How do I tell two knots I make apart?
\end{quote}

To tell two knots apart, we need to discuss the concept of sameness, that is,
what is equivalence in knots. Our concept of equivalence for knots is given by
\textbf{ambient isotopy}, and equal knots are said to be
\textbf{ambient isotopic}.

\begin{definition}{From Jablan and Sazdanovi{\'c}, Definition 1.20
  \textbf{\citep{jablanLinKnotKnotTheory2007}}}{def-ambient_isotopic}
  Knots $K$ and $K_1$ are \textbf{ambient isotopic} if there exists a
  continuous function $H: \R^3 \times[0,1] \rightarrow \R^3$ such that:
  \begin{itemize}
    \item{$h_0=H((x, y, z), 0)$ is the identity $\R^3 \rightarrow \R^3$,}
    \item{for all $t \in[0,1], h_t=H((x, y, z), t)$ is a
      homeomorphism $\R^3 \rightarrow \R^3$,}
    \item{if $h_1=H((x, y, z), 1)$, then $h_1(K)=K_1$.}
  \end{itemize}
\end{definition}When working with the three-dimensional model of a
knot, writing down explicit
ambient isotopies is, in general, quite difficult. As we did in
\nf{subsec-knot_def}, we can simplify the concept of equality by moving ambient
isotopy to an equivalence of knot diagrams. Taking the orthogonal projection
model for knot diagrams given in \nf{subsec-knot_def}, ambient isotopy can be
modeled as three Reidemeister moves on diagrams
\citep{reidemeisterElementareBegruendungKnotentheorie1927}. Meaning,
two knots are
ambient isotopic if and only if their diagrams are equal under a chain of
Reidemeister moves
\citep{reidemeisterElementareBegruendungKnotentheorie1927} and isotopies.

The first Reidemeister move we will define is the Type I move
\citep{reidemeisterElementareBegruendungKnotentheorie1927}. To carry
out the Type I
move (Figure~\ref{fig-knot_def-r1}), start by taking a portion of a
diagram with no
crossings, then add a half twist. When adding the twist, we have two choices;
twist into (left handed) or out of (right handed) the plane the diagram lies in.
In either, we can freely remove the new crossing by twisting in the opposite
direction.

\begin{figure}[H]
  \centering
  \includegraphics[width=0.3\linewidth]{files/R1-1068ad57841d108de41eaba727a15cd2.pdf}
  \caption[Executing the two flavors of type I move.]{Executing the
    two flavors of type I move on a knot diagram. On the left, we have
    a twist into the plane, also called a positive or left-handed
    twist. On the right,
  we have a twist into the plane, also called a negative or right-handed twist.}
  \label{fig-knot_def-r1}
\end{figure}

The next Reidemeister move is the Type II
move\citep{reidemeisterElementareBegruendungKnotentheorie1927}, seen in
Figure~\ref{fig-knot_def-r2}. When we carry out the type II move, we
need two strands, each
with no crossings. We then pull one strand on top of the other, inducing two new
crossings in our diagram. Similarly to the type I move, the type II move can be
freely undone by pulling the strands apart.

\begin{figure}[H]
  \centering
  \includegraphics[width=0.3\linewidth]{files/R2-4010ce2b2b2ae0fefc84be521d714e22.pdf}
  \caption[Executing the two type II moves.]{Executing the two type
    II moves with a pair of strands. In the top image, the
    bottom strand is pulled over the upper. In the bottom image, the
    bottom strand
  is pulled under the top strand.}
  \label{fig-knot_def-r2}
\end{figure}

The final Reidemeister move is the Type III
move\citep{reidemeisterElementareBegruendungKnotentheorie1927}. In
the type III move,
we take three strands, two that form a crossing and a third that lies in one of
three possible positions:

\begin{enumerate}
  \item above the over strand
  \item between the over and under strands
  \item below the under strand
\end{enumerate}

We now execute the type III by taking the third strand (not part of the center
crossing) and passing it across the center crossing. As with type I and type II,
we're free to reverse the type III move.

\begin{figure}[H]
  \centering
  \includegraphics[width=0.3\linewidth]{files/R3-053c00ec76f615349542d57514de7583.pdf}
  \caption[Executing the three type III moves.]{Executing the three
    type III moves with a set of three strands. Top
    to bottom, the third strand is: $\ \bullet$ on top of the
    crossing strands $\ \bullet$ between
  the crossing strands $\ \bullet$ under the crossing strands.}
  \label{fig-knot_def-r3}
\end{figure}

We should note here that with a concept of equivalence comes equivalence classes
of knot diagrams. Historically, of particular interest in the tabulation of
knots, are the knot diagrams that have minimal crossing number, we call these
\textbf{minimal diagrams}, knot diagrams where crossing number cannot be
decreased by Reidemeister moves.

%  prettier-ignore-start

\subsection{Prime Knots}\label{subsec-prime_knot}

%  prettier-ignore-end

With the goal of enumerating objects, we should be clear on what types of
objects should be enumerated and which should be left uncounted. We now describe
the class of knot that tabulators are interested in, the
\textbf{prime knots}. The
first step is to define an operation on knots, called the \textbf{connect sum}.

\begin{definition}{}{}For knots $J$ and $K$, the \textbf{connect sum}
  $J\#K$ is produced by:

  \begin{enumerate}
    \item Excising an arc from both $J$ and $K$
    \item Gluing endpoints of $J$ to endpoints of $K$ so no new
      crossings are added.
  \end{enumerate}

  \begin{figure}[H]
    \centering
    \includegraphics[width=\linewidth]{files/fig-prime_knot-conne-dc193ded4f495df17dda8990f19f876c.pdf}
    \caption[An example of the connect sum of two trefoil knots.]{An
    example of the connect sum of two trefoil knots.}
    \label{fig-prime_knot-connect_sum}
  \end{figure}

\end{definition}With the connect sum operation defined, we are now
prepared to give the
definitions for prime and composite knots.

\begin{definition}{}{}
  A knot is called \textbf{prime} if, in every decomposition into a
  connect sum, one
  of the factors is unknotted. Otherwise, the knot is called \textbf{composite}.

\end{definition}
%  prettier-ignore-start

\subsection{Knot Invariants}\label{subsec-invariant}

%  prettier-ignore-end

Our next topic of interest is that of a knot invariant. In general, an invariant
for an object is a datum that is computed deterministically for the object and
remains unchanged within an equivalence class. In the knot case, we
will take the
concept of equality to be that given in \nf{subsec-knot_equivalence}.
As discussed
in Section~\ref{sec-history-of-tabulation}, invariants play an
important role in computer
tabulation. We will now describe a simple invariant we first
introduced in \nf{subsec-knot_equivalence}.

% The remainder of this section will be devoted to the construction of
% two historically significant knot invariants, the minimal crossing number and
% the \textbf{Jones polynomial}.

\subsubsection{Minimal Crossing Number}

We saw at the end of \nf{subsec-knot_equivalence} the definition of
the minimal crossing number for a knot. That being the minimal number of
crossings needed to represent the knot as a diagram. Somewhat intuitively,
this number is an invariant for a knot. If a knot has minimal crossing number
$4$, we will never be able to represent it with three crossings, so it has to
be different from the trefoil knot (a knot with three crossings).  However, we
can see from the table of the first seven knot (seen in
Figure~\ref{fig-hist-7table}) that having
equivalent crossing number does not give us equivalent knots.

% \subsubsection{Jones Polynomial}

% In this section we will describe a more complex invariant for knots. While we
% won't use this construction for our tangle tabulation it provides an example
% for the kind of theory available for future work by undergraduates
% (see Section~\ref{sec-selection_projects}).
% The Jones polynomial (Definition~\ref{def-jones}) is due to Vaughan Jones
% \citep{jonesPolynomialInvariantKnots1985}, which he discovered in
% 1985 during his
% work on Von Neumann algebras. A more combinatorial approach to the
% construction,
% due to Kauffman \citep{kauffmanStateModelsJones1987}, is given here
% and uses the
% \textbf{Kauffman bracket}.

% \begin{definition}{From Lickorish, Definition 3.6
% \textbf{\citep{kauffmanStateModelsJones1987a,
% jonesPolynomialInvariantKnots1985,
% lickorishIntroductionKnotTheory1997}}}{def-jones}
% The Jones Polynomial $V\LP \mathscr{K}\RP$ of an oriented knot
% $\mathscr{K}$ is
% the Laurent polynomial\footnote{A Laurent polynomial is a
% polynomial that is allowed to have both positive
% and negative powers.} with integer coefficients in $t^{1/2}$. Defined by

% \begin{equation}
% V\LP \mathscr{K}\RP=\LP\LP-A\RP^{\ \m 3w(K)}\LA K\RA\RP_{t^{1/2}=A^{\ \m 2}}
% \end{equation}
% where $K$ is any oriented diagram for $\mathscr{K}$.

% \end{definition}
% %  prettier-ignore-start

% \paragraph{Kauffman Bracket}\label{subsec-kauff}

% %  prettier-ignore-end

% The Kauffman bracket is a function that takes knot diagrams as
% input and outputs
% a Laurent polynomial[\^lp] (Definition~\ref{kb-def-kaufb}).

% \begin{definition}{Kauffman, Definition 2.1
% \textbf{\citep{kauffmanStateModelsJones1987,
% lickorishIntroductionKnotTheory1997}}}{kb-def-kaufb}
% The \textbf{Kauffman bracket} of an unoriented knot $K$, $\LA
% K\RA$, is the Laurent
% polynomial with integer coefficients in $A$ given by the following relations:

% \begin{enumerate}
% \item $\LA\img{media/kauf_bkt/unknot} \RA=1$
% \item $\LA K \sqcup \img{media/kauf_bkt/unknot} \RA=\LP -A^2
% -A^{\m2}\RP \LA K \RA$
% \item $\LA \img{media/kauf_bkt/crossing/crossing_un} \RA=A\LA
% \img{media/kauf_bkt/type2/6a} \RA+A^{\ \m 1}\LA
% \img{media/kauf_bkt/type2/6b} \RA$
% \end{enumerate}

% \end{definition}Since the number one and the unknot serve important
% roles for polynomials and
% knots respectively, the first criterion is intuitive to select,
% when inventing a
% knot polynomial invariant. For now, we focus our discussion on the third
% criterion. The action we see in three is called \textbf{smoothing a
% crossing} and is
% difficult to see. We will now present an intuitive model (originally by Jones
% \citep{jonesJonesPolynomialDummies2014}) for what happens in the
% smoothing process.

% Consider a crossing as a tangle, imagine that the over strand of
% the crossing is
% the slot of a giant flathead screw, and attach a marker on either end of the
% slot. Taking a screwdriver, we may turn the screw clockwise or anti-clockwise.
% When the screw is turned, the markers (arrows in
% Figure~\ref{fig-jp-screw-model}) trace out
% two arcs on the page. We create two new tangles by placing arcs inside each
% tangle and connecting the endpoints joined by the marker trace. We can see, by
% inspection, that the resulting pictures match the components of the third
% condition of Definition~\ref{kb-def-kaufb}.

% \begin{figure}[H]
% \centering
% \includegraphics[width=0.5\linewidth]{files/fig-jp-screw-model_i-c5ffbb95a016d5b69741e75d01c6beb3.pdf}
% \caption[A diagram depicting the screwdriver model of crossing
% smoothing.]{A diagram depicting the screwdriver model of crossing
% smoothing. On top, we have
% the original, pre smoothed crossing. On the bottom left, we have an
% anti-clockwise
% smoothing. The result connects the top point to the left point and
% the bottom point
% to the right point. On the bottom right we have a clockwise smoothing. The
% result connects the top point to the right point and the bottom point
% to the left point.}
% \label{fig-jp-screw-model}
% \end{figure}

% We will now move to proving that $\LA\,\RA$ is invariant under Reidemeister
% moves. That is, when we start with two equivalent (by Reidemeister move) local
% knot diagrams, the brackets of the two diagrams are equivalent.
% First, we prove
% that $\LA\,\RA$ is invariant under the type II move. For a moment,
% we will write
% the third condition of Definition~\ref{kb-def-kaufb} as
% $\LA \img{media/kauf_bkt/crossing/crossing_un} \RA=A\LA
% \img{media/kauf_bkt/type2/6a} \RA+B\LA \img{media/kauf_bkt/type2/6b} \RA$
% and the second as
% $\LA K \sqcup \img{media/kauf_bkt/unknot} \RA=\LP -A^2 -A^{\m2}\RP \LA K \RA$.

% \begin{theorem}{}{thm-typeii_bkt}The equality of equation
% (\ref{thm-kb-math-t2}) holds.
% \begin{equation}
% \label{thm-kb-math-t2}
% \LA \img{media/kauf_bkt/type2/1}\RA=\LA \img{media/kauf_bkt/type2/6b} \RA
% \end{equation}

% \end{theorem}\begin{proof}We begin by smoothing the crossing on the
% right side of the diagram, we then smooth
% the other crossing, this process yields the following chain of equalities,
% combining like terms where appropriate.

% \begin{equation}
% \begin{aligned}
%  \bkt{media/kauf_bkt/type2/1}
% &=A\bkt{media/kauf_bkt/type2/2a}+\LP
%   B\RP\bkt{media/kauf_bkt/type2/2b}\\
% &=A \LP
%     A\bkt{media/kauf_bkt/type2/3a}+\LP
%     B\RP\bkt{media/kauf_bkt/type2/4}\RP
%  +\LP B\RP\LP A\bkt{media/kauf_bkt/type2/6b}+\LP
%     B\RP\bkt{media/kauf_bkt/type2/6a}\RP\\
% &=\LP A^2+ABC+B^2\RP\bkt{media/kauf_bkt/type2/6a}
%  +\LP AB\RP\bkt{media/kauf_bkt/type2/6b}\\
% \end{aligned}
% \end{equation}
% To ensure that the bracket polynomial is an invariant we need
% to select $A, \ B, \ C$ so that $A^2+ABC+B^2=0$ and $AB=1$. The
% conditions are satisfied by
% $A=A$, $B=A^{\m 1}$, and $C=\LP -A^2 -A^{\m2}\RP$ (justifying the
% second criteria
% of Definition~\ref{kb-def-kaufb}). We then compute from the
% beginning (\ref{sec-kb-math-t2ftb}),
% showing the desired result.

% \begin{equation}
% \label{sec-kb-math-t2ftb}
% \begin{aligned}
%  \bkt{media/kauf_bkt/type2/1}
% &=A\bkt{media/kauf_bkt/type2/2a}+\LP
% B\RP\bkt{media/kauf_bkt/type2/2b}\\
%  &=A \LP
% A\bkt{media/kauf_bkt/type2/3a}+\LP
% A^{\ \m 1}\RP\bkt{media/kauf_bkt/type2/4}\RP+\LP
% A^{\ \m 1}\RP\LP A\bkt{media/kauf_bkt/type2/6b}+\LP
% A^{\ \m 1}\RP\bkt{media/kauf_bkt/type2/6a}\RP\\
%  &= \LP
% A^2+A^{\ \m
% 2}\RP\bkt{media/kauf_bkt/type2/6a}+\bkt{media/kauf_bkt/type2/4}+\bkt{media/kauf_bkt/type2/6b}\\
%  &= \LP
% A^2+A^{\ \m 2}\RP\bkt{media/kauf_bkt/type2/6a}+\LP-A^{2}-A^{\ \m
% 2}\RP\bkt{media/kauf_bkt/type2/6a}+\bkt{media/kauf_bkt/type2/6b}\\
%  &=\bkt{media/kauf_bkt/type2/6b}\\
% \end{aligned}
% \end{equation}

% \end{proof}Now, utilizing the fact that $\LA\,\RA$ is invariant for
% type II, we will prove
% the same for type III.

% \begin{theorem}{}{thm-typeiii_bkt}The equality of equation
% (\ref{thm-kb-math-t3}) holds.
% \begin{equation}
% \label{thm-kb-math-t3}
% \LA \img{media/kauf_bkt/type3/1}\RA=\LA \img{media/kauf_bkt/type3/6} \RA
% \end{equation}

% \end{theorem}\begin{proof}We begin by smoothing the crossing at the
% center of each diagram in the
% equality, yielding

% \begin{equation}
% \label{eq-kbkt-t3-1}
% \bkt{media/kauf_bkt/type3/1}
% =A\bkt{media/kauf_bkt/type3/2b}+A^{\ \m 1}\bkt{media/kauf_bkt/type3/2a}
% \end{equation}

% \begin{equation}
% \label{eq-kbkt-t3-2}
% \bkt{media/kauf_bkt/type3/6}
% =A\bkt{media/kauf_bkt/type3/3b}+A^{\ \m 1}\bkt{media/kauf_bkt/type3/3a}
% \end{equation}
% Now, on the clockwise smoothing equation (\ref{eq-kbkt-t3-1}), we
% execute a type II
% move and obtain Figure~\ref{fig-inv-t3-c}.

% \begin{figure}[H]
% \centering
% \includegraphics[width=0.5\linewidth]{files/2b_color-1bfc5e5306fab67c2456967cbbdb0f6b.pdf}
% \caption[First smoothing of a type III move.]{On the left is a
% clockwise smoothing of the type III move type crossing
% Figure~\ref{fig-knot_def-r3}. On the right is an application of the
% type II move on the
% under strand.}
% \label{fig-inv-t3-c}
% \end{figure}

% Similarly, the anti-clockwise smoothing from equation
% (\ref{eq-kbkt-t3-2}) we execute a type II
% move to obtain Figure~\ref{fig-inv-t3-cc}.

% \begin{figure}[H]
% \centering
% \includegraphics[width=0.5\linewidth]{files/2a_color-5837b7c8e662b5e2c0f8448cce0ff0fa.pdf}
% \caption[Second smoothing of a type III move.]{On the left is an
% anti-clockwise smoothing of the type III move type crossing
% Figure~\ref{fig-knot_def-r3}. On the right is an application of the
% type II move on the
% under strand.}
% \label{fig-inv-t3-cc}
% \end{figure}

% Since the type II move is invariant for the bracket polynomial
% (Theorem~\ref{thm-typeii_bkt})
% the chain of equalities in equation (\ref{eq-kbkt-t3-final}) shows
% the desired result.

% \begin{equation}
% \label{eq-kbkt-t3-final}
% \begin{aligned}
% \bkt{media/kauf_bkt/type3/1}
% &=A\bkt{media/kauf_bkt/type3/2b}+A^{\ \m 1}\bkt{media/kauf_bkt/type3/2a}\\
% &=A\bkt{media/kauf_bkt/type3/3b}+A^{\ \m
% 1}\bkt{media/kauf_bkt/type3/3a}&&\text{(by type II move)} \\
% &=\bkt{media/kauf_bkt/type3/6}
% \end{aligned}
% \end{equation}

% \end{proof}Now for the final Reidemeister move, the type I move, we
% compute two chains of
% equality, (\ref{sec-kb-math-t11}) and (\ref{sec-kb-math-t12}).

% \begin{equation}
% \label{sec-kb-math-t11}
% \begin{aligned}
%  \bkt{media/kauf_bkt/type1/1b} & =
% A\bkt{media/kauf_bkt/type1/2a}+A^{\ \m 1}\bkt{media/kauf_bkt/type1/2b}\\
% & = A\bkt{media/kauf_bkt/type1/3}+A^{\ \m 1}\LP
% -A^2-A^{\ \m 2}\RP\bkt{media/kauf_bkt/type1/3}\\
% & = \LP A+
% -A^-A^{\ \m 3}\RP\bkt{media/kauf_bkt/type1/3}\\
% & = -A^{\ \m 3}\bkt{media/kauf_bkt/type1/3}\\
% \end{aligned}
% \end{equation}

% \begin{equation}
% \label{sec-kb-math-t12}
% \begin{aligned}
%  \bkt{media/kauf_bkt/type1/1} & =
% A\bkt{media/kauf_bkt/type1/2b}+A^{\ \m 1}\bkt{media/kauf_bkt/type1/2a}\\
%  & = A\LP
% -A^2-A^{\ \m 2}\RP\bkt{media/kauf_bkt/type1/3}+A^{\ \m
% 1}\bkt{media/kauf_bkt/type1/3}\\
% & = \LP
% -A^3-A^{\ \m 1}+A^{\ \m 1}\RP\bkt{media/kauf_bkt/type1/3}\\
% & = -A^{3}\bkt{media/kauf_bkt/type1/3}\\
% \end{aligned}
% \end{equation}

% Observe, with (\ref{sec-kb-math-t11}) and (\ref{sec-kb-math-t12}),
% that the bracket polynomial
% as it stands is not invariant under Reidemeister moves. We must augment the
% bracket polynomial to find the invariance we would like.

% \subparagraph{Writhe}

% Before we can develop the concept of the \textbf{writhe} of a knot diagram, we
% need to define an \textbf{orientation} for a knot diagram. Imagine
% the strands of a knot to be a roller
% coaster, it's clear that the rollercoster car has two choices for
% direction. One
% direction we intuitively call forward and its oposite backward. Just like most
% rollercosters, each component of a knot is a circle we can choose
% to traverse in
% one of two directions. On a knot diagram we call this choice of
% forward, an orientation,
% see (Figure~\ref{fig-writhe-orientation}).

% \begin{figure}[H]
% \centering
% \includegraphics[width=0.3\linewidth]{files/5_1-8d4df8b4b3c3dbcd7f8dfc4314714dc7.pdf}
% \caption[An orientation applied to the knot.]{An orientation
% applied to the knot 5\textsubscript{1}. Following the strand in the direction
% of the arrow, you'll find that you arrive where you started.}
% \label{fig-writhe-orientation}
% \end{figure}

% We now zoom in on our oriented diagram, and focus on each crossing. Since the
% whole diagram is oriented, each strand in the local crossing has an induced
% orientation, indicated by arrow heads on the strands. We arrange the crossing,
% by rotating, so that the over strand arrowhead points up, notice then that
% there are two possibilities for the under strand; pointing west or pointing
% east. We call the crossing with a west pointing under strand a positive (+)
% crossing (Figure~\ref{fig-inv-or-positive}), while an east pointing
% under strand is called a
% negative (-) crossing (Figure~\ref{fig-inv-or-negative}).

% \begin{figure}[H]
%     \centering
% \begin{subfigure}[c]{.2\textwidth}
% \centering
% \includegraphics[width=\textwidth]{files/crossing_+-a152d2a53dd5f6118d5896ef3bd810a3.pdf}
% \caption[A positive crossing.]{A positive crossing.}
% \label{fig-inv-or-positive}
% \end{subfigure}
% ~
% \begin{subfigure}[c]{.2\textwidth}
% \centering
% \includegraphics[width=\textwidth]{files/crossing_--eed911694e1575f898c0b8af79213f77.pdf}
% \caption[A negative crossing.]{A negative crossing.}
% \label{fig-inv-or-negative}
% \end{subfigure}
% \caption[The two crossing orientations.]{The two crossing orientations.}
% \end{figure}

% \begin{note}
% A hint for determining the orientation of a crossing in practice is
% a right-hand
% rule. After the knot has been oriented, imagine your right hand with your
% fingers extended and palm towards you. Align your hand with the over strand so
% that your thumb points with the arrowhead. Now if your fingers are
% pointing with the
% arrowhead of the under strand, the crossing is positive, if your fingers are
% opposite the arrowhead of the under strand, the crossing is negative.
% \end{note}

% Given an oriented knot diagram, we can now classify each crossing
% as either positive or
% negative, providing us with enough information to define the writhe of a knot.

% \begin{definition}{}{def-writhe}The writhe, $w\LP K\RP$, of an
% oriented knot diagram $K$ is defined to be:
% \begin{equation}
% \begin{aligned}
% w\LP K\RP&=\LP\text{count of positive crossings in } K\RP\\
% &-\LP\text{count of negative crossings in } K\RP
% \end{aligned}
% \end{equation}

% \end{definition}
% \begin{note}
% The writhe of a diagram is not an invariant for the knot. We can see this by
% adding a number of type I moves to a diagram, which will give us two different
% writhes for the same knot.
% \end{note}

% The writhe is fairly simple to calculate, we will now carry out an example
% calculation for knot 5\textsubscript{1}.

% We start by taking the oriented version of knot 5\textsubscript{1} seen in
% Figure~\ref{fig-writhe-orientation} and modifying the diagram with
% local pictures for each
% crossings seen in Figure~\ref{fig-writhe-local}.

% \begin{figure}[H]
% \centering
% \includegraphics[width=0.4\linewidth]{files/fig-writhe-local-4733d4e765087c4c45e8c8937260d970.pdf}
% \caption[An oriented five crossing knot.]{An oriented five crossing
% knot, each local picture around a vertex is magnified
% to show local crossing signs.}
% \label{fig-writhe-local}
% \end{figure}

% Classifying each crossing as positive or negative, we can count finding 5
% positive and 0 negative crossing giving

% \begin{equation}
% w\LP\img{media/kauf_bkt/orientation/5_1}\RP=5+0=5
% \end{equation}

% \subparagraph{Kauffman Bracket: Type I}

% With the writhe, we're ready to augment the bracket polynomial applied to the
% type I move. We do this with the addition of multiplication by a
% $-A^{\ \m 3w\LP K\RP}$ term, obtaining Theorem~\ref{thm-typei}.

% \begin{theorem}{}{thm-typei}The equality of equation
% (\ref{thm-kb-math-t1}) holds.
% \begin{equation}
% \label{thm-kb-math-t1}
%     \begin{aligned}
%         -A^{\ \m 3w\LP
%             \img{media/kauf_bkt/type1/1b}\RP}\bkt{media/kauf_bkt/type1/1b}
%             & = \bkt{media/kauf_bkt/type1/3}\\
%         -A^{\ \m 3w\LP
%             \img{media/kauf_bkt/type1/1}\RP}\bkt{media/kauf_bkt/type1/1}
%             & = \bkt{media/kauf_bkt/type1/3}\\
%     \end{aligned}
% \end{equation}

% \end{theorem}\begin{proof}We first compute our writhes
% \begin{equation}
%     \begin{aligned}
%             -w\LP \img{media/kauf_bkt/type1/1b}\RP=\ \m 1\\
%             -w\LP \img{media/kauf_bkt/type1/1}\RP=1\\
%     \end{aligned}
% \end{equation}
% now computing each flavor of type I move in turn
% \begin{equation}
% \begin{aligned}
%         -A^{\ \m 3w\LP
%     \img{media/kauf_bkt/type1/1b}\RP}\bkt{media/kauf_bkt/type1/1b}
%     & =
%     \LP-A^{3}\RP\LP-A^{\ \m 3}\bkt{media/kauf_bkt/type1/3}\RP \\
%         & =
%     \LP-A^{3}\RP\LP-A^{\ \m 3}\bkt{media/kauf_bkt/type1/3} \RP\\
%         & = \bkt{media/kauf_bkt/type1/3} \\
% \end{aligned}
% \end{equation}
% \begin{equation}
% \begin{aligned}
%         -A^{\ \m 3w\LP
%     \img{media/kauf_bkt/type1/1}\RP}\bkt{media/kauf_bkt/type1/1}
%     & =
%     \LP-A^{\ \m 3}\RP\LP-A^{3}\bkt{media/kauf_bkt/type1/3}\RP\\
%         & =
%     \LP-A^{\ \m 3}\RP\LP-A^{3}\bkt{media/kauf_bkt/type1/3} \RP\\
%         & = \bkt{media/kauf_bkt/type1/3} \\
% \end{aligned}
% \end{equation}
% the desired sets of equality.

% \end{proof}Having gained each equality needed for an invariant of
% knots, we finish the
% section with the following theorem.

% \begin{theorem}The Jones polynomial (Definition~\ref{def-jones}) is
% an invariant of oriented knots.

% \end{theorem}\begin{proof}Combining Theorem~\ref{thm-typeii_bkt},
% Theorem~\ref{thm-typeiii_bkt}, and Theorem~\ref{thm-typei},
% substituting $A$ for
% $t$ we obtain the result.

% \end{proof}
\subsection{Knot Notations}

The final topic to cover in our treatment of foundations for knot theory is
notational strategies for knots. In \nf{subsec-knot_def}, we came
across our first
notational strategy for knots, the knot diagrams. While diagrams are a great
human-readable way to note a knot, when the tasks of enumeration and computation
(by hand) are considered, knot diagrams quickly show deficiencies. These
deficiencies become intractable when a computer is brought into the picture. As
a remedy for this issue, knot theorists have invented several combinatorial
notations for knots. Perhaps the most historically important knot notation for
use in tabulation by computer is the Dowker-Thistlethwaite (DT) notation
(\nf{sec-proj-note_dt}), developed by its namesakes specifically for use in
computational tabulation. Each notational strategy used in knot theory has
strengths and weaknesses. For example, using DT notation for computation of the
Jones polynomial may be more cumbersome
than using the Planar Diagram
(PD) notation (\nf{sec-proj-note_pd}) for the same task, as PD directly encodes
crossings while DT encodes a walk on a strand. The remainder of this section
will be the development of the Conway notation, which lays the foundation for
the work in this thesis.

%  prettier-ignore-start

\subsubsection{Conway Notation}\label{sec-conway}

%  prettier-ignore-end

In Section~\ref{sec-history-of-tabulation}, we saw that Conway
claimed to have enumerated
knots up to 11 crossings in ``a few hours''. Conway accomplished this
by breaking
knots into building blocks he called tangles. This section gives an outline of
the tools he developed and used to achieve those ``few hours'' of amazing
efficiency.

%  prettier-ignore-start

\paragraph{Definition of a Tangle}\label{subsubsec-deftangles}

%  prettier-ignore-end

Our first step in unlocking Conway's tabulation secrets is the definition of a
tangle. We will give Conway's original definition followed by a description of
what this looks like for a three dimensional embedding for a knot.

\begin{definition}{Conway, Page 330
  \textbf{\citep{conwayEnumerationKnotsLinks1970}}}{}
  We define a \textbf{tangle} as a portion of a knot diagram from
  which there emerge just 4
  arcs pointing in the compass directions $NW, \ NE, \ SW, \ \text{and }SE$.

\end{definition}These boundaries that split knots at four points are
called \textbf{Conway circles},
and we call the points $NW,\ NE,\ SW,\ \text{and }SE$ \textbf{boundary points}.
Formally, we can consider a Conway circle to be a Jordan
curve\footnote{A Jordan curve is a simple closed curve. This can be
  thought of as a curve
drawn on a piece of paper that has: 1) No end points. 2) No self
intersections.}
meeting the knot diagram in exactly four points
\citep{bonahonNewGeometricSplittings2016}. In general, we prefer our
Conway circles
to actually be circles in the colloquial sense. Luckily, the circle and Jordan
curve constructions are equivalent. This can be seen by a straightforward
isotopy of one into the other, per Figure~\ref{fig-generic_jordan_iso}.

\begin{figure}[H]
\centering
\includegraphics[width=\linewidth]{files/conway_circ_isotopy-9a8c6805166c373f6a290fdb7e952bb9.pdf}
\caption[An isotopy turning Jordan curves into circles.]{An isotopy
turning Jordan curves into circles.}
\label{fig-generic_jordan_iso}
\end{figure}

We move our attention to the three dimensional analog for a Conway circle, the
\textbf{Conway sphere}. A Conway sphere, similar to the Conway
circle, is an $S^2$
that encapsulates a portion of a knot so that the knot intersects the sphere in
exactly four points. Here we see the first example of our preference for ambient
space to be $S^3$ as opposed to $\R^3$. When a knot in $S^3$ is split by a
Conway sphere, the ambient $S^3$ is decomposed into two $B^3$, each with a
portion of the knot. Meaning, a single Conway sphere splits a knot into a pair
of tangles.

%  prettier-ignore-start

\paragraph{Basic Tangles}\label{subsubsec-basic_tangles}

%  prettier-ignore-end

Often, when thinking about a new construction, we focus on the simplest object
that can be created with the construction. In the case of drawing Conway circles
to build tangles, the simplest tangles are a tangle with no crossings (the
0 tangle Figure~\ref{prelim-fig-basic_0}) and a tangle with a single
crossing (the +1
tangle Figure~\ref{prelim-fig-basic_1}).

\begin{figure}[H]
\centering
\begin{subfigure}[c]{0.3\textwidth}
\centering
\includegraphics[width=\textwidth]{files/0-4d1017ef91a5026c6771eb9f6ca80be3.pdf}
\caption[A tangle with no crossings.]{A tangle with no crossings,
called the 0 tangle.}
\label{prelim-fig-basic_0}
\end{subfigure}
\quad
\quad
\quad
\begin{subfigure}[c]{0.3\textwidth}
\centering
\includegraphics[width=\textwidth]{files/1-8c32584b33e2a4295948c6caf7fd3d15.pdf}
\caption[A tangle with a single crossing.]{A tangle with a single
crossing, called the 1 tangle.}
\label{prelim-fig-basic_1}
\end{subfigure}
\caption[Two basic tangles.]{Two basic tangles.}
\label{fig-basic_tangles}
\end{figure}
%  prettier-ignore-start

\paragraph{Rotation and Mirroring of Tangles}\label{subsubsec-tangle_flips}

%  prettier-ignore-end

Consider a \textbf{generic tangle}, as seen in
Figure~\ref{fig-generic_tangle}, where
orientation (the position of the NW point) of data in the interior of the Conway
circle is indicated by a broken $T$.

\begin{figure}[H]
\centering
\includegraphics[width=0.3\linewidth]{files/generic_tangle-a689b8d843d000777715cd84c327d565.pdf}
\caption[A generic tangle.]{A generic tangle with a broken T.}
\label{fig-generic_tangle}
\end{figure}

We can manipulate this tangle by the set of rotations, clockwise or
anti-clockwise. Each rotation in turn gives a new arrangement of the interior
data. We can also manipulate the tangle by the set of flips, one around the core
x-axis and one around the y-axis. Each flip gives an arrangement of the interior
data. Pairing flips with rotations gives the table seen in
Figure~\ref{fig-tangle_flips}.

\begin{figure}[H]
\centering
\includegraphics[width=0.7\linewidth]{files/fig-tangle_flips-3e9facf2c059882951c81a73f9e178fa.pdf}
\caption[A table with all unique rotations and flips for a generic
tangle.]{A table with all unique rotations and flips for a generic tangle.
From top to bottom in the first column:$\ \bullet$ No Flip$\ \bullet$
Flip around the north south axis.
From left to right in each row:$\ \bullet$ No rotation$\ \bullet$
rotation quarter turn clockwise$\ \bullet$ rotation half turn clockwise
$\ \bullet$ rotation three quarter turn clockwise$\ \bullet$ rotation
quarter turn clockwise}
\label{fig-tangle_flips}
\end{figure}

When we apply this set of flips and rotations to the basic tangles seen in
\nf{subsubsec-basic_tangles}, we obtain the two additional basic tangles seen in
Figure~\ref{fig-basic_tangles_extra}.

\begin{figure}[H]
\centering
\begin{subfigure}[c]{0.3\textwidth}
\centering
\includegraphics[width=\textwidth]{files/inf-6468e70ef1d9a2cf5e8722b2b4f72d73.pdf}
\caption[A tangle with no crossings.]{A tangle with no crossings,
called the $\infty$ tangle.}
\label{prelim-fig-basic_nc-inf}
\end{subfigure}
\quad
\quad
\quad
\begin{subfigure}[c]{0.3\textwidth}
\centering
\includegraphics[width=\textwidth]{files/m1-0b2e6635a4f891072ad82f405b21419d.pdf}
\caption[A tangle with a single crossing.]{A tangle with a single
crossing, called the $\m 1$ tangle.}
\label{prelim-fig-basic_c-m1}
\end{subfigure}
\caption[Two additional basic tangles.]{Two additional basic tangles.}
\label{fig-basic_tangles_extra}
\end{figure}
%  prettier-ignore-start

\paragraph{Operations on Tangles}\label{subsubsec-conway_calc}

%  prettier-ignore-end

In addition to the rotations and flips, Conway introduced a calculus on tangles
\citep{conwayEnumerationKnotsLinks1970}. This calculus allowed Conway
to build the
simple basic tangles into iteratively more complex tangles.

%  prettier-ignore-start

\paragraph{Minus Tangle}\label{subsubsec-conway_minus}

%  prettier-ignore-end

For a generic tangle $T$, we call the tangle generated from a clockwise rotation
and flip around the y-axis the negative of $T$, notated $-T$. Equivalently, this
can be thought of as rotating the tangle around the $NW$ and $SE$ axis
(Figure~\ref{fig-opo-minus}).

\begin{figure}[H]
\centering
\includegraphics[width=0.7\linewidth]{files/fig-opo-minus-1fdce40a8fa385c913446181d227f7bf.pdf}
\caption[The ``-'' of a tangle]{Rotating a tangle around the $NW$
$SE$ diagonal, yielding the negative of the
tangle.}
\label{fig-opo-minus}
\end{figure}

\begin{note}
Observe that the minus of the +1 tangle
(Figure~\ref{prelim-fig-basic_1}) is the $\m 1$
tangle (Figure~\ref{prelim-fig-basic_c-m1}).
\end{note}

%  prettier-ignore-start

\paragraph{Tangle Addition}\label{subsubsec-opo-plus}

%  prettier-ignore-end

For a pair of generic tangles, $A$ and $B$, we construct their sum $A+B$ by
first aligning $A$ and $B$ horizontally. We then connect the $NE$ and $SE$ of
$A$ to the $NW$ and $SW$ of $B$, as seen in Figure~\ref{fig-opo-plus}.

\begin{figure}[H]
\centering
\includegraphics[width=\linewidth]{files/fig-opo-plus-5f318d6baa2485438c40832900598a5b.pdf}
\caption[The sum of two generic tangles.]{The sum of two generic tangles.}
\label{fig-opo-plus}
\end{figure}

The class of tangles built by successive addition of the $\pm 1$ basic tangles
are called the \textbf{integral tangles}.

\paragraph{Tangle Multiplication}

For a pair of generic tangles, $A$ and $B$, we construct their product, $A*B$
(or $A\ B$) by first aligning $A$ and $B$ horizontally. We then take $-A$ and
sum the two resulting tangles, equivalent to $-A+B$, as seen in
Figure~\ref{fig-opo-times}.

\begin{figure}[H]
\centering
\includegraphics[width=\linewidth]{files/fig-opo-times-0b4f2e944f3018fb2228367498107e23.pdf}
\caption[The product of two generic tangles.]{The product of two
generic tangles.}
\label{fig-opo-times}
\end{figure}

\begin{note}
Notice that
\begin{equation}
A*0=-A+0=-A
\end{equation}
\end{note}

\paragraph{Tangle Ramification}

For a pair of generic tangles, $A$ and $B$, we construct their ramification
$A,B$ by first aligning $A$ and $B$ horizontally. We then take $-A$ and $-B$ and
sum the resulting tangles. This makes ramification equivalent to $-A -B$ or
$A0+B0$, as seen in Figure~\ref{fig-opo-ramification}.

\begin{figure}[H]
\centering
\includegraphics[width=\linewidth]{files/fig-opo-ramification-ab32acc0ffde9c4a27ffdeb51ac9edaa.pdf}
\caption[The ramification of two generic tangles.]{The ramification
of two generic tangles.}
\label{fig-opo-ramification}
\end{figure}

%  prettier-ignore-start

\paragraph{Indicating Precedence}\label{subsubsec-opo-precedence}

%  prettier-ignore-end

With a set of operations comes the desire to chain multiple operations together.
The precedence for operations on tangles is indicated by parentheses in the
obvious way, as seen in Figure~\ref{fig-opo-prec}.

\begin{figure}[H]
\centering
\includegraphics[width=\linewidth]{files/fig-opo-prec-d6df532c9a7248cf9c91aaf460b986e2.pdf}
\caption[Multiple tangle operations chained together.]{Multiple
operations chained together with precedence indicated by parentheses.}
\label{fig-opo-prec}
\end{figure}

%  prettier-ignore-start

\paragraph{The Flype}\label{subsubsec-opo-flype}

%  prettier-ignore-end

When working in this calculus of tangles, a common situation you find yourself
in is one where the 1 (or $\ \m 1$) tangle is added to a tangle. In this
situation, we can move the 1 crossing from one side of $T$ to the other by a
\textbf{flype}. To complete a flype, we grab the top (north) and
bottom (south) of
the tangle and rotate (opposite the handedness of the crossing) as in
Figure~\ref{fig-opo-flype}.

\begin{figure}[H]
\centering
\includegraphics[width=\linewidth]{files/flype-c1c5e6a071b468e3e84bc83a5a2255ad.pdf}
\caption[Demonstrating the flype move.]{A $(+)$-flype on the top and
a $( -)$-flype on the bottom. Note that the generic
tangle is flipped around the $x$-axis during the flype.}
\label{fig-opo-flype}
\end{figure}

\paragraph{Closures}

Since Conway's interest was in knots, he naturally needed ways to close up a
tangle to form a knot. In this section, we will introduce two ways that this can
be accomplished. The first closure method is the simple tangle closure, where
points on a tangle are connected. The second closure method is the insertion of
multiple tangles into a graph.

\subparagraph{Simple Tangle Closures}

The first closure method is the simple tangle closure. For a generic tangle $T$,
we have two options for how to simply close up the tangle. One option is to
connect a strand from $NW$ to $NE$ and a strand from $SW$ to $SE$
(Figure~\ref{fig-closure-num}), called the numerator closure. The
alternative is the
denominator closure, formed by connecting a strand from $NW$ to $SW$ and a
strand from $NE$ to $SE$ (Figure~\ref{fig-closure-den}). In both cases, we
introduce no additional crossings.

\begin{figure}[H]\label{fig-closure-prec}
\centering
\begin{subfigure}[c]{0.4\textwidth}
\centering
\includegraphics[height=\textwidth]{files/fig-closure-num-b689f8228388f8fb25c454811afe92c1.pdf}
\caption[The numerator closure of a tangle.]{The numerator closure of a tangle.}
\label{fig-closure-num}
\end{subfigure}
\quad
\quad
\quad
\begin{subfigure}[c]{0.4\textwidth}
\centering
\includegraphics[width=\textwidth]{files/fig-closure-den-82a12a03e602f75e4b812dd59cbaae3d.pdf}
\caption[The denominator closure of a tangle.]{The denominator
closure of a tangle.}
\label{fig-closure-den}
\end{subfigure}
\caption[The two simple closures of a tangle.]{The two simple
closures of a tangle.}

\end{figure}
%  prettier-ignore-start

\subparagraph{Tangle Insertions}\label{subsubsec-opo-insert}

%  prettier-ignore-end

With the calculus of tangles and simple closures, Conway was able to enumerate a
substantial number of knots, but not all. We should notice a common theme with
the calculus, when starting with basic tangles every operation forms
a bigon\footnote{A bigon is a polygon with two sides. In the same way
that an octagon has
eight sides or a trigon (triangle) has three.} between two tangles in the
knot shadow. We can collapse bigons in the knot shadow by deleting edges and
merging the two vertices of a bigon. For a knot formed by only the operations
on basic tangles and with simple closures, if we iteratively collapse
all bigons we obtain a
four-valent planar\footnote{A planar graph is one that can be drawn
in the plane without edges
overlapping.} graph\footnote{A graph is said to be four valent if
each vertex has four edge ends
connected to it. In Figure~\ref{fig-bigon_collapse}, the result is
four valent.} with one vertex, per
Figure~\ref{fig-bigon_collapse}. The class of knots who have a
presentation where bigons can
be collapsed to a single vertex with two self edges are called the
\textbf{algebraic
knots}.

\begin{figure}[H]
\centering
\includegraphics[width=\linewidth]{files/fig-bigon_collapse-61482270433167ccce09792d12223a13.pdf}
\caption[Bigon collapsing.]{The collapsing of the bigons in a knot
shadow from left to right: $\ \bullet$ A trefoil knot. $\ \bullet$ A
knot shadow for a trefoil knot with a
bigon highlighted. $\ \bullet$ The previously highlighted bigon
collapsed, and a new
bigon highlighted. $\ \bullet$ A graph with no bigons.}
\label{fig-bigon_collapse}
\end{figure}

To obtain a knot that has non-bigon connections between inputs, we will first
identify a four-valent planar graph that has non-bigon connections between
vertices. The class of graph that is most useful here are the polygon graphs.
For example, the $6^{**}$ graph (or octahedron) can be seen in
Figure~\ref{fig-6starstar}. Within the
$6^{**}$ graph, we notice triangular regions between vertices.

\begin{figure}[H]
\centering
\includegraphics[width=0.5\linewidth]{files/fig-6starstar-0873a710a7445b1d9dea336c329f5d83.pdf}
\caption[A 6 vertex polygon. ]{A four valent planar graph with six
vertices and triangular regions between
vertices. When the graph is placed on the surface of an $S^2$ we get the
octahedron.}
\label{fig-6starstar}
\end{figure}

The simplest thing we can do from here is consider the graph as a knot shadow,
and for each vertex, choose an over and under strand. While that method would
give us a knot, it is limiting. Less limiting is a process of tangle insertion.
In this process we consider each vertex a boundary of a Conway circle in which
we can place a tangle generated with the Conway calculus. When we insert the
tangle into the Conway circle (vertex in the graph), the
$NW,\,NE,\,SW,\,\text{and }SE$ points of the tangle are connected to marked
points of the Conway circle (the four edges of the vertex). An example of a
tangle insertion into $6^{**}$ can be seen in
Figure~\ref{fig-6starstar_insurtion}.

\begin{figure}[H]
\centering
\includegraphics[width=0.5\linewidth]{files/fig-6starstar_insurt-a3fdbf3ad9bd55c3bcc22b2d2b2dcff6.pdf}
\caption[Tangle insertions.]{Tangles inserted into the $6^{**}$
tangle, with Conway notation
\\$6^{**}\ast.1\ 2\ 2\ 3\ 1.1\ 2\ 2\ 3\ 1.1\ 2\ 2\ 3\ 1.1\ 2\ 2\ 3\ 1.1\ 2\ 2\ 3\ 1$
The $\ast$ labeled vertex defines the four boundary points of the
resulting tangle. }
\label{fig-6starstar_insurtion}
\end{figure}

When each vertex has a tangle inserted, the result is a knot. When $n$
vertices are left empty, the result is a tangle with $n$ boundary Conway
circles. If $n$ is 1, we have a tangle in the sense we've been discussing. To
reduce ambiguity, we mark in the graph with a $\ast$ in that empty vertex. On
each polygon graph, we select a canonical ordering of the vertices,
per Figure~\ref{fig-6starstar_ordered}. When
notating the tangle insertions, we list the subtangles we wish to insert in
graph canonical order, each separated by a period and an empty vertex indicated
with $\ast$. Following the terminology outlined by Conolly
\citep{connollyClassificationTabulation2string2021}, we call these
singularly marked
polygonal graphs \textbf{constellations}. We call the tangle diagrams
generated by
this approach the \textbf{polygonal diagrams}. If a tangle has no algebraic
representative we call it a \textbf{polygonal tangle}.

\begin{figure}[H]
\centering
\includegraphics[width=0.5\linewidth]{files/fig-6starstar_orderd-c5a9af08f3c2d72cc2ea67857cfdc271.pdf}
\caption[Order of a polygon.]{The $6^{\ast\ast}$ graph with an order applied.}
\label{fig-6starstar_ordered}
\end{figure}

\section{Foundations of Tangles}

So far we have used tangles as a building block for knots, we will switch gears
slightly to consider a tangle as our main object of interest. This section will
give the foundations of the theory of tangles needed for the remainder of this
thesis.

\subsection{Tangle Equivalence}

As we saw for knots in \nf{subsec-knot_equivalence}, if we want to
tell two tangles
apart, we first need to be able to identify when tangles are the same. Our
development of the concept of equality of tangles follows closely to that of
knots. From \nf{subsubsec-deftangles}, we have definitions for tangles in the
context of knot diagrams and three-dimensional embeddings of knots. This tells
us that the concepts of equality developed in
\nf{subsec-knot_equivalence}, namely
ambient isotopy and Reidemeister moves, will apply in the tangle case with two
key differences.

The first difference with tangles when compared to knots is that we restrict
ambient isotopy and Reidemeister moves from pushing a strand through the
Conway sphere, or Conway circle. The second difference is the handling of the
boundary points. There are two conventions for how to handle equality with the
boundary points; first allowing the boundary points to move on the Conway
sphere, and second fixing the boundary points on the Conway sphere.

\subsubsection{Moveable Boundary Points}

Our first case for handling the boundary of a tangle is allowing the boundary
points to move freely on the Conway sphere. In this case, a generic tangle, is
equivalent to each of its rotations and flips (\nf{subsubsec-tangle_flips}). In
addition to the rotation and flip equivalence, a moveable boundary allows us to
unwind the outermost integral components of a tangle
(Figure~\ref{fig-tangl_eq-unwinding}).

\begin{figure}[H]
\centering
\includegraphics[width=\linewidth]{files/fig-tangl_eq-unwindi-273d573f9a22c37519018e3a651b4f17.pdf}
\caption[Unwinding of integral tangles.]{The progressive unwinding of
integral tangles, leaving a basic 0 tangle.}
\label{fig-tangl_eq-unwinding}
\end{figure}

\begin{note}
In the moveable boundary case, there is only one basic tangle, the 0 tangle.
\end{note}

For this thesis, we will assume tangles have a fixed boundary unless explicitly
mentioned.

\subsubsection{Non-moveable Boundary Points}

The non-moveable case is the more straightforward of the two boundary concepts
of equality for tangles. In the fixed boundary world, we have four distinct
basic tangles $1,\ \ \m 1,\ 0,\ \text{and } \infty$ seen in
Figure~\ref{fig-basic_tangles}
and Figure~\ref{fig-basic_tangles_extra}. These are all distinct when
the boundary is fixed. In Figure~\ref{fig-tangl_eq-fixed}
we find two tangles, each with two crossings, but not equivalent by Reidemeister
moves.

\begin{figure}
\centering
\begin{subfigure}[t]{.45\textwidth}
\centering
\includegraphics[width=\textwidth]{files/fig-tang_eq-fixed-2-b02757c89b1831f1ec4678fa05e05464.pdf}
\caption[A horizontal integral tangle with two crossings.]{A
horizontal integral tangle with two crossings.}
\label{fig-tang_eq-fixed-2}
\end{subfigure}
~
\begin{subfigure}[t]{.45\textwidth}
\centering
\includegraphics[width=\linewidth]{files/fig-tang_eq-fixed-v2-cb810de4fba3b5df9ca4eab5dcecff9d.pdf}
\caption[A vertical integral tangle with two crossings.]{A vertical
integral tangle with two crossings.}
\label{fig-tang_eq-fixed-v2}
\end{subfigure}

\caption[Two nonequivalent tangles with two crossings.]{Two no
equivalent tangles with two crossings.}
\label{fig-tangl_eq-fixed}
\end{figure}
%  prettier-ignore-start

\subsection{Modified Tangle Operations}\label{subsec-tangle_operations}

%  prettier-ignore-end

In \nf{subsubsec-conway_calc} we saw Conway's calculus of tangles. While this
construction is powerful and flexible, it's rather unintuitive and cumbersome
when combined with computer methods. In this section, we will describe a
slightly modified, but equivalent, version of the calculus. This version of
tangle arithmetic is due to Kauffman and Goldman
\citep{goldmanRationalTangles1997}.

Instead of Conway's three operations on the two basic tangles 1 and 0, this
arithmetic needs the two basic operations and all four basic tangles
(Figure~\ref{fig-basic_tangles} and
Figure~\ref{fig-basic_tangles_extra}). The first operation in this
arithmetic is exactly Conway's horizontal sum, $+$
(Figure~\ref{fig-opo-plus}). The second
operation is the vertical sum, $\vee$, sometimes written
$*$\citep{goldmanRationalTangles1997} or
$+^\prime$\citep{kauffmanClassificationRationalKnots2002}. For
generic tangle $A$ and
$B$, $A\vee B$ is built analogously to the $+$ but stacking $A$ vertically on
top of $B$ instead of horizontally (Figure~\ref{fig-opo-vee}).

\begin{figure}[H]
\centering
\includegraphics[width=\linewidth]{files/fig-opo-vee-0108b868def1c0b26ad7e858e31be367.pdf}
\caption[The vertical sum of two generic tangles.]{The vertical sum
of two generic tangles.}
\label{fig-opo-vee}
\end{figure}

These two operations combined with parentheses as in
\nf{subsubsec-opo-precedence}
give a natural arithmetic structure to the combinations of tangles. We'll see in
later sections how this structure is easily encoded on a computer as a data
structure. We conclude the section by redefining the \textbf{algebraic tangles}.

\begin{definition}{Conway, Page
331\textbf{\citep{conwayEnumerationKnotsLinks1970}}}{def-algebraic}
Any tangle that can be produced by the two binary operations $+$ and $\vee$ on
the four basic tangles is called an \textbf{algebraic tangle}.

\end{definition}
%  prettier-ignore-start

\subsection{Integral Tangles}\label{subsec-integral_tangle}

%  prettier-ignore-end

We finish the chapter with a description of a class of tangles we first
encountered in \nf{subsubsec-opo-plus}. The integral tangles are the
simplest class
of tangle that are built from the basic operations on basic tangles. We start by
defining the horizontal integral tangles.

\begin{definition}{}{}A tangle built from the successive sum of $n +1$
tangles is called a
\textbf{horizontal integral} (or simply integral) $n$ tangle.
Similarly, a sum of $\m 1$ tangles is a
horizontal integral $\m n$ tangle.

\end{definition}It is convenient to notate the horizontal integral
tangles simply by their
corresponding integer, $\pm n$. A similar construction can be defined for the
$\vee$ operation, yielding the vertical integral tangles.

\begin{definition}{}{}A tangle built from the successive vertical sum of
$n+1$ tangles is called a
vertical integral $n$ tangle. Similarly, a vertical sum of $\m 1$ tangles is a
vertical integral $\m n$ tangle.

\end{definition}We will notate the vertical
tangles by $\pm\frac{1}{n}$.
