\chapter{Introduction}

There exists a common view of advanced mathematics as esoteric nonsense only
understandable by elite thinkers. While that sentiment is perhaps deserved by
some mathematical disciplines, knot theory is an area of advanced mathematics
that offers deep significance as well as general accessibility. A lay person can
be easily taught the basic structures of knot theory, while it takes years to
master the details.

%  prettier-ignore-start

\section{An Intuitive Grounding in Knot
Theory}\label{sec-intro-intuit_knot_theory}

%  prettier-ignore-end

A knot, when used in everyday life, is a tool whether it's the
``bunny ear'' knot
holding on your shoe, a decorative monkey's fist on your keychain, or a climbing
hitch securing yourself to a rock wall. Now, when thinking of these tool knots,
we should note one critical attribute, they're made of a single string with open
ends (Figure~\ref{fig-intro-everyday_knot}). Notice with this
construction, no matter how
``knotted'' the string, we can always pull on loops to remove the
knot, leaving us
with only an unknotted string.

\begin{figure}[H]
  \centering
  \includegraphics[width=0.8\linewidth]{files/everyday_knot-3d0a3f087e460add29008825581bbd36.pdf}
  \caption[An everyday knot with open ends.]{An everyday knot with open ends.}
  \label{fig-intro-everyday_knot}
\end{figure}

This leaves us with a somewhat unsatisfying construction, exactly one object, a
string. How might we add some spice to our construction? What if we turn the
string into a circle by gluing the ends as in
Figure~\ref{fig-intro-everyday_knot_closed}?
When we try to wiggle the closed up string around, trying to remove the knot, we
stretch and pull on the string, but regardless of what we do, we end up with a
knot. Constructing a physical model of
Figure~\ref{fig-intro-everyday_knot_closed} and
playing with it should convince you that the knot can't be removed. We call
these closed up versions of knots \textbf{mathematical knots}.

\begin{figure}[H]
  \centering
  \includegraphics[width=0.8\linewidth]{files/everyday_knot_closed-dc24f338fa5dd2633dcba3c6e31156ee.pdf}
  \caption[Closing the ends of the everyday knot.]{Closing the ends
    of the everyday knot in Figure~\ref{fig-intro-everyday_knot} to form a
  mathematical knot.}
  \label{fig-intro-everyday_knot_closed}
\end{figure}

With this, when compared to an everyday knot, we are already in a better place.
You may discover that then gluing the ends yields knots that can't be maneuvered
to look the same as Figure~\ref{fig-intro-everyday_knot_closed} or
Figure~\ref{fig-intro-circle}. Continuing to experiment with your physical
model, twisting the ends of the string around each other in different ways
(Figure~\ref{fig-intro-torus}).
\begin{figure}[H]
  \centering
  \includegraphics[width=0.3\linewidth]{files/circle-5a8ae67a7c8a7a4fc197ac67fe7ba70a.pdf}
  \caption[A representation of a simple loop.]{A representation of a
  simple loop made from gluing the ends of a rope together.}
  \label{fig-intro-circle}
\end{figure}

In fact, this twisting gives us a whole infinite family of odd crossing torus
knots, a sampling of which can be seen in Figure~\ref{fig-intro-torus_knots}.

\begin{figure}[H]
  \centering
  \includegraphics[width=0.5\textwidth]{files/torus_physical-fa039c19edb7484321bde650f450164b.pdf}
  \caption[A representation of twisting the ends of a rope.]{A
  representation of twisting the ends of a rope around itself before gluing.}
  \label{fig-intro-torus}
\end{figure}

\begin{figure}[H]
  \includegraphics[width=\textwidth]{files/torus-689573cf371cfec2180887a2654c1f43.pdf}
  \caption[Three knots built from twisting.]{Three knots built from
    the operation seen in Figure~\ref{fig-intro-torus}. From left
  to right: three, five, and seven crossing torus knots.}
  \label{fig-intro-torus_knots}
\end{figure}

Our experimentation with the physical model may conjure some questions, three
important questions we may ask are:

\begin{enumerate}
  \item ``How do I systematically construct knots?''
  \item ``How do I tell two knots I make apart?''
  \item ``How do I generate new knots?''
\end{enumerate}

Attempting to answer these questions, even just in part or with restrictions, is
the bread and butter of knot theory and the focus of the rest of this thesis.
Some great texts for continued reading on knot theory in order of accessibility
are: The Knot Book: An Elementary Introduction to the Mathematical Theory of
Knots by Adams \citep{adamsKnotBookElementary2004}, LinKnot: Knot
Theory by Computer
by Jablan and Sazdanovi{\'c} \citep{jablanLinKnotKnotTheory2007},
Knots and Links by
Rolfsen \citep{rolfsenKnotsLinks2003}, and An Introduction to Knot Theory by
Lickerish \citep{lickorishIntroductionKnotTheory1997}.

The remainder of this thesis quickly exchanges the idea of a knot for that of a
tangle, introduced by Conway \citep{conwayEnumerationKnotsLinks1970}.
A tangle can be
thought of as slamming a cookie cutter onto a knot diagram pinning down four
points. Once the knot is pinned down we then cut off the parts of the knot
laying outside the cookie cutter, seen in Figure~\ref{fig-intro-tangle_maker}.

\begin{figure}[H]
  \centering
  \includegraphics[width=\linewidth]{files/tangle_maker-d959b9e7c360bb1bd923d6177c09c044.pdf}
  \caption[The creation of a tangle from a knot.]{The creation of a
    tangle from a knot by cutting a section out of the knot but
  fixing four points.}
  \label{fig-intro-tangle_maker}
\end{figure}

%  prettier-ignore-start

\section{Brief Discussion on Applications}\label{sec-intro-applications}

%  prettier-ignore-end

As we saw in the Section~\ref{sec-intro-intuit_knot_theory},
mathematical knots can be easily constructed as physical objects. It should be
no surprise then that mathematical knots and tangles appear in the hard
sciences, particularly in the realms of physics, chemistry, and biology. In this
section, we will discuss one of the most commonly discussed applications of knot
theory.

\subsection{Tangles in DNA}

One of the fundamental features that identifies life as life is the ability to
self-replicate. In order to self-replicate, life must have a mechanism to pass
information to successive generations. Consider first the most basic self
replicating component of life, called a cell. The information of a cell is
stored as double-stranded DNA (dsDNA), as described by Crick, Franklin, Gosling,
and Watson
\citep{watsonMolecularStructureNucleic1953,
franklinMolecularConfigurationSodium1953},
a polymer consisting of two strands constructed from a sugar-phosphate connected
to one of the monomers \citep{watsonMolecularStructureNucleic1953}:

\begin{enumerate}
  \item Adenine (A)
  \item Guanine (G)
  \item Cytosine (C)
  \item Thymine (T)
\end{enumerate}

\begin{figure}[H]
  \centering
  \includegraphics[width=0.4\linewidth]{files/DNA_chemical_structu-d5653e324ba48edb00fc63567a2cf1c0.pdf}
  \caption[A schematic diagram demonstrating the structure of a dsDNA
  polymer.]{A schematic diagram demonstrating the structure of a
    dsDNA polymer (Ball CC BY-SA 2.5 via Wikimedia
  Commons\citep{madeleinepriceballDNAChemicalStructure2007})}
  \label{fig-intro-dna_chemical_structure}
\end{figure}

The two strands of dsDNA connect to each other to form the
``double-helix'' where
the monomers bind to each other with guanine (G) binding to cytosine (C), and
adenine (A) binding to thymine (T) \citep{watsonMolecularStructureNucleic1953}
(Figure~\ref{fig-intro-dna_chemical_structure}). When replicating,
the cell will duplicate
the dsDNA by splitting the dsDNA into two single strands with the enzyme DNA
helicase. Once the DNA is split, two new complementary single strands are
constructed to be paired with each of the original single strands.

\begin{figure}[H]
  \centering
  \includegraphics[width=\linewidth]{files/DNA_replication_en-7ba25b2f51c348cb7d031e6a172d6abd.pdf}
  \caption[A schematic diagram demonstrating the splitting of a
    double strand of DNA into
  two new double strands.]{A schematic diagram demonstrating the
    splitting of a double strand of DNA into
    two new double strands. (Ruiz Public domain via Wikimedia
  Commons\citep{ruizDNAReplicationFork})}
  \label{fig-intro-dna_split}
\end{figure}

The dsDNA of a cell needs to be physically stored inside the cell. Cell volume
is limited, so organizing the dsDNA to fit in that volume requires several
complex cellular mechanisms. One issue solved by these mechanisms is that of
local knotting, which becomes a problem when the cell attempts to replicate.
During the replication process, the DNA helicase begins splitting the dsDNA, but
when the DNA helicase reaches the locally knotted portion, it becomes stuck, the
replication will be unable to continue
\citep{albertsMolecularBiologyCell2022}, and
the cell will die. If this local knotting is allowed to happen unchecked, every
cell would eventually be unable to replicate and would ultimately die. Famously,
life finds a way, and one cellular mechanism that mitigates this local knotting
problem is the enzyme type II
topoisomerase\citep{albertsMolecularBiologyCell2022}.

The enzyme attempts to solve this local knotting by cutting one of
the dsDNA where
two segments cross, then moving the top double strand to the bottom, this action
can be seen in Figure~\ref{fig-intro-topo}.

\begin{figure}[H]
  \begin{subfigure}[t]{.35\textwidth}
    \centering
    \includegraphics[width=\textwidth]{files/Gyrase_structure_Dmi-1d125d277339d7c31499be4e7766a1ef.png}
    \caption[A schematic diagram of the enzyme type II
    topoisomerase.]{A schematic diagram of the enzyme type II
      topoisomerase. (Sutor CC BY-SA 4.0 via Wikimedia
    Commons\citep{sutorSchemeDNAGyrase})}
    \label{fig-intro-topo2schem}
  \end{subfigure}
  \hfill
  \centering
  \begin{subfigure}[t]{.6\textwidth}
    \centering
    \includegraphics[width=\textwidth]{files/topo3-31473ad625ab67bf8b08a2069a9bf0fb.pdf}
    \caption[Type II topoisomerase doing a crossing exchange.]{Type
      II topoisomerase doing a crossing exchange. From top left to bottom
  right: 1) A crossing of two dsDNA segments. 2) The enzyme grabs the
  under segment.
3) The enzyme splits the under segment. 4) The enzyme passes the over segment
through the gap. 5) The crossing with the segments exchanged.}
\label{fig-intro-topo}
\end{subfigure}
\caption[Schematic diagrams of the enzyme type II
topoisomerase.]{Schematic diagrams of the enzyme type II topoisomerase.}
\end{figure}

In mammals (and many other animal groups), dsDNA takes the form of long strings,
which can only become everyday knots. However, it was discovered by Dulbecco and
Vogt\citep{dulbeccoEVIDENCERINGSTRUCTURE1963,
weilCYCLICHELIXCYCLIC1963, vinogradTwistedCircularForm1965}
that in some viruses (Polyoma) the dsDNA is a closed loop, allowing it to form
into a mathematical knot (Figure~\ref{fig-intro-circ_dna_fig}) .

\begin{figure}
\begin{subfigure}[t]{.45\textwidth}
\includegraphics[width=\textwidth]{files/dna_circle-64775fa5ef9cccacd98295c0fecc35ba.pdf}
\caption[A schematic diagram of circular dsDNA.]{A schematic diagram
of circular dsDNA.}
\label{fig-intro-dna_circle}
\end{subfigure}
~
\begin{subfigure}[t]{.45\textwidth}
\centering
\includegraphics[width=\textwidth]{files/sem_knot-dee1106bb3b0e30c06d0f9d136d6f76f.png}
\caption[A scanning electron microscope image of knotted dsDNA.]{A
scanning electron microscope image of knotted dsDNA (Arsuaga CC
BY-ND\citep{arsuagaDNAKnotSeen2013}).}
\label{fig-intro-sem_knot-png}
\end{subfigure}
\caption[Schematic diagrams of circular and knotted dsDNA.]{Schematic
diagrams of circular and knotted dsDNA.}
\label{fig-intro-circ_dna_fig}
\end{figure}

From here one may ask, ``If the dsDNA is knotted and type II topoisomerase makes
a change, what kind of new knot can this make?'', this question was addressed
first by Ernst and Sumners in the 1990s \citep{ernstCalculusRationalTangles1990}
\citep{ernstTANGLEEQUATIONS1996}. Their approach considers the dsDNA
to be bounded by
two ``areas'', the first area is created by drawing a circle around the crossing
that type II topoisomerase is working on (right side of
Figure~\ref{fig-intro-tangle_equation_start}), and the second by
drawing a circle around the
remainder (left side of
Figure~\ref{fig-intro-tangle_equation_start}). From here, the
crossing change from type II topoisomerase can be modeled by changing the tangle
bound in the area on the right (Figure~\ref{fig-intro-tangle_calc}),
\citep{darcy3DVisualizationSoftware2008}.

\begin{note}
After the change, there may be many crossings that type II
topoisomerase could ``choose'' to work on next. A program
like TopoICE-X \citep{darcy3DVisualizationSoftware2008} (built into KnotPlot
\citep{schareinInteractiveTopologicalDrawing1998}) can help visualize
the results of
making these choices.
\end{note}

\begin{figure}
\begin{subfigure}[t]{.45\textwidth}
\centering
\includegraphics[width=\textwidth]{files/tangle_equation_star-893577b81dd84ec6fe27e7f4781c5418.pdf}
\caption[A knot diagram showing two areas containing knot data.]{A
knot diagram showing two areas containing knot data. The right side contains
the crossing that type II topoisomerase will work on.}
\label{fig-intro-tangle_equation_start}
\end{subfigure}
~
\begin{subfigure}[t]{.45\textwidth}
\centering
\includegraphics[width=\textwidth]{files/tangle_equation_resu-e4d9cd036ada95414b3b525d38b63b2d.pdf}
\caption[A knot diagram showing two areas containing knot data.]{A
knot diagram showing two areas containing knot data. The right side contains
the crossing that type II topoisomerase has worked on.}
\label{fig-intro-tangle_equation_result}
\end{subfigure}
\caption[A tangle model for a crossing change in a knot.]{A tangle
model for a crossing change in a knot.}
\label{fig-intro-tangle_calc}

\end{figure}
The modeling of the action of type II topoisomerase is just one of the many
applications for the theory of knots and tangles found in biology. For further
reading on applications, a good source is the ``Encyclopedia of knot
theory''\citep{adamsEncyclopediaKnotTheory2021}, a survey of many
subdisciplines of
knot theory with a chapter devoted to applications.

%  prettier-ignore-start

\section{Overview of this Thesis}\label{sec-intro-overview}

%  prettier-ignore-end

As discussed in the previous section, the goal of this thesis is to address, in
part, the three questions ``How do I systematically construct
knots?'', ``How do I
tell two knots I make apart?'', and ``How do I generate new knots?''
Particularly,
we will address a restricted version of these questions, for objects that can be
thought of as the building blocks of knots, the tangles introduced by Conway
\citep{conwayEnumerationKnotsLinks1970}. Through the
middle(Chapter~\ref{ch-tabulation} and Chapter~\ref{ch-software}) of
this thesis, we will
describe several strategies that have been employed to answer even further
restricted versions of these questions:

\begin{itemize}
\item ``How do I systematically construct rational tangles?'', ``How
do I tell two
rational tangles I make apart?'', and ``How do I generate new rational
tangles?''
\item ``How do I systematically construct Montesinos tangles?'' and
``How do I tell
two Montesinos tangles I make apart?'', and ``How do I generate new Montesinos
tangles?''
\item ``How do I systematically construct algebraic/arborescent
tangles?'' and ``How
do I tell two algebraic/arborescent tangles I make apart?'', and ``How do I
generate new algebraic/arborescent tangles?''
\end{itemize}

While we could answer these questions about tangles with pen and
paper brute force, as Tait, Little, Kirkman, Conway, and Caudron
\cite{taitTenfoldKnottiness1885,
kirkmanEnumerationDescriptionConstruction1885,
littleKnotsCensusOrder1885,conwayEnumerationKnotsLinks1970,caudron1982classification}
did for knots.  Beyond a
reasonable ``crossing number'' (Section~\ref{subsec-knot_def}), as
small as 8, the time
and effort needed makes pen and paper untenable. To achieve our goal in a
reasonable time, we will follow a similar framework to that of Hoste,
Thistlethwaite, Weeks, and Burton \citep{hosteFirst1701936Knots1998,
hosteEnumerationClassificationKnots2005,burtonNext350Million2020}.
Utilizing computer methods to generate a tangle table of algebraic/arborescent
tangles to twelve crossings. To effectively utilize computers for this
tabulation work requires the design and implementation of a knot theory specific
software toolbox.

\subsection{Chapter Summary}

We will now summarize each of the chapters of this thesis. Organizationally,
this thesis is partitioned into five chapters, with each chapter further divided
into sections, and subsections.

\subsubsection{Chapter 1: Introduction}

This introduction gave an intuitive description of the basics of knot theory,
and a discussion of an application for knots and tangles. We now finish with
this description of the content in the remainder of this thesis.

\subsubsection{Chapter 2: Preliminaries}

This chapter gives the preliminaries in knot and tangle theory needed
for the rest of the
thesis. Included are the historical background of tabulation in knot theory and
a grounding in knot and tangle theory. We will see definitions for knots and
tangles, as well as some notations and invariants for those knots and tangles.

\subsubsection{Chapter 3: Tabulation}

This chapter describes the theoretical methodology for the tabulation of
algebraic tangles. It is divided into two sections, the first
contains the method used for
tabulation of two well-understood classes of tangles: the rational and
Montesinos tangles. The second describes the methodology for the tabulation of
the more general type of tangle the arborescent (algebraic) tangles. For each
class of tangle we tabulated, a definition and classification is given, followed
by a theoretical generation strategy.

\subsubsection{Chapter 4: Software and its Engineering}

This chapter addresses the computational and engineering aspects of software for
mathematics research. The chapter begins with an overview of product management
and software engineering practices. In our discussion of software engineering we
develop a process designed for use by professional and undergraduate researchers
in computational knot theory. With our process, we create a product design for a
general purpose ``knot theory software toolbox''. The chapter concludes with the
software design (unit description) for the tools developed to realize the
solutions in the tabulation section.

\subsubsection{Chapter 5: Future Directions}

The final chapter of this thesis gives an overview of future work to be done in
the tangle tabulation domain. This takes two forms, first the direct next steps
for tangle tabulation at the professional research level. Second, we outline a
collection of topics and research directions at various levels appropriate for
undergraduate researchers.
